\documentclass[10 pt,usenames,dvipsnames, oneside]{article}
\usepackage{../../modelo-fracoes}
\graphicspath{{../../../Figuras/licao05/}}


\begin{document}

\begin{center}
  \begin{minipage}[l]{3cm}
\includegraphics[width=2cm]{../../../Figuras/logo}       
\end{minipage}\hfill
\begin{minipage}[r]{.8\textwidth}
 {\Large \scshape Atividade: }  
\end{minipage}
\end{center}
\vspace{.2cm}

\ifdefined\prof
%Caixa do Para o Professor
\begin{goals}
%Objetivos específicos
\begin{enumerate}
\item       Aplicar a ideia de obter um denominador comum entre duas frações dadas, com base no processo geométrico de subdivisão da unidade, em exercícios sem uma situação contextualizada e sem uma representação pictórica  previamente apresentada, ficando para o aluno construir tal representação.
\end{enumerate}

\tcblower

%Orientações e sugestões
\begin{itemize}
\item       Embora não sejam dadas situações contextualizadas, procure conduzir esta atividade com base em representações geométricas para as frações dadas e na determinação de uma subdivisão comum a partir dessas representações, como nas atividades 2 a 6. O objeto é justamente aplicar as ideias construídas a partir daquelas atividades em exercícios sem situações contextualizadas.
    \item  Nos casos que envolvem o número 1, deve-se relembrar   $1 = \frac{n}{n}$, qualquer que seja o número natural $n$.
\end{itemize}
\end{goals}

\bigskip
\begin{center}
{\large \scshape Atividade}
\end{center}
\fi

Em cada um dos itens a seguir, escreva frações iguais às frações dadas que tenham mesmo denominador. Para cada par de frações, destaque a subdivisão escolhida da unidade para determinar o denominador comum e represente essa subdivisão por meio de um desenho.

\begin{center}
  \begin{tabular}{m{0.25\textwidth}m{0.25\textwidth}m{0.25\textwidth}}

     a) $\frac{1}{3}$ e $\frac{2}{9}$  &   b) $\frac{3}{10}$ e $\frac{4}{5}$  &   c) 1 e $\frac{3}{7}$  \\
     \\
     d) $\frac{3}{5}$ e $\frac{8}{3}$  &   e) $\frac{7}{8}$ e $\frac{13}{12}$  &  f) $\frac{7}{4}$ e 5
  \end{tabular}
\end{center}

\ifdefined\prof
\begin{solucao}

  São respostas possíveis:
\begin{enumerate} 
    \item             $\frac{3}{9}$       e       $\frac{2}{9}$.        Subdivisão escolhida:       $\frac{1}{9}$       da unidade.
    \item             $\frac{3}{10}$       e       $\frac{8}{10}$.      Subdivisão escolhida:       $\frac{1}{10}$       da unidade.
    \item             $\frac{7}{7}$       e       $\frac{3}{7}$.        Subdivisão escolhida:       $\frac{1}{7}$       da unidade.
    \item             $\frac{9}{15}$       e       $\frac{40}{15}$.   Subdivisão escolhida:       $\frac{1}{15}$       da unidade.
    \item             $\frac{21}{24}$       e       $\frac{26}{24}$.    Subdivisão escolhida:       $\frac{1}{24}$       da unidade.
    \item             $\frac{7}{4}$       e       $\frac{20}{4}$.       Subdivisão escolhida:       $\frac{1}{4}$       da unidade.
\end{enumerate} %s


  Observação: Todos esses itens admitem outras respostas, uma vez que é possível escolher diferentes subdivisões da unidade, ou seja, outras fraçoes unitárias. Por exemplo, no item (e) temos como outra solução possível:   $\frac{42}{48}$   e   $\frac{52}{48}$. Subdivisão escolhida:   $\frac{1}{48}$   da unidade..

\end{solucao}
\fi

\end{document}