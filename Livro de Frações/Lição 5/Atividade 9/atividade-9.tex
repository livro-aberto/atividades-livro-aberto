\documentclass[10 pt,usenames,dvipsnames, oneside]{article}
\usepackage{../../modelo-fracoes}
\graphicspath{{../../../Figuras/licao05/}}


\begin{document}

\begin{center}
  \begin{minipage}[l]{3cm}
\includegraphics[width=2cm]{../../../Figuras/logo}       
\end{minipage}\hfill
\begin{minipage}[r]{.8\textwidth}
 {\Large \scshape Atividade: }  
\end{minipage}
\end{center}
\vspace{.2cm}

\ifdefined\prof
%Caixa do Para o Professor
\begin{goals}
%Objetivos específicos
\begin{enumerate}
\item     Relacionar a adição de frações com a sua representação como pontos na reta.
\end{enumerate}

\tcblower

%Orientações e sugestões
\begin{itemize}
\item     Esta atividade, assim como as atividades 10 e 11,   usam a ideia de que     $1 = \frac{n}{n}$, ou de forma mais geral, de que, se     $a$     é um número natural, então     $a = \frac{an}{n}$, para     $n$     diferente de 0.
  \item Essas atividades envolvem os chamados ''números mistos'' (números expressos por uma parte inteira e uma parte fracionária). No entanto, {\bf não há necessidade de apresentar essa nomenclatura aos alunos.}
\item A representação da reta na posição vertical foi emprega com o objetivo de destacar o fato de que os aspectos determinantes nesta forma de representação são a ordenação e a distância entre os pontos. A apresentação da reta numérica apenas na posição horizontal pode causar uma impressão de que apenas tal posição é aceitável.
\end{itemize}
\end{goals}

\bigskip
\begin{center}
{\large \scshape Atividade}
\end{center}
\fi

Miguel deseja calcular a soma $2 + \frac{1}{3}$. Para isso, marcou na reta numérica um ponto determinado pela justaposição do segmento correspondente a $2$ unidades com um segmento igual a $\frac{1}{3}$ da unidade, como na figura abaixo.

Miguel relacionou essa estratégia com o seguinte cálculo:
$$ 2 + \frac{1}{3} =  \frac{6}{3} + \frac{1}{3} = \frac{7}{3}$$


\begin{center}
 \begin{tikzpicture}[x=17mm,y=17mm]
  \draw[->] (0,-.25) -- (0,3.25);
  \foreach \x in {0,...,3}{
  \draw (-3pt,\x)--(3pt,\x);
  \node at (-7pt,\x) {\x};}
 \foreach \x in {2+1/3,2+2/3}\draw (-2pt,\x)--(2pt,\x);
 \draw[|-|] (9pt,2) -- (9pt,2+1/3);
 \node at (20pt,2+1/6) {$\dfrac{1}{3}$};
 \draw[->] (-35pt,2+1/3) -- (-9pt,2+1/3);
 \node at (-1.1,2+1/3) {$2 + \dfrac{1}{3}$};
 \fill[common] (0,2+1/3) circle (3pt);
 \draw[dotted] (9 pt, 2+1/3) -- (1.9, 2+1/3);

 \begin{scope}[shift={(2,0)}]
 %reta numerica vertical
 \draw[->] (0,-.25) -- (0,3.25);
  \foreach \x in {0,...,3}{
  \draw (-3pt,\x)--(3pt,\x);
  \node at (-7pt,\x) {\x};}
\foreach \x in {0,.3333,...,2.6666}\draw (-2pt,\x)--(2pt,\x);
  \fill[common] (0,2+1/3) circle (3pt);

% segmentos de 1/3 ao lado da reta

\foreach \x in {1,...,6}{
\draw[|-|] (9pt,\x/3+.01) -- (9pt,\x/3+1/3-.01);
\node at (20pt,\x/3+1/6) {{\small $\frac{1}{3}$}};}
\draw[|-|] (9pt,0) -- (9pt,1/3-.01);
\node at (20pt,1/6) {{\small $\frac{1}{3}$}};

%flecha e texto.
\draw[<-]  (30 pt, 2+1/6) -- (56pt, 2+1/6);
\node at (90pt, 2+1/6) {1 fração de $\dfrac{1}{3}$};
\node at (90pt, 1.6) {$+$};
\node at (90pt, 1) {6 frações de $\dfrac{1}{3}$};
%linhas tracejadas e chave
\foreach \x in {0,2} \draw[dotted] (25pt,\x) -- (45pt,\x);
\draw [thick, decoration={brace,mirror,raise=5}, decorate] (45pt,0) -- (45pt,2);
 \end{scope}
 \end{tikzpicture}
\end{center}

\ifdefined\prof
\clearpage
\else
\fi

\begin{enumerate}
 \item Em cada item a seguir, a partir da imagem repita o procedimento feito por Miguel e realize os cálculos.

\begin{center}
\scalebox{.85}
{
\begin{tabular}{>{\centering}m{.3\textwidth}>{\centering}m{.3\textwidth}>{\centering}m{.3\textwidth}}
 (A) & (B) & (C)\tabularnewline

 \begin{tikzpicture}[x=17mm,y=17mm]
  \draw[->] (0,-.5) -- (0,4.5);
  \foreach \x in {0,...,4}{
  \draw (-3pt,\x)--(3pt,\x);
  \node at (-7pt,\x) {\x};}
 \foreach \x in {3.25,3.5,3.75}\draw (-2pt,\x)--(2pt,\x);
 \fill[common] (0,3.25) circle (3pt);

 % setinha e texto
 \draw[->, overlay] (-35pt,3.25) -- (-9pt,3.25);
 \node[overlay] at (-1.1,3.25) {$3 + \dfrac{1}{4}$};

 \end{tikzpicture}
&

 \begin{tikzpicture}[x=17mm,y=17mm]
  \draw[->] (0,-.5) -- (0,5.5);
  \foreach \x in {0,...,5}{
  \draw (-3pt,\x)--(3pt,\x);
  \node at (-7pt,\x) {\x};}
 \draw (-2pt,4.5)--(2pt,4.5);
 \fill[common] (0,4.5) circle (3pt);

 % setinha e texto
 \draw[->, overlay] (-35pt,4.5) -- (-9pt,4.5);
 \node [overlay] at (-1.1,4.5) {$4 + \dfrac{1}{2}$};
  \end{tikzpicture}
 &
 \begin{tikzpicture}[x=17mm,y=17mm]
  \draw[->] (0,-.5) -- (0,3.5);
  \foreach \x in {0,...,3}{
  \draw (-3pt,\x)--(3pt,\x);
  \node at (-7pt,\x) {\x};}
 \draw (-2pt,2.6)--(2pt,2.6);
 \foreach \x in {2.2,2.4,...,2.8}\draw (-2pt,\x)--(2pt,\x);
 \fill[common] (0,2.6) circle (3pt);

 % setinha e texto
 \draw[->, overlay] (-35pt,2.6) -- (-9pt,2.6);
 \node [overlay] at (-1.1,2.6) {$2 + \dfrac{3}{5}$};

 \end{tikzpicture}
\end{tabular}
}
\end{center}

 \item Que valor é obtido se juntarmos 7 inteiros com dois terços?
\end{enumerate}

\ifdefined\prof
\begin{solucao}

\begin{enumerate}

\item
\adjustbox{valign=t}
{
\scalebox{.85}
{
\begin{tabular}{>{\centering}m{.3\textwidth}>{\centering}m{.3\textwidth}>{\centering}m{.3\textwidth}}
 (A) & (B) & (C)\tabularnewline

 \begin{tikzpicture}[x=17mm,y=17mm]
  \draw[->] (0,-.5) -- (0,4.5);
  \foreach \x in {0,...,4}{
  \draw (-3pt,\x)--(3pt,\x);
  \node at (-7pt,\x) {\x};}
 \foreach \x in {0.25,0.5,...,3.25}\draw (-2pt,\x)--(2pt,\x);
 \fill[common] (0,3.25) circle (3pt);

 % setinha e texto
 \draw[->, overlay] (-20pt,3.25) -- (-9pt,3.25);
 \node [overlay] at (-.8,3.25) {$3 + \dfrac{1}{4}$};




\foreach \x in {0,3} \draw[dotted, overlay] (10pt,\x) -- (20pt,\x);
\draw [thick, decoration={brace,mirror,raise=5}, decorate, overlay] (25pt,0) -- (25pt,3);
\node [overlay] at (70pt,1.5) {12 frações de $\frac{1}{4}$};

 \end{tikzpicture}
&

 \begin{tikzpicture}[x=17mm,y=17mm]
  \draw[->] (0,-.5) -- (0,5.5);
  \foreach \x in {0,...,5}{
  \draw (-3pt,\x)--(3pt,\x);
  \node at (-7pt,\x) {\x};}
 \foreach \x in {0.5,...,4.5}\draw (-2pt,\x)--(2pt,\x);
 \fill[common] (0,4.5) circle (3pt);


\foreach \x in {0,4} \draw[dotted, overlay] (10pt,\x) -- (20pt,\x);
\draw [thick, decoration={brace,mirror,raise=5}, decorate, overlay] (25pt,0) -- (25pt,4);
\node [overlay] at (70pt,2) {8 frações de $\frac{1}{2}$};


 % setinha e texto
 \draw[->, overlay] (-20pt,4.5) -- (-9pt,4.5);
 \node [overlay] at (-.8,4.5) {$4 + \dfrac{1}{2}$};
 \end{tikzpicture}

 &
 \begin{tikzpicture}[x=17mm,y=17mm]
  \draw[->] (0,-.5) -- (0,3.5);
  \foreach \x in {0,...,3}{
  \draw (-3pt,\x)--(3pt,\x);
  \node at (-7pt,\x) {\x};}
 \foreach \x in {0.2,.4,...,2.6} \draw (-2pt,\x)--(2pt,\x);
 \fill[common] (0,2.6) circle (3pt);

\foreach \x in {0,2} \draw[dotted, overlay] (10pt,\x) -- (20pt,\x);
\draw [thick, decoration={brace,mirror,raise=5}, decorate, overlay] (25pt,0) -- (25pt,2);
\node [overlay] at (70pt,1) {10 frações de $\frac{1}{5}$};


 % setinha e texto
 \draw[->, overlay] (-20pt,2.6) -- (-9pt,2.6);
 \node [overlay] at (-.8,2.6) {$2 + \dfrac{3}{5}$};

 \end{tikzpicture}
\end{tabular}
}
}

\item Repetindo o mesmo processo do item a) obtém-se $7 + \frac{2}{3} = \frac{21}{3} + \frac{2}{3} = \frac{23}{3}$.
\end{enumerate}


\end{solucao}
\fi

\end{document}