\documentclass[10 pt,usenames,dvipsnames, oneside]{article}
\usepackage{../../modelo-fracoes}
\graphicspath{{../../../Figuras/licao05/}}


\begin{document}

\begin{center}
  \begin{minipage}[l]{3cm}
\includegraphics[width=2cm]{../../../Figuras/logo}       
\end{minipage}\hfill
\begin{minipage}[r]{.8\textwidth}
 {\Large \scshape Atividade: }  
\end{minipage}
\end{center}
\vspace{.2cm}

\ifdefined\prof
%Caixa do Para o Professor
\begin{goals}
%Objetivos específicos
\begin{enumerate}
\item       Aplicar as ideias de obter um denominador comum entre duas frações dadas e de usar esse denominador para determinar adições e subtrações, com base no processo geométrico de subdivisão da unidade, em exercícios sem uma situação contextualizada.
\end{enumerate}

\tcblower

%Orientações e sugestões
\begin{itemize}
\item       Como na atividade anterior, embora não sejam dadas situações contextualizadas, procure conduzir esta atividade com base em representações geométricas para as frações dadas e na determinação de uma subdivisão comum a partir dessas representações, como nas atividades 2 a~6.
    \item  Nos casos que envolvem o número 1, deve-se relembrar   $1 = \frac{n}{n}$.
\end{itemize}
\end{goals}

\bigskip
\begin{center}
{\large \scshape Atividade}
\end{center}
\fi

Em cada um dos itens a seguir, faça a conta e uma ilustração que explique a maneira como você realizou o cálculo solicitado.

\begin{center}
  \begin{tabular}{m{0.25\textwidth}m{0.25\textwidth}m{0.25\textwidth}}
     a) $\frac{1}{3} - \frac{2}{9}$  &   b) $\frac{3}{10} + \frac{4}{5}$  &   c) $1 - \frac{3}{7}$
  \end{tabular}
\end{center}

\ifdefined\prof
\begin{solucao}

\begin{enumerate}
    \item             $\frac{1}{3} - \frac{2}{9} = \frac{3}{9} - \frac{2}{9} = \frac{1}{9}$.
    \item             $\frac{3}{10}+\frac{4}{5} = \frac{3}{10}+\frac{8}{10} =\frac{11}{10}$.
    \item             $1 - \frac{3}{7} = \frac{7}{7} - \frac{3}{7} = \frac{4}{7}$.
\end{enumerate}

\end{solucao}
\fi

\end{document}