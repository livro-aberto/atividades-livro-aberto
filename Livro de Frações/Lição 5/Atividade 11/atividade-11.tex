\documentclass[10 pt,usenames,dvipsnames, oneside]{article}
\usepackage{../../modelo-fracoes}
\graphicspath{{../../../Figuras/licao05/}}


\begin{document}

\begin{center}
  \begin{minipage}[l]{3cm}
\includegraphics[width=2cm]{../../../Figuras/logo}       
\end{minipage}\hfill
\begin{minipage}[r]{.8\textwidth}
 {\Large \scshape Atividade: }  
\end{minipage}
\end{center}
\vspace{.2cm}

\ifdefined\prof
%Caixa do Para o Professor
\begin{goals}
%Objetivos específicos
\begin{enumerate}
    \item       Determinar uma subtração de frações com a interpretação de completar;
    \item       Explorar a articulação entre número misto e subtração de frações.
\end{enumerate}

\tcblower

%Orientações e sugestões
\begin{itemize}
    \item       Como na \hyperref[chap5-ativ10]{atividade anterior}, observar que a visualização da representação na reta pode ajudar a destacar o fato de que se deve determinar       ``quanto falta''       de       $\frac{19}{7}$       para chegar a 2.
\end{itemize}
\end{goals}

\bigskip
\begin{center}
{\large \scshape Atividade}
\end{center}
\fi

Qual é o maior número, $\frac{19}{7}$ ou $2$? Quanto se deve acrescentar ao menor número para chegar ao maior?

\ifdefined\prof
\begin{solucao}

$\frac{19}{7} > \frac{14}{7} = 2$. Portanto, $\frac{19}{7}$ é maior e deve-se acrescentar 5/7 ao menor número para que o total se iguale ao maior número.

\end{solucao}
\fi

\end{document}