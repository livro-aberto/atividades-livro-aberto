\documentclass[10 pt,usenames,dvipsnames, oneside]{article}
\usepackage{../../modelo-fracoes}
\graphicspath{{../../../Figuras/licao05/}}


\begin{document}

\begin{center}
  \begin{minipage}[l]{3cm}
\includegraphics[width=2cm]{../../../Figuras/logo}       
\end{minipage}\hfill
\begin{minipage}[r]{.8\textwidth}
 {\Large \scshape Atividade: }  
\end{minipage}
\end{center}
\vspace{.2cm}

\ifdefined\prof
%Caixa do Para o Professor
\begin{goals}
%Objetivos específicos
\begin{enumerate}
    \item       Explorar a formulação de conjecturas envolvendo a estrutura algébrica dos conjuntos numéricos, visamos atingir não só reflexões a respeito de números racionais, mas também estimular a habilidade de argumentação em Matemática.
\end{enumerate}

\tcblower

%Orientações e sugestões
\begin{itemize}
    \item       Neste momento, não se espera ainda que os alunos justifiquem com rigor suas afirmações, mas sim que busquem ilustrar suas conjecturas a partir de exemplos.
    \item       Recomenda-se que o professor discuta cada item a partir das soluções dos alunos, destacando as respostas corretas com base nos exemplos propostos pelos estudantes
\end{itemize}
\end{goals}

\bigskip
\begin{center}
{\large \scshape Atividade}
\end{center}
\fi

Diga se as afirmações a seguir são verdadeiras ou falsas. Para as verdadeiras, explique com as suas palavras por que acha que são verdadeiras. Para as falsas, dê um exemplo que justifique a sua avaliação.
\begin{enumerate}
  \item     A soma de um número inteiro com uma fração não inteira pode sempre ser expressa por um número inteiro.
  \item     A diferença entre um número inteiro e uma fração não inteira pode sempre ser expressa por um número inteiro.
  \item     A soma de uma fração não inteira com uma fração não inteira é, necessariamente, uma fração não inteira.
  \item     A diferença entre uma fração não inteira e uma fração não inteira é, necessariamente, uma fração não inteira.
\end{enumerate}

\ifdefined\prof
\begin{solucao}

\begin{enumerate}
    \item       Falso. Exemplo: $3 + \frac{2}{5} = \frac{15}{5}+\frac{2}{5} = \frac{17}{5}$.  Há outras possibilidades de respostas.
    \item       Falso. Exemplo:       $7 - \frac{3}{4} = \frac{25}{4}$.
    \item       Falso. Exemplo:       $\frac{11}{6} + \frac{7}{6} = \frac{18}{6} = 3$.
    \item       Falso. Exemplo:       $\frac{3}{2} - \frac{1}{2} = \frac{2}{2} = 1$.
\end{enumerate}

\end{solucao}
\fi

\end{document}