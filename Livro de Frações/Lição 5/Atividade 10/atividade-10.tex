\documentclass[10 pt,usenames,dvipsnames, oneside]{article}
\usepackage{../../modelo-fracoes}
\graphicspath{{../../../Figuras/licao05/}}


\begin{document}

\begin{center}
  \begin{minipage}[l]{3cm}
\includegraphics[width=2cm]{../../../Figuras/logo}       
\end{minipage}\hfill
\begin{minipage}[r]{.8\textwidth}
 {\Large \scshape Atividade: }  
\end{minipage}
\end{center}
\vspace{.2cm}

\ifdefined\prof
%Caixa do Para o Professor
\begin{goals}
%Objetivos específicos
\begin{enumerate}
  \item     Determinar uma subtração de frações com a interpretação de completar.
\end{enumerate}

\tcblower

%Orientações e sugestões
\begin{itemize}
  \item     Explorar o fato de que não é incomum que o uso da palavra
\end{itemize}
\end{goals}

\bigskip
\begin{center}
{\large \scshape Atividade}
\end{center}
\fi

Quanto se deve acrescentar a $\frac{3}{8}$ para que se obtenha $\frac{27}{8}$?


\ifdefined\prof
\begin{solucao}

De 3 oitavos para se alcançar 27 oitavos faltam 24 oitavos, o que equivale a 3. De outro modo, $\frac{27}{8} - \frac{3}{8} = \frac{24}{8} = 3$. Isto indica que deve-se acrescentar a fração $\frac{3}{8}$ a 3 inteiros para obter-se $\frac{27}{8}$.

\end{solucao}
\fi

\end{document}