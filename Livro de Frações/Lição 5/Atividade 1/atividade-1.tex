\documentclass[10 pt,usenames,dvipsnames, oneside]{article}
\usepackage{../../modelo-fracoes}
\graphicspath{{../../../Figuras/licao05/}}


\begin{document}

\begin{center}
  \begin{minipage}[l]{3cm}
\includegraphics[width=2cm]{../../../Figuras/logo}       
\end{minipage}\hfill
\begin{minipage}[r]{.8\textwidth}
 {\Large \scshape Atividade: }  
\end{minipage}
\end{center}
\vspace{.2cm}

\ifdefined\prof
%Caixa do Para o Professor
\begin{goals}
%Objetivos específicos
\begin{enumerate}
\item     Perceber o papel de uma unidade comum para comparar, somar ou subtrair duas quantidades;
  \item     Resgatar as interpretações de juntar para a operação de adição, e de retirar e de comparar para a operação de subtração, anteriormente associadas às operações com números naturais;
  \item     Entender as operações de adição e de subtração com frações como extensões das respectivas operações com números naturais a partir do resgate dessas interpretações, isto é, como operações que dão conta de situações associadas às mesmas interpretações.
\end{enumerate}

\tcblower

%Orientações e sugestões
\begin{itemize}
  \item     Embora não se trabalhe diretamente com frações, a atividade enfoca processos de contagem a partir de uma unidade de referência, o que será fundamental para as operações de adição e de subtração com frações. Por exemplo, no caso da situação apresentada nesta atividade, a unidade comum empregada.
\end{itemize}
\end{goals}

\bigskip
\begin{center}
{\large \scshape Atividade}
\end{center}
\fi

Miguel e Alice estão participando de uma campanha da escola para coleta de óleo de cozinha. O objetivo é disponibilizar recipientes para que as pessoas depositem óleo. Depois esses recipientes serão destinados a empresas que usarão o óleo descartado para fazer sabão. Eles conseguiram diferentes recipientes e agora desejam saber qual tem maior capacidade.

\begin{center}
\begin{tabular}{ccc}
\includegraphics[width=100pt, keepaspectratio]{ativ1_fig01.png} &\quad \quad&\includegraphics[width=100pt, keepaspectratio]{ativ1_fig02.png}\\ \\
{\bf Recipiente 1:} trazido pela Alice & & {\bf Recipiente 2:} trazido pelo Miguel
\end{tabular}
\end{center}

Eles tiveram a seguinte ideia: encheram os dois recipientes com água para depois verificarem onde havia mais água. Para isso, usaram um copo d'água como unidade de medida.
\begin{itemize}
 \item O recipiente trazido por Alice foi enchido com 26 copos.
 \item O recipiente trazido por Miguel foi enchido com 40 copos.
\end{itemize}
Eles então observaram que a partir de {\bf uma unidade de medida comum} (nesse caso o copo), poderiam não só dizer qual recipiente tinha maior capacidade, mas também o quanto era maior e qual seria a capacidade dos dois recipientes juntos.
Usando a ideia de medida de Miguel e Alice, isto é, tomando o copo como unidade de medida, responda:
  \begin{enumerate}
   \item Qual recipiente tem maior capacidade?
   \item Qual é a capacidade dos dois recipientes juntos?
   \item Quanto de água se deve retirar do recipiente maior, para ter o mesmo volume de líquido que é possível colocar no recipiente menor?
  \end{enumerate}

\ifdefined\prof
\begin{solucao}

\begin{enumerate}
  \item O recipiente trazido por Miguel é maior, uma vez que precisou de mais copos para ser enchido ($40>26$).
  \item Usando o copo como unidade de medida, podemos indicar que a capacidade dos dois recipientes juntos é 66. Ou seja, 66 copos.
  \item Deve-se retirar 14 copos, pois $40 - 26=14$.
\end{enumerate}

\end{solucao}
\fi

\end{document}