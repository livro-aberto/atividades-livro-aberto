\documentclass[10 pt,usenames,dvipsnames, oneside]{article}
\usepackage{../../modelo-fracoes}
\graphicspath{{../../../Figuras/licao05/}}


\begin{document}

\begin{center}
  \begin{minipage}[l]{3cm}
\includegraphics[width=2cm]{../../../Figuras/logo}       
\end{minipage}\hfill
\begin{minipage}[r]{.8\textwidth}
 {\Large \scshape Atividade: }  
\end{minipage}
\end{center}
\vspace{.2cm}

\ifdefined\prof
%Caixa do Para o Professor
\begin{goals}
%Objetivos específicos
\begin{enumerate}
\item       Encontrar uma subdivisão comum entre as quantidades que permita efetuar as operações;
    \item       Perceber a não unicidade da subdivisão comum.
\end{enumerate}

\tcblower

%Orientações e sugestões
\begin{itemize}
\item       Como nas atividades anteriores e nas próximas desta lição, o uso obrigatório do MMC não é recomendado. Ao contrário, objetiva-se justamente provocar explicitamente a percepção de que       {\bf essa subdivisão não é única}. Assim, devem ser apresentadas diversas frações equivalentes às frações dadas na atividade, como por exemplo, as seguintes:

  $$\frac{6}{10} \,{\rm e} \, \frac{7}{10}$$
  $$\frac{12}{20} \,{\rm e} \, \frac{14}{20}$$
  $$\frac{24}{40} \,{\rm e} \, \frac{28}{40}$$

    \item       A partir dessas diferentes frações equivalentes, o professor deve procurar       {\bf articular com os estudantes a relação entre diferentes subdivisões com a sistematização de frações equivalentes.}       Deve-se retomar a reflexão iniciada na sessão       {\bf Organizando as Ideias}       de que escrever quantidades em relação a uma subdivisão comum corresponde a determinar frações equivalentes com um denominador comum.
\end{itemize}
\end{goals}

\bigskip
\begin{center}
{\large \scshape Atividade}
\end{center}
\fi

Tendo como unidade um mesmo retângulo, as representações das frações $\frac{3}{5}$ e $\frac{7}{10}$ estão ilustradas nas figuras a seguir.

\begin{center}
\begin{tikzpicture}[scale=4]
\fill[fill=common, fill opacity=.3] (0,0) rectangle (10,5);
\fill[attention] (0,0) rectangle (6,5);
\draw (0,0) rectangle (10,5);
\foreach \x in {2,4,...,8} \draw (\x,0) -- (\x, 5);

\begin{scope}[shift={(14,0)}]
\fill[fill=common, fill opacity=.3] (8,2.5) rectangle (10,5);
\fill[fill=common, fill opacity=.3] (6,0) rectangle (10,2.5);
\fill[light] (0,0) rectangle (6,5);
\fill[light] (6,2.5) rectangle (8,5);
\draw (0,0) rectangle (10,5);
\foreach \x in {2,4,...,8} \draw (\x,0) -- (\x, 5);
\draw (0,2.5) -- (10, 2.5);
\end{scope}
\end{tikzpicture}
\end{center}

\begin{enumerate} %s
  \item     Determine uma subdivisão da unidade que permita expressar essas quantidades por frações com um mesmo denominador. Represente tal subdivisão nas figuras acima.
  \item     Escreva frações iguais a     $\frac{3}{5}$     e a     $\frac{7}{10}$     a partir dessa subdivisão.
  \item     Existe alguma outra subdivisão, diferente da que você usou para responder os itens a) e b), com a qual também seja possível responder ao item b)? Se sim, qual?
  \item     Juntas, as regiões destacadas em vermelho e em bege determinam um região maior, menor ou igual a um retângulo? Explique.
\end{enumerate} %s

\ifdefined\prof
\begin{solucao}

\begin{enumerate}
 \item       Uma possível subdivisão comum é em 10 partes, portanto, igual a fração       $\frac{1}{10}$. Com essa subdivisão ambas as quantidades podem ser expressas por frações de denominador 10. Uma forma de observar tal fato é determinar, na primeira imagem, um segmento horizontal, de modo a dividir cada parte da partição já existente em duas partes iguais.

    \begin{center}
\begin{tikzpicture}[x=1mm,y=1mm,scale=2]
\fill[common,fill opacity=.3] (0,0) rectangle (10,5);
\fill[attention] (0,0) rectangle (6,5);

\draw (0,0) rectangle (10,5);
\foreach \x in {2,4,...,8} \draw (\x,0) -- (\x, 5);
\draw (0,2.5) -- (10,2.5);

\begin{scope}[shift={(14,0)}]
\fill[common,fill opacity=.3] (0,0) rectangle (10,5);
\fill[light] (0,0) rectangle (6,5);
\fill[light] (6,2.5) rectangle (8,5);
\draw (0,0) rectangle (10,5);
\foreach \x in {2,4,...,8} \draw (\x,0) -- (\x, 5);
\draw (0,2.5) -- (10, 2.5);
\end{scope}

\end{tikzpicture}
\end{center}

    \item             $\frac{3}{5} = \frac{6}{10}$. A fração       $\frac{7}{10}$       já está escrita a partir de décimos.
    \item       Sim, existem várias. Por exemplo,       $\frac{1}{10}$,       $\frac{1}{20}$       ou~$\frac{1}{70}$.
    \item       Como       $\frac{3}{5}+\frac{7}{10} = \frac{6}{10} + \frac{7}{10} = \frac{13}{10} > 1$, juntas, as regiões destacadas em vermelho e em bege determinam um região maior do que a do retângulo dado.
\end{enumerate}

\end{solucao}
\fi

\end{document}