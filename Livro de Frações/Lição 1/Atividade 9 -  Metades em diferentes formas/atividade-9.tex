\documentclass[10 pt,usenames,dvipsnames, oneside]{article}
\usepackage{../../modelo-fracoes}
\graphicspath{{../../../Figuras/licao01/}}


\begin{document}

\begin{center}
  \begin{minipage}[l]{3cm}
\includegraphics[width=2cm]{../../../Figuras/logo}       
\end{minipage}\hfill
\begin{minipage}[r]{.8\textwidth}
 {\Large \scshape Atividade: Reconhecendo metades em figuras}  
\end{minipage}
\end{center}
\vspace{.2cm}

\ifdefined\prof
\begin{goals}
\begin{enumerate}

    \item       Representar uma fração unitária (no caso, um meio ou metade) a partir de uma unidade não usual dada.
    \item       Estabelecer representações diferentes para a mesma fração unitária de uma mesma unidade.

\end{enumerate}
\tcblower

\begin{itemize} %s
   \item Esta é uma atividade que o aluno pode fazer individualmente. Mas caso os estudantes tenham dificuldades em encontrar três representações distintas, sugere-se que sejam socializadas suas respostas, quando provavelmente aparecerão mais do que três soluções distintas.
   \item Observe que a representação da unidade se dá de forma genérica, ainda em modelo contínuo, por uma figura não tradicional como retângulos e círculos, que é determinada pela justaposição de dois hexágonos regulares.
   \item       Como na atividade anterior, não se espera que, nesta atividade, o aluno use a medida para fazer a equipartição de maneira mais precisa. O objetivo é que o aluno faça a equipartição livremente e de forma coerente,  mesmo porque aqui não se recomenda o uso de material concreto para a realização da atividade. É esperado que o material concreto utilizado como apoio para as atividades anteriores já seja suficiente para que o estudante abstraia a ideia de equipartição e faça uso de sua imaginação apenas.
   \item Procure apresentar e discutir com a turma mais do que uma solução para cada item.
\end{itemize} %s

\end{goals}

\bigskip
\begin{center}
{\large \scshape Atividade}

\end{center}
\fi

\begin{enumerate} [label=\alph*)] %d
  \item     Pinte metade da figura.
\begin{center}
  \begin{tikzpicture}[x=1cm,y=1cm, scale=0.5]
  \draw[fill=common, fill opacity=.3] (3.,5.) -- (3.,3.) -- (4.7,2.) -- (6.46,3.) -- (8.2,2.) -- (9.9,3.) -- (9.9,5.) -- (8.2,6.)       -- (6.46,5.) -- (4.73,6.) -- cycle;
  \end{tikzpicture}
\end{center}

  \item     Pinte metade da figura de forma diferente da do item anterior.
\begin{center}
  \begin{tikzpicture}[x=1cm,y=1cm, scale=0.5]
  \draw[fill=common, fill opacity=.3] (3.,5.) -- (3.,3.) -- (4.7,2.) -- (6.46,3.) -- (8.2,2.) -- (9.9,3.) -- (9.9,5.) -- (8.2,6.)       -- (6.46,5.) -- (4.73,6.) -- cycle;
  \end{tikzpicture}
\end{center}

  \item     Pinte a metade da figura de forma diferente das dos dois itens anteriores.
\begin{center}
  \begin{tikzpicture}[x=1cm,y=1cm, scale=0.5]
  \draw[fill=common, fill opacity=.3] (3.,5.) -- (3.,3.) -- (4.7,2.) -- (6.46,3.) -- (8.2,2.) -- (9.9,3.) -- (9.9,5.) -- (8.2,6.)       -- (6.46,5.) -- (4.73,6.) -- cycle;
  \end{tikzpicture}
\end{center}
\end{enumerate}

\ifdefined\prof

\begin{solucao}
\tikzset{x=1cm,y=1cm}

Algumas das respostas possíveis para este problema são:
\begin{center}
\begin{tikzpicture}[scale=.4]
\draw[fill=attention] (3.,5.) -- (3.,3.) -- (4.7,2.) -- (6.46,3.) -- (6.46,5.) -- (4.73,6.) -- cycle;
\draw[fill=common, fill opacity=.3] (6.46,5.) -- (6.46,3.) -- (8.2,2.) -- (9.9,3.) -- (9.9,5.) -- (8.2,6.) -- cycle;
\end{tikzpicture}\hspace{.2cm}
\begin{tikzpicture}[scale=.4]
\draw[fill=common, fill opacity=.3] (3.,5.) -- (3.,3.) -- (4.7,2.) -- (6.46,3.) -- (6.46,5.) -- (4.73,6.) -- cycle;
\draw[fill=common, fill opacity=.3] (6.46,5.) -- (6.46,3.) -- (8.2,2.) -- (9.9,3.) -- (9.9,5.) -- (8.2,6.) -- cycle;
\draw[fill=attention] (4.732050807568877,6.) -- (3.,5.) -- (3.,3.) -- (4.732050807568877,2.) -- cycle;
\draw[fill=attention] (8.2,6.) -- (8.2,2.) -- (9.9,3.) -- (9.9,5.) -- cycle;
\end{tikzpicture}

\begin{tikzpicture}[scale=.4]
\draw[fill=common, fill opacity=.3] (3.,5.) -- (3.,3.) -- (4.7,2.) -- (6.46,3.) -- (6.46,5.) -- (4.73,6.) -- cycle;
\draw[fill=common, fill opacity=.3] (6.46,5.) -- (6.46,3.) -- (8.2,2.) -- (9.9,3.) -- (9.9,5.) -- (8.2,6.) -- cycle;
\draw[fill=attention] (3.,4.) -- (3.,3.) -- (4.732050807568877,2.) -- (6.464101615137757,3.) -- (8.196152422706634,2.) -- (9.928203230275509,3.) -- (9.928203230275509,4.) -- cycle;
\end{tikzpicture}\hspace{.2cm}
\begin{tikzpicture}[scale=.4]
\draw[fill=common, fill opacity=.3] (3.,5.) -- (3.,3.) -- (4.7,2.) -- (6.46,3.) -- (6.46,5.) -- (4.73,6.) -- cycle;
\draw[fill=common, fill opacity=.3] (6.46,5.) -- (6.46,3.) -- (8.2,2.) -- (9.9,3.) -- (9.9,5.) -- (8.2,6.) -- cycle;
\draw[fill=attention] (3,5) -- (3.,3.) -- (4.73,2.) -- (6.46,3.) -- (8.2,2.) -- (9.9,3.) -- cycle;
\end{tikzpicture}
\end{center}

\end{solucao}
\fi

\end{document}