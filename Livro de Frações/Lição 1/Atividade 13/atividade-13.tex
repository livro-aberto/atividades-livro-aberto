\documentclass[10 pt,usenames,dvipsnames, oneside]{article}
\usepackage{../../../modelo-fracoes}
\graphicspath{{../../../Figuras/licao01/}}


\begin{document}

\begin{center}
  \begin{minipage}[l]{3cm}
\includegraphics[width=2cm]{logo}    
\end{minipage}\hfill
\begin{minipage}[r]{.8\textwidth}
 {\Large \scshape Atividade: Meios, quartos e décimos em 1D, 2D e 3D}  
\end{minipage}
\end{center}
\vspace{.2cm}

\ifdefined\prof
\begin{goals}
\begin{enumerate}

   \item Reconhecer e diferenciar a representação das frações ``um meio'', ``um quarto'' e       ``um décimo'' em modelos diversos, baseados  ou não em equipartição.

\end{enumerate}
\tcblower
\tikzset{x=1cm, y=1cm}
\begin{itemize} %s
\item Esta é uma atividade que o aluno pode fazer individualmente.
\item Esta atividade pode ser desenvolvida com o uso do material para reprodução disponível no final do livro caso se entenda que seus estudantes ainda precisam deste recurso.
\item Como nas atividades anteriores, não se espera que os alunos usem a medida para confirmar a metade. O objetivo é que identifiquem a representação da metade (ou não) por sobreposição e justaposição dessas partes, decompondo e recompondo a figura.
\item No item m), pretende-se que o estudante considere o quadrado maior como unidade e perceba que a parte em vermelho corresponde à metade desse quadrado. Isso pode ser feito observando que todos os quadradinhos vermelhos podem ser agrupados para formar a metade superior do quadrado maior e os azuis agrupados para formar a parte de baixo, por exemplo.
\begin{center}
  \begin{tikzpicture}[scale=.5]
\draw[fill=common, fill opacity=.3] (0,0) rectangle (4,4);
 \filldraw[fill=attention, draw=black] (1,0) rectangle (2,1);
 \filldraw[fill=attention, draw=black] (3,0) rectangle (4,1);
 \filldraw[fill=attention, draw=black] (0,1) rectangle (1,2);
 \filldraw[fill=attention, draw=black] (2,1) rectangle (3,2);
 \filldraw[fill=attention, draw=black] (1,2) rectangle (2,3);
 \filldraw[fill=attention, draw=black] (3,2) rectangle (4,3);
 \filldraw[fill=attention, draw=black] (0,3) rectangle (1,4);
 \filldraw[fill=attention, draw=black] (2,3) rectangle (3,4);
\end{tikzpicture}
\quad
\begin{tikzpicture}[scale=.5]
  \draw[fill=common, fill opacity=.3] (0,0) rectangle (4,4);
  \filldraw[fill=attention, draw=black] (0,2) rectangle (4,4);
  \draw (0,0) grid (4,4);
\end{tikzpicture}
\end{center}

Neste item m), é prematuro e, portanto, inadequado usar o argumento de que a fração é um meio argumentando que dos 16 quadradinhos, existem 8 pintados de vermelho. Isso porque, nessa linha de raciocínio, a unidade seria o quadradinho e não o quadrado maior e, consequentemente, não se estaria fazendo uma divisão de uma unidade mas, sim, uma contagem em um modelo discreto de frações, abordagem essa que escolhemos evitar nesse momento.   


\item Incentive os alunos a argumentar, justificando a sua decisão. Para isso, podem, por exemplo, se apoiar em dobraduras ou no recorte das partes da figura.
\item Procure apresentar e discutir com a turma mais do que uma solução para cada item.
\end{itemize} %s


\end{goals}

\bigskip
\begin{center}
{\large \scshape Atividade}
\end{center}
\fi

Em cada uma das imagens, a parte em vermelho é uma fração da figura. Essas frações podem ser ``um meio'', ``um quarto'' ou ``um décimo'' da figura. Associe cada imagem à fração correspondente.


\begin{tasks}[label-width=2em](3)
\task \adjustbox{valign=t}{
\parbox[t][2cm][c]{2cm}{
\begin{tikzpicture}[scale=5]
 \draw[fill=common, fill opacity=.3] (0,0) rectangle (4,2);
 \filldraw[fill=attention, draw=black] (0,0) -- (4,2) -- (4,0) -- cycle;
\end{tikzpicture}}}
\task \adjustbox{valign=t}{
\parbox[t][2cm][c]{2cm}{
\begin{tikzpicture}[scale=5]
 \draw[fill=common, fill opacity=.3] (0,0) rectangle (4,2);
 \foreach \n in {0.4,0.8,1.2,...,3.2,3.6}{
 \draw (\n,0) -- (\n,2);}
 \filldraw[fill=attention, draw=black] (0.8,0) -- (0.8,2) -- (1.2,2) -- (1.2,0) -- cycle;
\end{tikzpicture}}}
\task \adjustbox{valign=t}{
\parbox[t][2cm][c]{2cm}{
\begin{tikzpicture}[scale=5]
 \draw[fill=common, fill opacity=.3] (0,0) rectangle (4,2);
 \filldraw[fill=attention, draw=black] (2,1) rectangle (4,2);
 \draw (0,1) -- (4,1);
 \draw (2,0) -- (2,2);
 \end{tikzpicture}}}
\task \adjustbox{valign=t}{
\parbox[t][2cm][c]{2cm}{
\begin{tikzpicture}[scale=5]
 \draw[fill=common, fill opacity=.3] (135:2) arc (135:405:2) -- (0,0) --cycle;
 \filldraw[fill=attention, draw=black] (45:2) arc (45:135:2) -- (0,0) -- cycle;
 \end{tikzpicture}}}
 \task \adjustbox{valign=t}{
\parbox[t][2cm][c]{2cm}{
\begin{tikzpicture}[scale=5]
 \draw[fill=common, fill opacity=.3] (0:2) -- (60:2) -- (90:1.73) -- (0,0) -- (180:2) -- (240:2) -- (300:2) --cycle;
 \filldraw[fill=attention, draw=black] (-2,0) -- (0,0) -- (0,1.73) -- (120:2) -- cycle;
\end{tikzpicture}}}
\task \adjustbox{valign=t}{
\parbox[t][2cm][c]{2cm}{
\begin{tikzpicture}[scale=5]
 \draw[fill=common, fill opacity=.3] (0,0) rectangle (5,1);
 \draw[fill=attention] (0,1) rectangle (5,2);
\end{tikzpicture}}}
\task \adjustbox{valign=t}{
\parbox[t][2cm][c]{2cm}{
\begin{tikzpicture}[scale=5]
 \draw[fill=common, fill opacity=.3] (0,.75) rectangle (2,3);
 \draw[fill=attention] (0,0) rectangle (2,.75);
\end{tikzpicture}}}
  \task \adjustbox{valign=t}{
\parbox[t][2cm][c]{2cm}{
\begin{tikzpicture}[scale=5]
 \draw[fill=common, fill opacity=.3] (236:2) arc (236:560:2);
 \draw[fill=attention] (200:2) arc (200:236:2) -- (0,0) -- cycle;
\end{tikzpicture}}}
\task \adjustbox{valign=t}{
\parbox[t][2cm][c]{2cm}{
\begin{tikzpicture}[scale=5]
 \draw[ultra thick,color=attention] (-1,0) arc (180:270:1);
 \draw (0,-1) arc (270:360:1) -- (1,0) arc (180:0:1);
 \end{tikzpicture}}}
\task \adjustbox{valign=t}{
\parbox[t][2cm][c]{2cm}{
\begin{tikzpicture}
\tikzset{
  annotated cuboid/.pic={
    \tikzset{%
      every edge quotes/.append style={midway, auto},
      /cuboid/.cd,
      #1
    }
    \draw [every edge/.append style={pic actions, densely dashed, opacity=0}, pic actions]
    (0,0,0) coordinate (o) -- ++(-\cubescale*\cubex,0,0) coordinate (a) -- ++(0,-\cubescale*\cubey,0) coordinate (b) edge coordinate [pos=1] (g) ++(0,0,-\cubescale*\cubez)  -- ++(\cubescale*\cubex,0,0) coordinate (c) -- cycle
    (o) -- ++(0,0,-\cubescale*\cubez) coordinate (d) -- ++(0,-\cubescale*\cubey,0) coordinate (e) edge (g) -- (c) -- cycle
    (o) -- (a) -- ++(0,0,-\cubescale*\cubez) coordinate (f) edge (g) -- (d) -- cycle;
 },
  /cuboid/.search also={/tikz},
  /cuboid/.cd,
  width/.store in=\cubex,
  height/.store in=\cubey,
  depth/.store in=\cubez,
  units/.store in=\cubeunits,
  scale/.store in=\cubescale,
  width=100,
  height=100,
  depth=100,
  units=cm,
  scale=.1,
}
    \draw[dotted] (2.95,-1.9) -- (25.5,-1.9);
    \pic [fill=attention, fill opacity=.8] at (0,0) {annotated cuboid={width=25, height=50, depth=8}};
    \pic [fill=common, fill opacity=.3] at (22.5,0) {annotated cuboid={width=225, height=50, depth=8}};
\end{tikzpicture}}}
\task \adjustbox{valign=t}{
\parbox[t][2cm][c]{2cm}{
\begin{tikzpicture}[scale=5]

  \draw[fill=common, fill opacity=.3] (0,2) rectangle (2,4);
 \draw[fill=common, fill opacity=.3] (0,0) rectangle (2,2);
 \draw[fill=common, fill opacity=.3] (2,0) rectangle (4,2);
 \draw[fill=common, fill opacity=.3] (2,2) rectangle (4,4);
 \draw[fill=attention] (1,1) rectangle (3,3);
 \draw (2,1) -- (2,3);
 \draw (1,2) -- (3,2);
 \end{tikzpicture}}}
\task \adjustbox{valign=t}{
\parbox[t][2cm][c]{2cm}{
\begin{tikzpicture}[scale=5]
\draw[fill=common, fill opacity=.3] (0,0) rectangle (4,4);
 \filldraw[fill=attention, draw=black] (1,0) rectangle (2,1);
 \filldraw[fill=attention, draw=black] (3,0) rectangle (4,1);
 \filldraw[fill=attention, draw=black] (0,1) rectangle (1,2);
 \filldraw[fill=attention, draw=black] (2,1) rectangle (3,2);
 \filldraw[fill=attention, draw=black] (1,2) rectangle (2,3);
 \filldraw[fill=attention, draw=black] (3,2) rectangle (4,3);
 \filldraw[fill=attention, draw=black] (0,3) rectangle (1,4);
 \filldraw[fill=attention, draw=black] (2,3) rectangle (3,4);
\end{tikzpicture}}}
\end{tasks}


\ifdefined\prof

\begin{solucao}

  \begin{tabular}{lll}
a) Um meio,&  b) Um décimo,& c) Um quarto,\\
d) Um quarto,& e) Um quarto,& f) Um meio,\\
g) Um quarto,& h) Um décimo,& i) Um quarto,\\
j) Um décimo,& l) Um quarto,& m) Um meio
\end{tabular}

\end{solucao}
\fi

\end{document}