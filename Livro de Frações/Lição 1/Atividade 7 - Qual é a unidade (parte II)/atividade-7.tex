\documentclass[10 pt,usenames,dvipsnames, oneside]{article}
\usepackage{../../modelo-fracoes}
\graphicspath{{../../../Figuras/licao01/}}


\begin{document}

\begin{center}
  \begin{minipage}[l]{3cm}
\includegraphics[width=2cm]{../../../Figuras/logo}       
\end{minipage}\hfill
\begin{minipage}[r]{.8\textwidth}
 {\Large \scshape Atividade: Atividade 7}  
\end{minipage}
\end{center}
\vspace{.2cm}

\ifdefined\prof
\begin{goals}
\begin{enumerate}

\item       Recompor a unidade a partir de uma fração unitária dada em modelos contínuos.
\item       Relacionar uma fração da unidade à quantidade necessária dessas partes para compor a unidade. Assim, por exemplo, é necessário reunir cinco       {\it quintas partes}       para recompor a unidade.

\end{enumerate}
\tcblower
\tikzset{x=1cm,y=1cm}
  \begin{itemize} %d
  \item Recomenda-se que a atividade seja desenvolvida em grupos de 3 a 5 alunos.
    \item No final deste volume estão disponíveis materiais para
reprodução para esta atividade.
    \item É importante ter em mente que existem várias soluções para cada item. Por exemplo, o primeiro item pode ser corretamente respondido por             \begin{tikzpicture}[scale=.2]
\draw [fill=common, fill opacity=.3] (0,0) arc (0:180:3) -- (0,0) -- cycle;
\draw (-3,0) -- (-3,3);
\end{tikzpicture}
 ou por
 \begin{tikzpicture}[scale=.2]
\draw [fill=common, fill opacity=.3] (0,0) arc (0:90:3) -- (-3,0) -- cycle;
\draw [fill=common, fill opacity=.3] (-3,0) arc (270:180:3) -- (-3,3) -- cycle;
\end{tikzpicture}.
\item O primeiro item pode ser corretamente respondido com
  \begin{tikzpicture}[scale=.2]
\draw [fill=common, fill opacity=.3] (0,0) arc (0:90:3) -- (-3,0) -- cycle;
\draw [fill=common, fill opacity=.3] (3.7,0) arc (0:90:3) -- (0.7,0) -- cycle;
\end{tikzpicture}
No entanto, para isso, espera-se que o estudante reconheça que essas duas partes separadas compõem uma só unidade. Sugerimos que o professor apenas discuta esse tipo de resposta, caso algum estudante a apresente. 
    \item Estimule os alunos a reconhecer (e a fazer) mais do que uma representação para a unidade em cada item.
    \item Estimule os alunos a, a partir da identificação da fração unitária, determinar a quantidade de partes necessárias para recompor a unidade.
\end{itemize} %d


\end{goals}

\bigskip
\begin{center}
{\large \scshape Atividade}
\end{center}
\fi


\begin{enumerate}[label=\alph*)]
\item Na tabela a seguir, a primeira coluna mostra uma figura que é uma fração de uma unidade. Na segunda, o nome que usamos para essa fração. Desenhe uma unidade na terceira coluna, unindo frações como essa.
  
  \def\h{1.8}
  
\begin{center}
  \begin{tabular}{|m{0.3\linewidth}|c|p{.3\linewidth}|}
  \hline
\centering Fração da unidade & \centering Nome da fração da unidade  & \centering Unidade\tabularnewline
\hline 
\centering \begin{tikzpicture}[scale=2]
 \draw [fill=common, fill opacity=.3] (0,0) arc (0:90:3) -- (-3,0) -- cycle;
\end{tikzpicture}
&\centering \parbox[c][\h cm][c]{0.01cm}{  } metade  &  \\
\hline
\end{tabular}
\end{center}

\item A seguir, complete cada linha da tabela como no item anterior.
  
\begin{center}
  \begin{tabular}{|m{0.3\linewidth}|c|p{.3\linewidth}|}
  \hline
\centering Fração da unidade & \centering Nome da fração da unidade  & \centering Unidade \tabularnewline
\hline
\centering \begin{tikzpicture}[scale=2]
 \draw [fill=common, fill opacity=.3] (0,0) arc (0:90:3) -- (-3,0) -- cycle;
\end{tikzpicture}
&\centering \parbox[c][\h cm][c]{0.01cm}{  } metade  &  \\
    \hline
\centering\begin{tikzpicture}[scale=2]
\draw [fill=common, fill opacity=.3] (0,0) arc (0:90:3) -- (-3,0) -- cycle;
\end{tikzpicture}        &\parbox[c][\h cm][c]{0.01cm}{  } \centering   um terço  &  \\
    \hline
\centering \begin{tikzpicture}[scale=2]
\draw [fill=common, fill opacity=.3] (0,0) arc (0:90:3) -- (-3,0) -- cycle;
\end{tikzpicture}        & \centering \parbox[c][\h cm][c]{0.01cm}{  } um quarto  &  \\
    \hline
\centering \begin{tikzpicture}[scale=2]
\draw [fill=common, fill opacity=.3] (0,0) rectangle (3,3);
\end{tikzpicture}
  & \centering \parbox[c][\h cm][c]{0.01cm}{  } metade  &  \\
    \hline
\centering \begin{tikzpicture}[scale=2]
\draw [fill=common, fill opacity=.3] (0,0) rectangle (3,3);
\end{tikzpicture}
  & \centering \parbox[c][\h cm][c]{0.01cm}{  } um terço  &  \\
    \hline
\centering \begin{tikzpicture}[scale=2]
\draw [fill=common, fill opacity=.3] (0,0) rectangle (3,3);
\end{tikzpicture}
 & \centering \parbox[c][\h cm][c]{0.01cm}{  } um quarto  &  \\
    \hline
\centering \begin{tikzpicture}[scale=2]
\draw  (0,0) -- (3,3);
\end{tikzpicture}
  & \centering \parbox[c][\h cm][c]{0.01cm}{  } metade  &  \\
    \hline
\centering \begin{tikzpicture}[scale=2]
\draw  (0,0) -- (3,3);
\end{tikzpicture}
  & \centering \parbox[c][\h cm][c]{0.01cm}{  } um terço  &  \\
    \hline
\centering \begin{tikzpicture}[scale=2]
\draw  (0,0) -- (3,3);
\end{tikzpicture}
  & \centering \parbox[c][\h cm][c]{0.01cm}{  } um quarto  &  \\
    \hline
\centering \begin{tikzpicture}[scale=2]
\draw [fill=common, fill opacity=.3] (90:2) -- (210:2) -- (330:2) -- cycle;
\end{tikzpicture}  & \centering \parbox[c][\h cm][c]{0.01cm}{  } metade  &  \\
    \hline
\centering \begin{tikzpicture}[scale=2]
\draw [fill=common, fill opacity=.3] (90:2) -- (210:2) -- (330:2) -- cycle;
\end{tikzpicture}  & \centering \parbox[c][\h cm][c]{0.01cm}{  } um terço  &  \\
    \hline
\centering \begin{tikzpicture}[scale=2]
\draw [fill=common, fill opacity=.3] (90:2) -- (210:2) -- (330:2) -- cycle;
\end{tikzpicture}  & \centering \parbox[c][\h cm][c]{0.01cm}{  } um quarto  &  \\
    \hline
  \end{tabular}
\end{center}

\end{enumerate}

\ifdefined\prof

\begin{solucao}

\tikzset{x=1cm,y=1cm}
Algumas possibilidades de respostas:
\begin{center}

 \def \sc {0.22}
 \def\h{1.8}

\setlength\tabulinesep{10pt} 
 % \renewcommand{\arraystretch}{2}
  \begin{longtable}{|m{0.2\linewidth}|c|>{\centering}m{0.4\linewidth}|}
  \hline
\centering Fração da unidade & \centering Nome da fração da unidade  & Unidade  \tabularnewline
\hline 
\centering \begin{tikzpicture}[scale=\sc, baseline=(current bounding box.center)]
 \draw [fill=common, fill opacity=.3] (0,0) arc (0:90:3) -- (-3,0) -- cycle;
\end{tikzpicture}
&\centering \parbox[c][\h cm][c]{0.01cm}{  } metade  & 


\begin{tikzpicture}[scale=\sc), baseline=(current bounding box.center)]
\draw [fill=common, fill opacity=.3] (0,0) arc (0:180:3) -- (0,0) -- cycle;
\draw (-3,0) -- (-3,3);
\end{tikzpicture}
\quad
\begin{tikzpicture}[scale=\sc, baseline=(current bounding box.center)]]
\draw [fill=common, fill opacity=.3] (0,0) arc (0:90:3) -- (-3,0) -- cycle;
\draw [fill=common, fill opacity=.3] (-3,0) arc (270:180:3) -- (-3,3) -- cycle;
\end{tikzpicture}
\quad
\begin{tikzpicture}[scale=\sc, baseline=(current bounding box.center)]
 \draw [fill=common, fill opacity=.3] (0,0) arc (0:90:3) -- (-3,0) -- cycle;
 \draw [fill=common, fill opacity=.3] (3,0) arc (0:90:3) -- (0,0) -- cycle;
\end{tikzpicture}


  \tabularnewline
    \hline
\centering\begin{tikzpicture}[scale=\sc]
\draw [fill=common, fill opacity=.3] (0,0) arc (0:90:3) -- (-3,0) -- cycle;
\end{tikzpicture}        &\parbox[c][\h cm][c]{0.01cm}{  } \centering   um terço  &


\begin{tikzpicture}[scale=\sc, baseline=(current bounding box.center)]
\draw [fill=common, fill opacity=.3] (0,0) arc (0:270:3) -- (-3,0) -- cycle;
\draw (-3,0) -- (-6,0);
\draw (-3,0) -- (-3,3);
\end{tikzpicture}
\begin{tikzpicture}[scale=\sc, baseline=(current bounding box.center), baseline=(current bounding box.center)]
\draw [fill=common, fill opacity=.3] (0,0) arc (0:90:3) -- (-3,0) -- cycle;
\draw [fill=common, fill opacity=.3] (-3,0) arc (270:180:3) -- (-3,3) -- cycle;
\draw [fill=common, fill opacity=.3] (-3,0) arc (180:270:3) -- (0,0) -- cycle;
\end{tikzpicture}
\begin{tikzpicture}[scale=\sc, baseline=(current bounding box.center)]
 \draw [fill=common, fill opacity=.3] (0,0) arc (0:90:3) -- (-3,0) -- cycle;
 \draw [fill=common, fill opacity=.3] (3,0) arc (0:90:3) -- (0,0) -- cycle;
 \draw [fill=common, fill opacity=.3] (0,3) arc (0:90:3) -- (-3,3) -- cycle;
\end{tikzpicture}



\tabularnewline

    \hline
\centering \begin{tikzpicture}[scale=\sc, baseline=(current bounding box.center)]
\draw [fill=common, fill opacity=.3] (0,0) arc (0:90:3) -- (-3,0) -- cycle;
\end{tikzpicture}        & \centering \parbox[c][\h cm][c]{0.01cm}{  } um quarto  & \begin{tikzpicture}[scale=\sc, baseline=(current bounding box.center)]
\draw [fill=common, fill opacity=.3] (0,0) arc (0:360:3) -- (-3,0) -- cycle;
\draw (-3,0) -- (-6,0);
\draw (-3,0) -- (-3,3);
\draw (-3,0) -- (-3,-3);
\end{tikzpicture}
\quad
\begin{tikzpicture}[scale=\sc, baseline=(current bounding box.center)]
\draw [fill=common, fill opacity=.3] (0,3) arc (90:270:3) -- (0,3) -- cycle;
 \draw [fill=common, fill opacity=.3] (3,3) arc (0:-90:3) -- (0,3) -- cycle;
 \draw [fill=common, fill opacity=.3] (0,0) arc (90:0:3) -- (0,-3) -- cycle;
 \draw  (0,0) -- (-3,0);
\end{tikzpicture}
 \tabularnewline
    \hline
\centering \begin{tikzpicture}[scale=\sc, baseline=(current bounding box.center)]
\draw [fill=common, fill opacity=.3] (0,0) rectangle (3,3);
\end{tikzpicture}
  & \centering \parbox[c][\h cm][c]{0.01cm}{  } metade  & \begin{tikzpicture}[scale=\sc, baseline=(current bounding box.center)]
\draw [fill=common, fill opacity=.3] (0,0) rectangle (3,3);
\draw [fill=common, fill opacity=.3] (3,0) rectangle (6,3);
\end{tikzpicture}
\quad
\begin{tikzpicture}[scale=\sc, baseline=(current bounding box.center)]
\draw [fill=common, fill opacity=.3] (0,0) rectangle (3,3);
\draw [fill=common, fill opacity=.3] (0,3) rectangle (3,6);
\end{tikzpicture}
 \tabularnewline
    \hline
\centering \begin{tikzpicture}[scale=\sc, baseline=(current bounding box.center)]
\draw [fill=common, fill opacity=.3] (0,0) rectangle (3,3);
\end{tikzpicture}
& \centering \parbox[c][\h cm][c]{0.01cm}{  } um terço  &
\begin{tikzpicture}[scale=\sc, baseline=(current bounding box.center)]
\draw [fill=common, fill opacity=.3] (0,0) rectangle (3,3);
\draw [fill=common, fill opacity=.3] (3,0) rectangle (6,3);
\draw [fill=common, fill opacity=.3] (6,0) rectangle (9,3);
\end{tikzpicture}
 \quad
\begin{tikzpicture}[scale=\sc, baseline=(current bounding box.center)]
\draw [fill=common, fill opacity=.3] (0,0) rectangle (3,3);
\draw [fill=common, fill opacity=.3] (0,3) rectangle (3,6);
\draw [fill=common, fill opacity=.3] (3,0) rectangle (6,3);
\end{tikzpicture}
 \quad
\begin{tikzpicture}[scale=\sc, baseline=(current bounding box.center)]
\draw [fill=common, fill opacity=.3] (0,0) rectangle (3,3);
\draw [fill=common, fill opacity=.3] (3,3) rectangle (6,6);
\draw [fill=common, fill opacity=.3] (6,0) rectangle (9,3);
\end{tikzpicture}
\tabularnewline
    \hline
\centering \begin{tikzpicture}[scale=\sc, baseline=(current bounding box.center)]
\draw [fill=common, fill opacity=.3] (0,0) rectangle (3,3);
\end{tikzpicture}
& \centering \parbox[c][\h cm][c]{0.01cm}{  } um quarto  &
\begin{tikzpicture}[scale=\sc, baseline=(current bounding box.center)]
\draw [fill=common, fill opacity=.3] (0,0) rectangle (3,3);
\draw [fill=common, fill opacity=.3] (3,0) rectangle (6,3);
\draw [fill=common, fill opacity=.3] (6,0) rectangle (9,3);
\draw [fill=common, fill opacity=.3] (9,0) rectangle (12,3);
\end{tikzpicture}
\quad
\begin{tikzpicture}[scale=\sc, baseline=(current bounding box.center)]
\draw [fill=common, fill opacity=.3] (0,0) rectangle (3,3);
\draw [fill=common, fill opacity=.3] (0,3) rectangle (3,6);
\draw [fill=common, fill opacity=.3] (3,0) rectangle (6,3);
\draw [fill=common, fill opacity=.3] (3,3) rectangle (6,6);
\end{tikzpicture}
\quad
\begin{tikzpicture}[scale=\sc, baseline=(current bounding box.center)]
\draw [fill=common, fill opacity=.3] (0,3) rectangle (3,6);
\draw [fill=common, fill opacity=.3] (0,0) rectangle (3,3);
\draw [fill=common, fill opacity=.3] (3,0) rectangle (6,3);
\draw [fill=common, fill opacity=.3] (6,0) rectangle (9,3);
\end{tikzpicture}
\tabularnewline
    \hline
\centering \begin{tikzpicture}[scale=\sc, baseline=(current bounding box.center)]
\draw  (0,0) -- (3,3);
\end{tikzpicture}
& \centering \parbox[c][\h cm][c]{0.01cm}{  } metade  &
\begin{tikzpicture}[scale=\sc, baseline=(current bounding box.center)]
\draw  (0,0) -- (3,3) -- (6,0);
\end{tikzpicture}
% \begin{tikzpicture}[scale=\sc]
% \draw  (0,0) -- (3,3);
% \draw  (2,0) -- (5,3);
% \end{tikzpicture}
\quad
\begin{tikzpicture}[scale=\sc, baseline=(current bounding box.center)]
\draw  (0,0) -- (3,3);
\draw  (3,0) -- (0,3);
\end{tikzpicture}
\tabularnewline
    \hline
\centering \begin{tikzpicture}[scale=\sc, baseline=(current bounding box.center)]
\draw  (0,0) -- (3,3);
\end{tikzpicture}
& \centering \parbox[c][\h cm][c]{0.01cm}{  } um terço  &
\begin{tikzpicture}[scale=\sc, baseline=(current bounding box.center)]
\draw  (0,0) -- (3,3) -- (6,0) -- (9,3);
\end{tikzpicture}
% \begin{tikzpicture}[scale=\sc]
% \draw  (0,0) -- (3,3);
% \draw  (2,0) -- (5,3);
% \draw  (4,0) -- (7,3);
% \end{tikzpicture}
 \quad
\begin{tikzpicture}[scale=\sc, baseline=(current bounding box.center)]
\draw  (0,0) -- (3,3);
\draw  (4,0) -- (1,3);
\draw  (2,0) -- (5,3);
\end{tikzpicture}
\tabularnewline
    \hline
\centering \begin{tikzpicture}[scale=\sc, baseline=(current bounding box.center)]
\draw  (0,0) -- (3,3);
\end{tikzpicture}
& \centering \parbox[c][\h cm][c]{0.01cm}{  } um quarto  &
\begin{tikzpicture}[scale=\sc, baseline=(current bounding box.center)]
\draw  (0,0) -- (3,3) -- (0,6) -- (-3,3) -- (0,0);
\end{tikzpicture}
% \begin{tikzpicture}[scale=\sc]
% \draw  (0,0) -- (3,3);
% \draw  (2,0) -- (5,3);
% \draw  (4,0) -- (7,3);
% \draw  (6,0) -- (9,3);
% \end{tikzpicture}
\quad
\begin{tikzpicture}[scale=\sc, baseline=(current bounding box.center)]
\draw  (0,0) -- (3,3);
\draw  (4,0) -- (1,3);
\draw  (2,0) -- (5,3);
\draw (2,0) -- (-1,3);
\end{tikzpicture}
\tabularnewline
\hline
\centering \begin{tikzpicture}[scale=\sc, baseline=(current bounding box.center)]
\draw [fill=common, fill opacity=.3] (90:2) -- (210:2) -- (330:2) -- cycle;
\end{tikzpicture}  
& \centering \parbox[c][\h cm][c]{0.01cm}{  } metade 
&
\begin{tikzpicture}[scale=\sc, baseline=(current bounding box.center)]
\draw [fill=common, fill opacity=.3] (90:2) -- (210:2) -- (330:2) -- cycle;
\draw [fill=common, fill opacity=.3, shift={(2*1.732,0)}] (90:2) -- (210:2) -- (330:2) -- cycle;
\end{tikzpicture}
\quad
\begin{tikzpicture}[scale=\sc, baseline=(current bounding box.center)]
\draw [fill=common, fill opacity=.3] (90:2) -- (210:2) -- (330:2) -- cycle;
\draw [fill=common, fill opacity=.3, rotate around={60:(90:2)}] (90:2) -- (210:2) -- (330:2) -- cycle;
\end{tikzpicture}
\tabularnewline
    \hline
\centering \begin{tikzpicture}[scale=\sc, baseline=(current bounding box.center)]
\draw [fill=common, fill opacity=.3] (90:2) -- (210:2) -- (330:2) -- cycle;
\end{tikzpicture}  
& \centering \parbox[c][\h cm][c]{0.01cm}{  } um terço  
&
\begin{tikzpicture}[scale=\sc, baseline=(current bounding box.center)]
\draw [fill=common, fill opacity=.3] (90:2) -- (210:2) -- (330:2) -- cycle;
\draw [fill=common, fill opacity=.3, shift={(2*1.732,0)}] (90:2) -- (210:2) -- (330:2) -- cycle;
\draw [fill=common, fill opacity=.3, shift={(4*1.732,0)}] (90:2) -- (210:2) -- (330:2) -- cycle;
\end{tikzpicture}
\quad
\begin{tikzpicture}[scale=\sc, baseline=(current bounding box.center)]
\draw [fill=common, fill opacity=.3] (90:2) -- (210:2) -- (330:2) -- cycle;
\draw [fill=common, fill opacity=.3, shift={(2*1.732,0)}] (90:2) -- (210:2) -- (330:2) -- cycle;
\draw [fill=common, fill opacity=.3, rotate around={120:(330:2)}] (90:2) -- (210:2) -- (330:2) -- cycle;
\end{tikzpicture}
 \quad
\begin{tikzpicture}[scale=\sc, baseline=(current bounding box.center)]
\draw [fill=common, fill opacity=.3] (90:2) -- (210:2) -- (330:2) -- cycle;
\draw [fill=common, fill opacity=.3, rotate around={60:(90:2)}] (90:2) -- (210:2) -- (330:2) -- cycle;
\draw [fill=common, fill opacity=.3, shift={(2*1.732,0)}] (90:2) -- (210:2) -- (330:2) -- cycle;
\end{tikzpicture}
\tabularnewline
    \hline
\centering \begin{tikzpicture}[scale=\sc, baseline=(current bounding box.center)]
\draw [fill=common, fill opacity=.3] (90:2) -- (210:2) -- (330:2) -- cycle;
\end{tikzpicture}  

& \centering \parbox[c][\h cm][c]{0.01cm}{  } um quarto


&
\begin{tikzpicture}[scale=\sc]
\draw [fill=common, fill opacity=.3] (90:2) -- (210:2) -- (330:2) -- cycle;
\draw [fill=common, fill opacity=.3, shift={(2*1.732,0)}] (90:2) -- (210:2) -- (330:2) -- cycle;
\draw [fill=common, fill opacity=.3, shift={(4*1.732,0)}] (90:2) -- (210:2) -- (330:2) -- cycle;
\draw [fill=common, fill opacity=.3, shift={(6*1.732,0)}] (90:2) -- (210:2) -- (330:2) -- cycle;
\end{tikzpicture}
\quad
\begin{tikzpicture}[scale=\sc]
\draw [fill=common, fill opacity=.3] (90:2) -- (210:2) -- (330:2) -- cycle;
\draw [fill=common, fill opacity=.3, rotate around={60:(90:2)}] (90:2) -- (210:2) -- (330:2) -- cycle;
\draw [fill=common, fill opacity=.3, shift={(1.732,3)}] (90:2) -- (210:2) -- (330:2) -- cycle;
\draw [fill=common, fill opacity=.3, shift={(1.732,3)}, rotate around={60:(90:2)}] (90:2) -- (210:2) -- (330:2) -- cycle;
\end{tikzpicture}
\quad
\begin{tikzpicture}[scale=\sc]
\draw [fill=common, fill opacity=.3] (90:2) -- (210:2) -- (330:2) -- cycle;
\draw [fill=common, fill opacity=.3, rotate around={60:(90:2)}] (90:2) -- (210:2) -- (330:2) -- cycle;
\draw [fill=common, fill opacity=.3, rotate around={120:(90:2)}] (90:2) -- (210:2) -- (330:2) -- cycle;
\draw [fill=common, fill opacity=.3, shift={(2*1.732,0)}] (90:2) -- (210:2) -- (330:2) -- cycle;
\end{tikzpicture}
\tabularnewline
\hline
\end{longtable}
\end{center}

\end{solucao}
\fi

\end{document}