\documentclass[10 pt,usenames,dvipsnames, oneside]{article}
\usepackage{../../modelo-fracoes}
\graphicspath{{../../../Figuras/licao01/}}


\begin{document}

\begin{center}
  \begin{minipage}[l]{3cm}
\includegraphics[width=2cm]{../../../Figuras/logo}       
\end{minipage}\hfill
\begin{minipage}[r]{.8\textwidth}
 {\Large \scshape Atividade: Qual é a unidade dessas figuras?}  
\end{minipage}
\end{center}
\vspace{.2cm}

\ifdefined\prof
\begin{goals}
\begin{enumerate}

\item Relacionar uma fração unitária à unidade correspondente por recomposição. Por exemplo, reconhecer que é necessário reunir cinco quintas partes para recompor a unidade.

\end{enumerate}
\end{goals}

\bigskip
\begin{center}
{\large \scshape Atividade}
\end{center}
\fi


\begin{enumerate}[label=\alph*)]
\item O triângulo cinza é um terço de uma das figuras coloridas. Qual é essa figura? Explique. 
\item O triângulo cinza é um quarto de alguma dessas figuras? Qual ou quais?
\item Agora é a sua vez: desenhe uma unidade da qual o triângulo cinza seja um quinto.  
\item Desafio: Que fração o triângulo cinza é do retângulo laranja? 
\end{enumerate}

\begin{center}  
\begin{tabular}[c]{ccc}

\begin{tikzpicture}[scale=20]
\fill [cbgray] (60:0) -- (120:1) -- (180:1) -- cycle;
\end{tikzpicture}
  &
%losango
\begin{tikzpicture}[scale=20,rotate=90]
\fill [cbpink] (60:0) -- (120:1) -- (180:1) -- ++(300:1) -- cycle;
\end{tikzpicture}
  &
%estrela de  seis pontas
\begin{tikzpicture}[scale=20]
\fill [cbpurple] (0,0) -- (60:1) -- (120:1) -- cycle;
\fill [cbpurple,shift={(.5,1.71/6)}] (60:0) -- (120:1) -- (180:1) -- cycle;
\end{tikzpicture}
    \\
%paralelogramo
\begin{tikzpicture}[scale=20]
\fill [cbgreen]  (120:1) -- ++(180:1) -- ++(240:1) -- ++(360:2) -- ++(60:1) -- cycle;
\end{tikzpicture}
  &
%retângulo
\begin{tikzpicture}[scale=20]
\fill [cborange] (0,0) -- (1,0) -- (1,1.732/2) -- (0,1.732/2) -- cycle;
\end{tikzpicture}
  &
    %trapézio:
\begin{tikzpicture}[scale=20]
\fill [cbpink] (60:0) -- (120:1) -- ++(180:1) -- +(240:1) -- cycle;
\end{tikzpicture}
  \\
  %hexágono
\begin{tikzpicture}[scale=20]
\fill [cbolive] (60:0) -- (120:1) -- ++(180:1) -- ++(240:1) -- ++(300:1) -- ++(360:1) -- cycle;
\end{tikzpicture}
&
%Bandeirinha
\begin{tikzpicture}[scale=20,xscale=-1]
\filldraw [cbyellow]  (120:1) -- ++(180:1) -- ++(240:1) -- ++(360:2) -- ++(60:1) -- cycle;
\fill [cbyellow,yscale=-1]  (120:1) -- ++(180:1) -- ++(240:1) -- ++(360:2) -- ++(60:1) -- cycle;
\end{tikzpicture}
&
%Triângulo ampliado
\begin{tikzpicture}[scale=40]
\fill [cbbrown] (60:0) -- (120:1) -- (180:1) -- cycle;
\end{tikzpicture}
\end{tabular}
\end{center}

\ifdefined\prof

\begin{solucao}



\end{solucao}
\fi

\end{document}