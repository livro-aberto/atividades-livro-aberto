\documentclass[10 pt,usenames,dvipsnames, oneside]{article}
\usepackage{../../modelo-fracoes}
\graphicspath{{../../../Figuras/licao01/}}


\begin{document}

\begin{center}
  \begin{minipage}[l]{3cm}
\includegraphics[width=2cm]{../../../Figuras/logo}       
\end{minipage}\hfill
\begin{minipage}[r]{.8\textwidth}
 {\Large \scshape Atividade: Comparando meio, quarto e oitavo}
 \end{minipage}
\end{center}
\vspace{.2cm}

\ifdefined\prof
\begin{goals}
\begin{enumerate}

    \item       Representar uma fração unitária a partir de uma unidade dada.
    \item       Reconhecer (e obter) um quarto como a metade da metade e um oitavo como a metade de um quarto.
    \item       Comparar as frações unitárias metade, um quarto e um oitavo de um mesmo quadrado.

\end{enumerate}
\tcblower
\tikzset{x=1cm,y=1cm}
  \begin{itemize} %s
    \item       Esta é uma atividade que o aluno pode fazer individualmente.
    \item       Não se espera que, nesta atividade, os alunos usem a medida para fazer a equipartição de maneira mais precisa. O objetivo é fazer a equipartição livremente e de forma coerente. Assim, por exemplo, podem ser aceitas como respostas:
\begin{center}
\begin{tikzpicture}[scale=.7]
 \draw[fill=common, fill opacity=.3] (0,0) rectangle (3,3);
 \draw[decorate, decoration={snake, amplitude= .2 mm}] (1.5,0) -- (1.5,3);
\end{tikzpicture}\hspace{1em}
e\hspace{1em}
\begin{tikzpicture}[scale=.7]
 \draw[fill=common, fill opacity=.3] (0,0) rectangle (3,3);
 \draw[decorate, decoration={snake, amplitude= .2 mm}] (0,0) -- (3,3);
\end{tikzpicture}
\end{center}
  Já as representações a seguir sugerem que os alunos precisam revisar os conceitos exigidos para a solução da atividade:
\begin{center}
  \begin{tikzpicture}[scale=.7]
 \draw[fill=common, fill opacity=.3] (0,0) rectangle (3,3);
 \draw[decorate, decoration={snake, amplitude= .2 mm}] (2.3,0) -- (2.3,3);
\end{tikzpicture}
\end{center}
    \item       A representação da unidade se dá de forma genérica por um quadrado.
    \item       Espera-se que os alunos reconheçam que, para obter um quarto da unidade, basta tomar a metade da metade. E que, para determinar um oitavo, pode-se dividir um quarto ao meio.
    \item       Recomenda-se que os alunos tenham em mãos três quadrados de papel iguais e que sejam orientados a fazer uso de dobradura para identificar as frações pedidas. Assim, por exemplo, a fração um quarto pode ser obtida por duas dobras do papel.
    \item Discuta com os estudantes que, quanto maior o número de partes iguais em que se particiona o quadrado, menor fica cada uma das partes.
    \item Procure apresentar e discutir com a turma mais do que uma solução para cada item.
    \item \textbf{As diferentes soluções apresentadas pelos alunos podem enriquecer a discussão}. A comparação entre, por exemplo, a metade do quadrado proveniente da dobradura pela diagonal e o quarto do quadrado proveniente da dobradura a partir de linhas paralelas aos lados (como um sinal de ``$+$'') pode não ser tão natural. Dificuldade semelhante pode ser observada na comparação entre esse mesmo quarto do quadrado e o oitavo do quadrado proveniente de uma sequência de dobraduras paralelas a um dos lados, determinando ``faixas paralelas''. Nesses casos, para executar a comparação, é necessário que os alunos reconheçam partes de formatos diferentes que correspondem a uma mesma fração do quadrado como iguais em quantidade (no caso, área). Assim, a comparação entre a metade do quadrado, obtida pela dobradura na diagonal, e o quarto do quadrado, obtido pela dobradura ``em sinal de $+$'', pode ser amparada pelo reconhecimento de que a metade em questão é igual em quantidade (área) à metade do quadrado obtida por uma única dobra paralela a um dos lados, que é o dobro do quarto do quadrado.

\begin{center}
  \begin{tikzpicture}[scale=.75]
         \draw[fill=common, fill opacity=.3] (0,0) rectangle (2,2);
         \draw[fill=attention] (0,1) rectangle (2,2);
        \end{tikzpicture}\quad\quad
        \begin{tikzpicture}[scale=.7]
         \draw[fill=common, fill opacity=.3] (0,0) rectangle (2,2);
         \draw[fill=attention] (0,0) -- (2,2) -- (0,2)--cycle;
        \end{tikzpicture}
\end{center}

\item Nossa experiência na implementação desta atividade com estudantes do $6^\circ$ ano mostrou que, após os alunos entenderem que se espera mesma quantidade e não mesmo formato, passaram a se divertir indo ao quadro para exibir equipartições diferentes daquelas já exibidas pelos colegas.
\end{itemize} %s


\end{goals}

\bigskip
\begin{center}
{\large \scshape Atividade}
\end{center}
\fi


\begin{enumerate} [label=\alph*)] %s
  \item     Pinte metade do quadrado a seguir.

  \begin{center}
 \begin{tikzpicture}[x=1mm,y=1mm, scale=.7]  \draw[fill=common, fill opacity=.3] (0,0) rectangle (20,20);  \end{tikzpicture}
  \end{center}

  \item     Pinte um quarto do quadrado a seguir.

  \begin{center}
 \begin{tikzpicture}[x=1mm,y=1mm, scale=.7]  \draw[fill=common, fill opacity=.3] (0,0) rectangle (20,20);  \end{tikzpicture}
  \end{center}

  \item     Pinte um oitavo do quadrado a seguir.

  \begin{center}
 \begin{tikzpicture}[x=1mm,y=1mm, scale=.7]  \draw[fill=common, fill opacity=.3] (0,0) rectangle (20,20);  \end{tikzpicture}
  \end{center}
  \item     Qual é a maior das frações do quadrado: metade, quarto ou oitavo?
\end{enumerate} %s


\ifdefined\prof
\clearpage
\begin{solucao}
\tikzset{x=1cm,y=1cm}
  Algumas soluções possíveis, convencionais e outras menos convencionais são:

        \begin{enumerate}[label=\alph*)]
         \item Metade:   \newline
         \begin{center}
        \begin{tikzpicture}[scale=.8]
         \draw[fill=common, fill opacity=.3] (0,0) rectangle (2,2);
         \draw[fill=attention] (0,1) rectangle (2,2);
        \end{tikzpicture}
        \begin{tikzpicture}[scale=.8]
         \draw[fill=common, fill opacity=.3] (0,0) rectangle (2,2);
         \draw[fill=attention] (0,0) -- (2,2) -- (0,2)--cycle;
        \end{tikzpicture}
        \begin{tikzpicture}[scale=.8]
         \draw[fill=common, fill opacity=.3] (0,0) rectangle (2,2);
         \draw[fill=attention] (0,2) -- (2,2) -- (1,1)--cycle;
         \draw[fill=attention] (0,0) -- (2,0) -- (1,1)--cycle;
        \end{tikzpicture}
        \begin{tikzpicture}[scale=.8]
         \draw[fill=common, fill opacity=.3] (0,0) rectangle (2,2);
         \draw[fill=attention] (0,1) rectangle (1,2);
         \draw[fill=attention] (1,0) rectangle (2,1);
        \end{tikzpicture}
        \end{center}
       \item Um quarto:  \newline
       \begin{center}
       \begin{tikzpicture}[scale=.8]
         \draw[fill=common, fill opacity=.3] (0,0) rectangle (2,2);
         \draw[fill=attention] (0,0) rectangle (.5,2);
         \foreach \x in {1,1.5} \draw (\x,0) -- (\x,2);
       \end{tikzpicture}
       \begin{tikzpicture}[scale=.8]
         \draw[fill=common, fill opacity=.3] (0,0) rectangle (2,2);
         \draw[fill=attention] (0,1) rectangle (1,2);
         \draw (1,0) -- (1,1);
         \draw (1,1)-- (2,1);
        \end{tikzpicture}
        \begin{tikzpicture}[scale=.8]
         \draw[fill=common, fill opacity=.3] (0,0) rectangle (2,2);
         \draw[fill=attention] (0,0) -- (1,1) -- (0,2)--cycle;
         \draw (1,1) -- (2,2);
         \draw (1,1) -- (2,0);
        \end{tikzpicture}
        \begin{tikzpicture}[scale=.8]
         \draw[fill=common, fill opacity=.3] (0,0) rectangle (2,2);
         \draw[fill=attention] (2,2) -- (1,2) -- (1,1)--cycle;
         \draw[fill=attention] (0,0) -- (1,0) -- (1,1)--cycle;
         \draw (0,2) -- (2,0);
        \end{tikzpicture}
        \end{center}
       \item  Um oitavo:  \newline
       \begin{center}
       \begin{tikzpicture}[scale=.8]
         \draw[fill=common, fill opacity=.3] (0,0) rectangle (2,2);
         \draw[fill=attention] (0,0) rectangle (.25,2);
         \foreach \x in {.5,.75,...,1.75} \draw (\x,0) -- (\x, 2);
       \end{tikzpicture}
       \begin{tikzpicture}[scale=.8]
         \draw[fill=common, fill opacity=.3] (0,0) rectangle (2,2);
         \draw[fill=attention] (0,1) rectangle (.5,2);
         \draw (0,1)--(2,1);
         \foreach \x in {.5,1,...,1.5} \draw (\x,0) -- (\x,2);
        \end{tikzpicture}
        \begin{tikzpicture}[scale=.8]
         \draw[fill=common, fill opacity=.3] (0,0) rectangle (2,2);
         \draw[fill=attention] (0,1) -- (1,1) -- (0,2)--cycle;
         \draw (0,0) -- (1,1);
        \end{tikzpicture}
        \begin{tikzpicture}[scale=.8]
         \draw[fill=common, fill opacity=.3] (0,0) rectangle (2,2);
         \draw[fill=attention] (2,2) -- (1,2) -- (1,1)--cycle;
         \draw (1,1) -- (1,0);
         \draw (0,0) -- (1,1);
         \end{tikzpicture}
         \end{center}
       \item  Dentre as opções apresentadas, a maior fração do quadrado é metade.
        \end{enumerate}

\end{solucao}
\fi

\end{document}