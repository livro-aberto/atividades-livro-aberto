\documentclass[10 pt,usenames,dvipsnames, oneside]{article}
\usepackage{../../modelo-fracoes}
\graphicspath{{../../../Figuras/licao01/}}


\begin{document}

\begin{center}
  \begin{minipage}[l]{3cm}
\includegraphics[width=2cm]{../../../Figuras/logo}       
\end{minipage}\hfill
\begin{minipage}[r]{.8\textwidth}
 {\Large \scshape Atividade: Frações do círculo, nome e comparação}  
\end{minipage}
\end{center}
\vspace{.2cm}

\ifdefined\prof
\begin{goals}
\begin{enumerate}

    \item Conhecer e compreender as expressões correspondentes às frações unitárias com denominadores de 5 a 10.
    \item Comparar frações da unidade por meio da representação visual de frações do círculo.
    \item Reconhecer a relação inversa entre o número de partes e o tamanho de cada parte.

\end{enumerate}
\tcblower

\begin{itemize} %s
    \item       Esta atividade pode ser resolvida individualmente, mas é essencial que seja discutida com toda a turma.
    \item       É provável que nem todos os alunos conheçam ou intuam as expressões correspondentes às frações propostas. Nesse caso, cabe ao professor apresentá-las e diferenciá-las.
    \item       Aproveite esta atividade para revisar e discutir o vocabulário que é objetivo nesta seção:       {\it unidade,}             {\it metade,}             {\it um meio,}             {\it um terço,}             {\it terça parte,}             {\it um quarto,}             {\it quarta parte,}             {\it um quinto,}             {\it quinta parte,}             {\it um sexto,}             {\it sexta parte,}             {\it um sétimo,}             {\it sétima parte,}             {\it um oitavo,}             {\it oitava parte,}             {\it um nono,}             {\it nona parte,}             {\it um décimo}       e       {\it décima parte}      .
\end{itemize} %s


\end{goals}

\bigskip
\begin{center}
{\large \scshape Atividade}
\end{center}
\fi

Nas figuras a seguir, um mesmo círculo azul aparece diferentemente dividido em regiões iguais, sendo algumas delas coloridas em vermelho.


\begin{center}
\begin{tabular*}{\textwidth}{ccccc}

\begin{tikzpicture}[x=1mm,y=1mm, scale=0.5]
      \draw[fill=common, fill opacity=.3] (0,0) circle (20);
      \draw[attention,fill] (0,0)
        -- ({7 * 360/9}:20) arc ({7 * 360/9}:{8 * 360/9}:20) -- (0,0);
    \foreach \x in {1,...,9}
      { \draw (0,0) -- ++({360 * \x / 9}:20); }
      \draw (0,0) circle (20);
    \node at (-20,16) {A)};
\end{tikzpicture}

&

\begin{tikzpicture}[x=1mm,y=1mm, scale=0.5]
      \draw[fill=common, fill opacity=.3] (0,0) circle (20);
      \draw[attention,fill] (0,0)
        -- ({3 * 360/8}:20) arc ({3 * 360/8}:{4 * 360/8}:20) -- (0,0);
    \foreach \x in {1,...,8}
      { \draw (0,0) -- ++({360 * \x / 8}:20); }
      \draw (0,0) circle (20);
    \node at (-20,16) {B)};
\end{tikzpicture}

&
%C)

\begin{tikzpicture}[x=1mm,y=1mm, scale=0.5]
    \draw[fill=attention] (0,0) circle (20);
    \node at (-20,16) {C)};
\end{tikzpicture}

&
%D)
\begin{tikzpicture}[x=1mm,y=1mm, scale=0.5]
      \draw[fill=common, fill opacity=.3] (0,0) circle (20);
      \draw[attention,fill] (0,0)
        -- ({1.5 * 360/6}:20) arc ({1.5 * 360/6}:{2.5 * 360/6}:20) -- (0,0);
    \foreach \x in {1.5,...,6.5}
      { \draw (0,0) -- ++({360 * \x / 6}:20); }
      \draw (0,0) circle (20);
    \node at (-20,16) {D)};
\end{tikzpicture}
&
%E)
\begin{tikzpicture}[x=1mm,y=1mm, scale=0.5]
      \draw[fill=common, fill opacity=.3] (0,0) circle (20);
      \draw[attention,fill] (0,0)
        -- (90 :20) arc (90:210:20) -- (0,0);
    \foreach \x in {1,...,3}
      { \draw (0,0) -- ++({90 + 360 * \x / 3}:20); }
      \draw (0,0) circle (20);
    \node at (-20,16) {E)};
\end{tikzpicture}
\\
%F)
\begin{tikzpicture}[x=1mm,y=1mm, scale=0.5]
      \draw[fill=common, fill opacity=.3] (0,0) circle (20);
      \draw[attention,fill] (0,0)
        -- ({10 + 5 * 360/10}:20) arc ({10 + 5 * 360/10}:{10+6 * 360/10}:20) -- (0,0);
    \foreach \x in {1,...,10}
      { \draw (0,0) -- ++({10 + 360 * \x / 10}:20); }
      \draw (0,0) circle (20);
    \node at (-20,16) {F)};
\end{tikzpicture}
&

%G)
\begin{tikzpicture}[x=1mm,y=1mm, scale=0.5]
      \draw[fill=common, fill opacity=.3] (0,0) circle (20);
      \draw[attention,fill] (0,0)-- ({90- 360/5}:20) arc ({90- 360/5}:90:20) -- (0,0);
  \foreach \x in {1,...,5}
      { \draw (0,0) -- ++({90 + 360 * \x / 5}:20); }
      \draw (0,0) circle (20);
  \node at (-20,16) {G)};
\end{tikzpicture}
&
%H)
\begin{tikzpicture}[x=1mm,y=1mm, scale=0.5]
  \draw[attention,fill] (0,0)-- (90:20) arc (90:-90:20) -- (0,0);
  \draw (0,0)-- (90:20) arc (90:-90:20) -- (0,0) --cycle;
  \draw[fill=common, fill opacity=.3] (0,0)-- (90:20) arc (90:270:20) -- (0,0) -- cycle;
  \draw (0,0) circle (20);
  \node at (-20,16) {H)};
\end{tikzpicture}
&
%I)
\begin{tikzpicture}[x=1mm,y=1mm, scale=0.5]
      \draw[fill=common, fill opacity=.3] (0,0) circle (20);
      \draw[attention,fill] (0,0)-- ({- 360/7}:20) arc ({- 360/7}:{-2 * 360/7}:20) -- (0,0);
    \foreach \x in {1,...,7}
      { \draw (0,0) -- ++({360 * \x / 7}:20); }
      \draw (0,0) circle (20);
    \node at (-20,16) {I)};
\end{tikzpicture}
&
%J)
\begin{tikzpicture}[x=1mm,y=1mm, scale=0.5]
      \draw[fill=common, fill opacity=.3] (0,0) circle (20);
      \draw[attention,fill] (0,0)-- (0:20) arc (0:{- 360/4}:20) -- (0,0);
    \foreach \x in {1,...,4}
      { \draw (0,0) -- ++({360 * \x / 4}:20); }
      \draw (0,0) circle (20);
    \node at (-20,16) {J)};
\end{tikzpicture}

 \end{tabular*}
\end{center}


\begin{enumerate} %[\quad a)] %s
  \item     Complete as sentenças a seguir identificando os círculos que as tornam verdadeiras.
\begin{enumerate}[label=\Roman*)] %[\quad I)] %d
      \item         A parte do círculo  colorida em vermelho na figura \begin{tikzpicture} \draw (0,0) -- (9,0);\end{tikzpicture} é um quinto do círculo.
      \item         A parte do círculo colorida em vermelho na figura \begin{tikzpicture} \draw (0,0) -- (9,0);\end{tikzpicture} é a sexta parte do círculo.
      \item         A parte do círculo colorida em vermelho na figura \begin{tikzpicture} \draw (0,0) -- (9,0);\end{tikzpicture} é um sétimo do círculo.
      \item         A parte do círculo colorida em vermelho na figura \begin{tikzpicture} \draw (0,0) -- (9,0);\end{tikzpicture} é um oitavo do círculo.
      \item         A parte do círculo colorida em vermelho na figura \begin{tikzpicture} \draw (0,0) -- (9,0);\end{tikzpicture} é a nona parte do círculo.
      \item         A parte do círculo colorida em vermelho na figura \begin{tikzpicture} \draw (0,0) -- (9,0);\end{tikzpicture} é um décimo do círculo.
\end{enumerate} %d
  \item     Dentre as frações do círculo destacadas em vermelho, identifique uma que seja menor do que um sexto do círculo.
  \item     Dentre as frações do círculo destacadas em vermelho, identifique uma que seja maior do que um nono do círculo.
  \item     Identifique uma fração do círculo que seja menor do que um sexto e maior do que um nono do círculo.
\end{enumerate} %s

\ifdefined\prof

\begin{solucao}

\begin{enumerate}[label=\alph*),wide,labelindent=0pt] %s
    \item       A correspondência adequada é:
\begin{enumerate} [label=\Roman*), labelindent=0pt] %d
        \item           A esta afirmação corresponde à figura G).
        \item           A esta afirmação corresponde à figura D).
        \item           A esta afirmação corresponde à figura I).
        \item           A esta afirmação corresponde à figura B).
        \item           A esta afirmação corresponde à figura A).
        \item           A esta afirmação corresponde à figura F).
\end{enumerate} %d

    \item       As frações um sétimo, um oitavo, um nono e um décimo do círculo são menores que um sexto do círculo. Qualquer uma delas está correta. Portanto qualquer uma delas serve como resposta.
    \item       As frações um meio, um terço, um quarto, um quinto, um sexto, um sétimo e um oitavo do círculo são maiores que um nono do círculo. Qualquer uma delas está correta.
    \item       As frações um sétimo e um oitavo do círculo são menores que um sexto e maiores que um nono do círculo.
\end{enumerate} %s

\end{solucao}
\fi

\end{document}