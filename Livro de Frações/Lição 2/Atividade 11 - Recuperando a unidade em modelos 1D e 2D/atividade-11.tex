\documentclass[10 pt,usenames,dvipsnames, oneside]{article}
\usepackage{../../modelo-fracoes}
\graphicspath{{../../../Figuras/licao02}}


\begin{document}

\begin{center}
  \begin{minipage}[l]{3cm}
\includegraphics[width=2cm]{../../../Figuras/logo}       
\end{minipage}\hfill
\begin{minipage}[r]{.8\textwidth}
 {\Large \scshape Atividade: Recuperando a unidade em modelos 1D e 2D}  
\end{minipage}
\end{center}
\vspace{.2cm}

\ifdefined\prof
\begin{goals}
\begin{enumerate}

    \item       Recompor a unidade a partir de uma fração dada em modelo contínuo e em linguagem simbólica, incluindo o caso de frações maiores que a unidade.
    \item       Relacionar a fração correspondente à parte apresentada à quantidade necessária dessas partes para compor a unidade. Assim, por exemplo, para compor a unidade a partir de       $\frac{2}{3}$ da unidade, basta repartir esta fração em 2 partes iguais (para recuperar a fração unitária       $\frac{1}{3}$) e, então, justapor 3 cópias de uma destas partes.


\end{enumerate}
\tcblower

  \begin{itemize} %d
    \item Recomenda-se que a atividade seja discutida em grupos de 3 a 5 alunos e respondida individualmente.
    %\item A exemplo das Atividades 5 e 7 da Lição 1, é importante ter em mente que existem várias soluções para cada item.
 \item A exemplo das Atividades 5 e 7 da Lição 1, é importante ter em mente que a forma da unidade pode variar pois a unidade pode ser representada de várias maneiras.
    \item Estimule os alunos a reconhecer (e a fazer) mais do que uma representação para a unidade em cada item.
    \item Caso seja necessário fazer alguma partição, não se espera nem se recomenda que os alunos usem alguma medida. Não se espera precisão, uma partição coerente será suficiente. 
\end{itemize} %d

\end{goals}

\bigskip
\begin{center}
{\large \scshape Atividade}
\end{center}
\fi

\begin{center}
  \begin{longtable}{|m{0.2\textwidth}|m{0.2\textwidth}|m{0.5\textwidth}|}
    \hline
     \centering Fração da unidade  & \centering  Figura correspondente à fração da unidade  & \centering Desenhe aqui a unidade  \tabularnewline
    \hline 
    \endhead
     \centering $\dfrac{1}{2}$  &\centering \parbox[c][1.1cm]{1.5cm}{\centering\begin{tikzpicture}
                                    \draw[fill=common, fill opacity=.3] (0,0) rectangle (12,6);
                                   \end{tikzpicture}}
 &  \\
    \hline
     \centering $\dfrac{4}{2}$  &   \centering \parbox[c][1.1cm]{1.5cm}{\centering\begin{tikzpicture}
                                    \draw[fill=common, fill opacity=.3] (0,0) rectangle (12,6);
                                   \end{tikzpicture}}
                                   &  \\
    \hline
     \centering $\dfrac{3}{2}$  &  \centering \parbox[c][1.1cm]{1.5cm}{\centering\begin{tikzpicture}
                                    \draw[fill=common, fill opacity=.3] (0,0) rectangle (12,6);
                                   \end{tikzpicture}}
                                   &  \\
    \hline
     \centering $\dfrac{2}{3}$  &  \centering \parbox[c][1.1cm]{1.5cm}{\centering\begin{tikzpicture}
                                    \draw[fill=common, fill opacity=.3] (0,0) rectangle (12,6);
                                   \end{tikzpicture}}
                                   &  \\
    \hline
     \centering $\dfrac{1}{2}$  &  \centering \parbox[c][1.1cm]{1.5cm}{\centering\begin{tikzpicture}
                                    \draw[fill=common, fill opacity=.3] (0,0) arc (0:180:6) -- cycle;
                                   \end{tikzpicture}}  &  \\
    \hline
      \centering $\dfrac{4}{2}$  &  \centering \parbox[c][1.1cm]{1.5cm}{\centering\begin{tikzpicture}
                                    \draw[fill=common, fill opacity=.3] (0,0) arc (0:180:6) -- cycle;
                                   \end{tikzpicture}} &  \\
    \hline
      \centering $\dfrac{3}{2}$  &  \centering \parbox[c][1.1cm]{1.5cm}{\centering\begin{tikzpicture}
                                    \draw[fill=common, fill opacity=.3] (0,0) arc (0:180:6) -- cycle;
                                   \end{tikzpicture}}  &  \\
    \hline
      \centering $\dfrac{2}{3}$  &  \centering \parbox[c][1.1cm]{1.5cm}{\centering\begin{tikzpicture}
                                    \draw[fill=common, fill opacity=.3] (0,0) arc (0:180:6) -- cycle;
                                   \end{tikzpicture}} &  \\
    \hline
      \centering $\dfrac{1}{2}$  &  \centering \parbox[c][1.1cm]{1.5cm}{\centering\begin{tikzpicture}
                                    \draw[fill=common, fill opacity=.3] (0,0) rectangle (12,1);
                                   \end{tikzpicture}}  &  \\
    \hline
      \centering $\dfrac{4}{2}$  &  \centering \parbox[c][1.1cm]{1.5cm}{\centering\begin{tikzpicture}
                                    \draw[fill=common, fill opacity=.3] (0,0) rectangle (12,1);
                                   \end{tikzpicture}}  &  \\
    \hline
      \centering $\dfrac{3}{2}$  &  \centering \parbox[c][1.1cm]{1.5cm}{\centering\begin{tikzpicture}
                                    \draw[fill=common, fill opacity=.3] (0,0) rectangle (12,1);
                                   \end{tikzpicture}}  &  \\
    \hline
      \centering $\dfrac{2}{3}$  &  \centering \parbox[c][1.1cm]{1.5cm}{\centering\begin{tikzpicture}
                                    \draw[fill=common, fill opacity=.3] (0,0) rectangle (12,1);
                                   \end{tikzpicture}}   &  \\
    \hline
        \centering $\dfrac{1}{2}$  &  \centering \parbox[c][1.1cm]{1.5cm}{\centering \begin{tikzpicture}
                                      \draw[fill=common, fill opacity=.3] (0:4) -- (60:4)--(120:4)-- (180:4)--(240:4)--(300:4)--cycle;
                                     \end{tikzpicture} } &  \\
     \hline
     \centering $\dfrac{4}{2}$  &  \centering \parbox[c][1.1cm]{1.5cm}{\centering \begin{tikzpicture}
                                    \draw[fill=common, fill opacity=.3] (0:4) -- (60:4)--(120:4)-- (180:4)--(240:4)--(300:4)--cycle;
                                   \end{tikzpicture} } &  \\
     \hline
       \centering $\dfrac{3}{2}$  &  \centering \parbox[c][1.1cm]{1.5cm}{\centering \begin{tikzpicture}
                                    \draw[fill=common, fill opacity=.3] (0:4) -- (60:4)--(120:4)-- (180:4)--(240:4)--(300:4)--cycle;
                                   \end{tikzpicture} } &  \\
    \hline
      \centering $\dfrac{2}{3}$  &  \centering \parbox[c][1.1cm]{1.5cm}{\centering \begin{tikzpicture}
                                    \draw[fill=common, fill opacity=.3] (0:4) -- (60:4)--(120:4)-- (180:4)--(240:4)--(300:4)--cycle;
                                   \end{tikzpicture} } &  \\
    \hline
  \end{longtable}
\end{center}

\ifdefined\prof

\begin{solucao}
\begin{center}
  \begin{longtable}{|m{0.2\textwidth}|m{0.3\textwidth}|c|}
    \hline
     \centering Fração da unidade  & \centering  Figura correspondente à fração da unidade  & Unidade \\
    \hline 
     \centering $\dfrac{1}{2}$ & \centering \begin{tikzpicture}[x=1mm,y=1mm]
                                    \draw[fill=common, fill opacity=.3] (0,0) rectangle (12,6);
                                   \end{tikzpicture}
        &\parbox[c][1.1cm][c]{24mm}{\begin{tikzpicture}[x=1mm,y=1mm]
 \draw[fill=common, fill opacity=.3,  ] (0,0) rectangle (24,6);
 \end{tikzpicture}} \\
    \hline
     \centering $\dfrac{4}{2}$ &   \centering \begin{tikzpicture}[x=1mm,y=1mm]
                                    \draw[fill=common, fill opacity=.3,  ] (0,0) rectangle (12,6);
                                   \end{tikzpicture}
                                   & \parbox[c][1.1cm][c]{24mm}{\centering \begin{tikzpicture}[x=1mm,y=1mm]
                                    \draw[fill=common, fill opacity=.3] (0,0) rectangle (6,6);
                                   \end{tikzpicture}} \\
    \hline
     \centering $\dfrac{3}{2}$   &  \centering \begin{tikzpicture}[x=1mm,y=1mm]
                                    \draw[fill=common, fill opacity=.3,  ] (0,0) rectangle (12,6);
                                   \end{tikzpicture}
                                   & \parbox[c][1.1cm][c]{24mm}{\centering \begin{tikzpicture}[x=1mm,y=1mm]
                                    \draw[fill=common, fill opacity=.3] (0,0) rectangle (8,6);
                                   \end{tikzpicture}}  \\
    \hline
     \centering $\dfrac{2}{3}$ &  \centering \begin{tikzpicture}[x=1mm,y=1mm]
                                    \draw[fill=common, fill opacity=.3,  ] (0,0) rectangle (12,6);
                                  \end{tikzpicture}
                                  & \parbox[c][1.1cm][c]{24mm}{\centering \begin{tikzpicture}[x=1mm,y=1mm]
\draw[fill=common, fill opacity=.3] (0,0) rectangle (18,6);
\end{tikzpicture}} \\
\hline
\centering $\dfrac{1}{2}$ &  \centering \begin{tikzpicture}[x=1mm,y=1mm]
\draw[fill=common, fill opacity=.3,  ] (0,0) arc (0:180:6) -- cycle;
\end{tikzpicture}
& \parbox[c][1.3cm][c]{24mm}{\centering \begin{tikzpicture}[x=1mm,y=1mm]
\draw[fill=common, fill opacity=.3] (0,0) circle (6);
\end{tikzpicture}}
\\
\hline
\centering $\dfrac{4}{2}$ &  \centering \begin{tikzpicture}[x=1mm,y=1mm]
\draw[fill=common, fill opacity=.3,  ] (0,0) arc (0:180:6) -- cycle;
\end{tikzpicture}
& \parbox[c][1.1cm][c]{24mm}{\centering \begin{tikzpicture}[x=1mm,y=1mm]
\draw[fill=common, fill opacity=.3] (-6,6) arc (90:180:6) -- (-6,0)-- cycle;
\end{tikzpicture}}
\\
\hline
\centering $\dfrac{3}{2}$ &  \centering \begin{tikzpicture}[x=1mm,y=1mm]
\draw[fill=common, fill opacity=.3,  ] (0,0) arc (0:180:6) -- cycle;
\end{tikzpicture}
&
\parbox[c][1.1cm][c]{24mm}{\centering \begin{tikzpicture}[x=1mm,y=1mm]
\draw[fill=common, fill opacity=.3,  ] (0,0) -- (60:6) arc (60:180:6) -- cycle;
\end{tikzpicture}}
\\
\hline
\centering $\dfrac{2}{3}$ &  \centering \begin{tikzpicture}[x=1mm,y=1mm]
\draw[fill=common, fill opacity=.3,  ] (0,0) arc (0:180:6) -- cycle;
                                   \end{tikzpicture} & \parbox[c][1.3cm][c]{24mm}{\centering \begin{tikzpicture}[x=1mm,y=1mm]
\draw[fill=common, fill opacity=.3] (0:6) arc (0:270:6) -- (0,0) -- cycle;
\end{tikzpicture}} \\
    \hline
\centering $\dfrac{1}{2}$ & \centering \begin{tikzpicture}[x=1mm,y=1mm]
 \draw[fill=common, fill opacity=.3,  ] (0,0) rectangle (12,1);
                                  \end{tikzpicture}
& \parbox[c][1.1cm][c]{24mm}{\centering \begin{tikzpicture}[x=1mm,y=1mm]
                                    \draw[fill=common, fill opacity=.3] (0,0) rectangle (24,1);
                                   \end{tikzpicture}} \\
    \hline
    \centering $\dfrac{4}{2}$ &  \centering \begin{tikzpicture}[x=1mm,y=1mm]
 \draw[fill=common, fill opacity=.3,  ] (0,0) rectangle (12,1);
                                  \end{tikzpicture}
& \parbox[c][1.1cm][c]{24mm}{\centering \begin{tikzpicture}[x=1mm,y=1mm]
                                    \draw[fill=common, fill opacity=.3] (0,0) rectangle (6,1);
                                   \end{tikzpicture}} \\
    \hline
      \centering $\dfrac{3}{2}$ &  \centering \begin{tikzpicture}[x=1mm,y=1mm]
\draw[fill=common, fill opacity=.3,  ] (0,0) rectangle (12,1);                 \end{tikzpicture}                                                                                            & \parbox[c][1.1cm][c]{24mm}{\centering \begin{tikzpicture}[x=1mm,y=1mm]
                                    \draw[fill=common, fill opacity=.3] (0,0) rectangle (8,1);
                                   \end{tikzpicture}} \\
    \hline
\centering $\dfrac{2}{3}$ &  \centering \begin{tikzpicture}[x=1mm,y=1mm]
\draw[fill=common, fill opacity=.3,  ] (0,0) rectangle (12,1);
\end{tikzpicture} & \parbox[c][1.1cm][c]{24mm}{\centering \begin{tikzpicture}[x=1mm,y=1mm]
                                    \draw[fill=common, fill opacity=.3,  ] (0,0) rectangle (18,1);
                                   \end{tikzpicture}}  \\
    \hline
        \centering $\dfrac{1}{2}$ &  \centering \begin{tikzpicture}[x=1mm,y=1mm]
                                      \draw[fill=common, fill opacity=.3,  ] (0:4) -- (60:4)--(120:4)-- (180:4)--(240:4)--(300:4)--cycle;
                                    \end{tikzpicture}                                                                                            & \parbox[c][1.1cm][c]{24mm}{\centering \begin{tikzpicture}[x=1mm,y=1mm]
%\draw[fill=common, fill opacity=.3] (0:4) -- (60:4) -- (120:4) -- (180:4) -- (240:4) -- (300:4) -- cycle;
%\draw[fill=common, fill opacity=.3, shift={(-6,{2*sqrt(3)})}] (180:4) -- (0:4) -- (300:4) -- (240:4)--cycle;
%\draw[fill=common, fill opacity=.3, shift={(-6,{-2*sqrt(3)})}] (180:4) -- (0:4) -- (60:4) -- (120:4)--cycle;
                                    \draw[fill=common, fill opacity=.3] (-10,-3.42) -- (240:4) -- (300:4) -- (0:4) -- (60:4) -- (120:4) --+ (-8,0) -- (-8,0) -- cycle;
                                    \draw [dashed, gray] (180:8) -- (180:4) -- (240:4);
\draw [dashed, gray] (180:4) -- (120:4);
                                  \end{tikzpicture}}\\

%                                      ou \begin{tikzpicture}[x=1mm,y=1mm]
%                                     \draw[fill=common, fill opacity=.3] (0:4) -- (60:4)--(120:4)-- (180:4)--(240:4)--(300:4)--cycle;
%                                     \draw[fill=common, fill opacity=.3, shift={(8,0)} ] (0:4) -- (60:4)--(120:4)-- (180:4)--(240:4)--(300:4)--cycle;
%                                     \end{tikzpicture}
     \hline
     \centering $\dfrac{4}{2}$ &  \centering \begin{tikzpicture}[x=1mm,y=1mm]
                                    \draw[fill=common, fill opacity=.3,  ] (0:4) -- (60:4)--(120:4)-- (180:4)--(240:4)--(300:4)--cycle;
                                   \end{tikzpicture} & \parbox[c][1.1cm][c]{24mm}{\centering \begin{tikzpicture}[x=1mm,y=1mm]
                                   \draw[fill=common, fill opacity=.3] (180:4) -- (0:4) -- (60:4) -- (120:4)--cycle;
\end{tikzpicture}}  \\
\hline
\centering $\dfrac{3}{2}$ &  \centering  \begin{tikzpicture}[x=1mm,y=1mm]
\draw[fill=common, fill opacity=.3,  ] (0:4) -- (60:4)--(120:4)-- (180:4)--(240:4)--(300:4)--cycle;
\end{tikzpicture}
& \parbox[c][1.1cm][c]{24mm}{\centering \begin{tikzpicture}[x=1mm,y=1mm]
\draw[fill=common, fill opacity=.3] (0,0) -- (0:4) -- (60:4) -- (120:4)-- (180:4) -- (240:4) -- cycle;
%\draw[fill=common, fill opacity=.3] (240:4)  -- (300:4)-- (0:4)-- (0,0) --cycle;
%\draw[ ] (0:4) -- (60:4)--(120:4)-- (180:4)--(240:4)--(300:4)--cycle;
%\draw (0,0) -- (120:4);
\end{tikzpicture}} \\
    \hline
      \centering $\dfrac{2}{3}$ &  \centering \begin{tikzpicture}[x=1mm,y=1mm]
                                    \draw[fill=common, fill opacity=.3,  ] (0:4) -- (60:4)--(120:4)-- (180:4)--(240:4)--(300:4)--cycle; \end{tikzpicture}                                         &  \parbox[c][1.1cm][c]{24mm}{\centering \begin{tikzpicture}[x=1mm,y=1mm]
%\draw[fill=common, fill opacity=.3] (0:4) -- (60:4)--(120:4)-- (180:4) -- (240:4) -- (300:4) -- cycle;
%\draw[fill=common, fill opacity=.3, shift={(-6,{-2*sqrt(3)})}] (180:4) -- (0:4) -- (60:4) -- (120:4)--cycle;
%\draw[ ] (0:4) -- (60:4)--(120:4)-- (180:4)--(240:4)--(300:4)--cycle;
\end{tikzpicture}}  \\
    \hline
  \end{longtable}
\end{center}

\end{solucao}
\fi

\end{document}