\documentclass[10 pt,usenames,dvipsnames, oneside]{article}
\usepackage{../../../modelo-fracoes}
\graphicspath{{../../../Figuras/licao02/}}


\begin{document}

\begin{center}
  \begin{minipage}[l]{3cm}
\includegraphics[width=2cm]{logo}    
\end{minipage}\hfill
\begin{minipage}[r]{.8\textwidth}
 {\Large \scshape Atividade: Mesma fração, diferentes quantidades}  
\end{minipage}
\end{center}
\vspace{.2cm}

\ifdefined\prof
\begin{goals}
\begin{enumerate}

    \item       Perceber que uma mesma fração (no caso, $\frac{1}{2}$) de unidades diferentes pode resultar em quantidades diferentes.

\end{enumerate}
\tcblower

  \begin{itemize} %s
    \item       Esta é uma atividade que o aluno pode fazer individualmente, mas é essencial que seja discutida com toda a turma.
    \item       Como fechamento da atividade, é importante enfatizar que  uma mesma fração de unidades diferentes pode resultar em quantidades diferentes. Sugere-se ressaltar aos estudantes que, no contexto ``frações de'', é fundamental saber a que o ``de'' se refere, isto é, qual é a unidade que está sendo considerada. Neste sentido, esta atividade está fortemente relacionada com a Atividade 8.
\end{itemize} %s

\end{goals}

\bigskip
\begin{center}
{\large \scshape Atividade}
\end{center}
\fi

(NAEP, 1992) Pense cuidadosamente nesta questão. Escreva uma resposta completa. Você pode usar desenhos, palavras e números para explicar sua resposta. Certifique-se de mostrar todo o seu raciocínio.

José comeu $\frac{1}{2}$ de uma pizza. Ella comeu $\frac{1}{2}$ de uma outra pizza. José disse que ele comeu mais pizza do que Ella, mas Ella diz que eles comeram a mesma quantidade. Use palavras, figuras ou números para mostrar que José pode estar certo.
%\vspace*{-.5cm}

\ifdefined\prof

\begin{solucao}

  José está certo se a pizza da qual comeu metade for maior do que a pizza da qual Ella comeu metade, como ilustra a figura a seguir.
  \begin{center}
   \begin{tikzpicture}[x=1mm,y=1mm]
    \draw (0,0) circle (4);
    \draw[fill=attention] (90:4) arc (90:270:4) --  cycle;
    \draw[fill=common, fill opacity=.3] (90:4) arc (90:-90:4) --  cycle;
    \node at (0,8) {Pizza de Ella};
    \begin{scope}[shift={(30,0)}]
    \draw (0,0) circle (8);
    \draw[fill=attention] (90:8) arc (90:270:8) --  cycle;
    \draw[fill=common, fill opacity=.3] (90:8) arc (90:-90:8) --  cycle;
    \node at (0,12) {Pizza de José};
    \end{scope}
   \end{tikzpicture}

  \end{center}

\end{solucao}
\fi

\end{document}