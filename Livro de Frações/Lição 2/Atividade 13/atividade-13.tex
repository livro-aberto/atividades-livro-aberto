\documentclass[10 pt,usenames,dvipsnames, oneside]{article}
\usepackage{../../../modelo-fracoes}
\graphicspath{{../../../Figuras/licao02/}}


\begin{document}

\begin{center}
  \begin{minipage}[l]{3cm}
\includegraphics[width=2cm]{logo}    
\end{minipage}\hfill
\begin{minipage}[r]{.8\textwidth}
 {\Large \scshape Atividade: Mesma quantidade, diferentes frações}  
\end{minipage}
\end{center}
\vspace{.2cm}

\ifdefined\prof
\begin{goals}
\begin{enumerate}

    \item Reconhecer que uma mesma quantidade pode ser expressa por frações diferentes dependendo da unidade escolhida.

\end{enumerate}
\tcblower

  \begin{itemize} %s
  \item Uma resposta possível é que dois (ou até os três) estudantes tenham errado em suas afirmações.    Muito provavelmente, nesse caso, os estudantes estão considerando a mesma unidade - bolos de mesmo tamanho. 
Como o objetivo da questão é levar os alunos a concluírem que unidades diferentes podem determinar a identificação de frações diferentes para uma mesma quantidade, cabe ao professor instigar essa discussão com os alunos. Espera-se que leve-os a considerar o caso de os bolos originais terem tamanhos diferentes. Isso pode ser feito com perguntas como, por exemplo: É possível uma tal situação acontecer estando os três amigos certos?

Será que os bolos eram iguais? Será que tinham o mesmo tamanho? 
\end{itemize}

\end{goals}

\bigskip
\begin{center}
{\large \scshape Atividade}
\end{center}
\fi

Anita, Gustavo e Henrique descobriram que todos tinham levado bolo para o lanche.
\begin{itemize}[label=---]
\item Anita falou: ``Mamãe colocou metade do bolo no meu lanche.''
\item Gustavo falou: ``Eu trouxe um terço do bolo que minha tia fez.''
\item Henrique falou: ``Eu trouxe apenas um quinto do bolo que minha mãe preparou!''
\end{itemize}
Para surpresa de todos, ao retirarem seus lanches da mochila, descobriram que todos traziam-no em  embalagens iguais, portanto traziam a mesma quantidade de bolo.

Como você explica tal situação?

\ifdefined\prof

\begin{solucao}

Se os bolos tivessem tamanhos iguais, um quinto do bolo teria menor quantidade que um terço do bolo, que por sua vez corresponderia a menos bolo que metade. Como todos os estudantes tinham a mesma quantidade, os bolos eram diferentes.
  
\end{solucao}
\fi

\end{document}