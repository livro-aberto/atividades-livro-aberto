\documentclass[10 pt,usenames,dvipsnames, oneside]{article}
\usepackage{../../../modelo-fracoes}
\graphicspath{{../../../Figuras/licao02/}}


\begin{document}

\begin{center}
  \begin{minipage}[l]{3cm}
\includegraphics[width=2cm]{logo}    
\end{minipage}\hfill
\begin{minipage}[r]{.8\textwidth}
 {\Large \scshape Atividade: Comparando um quarto e um oitavo}  
\end{minipage}
\end{center}
\vspace{.2cm}

\ifdefined\prof
\begin{goals}
\begin{enumerate}

  \item Expressar as frações $\frac{1}{4}$ e $\frac{1}{8}$ em símbolos matemáticos.
  \item Comparar as frações um quarto e um oitavo a partir de modelos visuais.

\end{enumerate}
\tcblower

\begin{itemize} %s
    \item Esta atividade pode ser resolvida individualmente, mas é essencial que seja discutida com toda a turma.
    \item Os estudantes compararam as frações um quarto e um oitavo em modelos retangulares na Atividade 8 da Lição 1, isso justifica a opção por um modelo circular nesta atividade.
    \item Em particular, incentive os alunos a argumentar, justificando a sua resposta.
    \item Conduza a discussão de modo a conseguirem reconhecer a relação inversa entre denominador (número de partes) e o tamanho de cada parte: quanto maior o denominador, maior é o número de partes em que foi repartida a pizza, logo menor o tamanho da parte.
      \item Embora não seja o objetivo da atividade, algum estudante pode reconhecer que uma fatia da primeira pizza tem o dobro da quantidade de uma fatia da segunda  pizza, ou seja, são necessárias duas fatias da segunda pizza para ter-se a mesma quantidade de pizza que uma fatia da primeira pizza.
\end{itemize}

\end{goals}

\bigskip
\begin{center}
{\large \scshape Atividade}
\end{center}
\fi

Um grupo de amigos está dividindo duas pizzas circulares iguais, isto é, de mesmo tamanho. A primeira pizza foi cortada em 4 fatias de mesmo tamanho. A segunda pizza foi repartida em 8 pedaços iguais.
\begin{enumerate} [label=\alph*)] %s
\item Uma fatia da primeira pizza é que fração dessa pizza? Responda usando símbolos matemáticos.
\item Uma fatia da segunda pizza é que fração dessa pizza? Responda usando símbolos matemáticos.
\item Qual fatia tem mais quantidade de pizza: uma fatia da primeira ou uma fatia da segunda? Explique usando uma figura.
\end{enumerate} %s

\ifdefined\prof
\clearpage
\begin{solucao}
\begin{multicols}{2}
\begin{enumerate} [label=\alph*)] %s
    \item $\frac{1}{4}$.
    \item $\frac{1}{8}$.
    \item Uma fatia da primeira pizza é maior do que uma fatia da segunda pizza: como partimos a segunda pizza em mais partes, cada fatia tem menos pizza.

\end{enumerate} %s
\begin{center}
       \begin{tikzpicture}[x=1mm,y=1mm, scale=.7]
        \draw[fill=common, fill opacity=.3] (0,0) circle (10);
        \fill[attention] (180:10) arc (180:270:10) -- (0,0) -- cycle;
        \foreach \x in {0,90}\draw (\x:10) -- (\x:-10);
       \end{tikzpicture} \quad \quad
       \begin{tikzpicture}[x=1mm,y=1mm, scale=.7]
        \draw[fill=common, fill opacity=.3] (0,0) circle (10);
        \fill[attention] (180:10) arc (180:270:10) -- (0,0) -- cycle;
        \foreach \x in {0,45,90, 135}\draw (\x:10) -- (\x:-10);
       \end{tikzpicture}
\end{center}
\end{multicols}
\end{solucao}
\fi

\end{document}