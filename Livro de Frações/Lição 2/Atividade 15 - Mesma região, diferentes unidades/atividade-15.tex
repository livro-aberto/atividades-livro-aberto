\documentclass[10 pt,usenames,dvipsnames, oneside]{article}
\usepackage{../../modelo-fracoes}
\graphicspath{{../../../Figuras/licao02}}


\begin{document}

\begin{center}
  \begin{minipage}[l]{3cm}
\includegraphics[width=2cm]{../../../Figuras/logo}       
\end{minipage}\hfill
\begin{minipage}[r]{.8\textwidth}
 {\Large \scshape Atividade: Mesma região, diferentes unidades}  
\end{minipage}
\end{center}
\vspace{.2cm}

\ifdefined\prof
\begin{goals}
\begin{enumerate}

    \item Reconhecer que uma mesma quantidade pode ser expressa por frações diferentes dependendo da unidade escolhida.
    \item Utilizar linguagem simbólica para referir-se a uma fração~$\frac{a}{b}$.

\end{enumerate}
\tcblower

  \begin{itemize} %s
  \item Recomenda-se que a atividade seja desenvolvida em grupos de 3 a 5 alunos.
    \item Recomenda-se que os itens sejam propostos em blocos (de três em três, por exemplo) intercalados com a correção. Tendo em vista que se o estudante não atingiu os objetivos da atividade nos primeiros itens, ele provavelmente não conseguirá fazê-lo nas seguintes sem a intervenção do professor.
    \item As diversas soluções apresentadas devem ser discutidas com a
turma inteira. É possível que os alunos utilizem frações equivalentes como
resposta para um mesmo item. Por exemplo, no item f), as frações $\frac{3}{6}$ e $\frac{1}{2}$ são respostas corretas. Nesses casos, dê a oportunidade para que cada aluno explique como chegou a sua resposta. Os alunos perceberão que uma mesma quantidade pode ser descrita por frações com numeradores e denominadores diferentes. Isso vai prepará-los para o assunto frações equivalentes, que será tratado na Lição 4.
    \item Observe que, no contexto ``frações de'', é fundamental saber a que o ``de'' se refere, isto é, qual é a unidade que está sendo considerada. Assim, no final da atividade, é importante enfatizar para os alunos que uma mesma quantidade pode ser descrita por frações diferentes com unidades diferentes.
\end{itemize} %s

\end{goals}

\bigskip
\begin{center}
{\large \scshape Atividade}
\end{center}
\fi

Complete as sentenças a seguir com uma fração adequada (usando símbolos matemáticos). Perceba que uma mesma região pintada pode ser descrita por frações diferentes, dependendo da unidade considerada.

%definição da região limitada pelos 2 hexágonos encaixados.
\def \tripinha{ (30:4) -- (90:4) -- (150:4)--(210:4)--(270:4)--(330:4) [shift={({4*sqrt(3)},0)}] --(270:4) -- (330:4) -- (30:4) -- (90:4)--(150:4)--cycle;}
%definição da região limitada pelos 3 hexágonos encaixados.
\def \tripa{ (30:4) -- (90:4) -- (150:4)--(210:4)--(270:4)--(330:4) [shift={({4*sqrt(3)},0)}] --(270:4) -- (330:4) [shift={({4*sqrt(3)},0)}]--  (270:4) -- (330:4) -- (30:4) -- (90:4)--(150:4) [shift={({-4*sqrt(3)},0)}] -- (90:4) -- (150:4)--cycle;}

\begin{enumerate} [label=\alph*)] %s
%a
\item     A região pintada em vermelho em
\begin{tikzpicture}[scale=1.3]
 \draw \tripa;
\begin{scope}
 \clip \tripa;
\draw[fill=attention] (-4,-4) rectangle (0,4);
\draw[fill=common, fill opacity=.3] (0,-4) rectangle (20,4);
\end{scope}
\end{tikzpicture}
é
\begin{tikzpicture} \draw (0,0)--(20,0);\end{tikzpicture}
de
\begin{tikzpicture}[scale=1.3]
\draw[fill=common, fill opacity=.3] (30:4) -- (90:4) -- (150:4) -- (210:4) -- (270:4) -- (330:4)--cycle;
\end{tikzpicture}.
 %b
\item     A região pintada em vermelho em
\begin{tikzpicture}[scale=1.3]
 \draw \tripa;
\begin{scope}
 \clip \tripa;
\draw[fill=attention] (-4,-4) rectangle (0,4);
\draw[fill=common, fill opacity=.3] (0,-4) rectangle (20,4);
\end{scope}
\end{tikzpicture}
é
\begin{tikzpicture} \draw (0,0)--(20,0);\end{tikzpicture}
de
\begin{tikzpicture}[scale=1.3]
 \draw[fill=common, fill opacity=.3] (30:4) -- (90:4) -- (150:4)--(210:4)--(270:4)--(330:4) [shift={({4*sqrt(3)},0)}] --(270:4) -- (330:4) --(30:4) -- (90:4) -- (150:4)--cycle;
\end{tikzpicture}.

%c
\item     A região pintada em vermelho em \begin{tikzpicture}[scale=1.3]
 \draw \tripa;
\begin{scope}
 \clip \tripa;
\draw[fill=attention] (-4,-4) rectangle (0,4);
\draw[fill=common, fill opacity=.3] (0,-4) rectangle (20,4);
\end{scope}
\end{tikzpicture}
     é \begin{tikzpicture} \draw (0,0)--(20,0);\end{tikzpicture}    de
\begin{tikzpicture}[scale=1.3]
\draw[fill=common, fill opacity=.3] \tripa;
\end{tikzpicture}.

%d
 \item     A região pintada em vermelho em  \begin{tikzpicture}[scale=1.3]
 \draw \tripa;
\begin{scope}
 \clip \tripa;
\draw[fill=attention] (-4,-4) rectangle ({4*sqrt(3)},4);
\draw[fill=common, fill opacity=.3] ({4*sqrt(3)},-4) rectangle ({10*sqrt(3)},4);
\end{scope}
\end{tikzpicture}  é \begin{tikzpicture} \draw (0,0)--(20,0);\end{tikzpicture}    de
\begin{tikzpicture}[scale=1.3]
\draw[fill=common, fill opacity=.3] (30:4) -- (90:4) -- (150:4) -- (210:4) -- (270:4) -- (330:4)--cycle;
\end{tikzpicture}.
%e
\item     A região pintada em vermelho em   \begin{tikzpicture}[scale=1.3]
 \draw \tripa;
\begin{scope}
 \clip \tripa;
\draw[fill=attention] (-4,-4) rectangle ({4*sqrt(3)},4);
\draw[fill=common, fill opacity=.3] ({4*sqrt(3)},-4) rectangle ({10*sqrt(3)},4);
\end{scope}
\end{tikzpicture}    é \begin{tikzpicture} \draw (0,0)--(20,0);\end{tikzpicture}    de
\begin{tikzpicture}[scale=1.3]
 \draw[fill=common, fill opacity=.3] (30:4) -- (90:4) -- (150:4)--(210:4)--(270:4)--(330:4) [shift={({4*sqrt(3)},0)}] --(270:4) -- (330:4) --(30:4) -- (90:4) -- (150:4)--cycle;
\end{tikzpicture}.

%f
\item     A região pintada em vermelho em  \begin{tikzpicture}[scale=1.3]
 \draw \tripa;
\begin{scope}
 \clip \tripa;
\draw[fill=attention] (-4,-4) rectangle ({4*sqrt(3)},4);
\draw[fill=common, fill opacity=.3] ({4*sqrt(3)},-4) rectangle ({10*sqrt(3)},4);
\end{scope}
\end{tikzpicture}   é \begin{tikzpicture} \draw (0,0)--(20,0);\end{tikzpicture}    de
\begin{tikzpicture}[scale=1.3]
\draw[fill=common, fill opacity=.3] \tripa;
\end{tikzpicture}.

%g
\item     A região pintada em vermelho em
\begin{tikzpicture}[scale=1.3]
 \draw \tripa;
\begin{scope}
 \clip \tripa;
\draw[fill=attention] (-4,-4) rectangle ({8*sqrt(3)},4);
\draw[fill=common, fill opacity=.3] ({8*sqrt(3)},-4) rectangle ({10*sqrt(3)},4);
\end{scope}
\end{tikzpicture}
é \begin{tikzpicture} \draw (0,0)--(20,0);\end{tikzpicture}    de
\begin{tikzpicture}[scale=1.3]
\draw[fill=common, fill opacity=.3] (30:4) -- (90:4) -- (150:4) -- (210:4) -- (270:4) -- (330:4)--cycle;
\end{tikzpicture}.
%h
\item     A região pintada em vermelho em  \begin{tikzpicture}[scale=1.3]
 \draw \tripa;
\begin{scope}
 \clip \tripa;
\draw[fill=attention] (-4,-4) rectangle ({8*sqrt(3)},4);
\draw[fill=common, fill opacity=.3] ({8*sqrt(3)},-4) rectangle ({10*sqrt(3)},4);
\end{scope}
\end{tikzpicture} é \begin{tikzpicture} \draw (0,0)--(20,0);\end{tikzpicture}    de
\begin{tikzpicture}[scale=1.3]
 \draw[fill=common, fill opacity=.3] (30:4) -- (90:4) -- (150:4)--(210:4)--(270:4)--(330:4) [shift={({4*sqrt(3)},0)}] --(270:4) -- (330:4) --(30:4) -- (90:4) -- (150:4)--cycle;
\end{tikzpicture}.

%i
\item     A região pintada em vermelho em
\begin{tikzpicture}[scale=1.3]
 \draw \tripa;
\begin{scope}
 \clip \tripa;
\draw[fill=attention] (-4,-4) rectangle ({8*sqrt(3)},4);
\draw[fill=common, fill opacity=.3] ({8*sqrt(3)},-4) rectangle ({10*sqrt(3)},4);
\end{scope}
\end{tikzpicture}
é \begin{tikzpicture} \draw (0,0)--(20,0);\end{tikzpicture}    de
\begin{tikzpicture}[scale=1.3]
\draw[fill=common, fill opacity=.3] \tripa;
\end{tikzpicture}.
%j
\item     A região pintada em vermelho em
  \begin{tikzpicture}[scale=1.3]
 \draw \tripa;
\begin{scope}
 \clip \tripa;
\draw[fill=attention] (-4,-4) rectangle ({12*sqrt(3)},4);
\end{scope}
\end{tikzpicture}
é \begin{tikzpicture} \draw (0,0)--(20,0);\end{tikzpicture}    de
\begin{tikzpicture}[scale=1.3]
\draw[fill=common, fill opacity=.3] (30:4) -- (90:4) -- (150:4) -- (210:4) -- (270:4) -- (330:4)--cycle;
\end{tikzpicture}.
%k
\item     A região pintada em vermelho em
\begin{tikzpicture}[scale=1.3]
 \draw \tripa;
\begin{scope}
 \clip \tripa;
\draw[fill=attention] (-4,-4) rectangle ({12*sqrt(3)},4);
\end{scope}
\end{tikzpicture}
é \begin{tikzpicture} \draw (0,0)--(20,0);\end{tikzpicture}    de
\begin{tikzpicture}[scale=1.3]
 \draw[fill=common, fill opacity=.3] (30:4) -- (90:4) -- (150:4)--(210:4)--(270:4)--(330:4) [shift={({4*sqrt(3)},0)}] --(270:4) -- (330:4) --(30:4) -- (90:4) -- (150:4)--cycle;
\end{tikzpicture}.

%l
\item     A região pintada em vermelho em
  \begin{tikzpicture}[scale=1.3]
 \draw \tripa;
\begin{scope}
 \clip \tripa;
\draw[fill=attention] (-4,-4) rectangle ({12*sqrt(3)},4);
\end{scope}
\end{tikzpicture}
é \begin{tikzpicture} \draw (0,0)--(20,0);\end{tikzpicture}    de
\begin{tikzpicture}[scale=1.3]
\draw[fill=common, fill opacity=.3] \tripa;
\end{tikzpicture}.

\end{enumerate} %s

\ifdefined\prof

\begin{solucao}

\noindent\begin{tabular}{m{.12\textwidth}m{.12\textwidth}m{.12\textwidth}m{.14\textwidth}m{.14\textwidth}m{.14\textwidth}}
  a) $\dfrac{1}{2}$. &  b) $\dfrac{1}{4}$. & c) $\dfrac{1}{6}$. & d) $\dfrac{3}{2}$. &  e) $\dfrac{3}{4}$. &  f) $\dfrac{1}{2}$ ou $\dfrac{3}{6}$.   \\
\\
g) $\dfrac{5}{2}$. &  h) $\dfrac{5}{4}$. & i) $\dfrac{5}{6}$. & j) $3$ ou $\dfrac{6}{2}$.
    &  k) $\dfrac{3}{2}$ ou $\dfrac{6}{4}$. &  l) $1$ ou $\dfrac{3}{3}$ ou $\dfrac{6}{6}$.
\end{tabular} %s

\end{solucao}
\fi

\end{document}