\documentclass[10 pt,usenames,dvipsnames, oneside]{article}
\usepackage{../../modelo-fracoes}
\graphicspath{{../../../Figuras/licao02}}


\begin{document}

\begin{center}
  \begin{minipage}[l]{3cm}
\includegraphics[width=2cm]{../../../Figuras/logo}       
\end{minipage}\hfill
\begin{minipage}[r]{.8\textwidth}
 {\Large \scshape Atividade: Maior, menor ou igual a meio?}  
\end{minipage}
\end{center}
\vspace{.2cm}

\ifdefined\prof
\begin{goals}
\begin{enumerate}

    \item       Comparar frações com relação a uma fração de referência (no caso, a fração       $\frac{1}{2}$) usando modelos contínuos (área).

\end{enumerate}
\tcblower

\begin{itemize} %s
\item Essa é uma atividade que o aluno pode fazer individualmente.
\item Incentive seus alunos a darem justificativas para suas respostas, mesmo que informais.
\end{itemize} %s

\end{goals}

\bigskip
\begin{center}
{\large \scshape Atividade}
\end{center}
\fi

Para cada figura a seguir, indique a fração da figura que está pintada de vermelho. Esta fração é maior, menor ou exatamente igual a $\frac{1}{2}$ da figura?
\bigskip
\begin{center}
\begin{multicols}{3}
\begin{enumerate}
\item\adjustbox{valign=t}{

\begin{tikzpicture}
\draw[fill=common, fill opacity=.3] (0,0) circle (10);
 \foreach \x in {0,72,...,288}{
 \draw[fill=attention] (0,0) -- (\x:10) arc (\x:\x+36:10) --cycle;
 \draw (\x:10) -- (\x:-10);}
\end{tikzpicture} }

\item\adjustbox{valign=t}{

\begin{tikzpicture}
\draw[fill=common, fill opacity=.3] (0,0) rectangle (14,20);
\draw[fill=attention] (0,4) rectangle (7,20);
\foreach \y in {4,8,12,16}{
\draw (0,\y)--(14,\y);}
\draw (7,0) -- (7,20);
\end{tikzpicture} }

\item\adjustbox{valign=t}{

\begin{tikzpicture}
\draw[fill=common, fill opacity=.3] (0,0) rectangle (30,20);
\fill[attention] (0,0) rectangle (18,20);
 \foreach \x in {3,6,...,27}{
 \draw (\x,0)--(\x,20);}
\end{tikzpicture} }
\end{enumerate}
\end{multicols}
\end{center}

\ifdefined\prof

\begin{solucao}

\begin{enumerate} [label=\alph*)] %s
    \item A parte pintada é igual a $\frac{5}{10}$ da figura, ou seja, metade da figura.
    \item A parte pintada é igual a $\frac{4}{10}$, que é menor do que       $\frac{5}{10}$ da figura, metade da figura.
    \item  A parte pintada é igual a $\frac{6}{10}$ e é maior do que       $\frac{5}{10}$ da figura, que é a metade da figura.
\end{enumerate} %s

\end{solucao}
\fi

\end{document}