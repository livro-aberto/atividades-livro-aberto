\documentclass[10 pt,usenames,dvipsnames, oneside]{article}
\usepackage{../../../modelo-fracoes}
\graphicspath{{../../../Figuras/licao02/}}


\begin{document}

\begin{center}
  \begin{minipage}[l]{3cm}
\includegraphics[width=2cm]{logo}    
\end{minipage}\hfill
\begin{minipage}[r]{.8\textwidth}
 {\Large \scshape Atividade: Juntando frações de uma pizza}  
\end{minipage}
\end{center}
\vspace{.2cm}

\ifdefined\prof
\begin{goals}
\begin{enumerate}

\item Estender o conceito de frações para expressar quantidades que correspondam a mais do que uma fração unitária, a partir da junção de duas ou mais partes correspondentes às frações unitárias de mesmo denominador.
\item Reconhecer a necessidade de apresentar uma expressão verbal que identifique a quantidade correspondente à junção de duas ou

\end{enumerate}
\tcblower

  \begin{itemize} %s
\item Recomenda-se que a atividade seja realizada individualmente e que os alunos compartilhem suas respostas. 
\item É possível que os alunos utilizem expressões variadas para nomear as quantidades de pizza que cada amigo comeu. Por exemplo, ``três de quatro fatias'', ``três fatias de um quarto de pizza'', ``três quartos de pizza'', dentre outras. É importante que a discussão conduza os alunos ao uso de quartos, sextos e oitavos: ``três quartos'', ``cinco sextos'', ``cinco oitavos'', etc. 
\end{itemize} %s


\end{goals}

\bigskip
\begin{center}
{\large \scshape Atividade}
\end{center}
\fi

Luiza, João e Mariele foram a uma pizzaria. Cada um pediu uma pizza do seu sabor preferido. 
Luiza cortou sua pizza em 4 fatias; João cortou sua pizza em 6 fatias e Mariele cortou sua pizza em 8 fatias.
%Veja o quanto restou de pizza após os amigos estarem satisfeitos:
O esquema a seguir indica o quanto restou de pizza após os amigos estarem satisfeitos:

\begin{center}
\begin{tikzpicture}[scale=.7]
\draw [dashed] (0,0) circle (2cm);
\filldraw[fill=common] (0,0) -- (120: 2cm) arc (120:240:2cm)--cycle;
\foreach \t in {0,60,300}{
  \draw [dashed] (0,0) -- (\t: 2cm);
}
\draw (0,0) -- (180:2cm);
\end{tikzpicture} \hfill
\begin{tikzpicture}[scale=.7]
\draw [dashed] (0,0) circle (2cm);
  \filldraw[fill=common] (0,0) -- (135: 2cm) arc (135:225:2cm)--cycle;

\foreach \t in {0,45,...,315}{
  \draw [dashed] (0,0) -- (\t: 2cm);
}
\draw  (0,0) -- (135: 2cm);
\draw  (0,0) -- (180: 2cm);
\draw  (0,0) -- (225: 2cm);
\end{tikzpicture}\hfill
\begin{tikzpicture}[scale=.7]
\draw [dashed] (0,0) circle (2cm);
  \filldraw[fill=common] (0,0) -- (180: 2cm) arc (180:270:2cm)--cycle;

\foreach \t in {0,90}{
  \draw [dashed] (0,0) -- (\t: 2cm);
}
\draw (0,0) -- (180: 2cm);
\draw (0,0) -- (270: 2cm);
\end{tikzpicture}
\end{center}

\begin{enumerate} [label=\alph*)] %d
\item   Identifique a pizza de cada um dos amigos.
\item   Em cada caso, que fração da pizza representa uma fatia?
\item   Escreva a quantidade de pizza que cada amigo comeu utilizando fração?
\end{enumerate}

\ifdefined\prof
\clearpage
\begin{solucao}

\begin{enumerate} [label=\alph*)] %d
\item Da esquerda para a direita as pizzas são de João, Mariele e Luiza.
\item Na pizza de João uma fatia representa um sexto da pizza. \newline
  Na pizza de Mariele uma fatia representa um oitavo da pizza. \newline
  Na pizza de Luiza uma fatia representa um quarto da pizza.
\item João comeu quatro sextos de sua pizza; Mariele, seis oitavos; e Luiza comeu três quartos de sua pizza.
\end{enumerate} %d

\end{solucao}
\fi

\end{document}