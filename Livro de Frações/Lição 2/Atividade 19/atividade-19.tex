\documentclass[10 pt,usenames,dvipsnames, oneside]{article}
\usepackage{../../../modelo-fracoes}
\graphicspath{{../../../Figuras/licao02/}}


\begin{document}

\begin{center}
  \begin{minipage}[l]{3cm}
\includegraphics[width=2cm]{logo}    
\end{minipage}\hfill
\begin{minipage}[r]{.8\textwidth}
 {\Large \scshape Atividade: Três quintos, três meios ou três?}  
\end{minipage}
\end{center}
\vspace{.2cm}

\ifdefined\prof
\begin{goals}
\begin{enumerate}

    \item       Perceber a importância da unidade na representação de quantidades.

\end{enumerate}
\tcblower

\begin{itemize} %s
    \item       Esta é uma atividade que o aluno pode fazer individualmente, mas é essencial que seja discutida com toda a turma.
    \item       No final da atividade, é importante enfatizar para seus alunos a propriedade matemática que esta atividade quer destacar, ou seja, que uma mesma quantidade pode ser expressa por frações diferentes dependendo da unidade escolhida. Observe para eles que, no contexto       ``frações de'', é fundamental saber a que o       ``de''     se refere, isto é, qual é a unidade que está sendo considerada. Neste sentido, esta atividade está fortemente relacionada com as Atividades 8 e 13.
\end{itemize} %s

\end{goals}

\bigskip
\begin{center}
{\large \scshape Atividade}
\end{center}
\fi

Júlia, Davi e Laura estavam estudando a figura a seguir.
\begin{center}
\begin{tikzpicture}[scale=4]
%\fill(1.,3.) -- (6.,3.) -- (6.,0.02) -- (1.,0.) -- cycle;
\fill[attention] (1.,3.) -- (4.,3.) -- (4.,0.012) -- (1.,0.) -- cycle;
\fill[common, fill opacity=.3] (4,0) rectangle (6,3);
%\fill[line width=0.pt,color=black,fill=black,fill opacity=1.0] (4.,3.) -- (4.,0.012) -- (6.,0.04) -- (6.,3.) -- cycle;
\draw (1.,3.)-- (1.,0.);
\draw (2.,0.)-- (2.,3.);
\draw (3.,3.)-- (3.,0.);
\draw (4.,0.)-- (4.,3.);
\draw (5.,3.)-- (5.,0.);
\draw (6.,0.)-- (6.,3.);
\draw (1.,3.)-- (6.,3.);
\draw (6.,3.)-- (6.,0.02);
\draw (6.,0.02)-- (1.,0.);
\draw (1.,0.)-- (1.,3.);
\end{tikzpicture}
\end{center}
Júlia disse: ``A parte em vermelho representa $\frac{3}{5}$.''. Davi retrucou: ``Não, não! A parte em vermelho representa $\frac{3}{2}$!''. Laura, então acrescentou: ``Eu acho que a parte em vermelho representa $3$!''. Quem está certo? Júlia, Davi ou Laura? Explique!

\ifdefined\prof

\begin{solucao}

  As afirmações de Júlia, Davi e Laura estão incompletas, pois eles não informaram a   {\bf unidade}. De fato, dependendo da escolha da unidade, cada um deles pode estar certo e os demais errados. Por exemplo, se a unidade for
\begin{center}
  \begin{tikzpicture}[x=1mm,y=1mm]
 \draw[fill=common, fill opacity=.3, scale=4] (0,0) rectangle (5,3);
\end{tikzpicture}
\end{center}
então a parte pintada de vermelho em
\begin{center}
\begin{tikzpicture}[x=1mm,y=1mm]
 \draw[fill=attention, scale=4] (0,0) rectangle (3,3);
 \draw[fill=common, fill opacity=.3, scale=4] (3,0) rectangle (5,3);
 \foreach \x in {1,2,4} \draw[scale=4] (\x,0) -- (\x,3);
\end{tikzpicture}
\end{center}
de fato corresponde a   $\frac{3}{5}$ desta unidade, de modo que, nesta situação, Júlia está certa e David e Laura estão errados. Contudo, se a unidade for
\begin{center}
\begin{tikzpicture}[x=1mm,y=1mm]
 \draw[fill=common, fill opacity=.3, scale=4] (0,0) rectangle (2,3);
\end{tikzpicture}
\end{center}
então a parte pintada de vermelho em
\begin{center}
\begin{tikzpicture}[x=1mm,y=1mm]
 \draw[fill=attention, scale=4] (0,0) rectangle (3,3);
 \draw[fill=common, fill opacity=.3, scale=4] (3,0) rectangle (5,3);
 \foreach \x in {1,2,4} \draw[scale=4] (\x,0) -- (\x,3);
\end{tikzpicture}
\end{center}
  corresponde a   $\frac{3}{2}$ desta unidade,  de modo que, nesta situação, David está certo e Júlia e Laura estão errados. Finalmente, se a unidade for
\begin{center}
\begin{tikzpicture}[x=1mm,y=1mm]
 \draw[fill=common, fill opacity=.3, scale=4] (0,0) rectangle (1,3);
\end{tikzpicture}
\end{center}
então a parte pintada de vermelho em
\begin{center}
\begin{tikzpicture}[x=1mm,y=1mm]
 \draw[fill=attention, scale=4] (0,0) rectangle (3,3);
 \draw[fill=common, fill opacity=.3, scale=4] (3,0) rectangle (5,3);
 \foreach \x in {1,2,4} \draw[scale=4] (\x,0) -- (\x,3);
\end{tikzpicture}
\end{center}
  corresponde a   $3$ desta unidade e, neste caso, Laura está certa e David e Júlia estão errados.

\end{solucao}
\fi

\end{document}