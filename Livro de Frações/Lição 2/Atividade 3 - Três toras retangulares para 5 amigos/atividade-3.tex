\documentclass[10 pt,usenames,dvipsnames, oneside]{article}
\usepackage{../../modelo-fracoes}
\graphicspath{{../../../Figuras/licao02}}


\begin{document}

\begin{center}
  \begin{minipage}[l]{3cm}
\includegraphics[width=2cm]{../../../Figuras/logo}       
\end{minipage}\hfill
\begin{minipage}[r]{.8\textwidth}
 {\Large \scshape Atividade: Três tortas retangulares para 5 amigos}  
\end{minipage}
\end{center}
\vspace{.2cm}

\ifdefined\prof
\begin{goals}
\begin{enumerate}

\item Reconhecer a necessidade de apresentar uma expressão verbal que identifique a quantidades correspondentes a $n$ quintos.
\item Compreender e usar a expressão ``$n$ quintos de'' como uma forma de identificar a quantidade equivalente a $n$ partes da equipartição da unidade em quintos, incluindo os casos em que $n$ é maior do que cinco.
\item Comparar frações de mesmo denominador em uma situação.

\end{enumerate}
\tcblower

  \begin{itemize} %s
  \item Recomenda-se que a atividade seja desenvolvida em grupos de 3 a 5 alunos.
  \item Nesta atividade, é importante que os alunos possam ter cópias de figuras ilustrativas da torta para dividir e poder avaliar e decidir suas respostas. Faça cópias das páginas para reprodução e entregue uma para cada grupo.
  \item       As diversas soluções apresentadas pelos diferentes grupos devem ser discutidas com a turma inteira.
  \item       Em particular, no Item a), não se espera, nem se recomenda, que a representação feita pelos alunos seja amparada por medida. O objetivo é que façam a equipartição livremente e de forma coerente. Assim, por exemplo, pode ser aceita como resposta a solução indicada na figura a seguir.


\begin{center}
 \begin{tikzpicture}[yscale=.45,xscale=.4, x=1mm,y=1mm, rotate=90]
 \draw (0,0) rectangle (30,60);
 \foreach \x in {12,24,...,48} \draw (0,\x) -- (30,\x);
 \node[rotate=90, attention] at (15,54) {{\small Amarildo}};
 \node[rotate=90, attention] at (15,42) {{\small Beto}};
 \node[rotate=90, attention] at (15,30) {{\small Carlos}};
 \node[rotate=90, attention] at (15,18) {{\small Davi}};
 \node[rotate=90, attention] at (15,6) {{\small Edison}};
\end{tikzpicture}

\begin{tikzpicture}[yscale=.45,xscale=.4, x=1mm,y=1mm, rotate=90]
 \draw (0,0) rectangle (30,60);
 \foreach \x in {12,24,...,48} \draw (0,\x) -- (30,\x);
 \node[rotate=90, attention] at (15,54) {{\small Amarildo}};
 \node[rotate=90, attention] at (15,42) {{\small Beto}};
 \node[rotate=90, attention] at (15,30) {{\small Carlos}};
 \node[rotate=90, attention] at (15,18) {{\small Davi}};
 \node[rotate=90, attention] at (15,6) {{\small Edison}};
\end{tikzpicture}

\begin{tikzpicture}[yscale=.45,xscale=.4, x=1mm,y=1mm, rotate=90]
 \draw (0,0) rectangle (30,60);
 \foreach \x in {12,24,...,48} \draw (0,\x) -- (30,\x);
 \node[rotate=90, attention] at (15,54) {{\small Amarildo}};
 \node[rotate=90, attention] at (15,42) {{\small Beto}};
 \node[rotate=90, attention] at (15,30) {{\small Carlos}};
 \node[rotate=90, attention] at (15,18) {{\small Davi}};
 \node[rotate=90, attention] at (15,6) {{\small Edison}};
\end{tikzpicture}
\end{center}


    \item       Em suas respostas, é possível que os alunos utilizem expressões variadas para nomear as partes das tortas em cada divisão e para as quantidades de torta que cada irmão recebe. Por exemplo,       ``três dos quinze pedaços'',       ``três pedaços de um quinto de torta'', dentre outras. É importante que a discussão conduza os alunos ao uso de quintos:       ``três quintos'',       ``seis quintos'',       ``quinze quintos'', etc.
    \item       Espera-se que, ao final da atividade, o aluno reconheça o significado das expressões dois quintos e três quintos, mesmo que não o faça espontaneamente (usando, por exemplo, especificações como       ``dois pedaços''     ou       ``duas fatias'') e seja necessária a intervenção do professor. {\bf O professor deve fazer e incentivar o uso da terminologia de frações que se quer estabelecer nesta lição.}
   % \item Cabe ressaltar que no Item b) (II) já está implícita a adição, na medida em que é esperado do estudante juntar duas frações não unitárias. No entanto, não é recomendável aqui que se chame atenção do estudante para a adição ou que seja utilizada a notação ``+'', mas sim que se enfatize o conceito de fração não unitária: juntou-se, afinal, 6 pedaços iguais a um quinto de torta.
    \item       Nos Itens c) e d), não basta uma resposta       ``Sim''     ou       ``Não''. É importante estimular os seus alunos a darem uma justificativa.
\end{itemize} %s


\end{goals}

\bigskip
\begin{center}
{\large \scshape Atividade}
\end{center}
\fi

Um grupo de cinco amigos (Amarildo, Beto, Carlos, Davi e Edilson) encomendou três tortas salgadas, de mesmo tamanho retangular, como na ilustração para uma comemoração.

\begin{center}
 \begin{tikzpicture}[scale=0.6]
  \draw (0,0) rectangle (60,30);
  \draw (70,0) rectangle (130,30);
  \draw (140,0) rectangle (200,30);
 \end{tikzpicture}

\end{center}

\begin{enumerate} [label=\alph*)] %s
  \item     Como dividir as três tortas de modo que cada amigo receba a mesma quantidade de torta? Faça um desenho no seu caderno mostrando sua proposta de divisão. Indique qual parte é de qual amigo!
  \item     Considerando-se uma torta como unidade, como você nomearia, usando frações, a quantidade de torta que:
\begin{enumerate} [label=\Roman*)] %d
      \item         Amarildo recebeu?
      \item         Amarildo e Beto receberam juntos?
      \item         Amarildo, Beto e Carlos receberam juntos?
      \item         Amarildo, Beto, Carlos e Davi receberam juntos?
      \item         Amarildo, Beto, Carlos, Davi e Edilson receberam juntos?
\end{enumerate} %d

  \item     A quantidade de torta que cada amigo recebeu é menor do que um quinto de torta? E do que dois quintos de torta? Explique sua resposta.
  \item     A quantidade de torta que cada amigo recebeu é maior do que três quintos de torta? E do que quatro quintos de torta? Explique sua resposta.
\end{enumerate} %s

\ifdefined\prof

\begin{solucao}

\begin{enumerate} [label=\alph*)] %d
    \item       Uma resposta possível: dividir cada uma das três tortas em 5 partes iguais e, então, com as 15 partes disponíveis, distribuir 3 partes para cada amigo, como mostra a figura a seguir

\begin{center}
 \begin{tikzpicture}[yscale=.45,xscale=.4, x=1mm,y=1mm, rotate=90]
 \draw (0,0) rectangle (30,60);
 \foreach \x in {12,24,...,48} \draw (0,\x) -- (30,\x);
 \node[rotate=90, attention] at (15,54) {{\small Amarildo}};
 \node[rotate=90, attention] at (15,42) {{\small Beto}};
 \node[rotate=90, attention] at (15,30) {{\small Carlos}};
 \node[rotate=90, attention] at (15,18) {{\small Davi}};
 \node[rotate=90, attention] at (15,6) {{\small Edison}};
\end{tikzpicture}\hspace{1em}
\begin{tikzpicture}[yscale=.45,xscale=.4, x=1mm,y=1mm, rotate=90]
 \draw (0,0) rectangle (30,60);
 \foreach \x in {12,24,...,48} \draw (0,\x) -- (30,\x);
 \node[rotate=90, attention] at (15,54) {{\small Amarildo}};
 \node[rotate=90, attention] at (15,42) {{\small Beto}};
 \node[rotate=90, attention] at (15,30) {{\small Carlos}};
 \node[rotate=90, attention] at (15,18) {{\small Davi}};
 \node[rotate=90, attention] at (15,6) {{\small Edison}};
\end{tikzpicture}\hspace{1em}
\begin{tikzpicture}[yscale=.45,xscale=.4, x=1mm,y=1mm, rotate=90]
 \draw (0,0) rectangle (30,60);
 \foreach \x in {12,24,...,48} \draw (0,\x) -- (30,\x);
 \node[rotate=90, attention] at (15,54) {{\small Amarildo}};
 \node[rotate=90, attention] at (15,42) {{\small Beto}};
 \node[rotate=90, attention] at (15,30) {{\small Carlos}};
 \node[rotate=90, attention] at (15,18) {{\small Davi}};
 \node[rotate=90, attention] at (15,6) {{\small Edison}};
\end{tikzpicture}
\end{center}

Outra resposta possível: dividir cada uma das três tortas em cinco partes iguais e distribuir três partes \textit{consecutivas} para cada amigo.

\begin{center}
 \begin{tikzpicture}[yscale=.45,xscale=.4, x=1mm,y=1mm, rotate=90]
 \draw (0,0) rectangle (30,60);
 \foreach \x in {12,24,...,48} \draw (0,\x) -- (30,\x);
 \node[rotate=90, attention] at (15,54) {{\small Amarildo}};
 \node[rotate=90, attention] at (15,42) {{\small Amarildo}};
 \node[rotate=90, attention] at (15,30) {{\small Amarildo}};
 \node[rotate=90, attention] at (15,18) {{\small Beto}};
 \node[rotate=90, attention] at (15,6) {{\small Beto}};
\end{tikzpicture}\hspace{1em}
\begin{tikzpicture}[yscale=.45,xscale=.4, x=1mm,y=1mm, rotate=90]
 \draw (0,0) rectangle (30,60);
 \foreach \x in {12,24,...,48} \draw (0,\x) -- (30,\x);
 \node[rotate=90, attention] at (15,54) {{\small Beto}};
 \node[rotate=90, attention] at (15,42) {{\small Carlos}};
 \node[rotate=90, attention] at (15,30) {{\small Carlos}};
 \node[rotate=90, attention] at (15,18) {{\small Carlos}};
 \node[rotate=90, attention] at (15,6) {{\small Davi}};
 \node[rotate=90, attention] at (15,6) {{\small\phantom{Amarildo}}};
\end{tikzpicture}\hspace{1em}
\begin{tikzpicture}[yscale=.45,xscale=.4, x=1mm,y=1mm, rotate=90]
 \draw (0,0) rectangle (30,60);
 \foreach \x in {12,24,...,48} \draw (0,\x) -- (30,\x);
 \node[rotate=90, attention] at (15,54) {{\small Davi}};
 \node[rotate=90, attention] at (15,42) {{\small Davi}};
 \node[rotate=90, attention] at (15,30) {{\small Edison}};
 \node[rotate=90, attention] at (15,18) {{\small Edison}};
 \node[rotate=90, attention] at (15,6) {{\small Edison}};
  \node[rotate=90, attention] at (15,6) {{\small\phantom{Amarildo}}};

\end{tikzpicture}
\end{center}


%Outra resposta possível: juntar as três tortas como na figura

% \begin{tikzpicture}[yscale=.45,xscale=.4, x=1mm,y=1mm]
%  \fill[cbyellow] (0,0) rectangle (60,30);
%   \fill[cborange] (60,0) rectangle (120,30);
%   \fill[cbbrown] (120,0) rectangle (180,30);
%   \foreach \x in {30,90,150}  \node [] at (\x,15) {\textbf{\small torta}};
% \end{tikzpicture}

% e dividir essa reunião em cinco partes iguais e distribuir uma dessas partes para cada amigo.

% \begin{tikzpicture}[yscale=.45,xscale=.4, x=1mm,y=1mm]
%   \fill[cbyellow] (0,0) rectangle (60,30);
%   \fill[cborange] (60,0) rectangle (120,30);
%   \fill[cbbrown] (120,0) rectangle (180,30);
%   \foreach \x in {36,72,108,144} \draw [dashed, very thick] (\x,-8) -- (\x,38); % quatro cortes.
%   %\foreach \x in {60,120} \draw[thin] (\x,0) -- (\x,30); % fronteira entre as tortas.
% \foreach \x/\n in {18/Amarildo, 54/Beto, 90/Carlos, 126/Davi, 162/Edison}  \node [ ] at (\x,15) {\textbf{\small \n}};
% \end{tikzpicture}
    \item
\begin{enumerate}[label=\Roman*)]
          \item Três quintos.
          \item Seis quintos (ou uma torta inteira e um quinto de torta).
          \item Nove quintos (ou uma torta inteira e quatro quintos de torta).
          \item Doze quintos (ou duas tortas inteiras e dois quintos de torta).
          \item Quinze quintos (ou três tortas inteiras).
\end{enumerate}

    \item     A quantidade de torta que cada amigo recebeu não pode ser menor do que um quinto de torta pois, se isto acontecesse, a quantidade total de torta recebida pelos cinco amigos seria menor do que cinco quintos de torta, isto é, seria menor do que uma torta inteira, o que não é o caso. Um argumento análogo mostra que a quantidade de torta que cada amigo recebeu não pode ser menor do que dois quintos de torta.
    \item      A quantidade de torta que cada amigo recebeu não pode ser maior do que três quintos de torta pois, se isto acontecesse, a quantidade total de torta recebida pelos cinco amigos seria maior do que quinze quintos de torta, isto é, seria maior do que três tortas inteiras, o que não é o caso. Um argumento análogo mostra que a quantidade de torta que cada amigo recebeu não pode ser maior do que quatro quintos de torta.
\end{enumerate} %d

\end{solucao}
\fi

\end{document}