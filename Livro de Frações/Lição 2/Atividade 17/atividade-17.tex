\documentclass[10 pt,usenames,dvipsnames, oneside]{article}
\usepackage{../../../modelo-fracoes}
\graphicspath{{../../../Figuras/licao02/}}


\begin{document}

\begin{center}
  \begin{minipage}[l]{3cm}
\includegraphics[width=2cm]{logo}    
\end{minipage}\hfill
\begin{minipage}[r]{.8\textwidth}
 {\Large \scshape Atividade: Três partes pintadas e quatro não}  
\end{minipage}
\end{center}
\vspace{.2cm}

\ifdefined\prof
\begin{goals}
\begin{enumerate}

    \item Perceber que, se uma unidade foi equiparticionada em $n + m$ partes iguais, das quais $n$ foram pintadas, então $\frac{n}{m}$ {\bf não especifica} a fração da unidade que foi pintada.

\end{enumerate}
\tcblower

\begin{itemize} %s
    \item       Esta é uma atividade que o aluno pode fazer individualmente, mas é essencial que seja discutida com toda a turma.
    \item       O tipo de situação descrita na atividade destaca um equívoco comum entre os alunos. Assim, esta atividade é uma oportunidade para reforçar os papéis do denominador e do numerador na notação simbólica matemática para frações: o denominador especifica o número de partes iguais em que a unidade foi dividida e o numerador especifica o número de cópias que foram tomadas de uma destas partes.
      \item Considere perguntar aos alunos que fração a parte vermelha é da parte azul.

\end{itemize} %s

\end{goals}

\bigskip
\begin{center}
{\large \scshape Atividade}
\end{center}
\fi

A figura a seguir tem 3 partes pintadas de vermelho e 4 partes pintadas de azul. É correto afirmar que a parte pintada de vermelho corresponde a $\frac{3}{4}$ da figura? Explique.
\begin{center}
\begin{tikzpicture}[scale=3.5]%[line cap=round,line join=round,>=triangle 45,x=1.0cm,y=1.0cm]
\fill[fill=attention] (0.,3.) -- (3.,3.) -- (3.,0.) -- (0.,0.) -- cycle;
\fill[common, fill opacity=.3] (3,0) rectangle (7,3);
\draw (1.,3.)-- (1.,0.);
\draw (2.,0.)-- (2.,3.);
\draw (3.,3.)-- (3.,0.);
\draw (4.,0.)-- (4.,3.);
\draw (5.,3.)-- (5.,0.);
\draw (6.,0.)-- (6.,3.);
\draw (7.,3.)-- (7.,0.);
\draw (0.,3.)-- (7.,3.);
\draw (7.,3.)-- (7.,0.);
\draw (7.,0.)-- (0.,0.);
\draw (0.,0.)-- (0.,3.);
\end{tikzpicture}
\end{center}

\ifdefined\prof

\begin{solucao}

A parte pintada de vermelho   {\bf não}   corresponde a   $\frac{3}{4}$ da figura. Ela corresponde a $\frac{3}{7}$ da figura, pois a figura foi dividida em   $7$ partes iguais das quais $3$ foram pintadas.

\end{solucao}
\fi

\end{document}