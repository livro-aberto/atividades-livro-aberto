\documentclass[10 pt,usenames,dvipsnames, oneside]{article}
\usepackage{../../../modelo-fracoes}
\graphicspath{{../../../Figuras/licao02/}}


\begin{document}

\begin{center}
  \begin{minipage}[l]{3cm}
\includegraphics[width=2cm]{logo}    
\end{minipage}\hfill
\begin{minipage}[r]{.8\textwidth}
 {\Large \scshape Atividade: Cabe toda água num copo só?
}  
\end{minipage}
\end{center}
\vspace{.2cm}

\ifdefined\prof
\begin{goals}
\begin{enumerate}

    \item       Analisar e resolver um problema no contexto da reunião de partes correspondentes a frações unitárias com mesmo denominador.

\end{enumerate}
\tcblower

    \begin{itemize} %s
    \item Essa é uma atividade que o aluno pode fazer individualmente.
    \item As diversas soluções apresentadas devem ser discutidas com a turma inteira.
    \item Caso os estudantes se refiram a cada uma das oito partes em que foi dividido cada um dos três copos de maneira diferente de ``oitavos'', pergunte a eles ``\textit{a que fração do copo essa parte corresponde?}'' e então peça que se esforcem para usar ``oitavos'' ao invés da expressão que estavam usando.
    \item Avalie a necessidade de apresentar os seguintes problemas preliminares:
      \begin{enumerate}
      \item Indique a fração da capacidade do copo que está com água em cada um dos três copos.
      \item Qual é a fração da capacidade do copo que corresponde a toda água que está nos três copos?
      \end{enumerate}
    \item É possível que os alunos utilizem frações equivalentes como resposta para um mesmo item. Por exemplo, para o copo (3), as frações $\frac{4}{8}$, $\frac{2}{4}$ e $\frac{1}{2}$ são respostas corretas. Nesses casos, dê a oportunidade para que cada aluno explique como chegou à sua resposta. Procedendo desta maneira, mesmo que de forma pontual, os alunos perceberão que uma mesma quantidade pode ser descrita por frações com numeradores e denominadores diferentes, um preparo para o assunto ``frações equivalentes'' que será tratado na Lição 4.
\end{itemize} %s

\end{goals}

\bigskip
\begin{center}
{\large \scshape Atividade}
\end{center}
\fi

A figura mostra três copos idênticos.  É possível armazenar a água dos três copos em um único copo sem que transborde? Explique usando frações.

\begin{center}
\begin{tikzpicture}[scale=0.3, x=1cm,y=1cm]

% Definição do eixo vertical das elipses
\def\EixoM{0.5}

% colorindo o primeiro cilindro
\fill[common] (2,0) ellipse (2 and \EixoM);
\fill[common] (0,0) rectangle (4,3);
\fill[common] (2,3) ellipse (2 and \EixoM);

% colorindo o segundo cilindro
\fill[common] (8,0) ellipse (2 and \EixoM);
\fill[common] (6,0) rectangle (10,2);
\fill[common] (8,2) ellipse (2 and \EixoM);

% colorindo o terceiro
\fill[common] (14,0) ellipse (2 and \EixoM);
\fill[common] (12,0) rectangle (16,4);
\fill[common] (14,4) ellipse (2 and \EixoM);

% shift horizontal nos cilindros definido por \x
\foreach \x in {0,6,12}{
\draw (\x,0)--(\x,8);
\draw (\x + 4,0)--(\x + 4,8);
% shift vertical nos arcos de elipse definido por \y
\foreach \y in {0,1,...,7}{
\pgfpathmoveto{\pgfpoint{\x cm}{\y cm}}
\pgfpatharc{-180}{0}{2cm and \EixoM cm}
\pgfusepath{draw}}
\draw (\x + 2,8) ellipse (2 and \EixoM);}

\node at (2,-2) {(1)};
\node at (8,-2) {(2)};
\node at (14,-2) {(3)};
\end{tikzpicture}

\end{center}

\ifdefined\prof

\begin{solucao}

 Não é possível armazenar a água dos três copos em um único copo sem que o mesmo transborde, pois a água do primeiro copo ocupa 3 oitavos de sua capacidade, a água do segundo copo ocupa 2 oitavos de sua capacidade e a água do terceiro copo ocupa 4 oitavos de sua capacidade, a água dos três copos, juntos, ocupa       $3 + 2 + 4 = 9$ oitavos da capacidade do copo e qualquer copo só consegue armazenar no máximo $8$ oitavos de sua capacidade.

  Caso tenha decidido aceitar a sugestão das perguntas preliminares, as respostas são:
\begin{enumerate} [label=\alph*)] %s
    \item       (1):       $\dfrac{3}{8}$. (2):       $\dfrac{2}{8}$. (3):       $\dfrac{4}{8}$.
    \item             $\dfrac{9}{8}$.
\end{enumerate} %s

\end{solucao}
\fi

\end{document}