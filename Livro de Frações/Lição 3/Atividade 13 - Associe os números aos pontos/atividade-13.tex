
\documentclass[10 pt,usenames,dvipsnames, oneside]{article}
\usepackage{../../modelo-fracoes}
\graphicspath{{../../../Figuras/licao03/}}


\begin{document}

\begin{center}
  \begin{minipage}[l]{3cm}
\includegraphics[width=2cm]{../../../Figuras/logo}       
\end{minipage}\hfill
\begin{minipage}[r]{.8\textwidth}
 {\Large \scshape Atividade: Associe os números aos pontos}  
\end{minipage}
\end{center}
\vspace{.2cm}

\ifdefined\prof
%Caixa do Para o Professor
\begin{goals}
%Objetivos específicos
\begin{enumerate}
\item Representar frações unitárias na reta numérica.
\item Comparar frações unitárias na reta numérica.
\end{enumerate}

\tcblower

%Orientações e sugestões
\begin{itemize}
\item  Observe e discuta com seus alunos que, no caso das frações de numerador igual a $1$ (frações unitárias), quanto maior o denominador, menor a fração. Portanto, sua representação na reta numérica está mais perto do zero.
\item  Aproveite para propor e discutir com seus alunos algumas reflexões tais como:
   \begin{enumerate}[label=\alph*)]
    \item Alguma fração com numerador igual a 1 pode ter sua representação na reta numérica entre $\frac{1}{2}$ e 1?
    \item Qual fração é maior, $\frac{1}{4}$ ou $\frac{1}{10}$?
    \item Que fração tem sua representação na reta numérica mais próxima de 0, $\frac{1}{5}$ ou $\frac{1}{6}$?
   \end{enumerate}
\end{itemize}
\end{goals}

\bigskip
\begin{center}
{\large \scshape Atividade}
\end{center}
\fi

Associe, como no exemplo, cada  fração à sua representação na reta numérica.

\noindent\begin{tabular}{m{.09\textwidth}m{.08\textwidth}m{.08\textwidth}m{.08\textwidth}m{.08\textwidth}m{.08\textwidth}m{0.08\textwidth}m{.08\textwidth}m{0.08\textwidth}}
(A) $\frac{1}{2}$ & (B) $\frac{1}{3}$ &  (C) $\frac{1}{4}$  & (D) $\frac{1}{5}$ & (E) $\frac{1}{6}$  & (F) $\frac{1}{7}$  & (G) $\frac{1}{8}$  & (H) $\frac{1}{9}$  & (I) $\frac{1}{10}$
\end{tabular}

\begin{center}
\begin{longtable}{m{.2\textwidth}m{.5\textwidth}}

\centering \begin{tikzpicture}
 \draw (0,0) -- (20,0);
\end{tikzpicture}
 &
\parbox[t][1cm][c]{8cm}{
\begin{tikzpicture}[x=50mm,y=50mm]
\draw[->] (-0.3,0) -- (1.3,0) ; %reta anterior
\foreach \x in {0,.1,1}{ \draw (\x,3pt) -- (\x,-3pt);}
 \node[below] at (0,0) {0};
 \node[below] at (1,0) {1};
 \fill[common] (.1,0) circle (3pt);
\end{tikzpicture}}
\\

 \centering \begin{tikzpicture}
 \draw (0,0) -- (20,0);
 \node[above] at (10,0){(A)};
\end{tikzpicture}
 &
\parbox[t][1cm][c]{8cm}{
\begin{tikzpicture}[x=50mm,y=50mm]
\draw[->] (-0.3,0) -- (1.3,0) ; %reta anterior
\foreach \x in {0,1}{ \draw (\x,3pt) -- (\x,-3pt);}
 \node[below] at (0,0) {0};
 \node[below] at (1,0) {1};
 \fill[common] (.5,0) circle (3pt);
\end{tikzpicture}
}\\

\centering \begin{tikzpicture}
 \draw (0,0) -- (20,0);
\end{tikzpicture}
  &
\parbox[t][1cm][c]{8cm}{
\begin{tikzpicture}[x=50mm,y=50mm]
\draw[->] (-0.3,0) -- (1.3,0) ; %reta anterior
\foreach \x in {0,1}{ \draw (\x,3pt) -- (\x,-3pt);}
 \node[below] at (0,0) {0};
 \node[below] at (1,0) {1};
 \fill[common] (1/3,0) circle (3pt);
\end{tikzpicture}
}\\

\centering \begin{tikzpicture}
 \draw (0,0) -- (20,0);
\end{tikzpicture}
 &
\parbox[t][1cm][c]{8cm}{
\begin{tikzpicture}[x=50mm,y=50mm]
\draw[->] (-0.3,0) -- (1.3,0) ; %reta anterior
\foreach \x in {0,1}{ \draw (\x,3pt) -- (\x,-3pt);}
 \node[below] at (0,0) {0};
 \node[below] at (1,0) {1};
 \fill[common] (1/9,0) circle (3pt);
\end{tikzpicture}
}\\

\centering \begin{tikzpicture}
 \draw (0,0) -- (20,0);
\end{tikzpicture}
 &
\parbox[t][1cm][c]{8cm}{
\begin{tikzpicture}[x=50mm,y=50mm]
\draw[->] (-0.3,0) -- (1.3,0) ; %reta anterior
\foreach \x in {0,1}{ \draw (\x,3pt) -- (\x,-3pt);}
 \node[below] at (0,0) {0};
 \node[below] at (1,0) {1};
 \fill[common] (1/7,0) circle (3pt);
\end{tikzpicture}
}\\

\centering \begin{tikzpicture}
 \draw (0,0) -- (20,0);
\end{tikzpicture}
 &
\parbox[t][1cm][c]{8cm}{
\begin{tikzpicture}[x=50mm,y=50mm]
\draw[->] (-0.3,0) -- (1.3,0) ; %reta anterior
\foreach \x in {0,1}{ \draw (\x,3pt) -- (\x,-3pt);}
 \node[below] at (0,0) {0};
 \node[below] at (1,0) {1};
 \fill[common] (.25,0) circle (3pt);
\end{tikzpicture}
}\\

\centering \begin{tikzpicture}
 \draw (0,0) -- (20,0);
\end{tikzpicture}
 &
\parbox[t][1cm][c]{8cm}{
\begin{tikzpicture}[x=50mm,y=50mm]
\draw[->] (-0.3,0) -- (1.3,0) ; %reta anterior
\foreach \x in {0,1}{ \draw (\x,3pt) -- (\x,-3pt);}
 \node[below] at (0,0) {0};
 \node[below] at (1,0) {1};
 \fill[common] (1/6,0) circle (3pt);
\end{tikzpicture}
}\\

\centering \begin{tikzpicture}
 \draw (0,0) -- (20,0);
\end{tikzpicture}
&
\parbox[t][1cm][c]{8cm}{
\begin{tikzpicture}[x=50mm,y=50mm]
\draw[->] (-0.3,0) -- (1.3,0) ; %reta anterior
\foreach \x in {0,1}{ \draw (\x,3pt) -- (\x,-3pt);}
 \node[below] at (0,0) {0};
 \node[below] at (1,0) {1};
 \fill[common] (1/5,0) circle (3pt);
\end{tikzpicture}
}\\

\centering \begin{tikzpicture}
 \draw (0,0) -- (20,0);
\end{tikzpicture}
&
\parbox[t][1cm][c]{8cm}{
\begin{tikzpicture}[x=50mm,y=50mm]
\draw[->] (-0.3,0) -- (1.3,0) ; %reta anterior
\foreach \x in {0,1}{ \draw (\x,3pt) -- (\x,-3pt);}
 \node[below] at (0,0) {0};
 \node[below] at (1,0) {1};
 \fill[common] (1/8,0) circle (3pt);
\end{tikzpicture}
}\\

\end{longtable}
\end{center}


\ifdefined\prof
\begin{solucao}

As respostas são na ordem I, A, B, H, F, C, E, D e G.

\end{solucao}
\fi

\end{document}