
\documentclass[10 pt,usenames,dvipsnames, oneside]{article}
\usepackage{../../modelo-fracoes}
\graphicspath{{../../../Figuras/licao03/}}


\begin{document}

\begin{center}
  \begin{minipage}[l]{3cm}
\includegraphics[width=2cm]{../../../Figuras/logo}       
\end{minipage}\hfill
\begin{minipage}[r]{.8\textwidth}
 {\Large \scshape Atividade: Marque as frações na reta e compare}  
\end{minipage}
\end{center}
\vspace{.2cm}

\ifdefined\prof
%Caixa do Para o Professor
\begin{goals}
%Objetivos específicos
\begin{enumerate}
\item       Representar frações na reta numérica.
\item       Comparar frações a partir de sua representação na reta numérica.
\end{enumerate}

\tcblower

%Orientações e sugestões
\begin{itemize}
    \item       A associação das frações aos pontos correspondentes exigirá que os alunos façam comparações de diferentes tipos. Valorize e discuta as diversas estratégias apresentadas pelos alunos.
    \item Observe que apenas os pontos correspondentes ao 0 e ao 2 estão destacados na reta. Fique atento se seu aluno consegue identificar corretamente a unidade (explícita ou implicitamente). Essa identificação tem papel fundamental na reta numérica. Respostas como as que estão a seguir revelam tal dificuldade.
      \begin{center}
      \begin{tikzpicture}[x=12mm,y=15mm]
\draw[->] (-0.5,0) -- (5.5,0) ; %reta anterior
\foreach \x in {.5, 1, 1.5, 2.5}{ \fill[common] (\x,0) circle (2pt);}
\foreach \x in {0,2}
{
  \draw (\x,-3pt) -- (\x,3pt);
  \node [below] at (\x,0) {\x};
}
\foreach \x in {1,3,5} \node[above] at (\x/2,0) {$\frac{\x}{4}$};
\node[below] at (1, 0) {$\frac{1}{2}$};

\begin{scope}[yshift=-40]
  \draw[->] (-0.5,0) -- (5.5,0) ; %reta anterior
\foreach \x in {.2, .4, .6, 1}{ \fill[common] (\x,0) circle (2pt);}
\foreach \x in {0,2}
{
  \draw (\x,-3pt) -- (\x,3pt);
  \node [below] at (\x,0) {\x};
}
\foreach \x in {1,3,5} \node[above] at (\x/5,0) {$\frac{\x}{4}$};
\node[below] at (.4, 0) {$\frac{1}{2}$};
 \draw (.8,-3pt) -- (.8,3pt);
\node[below] at (.8, 0) {1};
\end{scope}
      \end{tikzpicture}
      \end{center}
    \item       O último item desta atividade admite várias respostas. Explore as soluções dadas pelos seus alunos. Aproveite para discutir estratégias variadas de comparação. Por exemplo, decidir que $\frac{7}{2} < \frac{11}{3} < 4$ pode ser justificado da seguinte maneira: como $4 = \frac{8}{2} = \frac{12}{3}$, na reta, os pontos correspondentes aos números $\frac{7}{2}$ e $\frac{11}{3}$ estarão ambos antes do 4 e, como  $\frac{1}{3} < \frac{1}{2}$, a marcação de $\frac{7}{2}$ estará mais distante de 4 do que a de $\frac{11}{3}$, ou seja, $\frac{7}{2}$ é menor do que~$\frac{11}{3}$.
\end{itemize}
\end{goals}

\bigskip
\begin{center}
{\large \scshape Atividade}
\end{center}
\fi

Na reta numérica a seguir:

  
\begin{center}
\begin{tikzpicture}[x=22mm,y=15mm]
\draw[->] (-0.5,0) -- (5.5,0) ; %reta anterior
\foreach \x in {0,2}{ \draw (\x,3pt) -- (\x,-3pt);}
\node[below] at (0,0) {0};
\node[below] at (2,0) {2};
\end{tikzpicture}
\end{center}

\begin{enumerate}
  \item     Marque     $\frac{1}{2}$. Justifique sua resposta.
  \item     Marque     $\frac{1}{4}$,     $\frac{3}{4}$     e     $\frac{5}{4}$    . Explique como raciocinou para fazer essas marcações.

  \item $\frac{1}{4}$     é maior ou menor do que     $\frac{1}{2}$    ?
  \item $\frac{3}{4}$ é maior ou menor do que $\frac{1}{2}$    ?
  \item $\frac{5}{4}$   é maior ou menor do que 1?
  \item Escreva as frações marcadas na reta em ordem crescente, completando os espaços a seguir:

$$0 <  \dfrac{\text{\Large $\square$}}{\text{\Large $\square$}} <  \dfrac{\text{\Large $\square$}}{\text{\Large $\square$}} < \dfrac{\text{\Large $\square$}}{\text{\Large $\square$}} < 1 < \dfrac{\text{\Large $\square$}}{\text{\Large $\square$}}< 2.$$

\item Volte à reta e registre outras três frações que atendam às seguintes condições:
  \begin{enumerate}[label=\Alph*)]
  \item A primeira deve ser maior do que $3$ e menor do que $4$.
 \item A segunda deve ser maior do que $\frac{7}{2}$.
 \item A terceira deve ser maior do que $\frac{17}{4}$ e menor do que $5$.
\end{enumerate}
\end{enumerate} %s

\ifdefined\prof
\begin{solucao}

\begin{center}
\begin{tikzpicture}[x=12mm,y=15mm]
\draw[->] (-0.5,0) -- (5.5,0) ; %reta anterior
\foreach \x in {0,.25, .5, .75,1, 1.25, 2, 3, 3.5, 3.333, 11/3, 15/4, 4, 17/4, 4.5, 14/3, 5}{ \fill[common] (\x,0) circle (2pt);}
\foreach \x in {1,3,5,15,17} \node[above] at (\x/4,4pt) {$\frac{\x}{4}$};
\foreach \x in {10,11,14} \node[below] at (\x/3, -4pt) {$\frac{\x}{3}$};
\foreach \x in {1,7,9} \node[above] at (\x/2, 4pt) {$\frac{\x}{2}$};
\node[below] at (0,0) {0};
\node[below] at (1,0) {1};
\node[below] at (2,0) {2};
\node[below] at (3,0) {3};
\node[below] at (4,0) {4};
\node[below] at (5,0) {5};

\end{tikzpicture}
\end{center}


\begin{enumerate}
    \item      O número 1 é o ponto que está a mesma distância do 0 e do 2, ou seja, o ``ponto médio'' entre o 0 e o 2. O número $\frac{1}{2}$ é o ponto médio entre o 0 e o 1.

\item  As marcações do $\frac{1}{4}$, $\frac{3}{4}$ e $\frac{5}{4}$ podem ser feitas de maneira semelhante (i) $\frac{1}{4}$ é o ponto médio entre $\frac{1}{2}$ e 1, (ii) $\frac{3}{4}$ é o ponto médio entre $\frac{1}{2}$ e 1, (iii) $\frac{5}{4}$ é o ponto médio entre 1 e $\frac{3}{2}$.
    \item       $\frac{1}{4}$       é menor do que       $\frac{1}{2}$       porque $\frac{1}{4}$ está antes na reta numérica.
    \item       $\frac{3}{4}$       é maior do que        $\frac{1}{2}$       porque $\frac{3}{4}$ está mais adiante na reta numérica.
    \item       $\frac{5}{4}$       é menor do que 1 porque $\frac{5}{4}$ está mais adiante na reta numérica.
    \item       $$0 < \frac{1}{4} < \frac{1}{2}< \frac{3}{4} < 1 < \frac{5}{4} < 2.$$
    \item \mbox{ }
      
      \begin{enumerate}[label=\Alph*)]
    \item Há várias respostas possíveis. Por exemplo,       $3 < \frac{7}{2} < 4$,       $3 < \frac{15}{4} < 4$         ou         $3 < \frac{10}{3} < 4$.
    \item       Há várias respostas possíveis. Por exemplo,       $\frac{7}{2} < \frac{15}{4} < 4$ ou $\frac{7}{2} < \frac{11}{3} < 4$.
    \item       Há várias respostas possíveis. Por exemplo,       $\frac{17}{4} < \frac{9}{2} < 5$, $\frac{17}{4} < \frac{19}{4} < 5$ ou $\frac{17}{4} < \frac{14}{3} < 5$.
      \end{enumerate}
     
\end{enumerate} %s

\end{solucao}
\fi

\end{document}