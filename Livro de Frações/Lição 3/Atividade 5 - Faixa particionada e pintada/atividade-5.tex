
\documentclass[10 pt,usenames,dvipsnames, oneside]{article}
\usepackage{../../modelo-fracoes}
\graphicspath{{../../../Figuras/licao03/}}


\begin{document}

\begin{center}
  \begin{minipage}[l]{3cm}
\includegraphics[width=2cm]{../../../Figuras/logo}       
\end{minipage}\hfill
\begin{minipage}[r]{.8\textwidth}
 {\Large \scshape Atividade: Faixa particionada e pintada}  
\end{minipage}
\end{center}
\vspace{.2cm}

\ifdefined\prof
%Caixa do Para o Professor
\begin{goals}
%Objetivos específicos
\begin{enumerate}
\item Identificar na reta numérica pontos correspondentes a frações apresentadas em modelos contínuo. Especificamente    $\frac{0}{5}$,  $\frac{1}{5}$,       $\frac{2}{5}$,       $\frac{3}{5}$, $\frac{4}{5}$ e $\frac{5}{5}$.
\end{enumerate}

\tcblower

%Orientações e sugestões
\begin{itemize}
\item A atividade evidencia a diferença entre representar uma fração a partir de um modelo contínuo de área (no caso, uma faixa retangular) e na reta numérica. No primeiro caso, a fração é identificada por uma região colorida. Na reta, a fração é associada a um único ponto, um número.
\item Observe que as frações destacadas nas faixas foram alinhadas e ordenadas visando à correspondência com a representação na reta.  Assim, por exemplo, $\frac{1}{5}$ é representado por
\begin{tikzpicture}[scale=.3]
\fill [fill=common, fill opacity=.3] (0,0) rectangle (10,1);
\fill [attention] (0,0) rectangle (2,1);
\draw [step=2] (0,0) grid (10,1);
\draw (0,1) -- (10,1);
\end{tikzpicture}
e não por       \begin{tikzpicture}[scale=.3]
\fill [fill=common, fill opacity=.3] (0,0) rectangle (10,1);
\fill [attention] (4,0) rectangle (6,1);
\draw [step=2] (0,0) grid (10,1);
\draw (0,1) -- (10,1);
\end{tikzpicture}
O alinhamento à esquerda faz a associação ao zero e atende à ordem estabelecida na reta.
\item No item b) espera-se que os alunos identifiquem a faixa sem pintar à fração $\frac{0}{5}$, ou seja, a zero, e a faixa inteiramente pintada à fração $\frac{0}{5}$, ou seja, à unidade, portanto, igual a 1. Cabe aqui registrar as igualdades $\frac{0}{5}=0$ e $\frac{5}{5}=1$.

\end{itemize}
\end{goals}

\bigskip
\begin{center}
{\large \scshape Atividade}
\end{center}
\fi

A faixa a seguir está dividida em 5 partes iguais.

\begin{center}
 \begin{tikzpicture}[x=56.25mm,y=56.25mm]
\foreach \x in {0,1,2,...,4}{
\draw[fill=common, fill opacity=.3] (\x/5,0) rectangle (\x/5 + 1/5,.1);
\draw[step=.2] (0,0) grid (1,.1);
\draw (0,.1)-- (1,.1);}
 \end{tikzpicture}
\end{center}


\begin{enumerate} %s
  \item    Considerando a faixa como unidade,  complete a reta numérica  escrevendo frações  correspondentes às regiões coloridas em vermelho. 

\begin{center}
  \begin{tikzpicture}[x=56.25mm,y=56.25mm]

\foreach \x in {0,1,2,...,5}{
\fill[fill=common, fill opacity=.3, shift={(0,-\x*.15)}] (\x/5,0) rectangle (1,.1);
\draw[fill=attention, shift={(0,-\x*.15)}] (0,0) rectangle (\x/5,.1);
\draw[step=.2, shift={(0,-\x*.15)}] (0,0) grid (1,.1);
\draw[shift={(0,-\x*.15)}] (0,.1)-- (1,.1);}

\begin{scope}[shift={(0,-.9)}]
\draw[->] (-0.1,0) -- (1.1,0) ; %edit here for the axis
\foreach \x in  {0,1} % edit here for the vertical lines
\draw[shift={(\x,0)},color=black] (0,3pt) -- (0pt,-3pt) node[below] {$\x$};

\foreach \x in  {.2,.4,.6,.8} % edit here for the vertical lines
\draw[shift={(\x,0)},color=black] (0,3pt) -- (0pt,-3pt) node[below] {{\huge $\dfrac{\square}{\square}$}};
\end{scope}

\end{tikzpicture}
 \end{center}

  \item     E as faixas sem pintura vermelha e toda pintada de vermelho? Escreva as frações correspondentes a elas.

\end{enumerate} %s

\ifdefined\prof
\begin{solucao}

\begin{enumerate}

\item \adjustbox{valign=t}
{
 \begin{tikzpicture}[x=56.25mm,y=56.25mm]


 \foreach \x in {0,1,2,...,5}{
\fill[fill=common, fill opacity=.3, shift={(0,-\x*.15)}] (\x/5,0) rectangle (1,.1);
\draw[fill=attention, shift={(0,-\x*.15)}] (0,0) rectangle (\x/5,.1);
\draw[step=.2, shift={(0,-\x*.15)}] (0,0) grid (1,.1);
\draw[shift={(0,-\x*.15)}] (0,.1)-- (1,.1);}


 \begin{scope}[yshift=-135]
\draw[->] (-0.1,0) -- (1.1,0) ; %edit here for the axis

%\draw[shift={(0,0)},color=black] (0,3pt) -- (0pt,-3pt) node[below] {0};

\foreach \x in  {1,...,4} % edit here for the vertical lines
\fill[shift={(\x/5,0)},common] (0,0) circle (3pt) node[below, black] {{\Large $\frac{\x}{5}$}};
\foreach \x in {0,1}
\fill[shift={(\x,0)},common] (0,0) circle (3pt) node[below, black] {\x};
 \end{scope}

\end{tikzpicture}
}

\item A faixa sem pintar é igual a fração $\frac{0}{5}$, ou seja, é igual a zero. A faixa inteiramente pintada é igual a fração $\frac{5}{5}$, ou seja, é igual à unidade, ou igual a 1. Portanto, $\frac{0}{5}=0$ e $\frac{5}{5}=1$.

\end{enumerate}

\end{solucao}
\fi

\end{document}