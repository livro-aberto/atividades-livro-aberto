
\documentclass[10 pt,usenames,dvipsnames, oneside]{article}
\usepackage{../../modelo-fracoes}
\graphicspath{{../../../Figuras/licao03/}}


\begin{document}

\begin{center}
  \begin{minipage}[l]{3cm}
\includegraphics[width=2cm]{../../../Figuras/logo}       
\end{minipage}\hfill
\begin{minipage}[r]{.8\textwidth}
 {\Large \scshape Atividade: Frações dos retângulos coloridos}  
\end{minipage}
\end{center}
\vspace{.2cm}

\ifdefined\prof
%Caixa do Para o Professor
\begin{goals}
%Objetivos específicos
\begin{enumerate}
\item       Relacionar a representação de frações unitárias em modelo de área retangular com a representação dessas frações na reta numérica.
\end{enumerate}

\tcblower

%Orientações e sugestões
\begin{itemize}
\item  Cada aluno deve receber o material para o desenvolvimento da atividade, que consiste em uma folha, disponível para reprodução no final do livro. Será necessário traçar uma reta. Distribua uma folha para isso ou oriente os alunos a fazê-lo no caderno, usando em ambos os casos a maior dimensão do papel.
\item Recomenda-se que os alunos manuseiem o material a ser reproduzido. É importante que reconheçam que todos os retângulos coloridos são iguais (congruentes), o que pode ser constatado pela sobreposição. O retângulo representa a unidade. Além disso, é importante que percebam que cada um dos retângulos coloridos (ou a unidade) tem uma equipartição indicada, representando frações unitárias diferentes. Por exemplo, cada parte do retângulo vermelho representa       $\dfrac{1}{5}$       da unidade.
\item       Algumas das frações indicadas para serem representadas na reta numérica são maiores que uma unidade. Nesses casos, oriente seus alunos a fazer a justaposição das partes dos retângulos correspondentes. Por exemplo, para representar       $\dfrac{12}{7}$       será necessário justapor um retângulo a cinco partes do retângulo azul.
\end{itemize}
\end{goals}

\bigskip
\begin{center}
{\large \scshape Atividade}
\end{center}
\fi

Você recebeu uma folha com 11 retângulos coloridos de mesmo tamanho. Cada retângulo está dividido em uma determinada quantidade de partes iguais.
  
\begin{enumerate}
\item Complete os retângulos escrevendo em cada uma das partes a fração correspondente, como no exemplo: o retângulo laranja está dividido em duas partes iguais, então cada parte é $\dfrac{1}{2}$.

\begin{center}
\begin{tikzpicture}[scale=.5, rotate=90]
\draw[fill=gray] (0,0) rectangle (60,12);


\begin{scope}[yshift=-40pt]
\draw[fill=light] (0,0) rectangle (60,12);
\draw (30,0) -- (30,12);
\node at (15,6) {{\small $\dfrac{1}{2}$}};
\node at (45,6) {{\small $\dfrac{1}{2}$}};


   \begin{scope}[yshift=-40pt]
\draw[fill=pink] (0,0) rectangle (60,12);
\foreach \x in {1,2} \draw (\x*60/3,0) -- (\x*60/3,12);


      \begin{scope}[yshift=-40pt]
\draw[fill=special] (0,0) rectangle (60,12);
\foreach \x in {1,2,3} \draw (\x*60/4,0) -- (\x*60/4,12);


   \begin{scope}[yshift=-40pt]
\draw[fill=attention] (0,0) rectangle (60,12);
\foreach \x in {1,...,4} \draw (\x*60/5,0) -- (\x*60/5,12);


   \begin{scope}[yshift=-40pt]
\draw[fill=common] (0,0) rectangle (60,12);
\foreach \x in {1,...,5} \draw (\x*60/6,0) -- (\x*60/6,12);


    \begin{scope}[yshift=-40pt]
\draw[fill=CornflowerBlue] (0,0) rectangle (60,12);
\foreach \x in {1,...,6} \draw (\x*60/7,0) -- (\x*60/7,12);


   \begin{scope}[yshift=-40pt]
\draw[fill=dark] (0,0) rectangle (60,12);
\foreach \x in {1,...,7} \draw (\x*60/8,0) -- (\x*60/8,12);


   \begin{scope}[yshift=-40pt]
\draw[fill=Fuchsia] (0,0) rectangle (60,12);'
\foreach \x in {1,...,8} \draw (\x*60/9,0) -- (\x*60/9,12);


 \begin{scope}[yshift=-40pt]
 \draw[fill=NavyBlue] (0,0) rectangle (60,12);
 \foreach \x in {1,...,9} \draw (\x*60/10,0) -- (\x*60/10,12);


\begin{scope}[yshift=-40pt]
    \draw[fill=BlueViolet] (0,0) rectangle (60,12);
\foreach \x in {1,...,15} \draw (\x*60/16,0) -- (\x*60/16,12);
\end{scope}\end{scope}\end{scope}\end{scope}\end{scope}\end{scope}\end{scope}\end{scope}\end{scope}\end{scope}
    \end{tikzpicture}
 \end{center}
 
\item  Recorte os retângulos da folha que você recebeu e use-os para representar os números a seguir em uma reta numérica construída por você.

  \begin{center}
$0$, $1$, $\dfrac{1}{2}$, $\dfrac{1}{3}$, $\dfrac{1}{4}$, $\dfrac{3}{4}$, $\dfrac{3}{5}$, $\dfrac{5}{6}$, $\dfrac{7}{6}$, $\dfrac{6}{7}$,  $\dfrac{10}{7}$,  $\dfrac{12}{7}$,  $\dfrac{10}{8}$,  $\dfrac{12}{8}$, $\dfrac{10}{9}$, $\dfrac{12}{9}$, $\dfrac{10}{10}$, $\dfrac{20}{16}$
  \end{center}


\end{enumerate}

\ifdefined\prof
\begin{solucao}

\begin{enumerate}
\item Da esquerda para a direita as frações são 1, $\dfrac{1}{2}$, $\dfrac{1}{3}$, $\dfrac{1}{4}$, $\dfrac{1}{5}$, $\dfrac{1}{6}$, $\dfrac{1}{7}$, $\dfrac{1}{8}$, $\dfrac{1}{9}$, $\dfrac{1}{10}$ e $\dfrac{1}{16}$.


\item \mbox{ }

\begin{center}\begin{tikzpicture}[x=37mm,y=35mm]
\draw[->] (-.1,0) -- (2.1,0);
\foreach \x in {0,1,2} {\draw (\x,-3pt) -- (\x,3 pt); \node at (\x,10pt) {\x};}
%\node at (60,-20pt) {1};
\foreach \x in {2,3,4} {\fill[common] (1/\x,0) circle (2 pt);\node at (1/\x,-10pt) {$\dfrac{1}{\x}$};}
\fill[common] (.75,0) circle (2 pt);\node at (.75,-10pt) {$\dfrac{3}{4}$};
\fill[common] (.6,0) circle (2 pt);\node at (.6,-10pt) {$\dfrac{3}{5}$};
\fill[common] (5/6,0) circle (2 pt);\node at (5/6,-10pt) {$\dfrac{5}{6}$};
\fill[common] (7/6,0) circle (2 pt);\node at (7/6,-10pt) {$\dfrac{7}{6}$};
\fill[common] (6/7,0) circle (2 pt);\node at (6/7,10pt) {$\dfrac{6}{7}$};
\fill[common] (10/7,0) circle (2 pt);\node at (10/7,10pt) {$\dfrac{10}{7}$};
\fill[common] (12/7,0) circle (2 pt);\node at (12/7,10pt) {$\dfrac{12}{7}$};
\fill[common] (10/8,0) circle (2 pt);\node at (10/8,-10pt) {$\dfrac{10}{8}$};
\fill[common] (12/8,0) circle (2 pt);\node at (12/8,-10pt) {$\dfrac{12}{8}$};
\fill[common] (10/9,0) circle (2 pt);\node at (10/9,10pt) {$\dfrac{10}{9}$};
\fill[common] (12/9,0) circle (2 pt);\node at (12/9,-10pt) {$\dfrac{12}{9}$};
\fill[common] (1,0) circle (2 pt);\node at (1,-10pt) {$\dfrac{10}{10}$};
\fill[common] (20/16,0) circle (2 pt);\node at (20/16,10pt) {$\dfrac{20}{16}$};
\end{tikzpicture}
\end{center}
\end{enumerate}

\end{solucao}
\fi

\end{document}