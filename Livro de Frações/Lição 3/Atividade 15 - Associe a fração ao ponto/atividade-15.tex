
\documentclass[10 pt,usenames,dvipsnames, oneside]{article}
\usepackage{../../modelo-fracoes}
\graphicspath{{../../../Figuras/licao03/}}


\begin{document}

\begin{center}
  \begin{minipage}[l]{3cm}
\includegraphics[width=2cm]{../../../Figuras/logo}       
\end{minipage}\hfill
\begin{minipage}[r]{.8\textwidth}
 {\Large \scshape Atividade: Associe a fração ao ponto}  
\end{minipage}
\end{center}
\vspace{.2cm}

\ifdefined\prof
%Caixa do Para o Professor
\begin{goals}
%Objetivos específicos
\begin{enumerate}
\item       Comparar frações com o mesmo numerador ou com o mesmo denominador e a partir de um referencial.
\end{enumerate}

\tcblower

%Orientações e sugestões
\begin{itemize}
\item Estimule seus alunos a explicarem suas respostas.
\item A associação das frações aos pontos correspondentes na reta numérica exigirá que os alunos façam comparações de diferentes tipos: frações com mesmo numerador, frações com mesmo denominador e comparação a partir de um referencial (mais detalhes no Para o Professor do início da Lição). Valorize e discuta as diversas estratégias apresentadas por eles.
\item       Por exemplo, uma vez que os pontos correspondentes ao 0, ao 1 e ao $\frac{1}{2}$ já estão destacados, é natural que as primeiras frações a serem associadas a pontos na reta numérica sejam       $\frac{1}{4}$       e       $\frac{3}{4}$. Em seguida, reconhecendo que       $\frac{1}{8}$       corresponde à metade de       $\frac{1}{4}$,  as frações       $\frac{3}{8}$       e       $\frac{5}{8}$       podem ser as próximas.  Na sequência, o aluno pode reconhecer que       $\frac{4}{5}$       e       $\frac{9}{10}$       são menores do que a unidade e que       $\frac{9}{8}$       e       $\frac{11}{10}$       são maiores.  Entre       $\frac{4}{5}$       e       $\frac{9}{10}$,       $\frac{9}{10}$       pode ser identificada como maior por faltar  apenas       $\frac{1}{10}$       para compor a unidade , enquanto que para       $\frac{4}{5}$       falta       $\frac{1}{5}$       da unidade. Por fim, por raciocínio análogo, a fração       $\frac{9}{8}$       pode ser identificada como maior do que       $\frac{11}{10}$: Sabe-se que $\frac{1}{8}$ é maior do que $\frac{1}{10}$ e que $\frac{9}{8}$  é  $\frac{1}{8}$       maior do que a unidade, enquanto que       $\frac{11}{10}$       é        $\frac{1}{10}$       maior. Portanto, $\frac{9}{8}$    é maior do que       $\frac{11}{10}$.
\item       Se achar necessário, discuta a comparação entre alguns pares das frações apresentadas antes de os alunos resolverem a atividade. Por exemplo, peça-lhes que comparem       $\frac{3}{8}$        e       $\frac{5}{8}$, que são frações com o mesmo denominador. Ou que comparem       $\frac{9}{8}$       e       $\frac{9}{10}$, frações com o mesmo numerador.
\item       O aluno pode responder simplesmente       ``ligando''       os cartões com as frações aos pontos correspondentes na reta numérica. No entanto, recomenda-se que o professor oriente-os a escreverm as frações abaixo dos pontos correspondentes na reta numérica, a exemplo do 0, do 1, e do       $\frac{1}{2}$.
\end{itemize}
\end{goals}

\bigskip
\begin{center}
{\large \scshape Atividade}
\end{center}
\fi

Na reta numérica a seguir estão identificados os pontos correspondentes aos números 0, 1 e $\frac{1}{2}$. Os demais pontos correspondem aos números   apresentados a seguir. Associe cada fração ao ponto correspondente na reta numérica.

\begin{center}
\begin{tikzpicture}[x=.3cm,y=.3cm,]
\draw[->] (-2,0) -- (47,0);

\fill[common] (0,0) circle (3 pt);		%0
\fill[common] (10,0) circle (3 pt);
\fill[common] (15,0) circle (3 pt);
\fill[common] (20,0) circle (3 pt);
\fill[common] (25,0) circle (3 pt);
\fill[common] (30,0) circle (3 pt);
\fill[common] (32,0) circle (3 pt);
\fill[common] (36,0) circle (3 pt);
\fill[common] (40,0) circle (3 pt);
\fill[common] (44,0) circle (3 pt);
\fill[common] (45,0) circle (3 pt);

\node[below]  at (0,-1) {0};			%0
\node[below]  at (20,-1) {$\frac{1}{2}$};	%1/2
\node[below]  at (40,-1) {1};			%1

\tikzstyle{gray_block} = [draw,outer sep=3,inner sep=3,minimum size=1,line width=1, very thick, draw=black!55, top color=white,bottom color=black!20]
\node [gray_block] at (10.5,5) {$\frac{1}{4}$};
\node [gray_block] at (14.5,5) {$\frac{3}{4}$};
\node [gray_block] at (18.5,5) {$\frac{4}{5}$};
\node [gray_block] at (22.5,5) {$\frac{3}{8}$};
\node [gray_block] at (26.5,5) {$\frac{5}{8}$};
\node [gray_block] at (30.5,5) {$\frac{9}{8}$};
\node [gray_block] at (35,5) {$\frac{9}{10}$};
\node [gray_block] at (40,5) {$\frac{11}{10}$};
\end{tikzpicture}
\end{center}

\ifdefined\prof
\begin{solucao}

\begin{center}
\begin{tikzpicture}[x=.3cm,y=.3cm, every node/.style={scale=1.25}]

\draw  (-2,0) -- (45,0);
\draw[->]  (45,0) -- (47,0);


\node[below,yshift=-.1cm]  at (0,0) {0};							%0
\node[below,yshift=-.1cm]  at (20,0) {$\frac{1}{2}$};			%1/2
\node[below,yshift=-.1cm]  at (40,0) {1};						%1

\node[below,yshift=-.1cm]  at (10,0) {$\frac{1}{4}$};			%1/4
\node[below,yshift=-.1cm]  at (15,0) {$\frac{3}{8}$};			%3/8

\node[below,yshift=-.1cm]  at (25,0) {$\frac{5}{8}$};			%5/8
\node[below,yshift=-.1cm]  at (30,0) {$\frac{3}{4}$};			%3/4
\node[below,yshift=-.1cm]  at (32,0) {$\frac{4}{5}$};			%4/5
\node[below,yshift=-.1cm]  at (36,0) {$\frac{9}{10}$};			%9/10

\node[below,yshift=-.1cm]  at (43.5,0) {$\frac{11}{10}$};		%11/10
\node[below,yshift=-.1cm]  at (45.5,0) {$\frac{9}{8}$};			%9/8

\foreach \x in {0,10,15,20,25,30,32,36,40,44,45} \node [fill, circle, inner sep=2pt, common] at (\x,0) {};
\end{tikzpicture}

\end{center}

\end{solucao}
\fi

\end{document}