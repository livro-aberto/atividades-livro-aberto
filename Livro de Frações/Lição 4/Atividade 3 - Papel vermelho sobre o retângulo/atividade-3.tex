
\documentclass[10 pt,usenames,dvipsnames, oneside]{article}
\usepackage{../../modelo-fracoes}
\graphicspath{{../../../Figuras/licao04/}}


\begin{document}

\begin{center}
  \begin{minipage}[l]{3cm}
\includegraphics[width=2cm]{../../../Figuras/logo}       
\end{minipage}\hfill
\begin{minipage}[r]{.8\textwidth}
 {\Large \scshape Atividade: }  
\end{minipage}
\end{center}
\vspace{.2cm}

\ifdefined\prof
%Caixa do Para o Professor
\begin{goals}
%Objetivos específicos
\begin{enumerate}
\item       Reconhecer que as frações       $\frac{3}{4}$       e
$\frac{12 \times 3}{12 \times 4}$       são iguais a partir da observação das
representações destas frações em modelos de área sem a contagem um a um das
partes que compõem as subdivisões destas representações.
\end{enumerate}

\tcblower

%Orientações e sugestões
\begin{itemize}
\item       Recomenda-se que, nesta atividade, os alunos trabalhem
individualmente ou em duplas. No entanto, é fundamental que os alunos sejam
estimulados a explicar o raciocínio realizado.
\item       O propósito de encobrir as divisões do retângulo é para evitar
que os alunos façam a contagem das partes uma a uma e que, assim, sejam
estimulados a perceber a estrutura multiplicativa       $12 \times 3$       e
$12 \times 4$       na divisão do retângulo.
\item       É importante, ao final da atividade, observar para os alunos que
uma mesma parte do retângulo (a área da região pintada de azul) está sendo
descrita por frações com numeradores e denominadores diferentes (isto é, por
frações equivalentes), mas que, não obstante, por expressarem uma mesma
quantidade, estas frações são iguais. 
\end{itemize}
\end{goals}

\bigskip
\begin{center}
{\large \scshape Atividade}
\end{center}
\fi

\textit{(Garcez, 2013)}

\begin{enumerate} %s
  \item     O retângulo desenhado a seguir está dividido em     $4$ partes iguais, das quais     $3$ estão pintadas de azul. Que fração do retângulo está pintada de azul?
  \begin{center}
\begin{tikzpicture}[scale=.8]
\draw[fill=common] (0,0) rectangle (30,30);
\draw (30,0) rectangle (40,30);
\draw  (10,0)-- (10,30);
\draw  (20,0)-- (20,30);
\end{tikzpicture}
\end{center}

\item     O retângulo do item anterior foi dividido com o acréscimo de onze linhas horizontais igualmente espaçadas e ele está parcialmente coberto com um papel vermelho que impede a visualização dos retângulos menores que compõem a nova equipartição. Com essa nova divisão, em quantas partes ficou dividido o retângulo inicial? Quantas dessas partes estão pintadas de azul? Que fração do retângulo está pintada de azul?     \mbox{} \newline
\end{enumerate} %s

\begin{center}
\begin{tikzpicture}[scale=.8]
\draw[fill=common] (0,0) rectangle (30,30);
\draw (30,0) rectangle (40,30);
\draw  (10,0)-- (10,30);
\draw  (20,0)-- (20,30);
\foreach \x in {1,...,11}{
\draw (0,\x*30/12) -- (40,\x*30/12);}
\fill[attention] (5,-10) rectangle (55,29);
\end{tikzpicture}
\end{center}

\ifdefined\prof
\begin{solucao}

\begin{enumerate}
\item             $\frac{3}{4}$.
\item       Com a nova divisão, o retângulo inicial ficou dividido em       $12
\times 4 = 48$ retângulos menores, dos quais       $12 \times 3 = 36$       estão
pintados de azul. Assim, a fração do retângulo inicial que está pintada de azul é
$\frac{12 \times 3}{12 \times 4} = \frac{36}{48}$.
\end{enumerate}

\end{solucao}
\fi

\end{document}