
\documentclass[10 pt,usenames,dvipsnames, oneside]{article}
\usepackage{../../modelo-fracoes}
\graphicspath{{../../../Figuras/licao04/}}


\begin{document}

\begin{center}
  \begin{minipage}[l]{3cm}
\includegraphics[width=2cm]{../../../Figuras/logo}       
\end{minipage}\hfill
\begin{minipage}[r]{.8\textwidth}
 {\Large \scshape Atividade: Fração unitária}  
\end{minipage}
\end{center}
\vspace{.2cm}

\ifdefined\prof
%Caixa do Para o Professor
\begin{goals}
%Objetivos específicos
\begin{enumerate}
\item       Analisar quando uma fração é igual a uma fração unitária.
\end{enumerate}

\tcblower

%Orientações e sugestões
\begin{itemize}
\item       Esta é uma atividade que pode ser desenvolvida individualmente.
Contudo, é fundamental que os alunos sejam estimulados a explicar o raciocínio
realizado.
    \item       O item c) relaciona-se com a Atividade \textit{Repartindo sanduíches no piquenique}: como não é possível,
em uma equipartição de uma região retangular, escolher uma quantidade de partes
que corresponda à metade desta região se a quantidade total de partes for um
número ímpar, não existe uma fração de denominador ímpar que seja igual à fração
      $\dfrac{1}{2}$.
    \item       Observe para seus alunos que as frações estudadas na Lição 1 são
justamente as frações unitárias e que, pela Lição 2, toda fração é a
justaposição de frações unitárias. Em outras palavras, as frações unitárias
constituem a estrutura básica a partir da qual as demais frações são obtidas.
\end{itemize}
\end{goals}

\bigskip
\begin{center}
{\large \scshape Atividade}
\end{center}
\fi

Uma fração é dita {\bf unitária} se o seu numerador é igual a $1$.
\begin{enumerate}
\item  Quais das frações a seguir são iguais a alguma fração unitária? Justifique sua resposta.

\begin{center}
\begin{tabular}{m{.13\textwidth}m{.13\textwidth}m{.13\textwidth}m{.13\textwidth}}
$\dfrac{4}{20}$, & $\dfrac{21}{7}$, & $\dfrac{4}{30}$, & $\dfrac{6}{18}$.
\end{tabular}
\end{center}

\item  Uma fração com numerador maior do que o denominador pode ser igual a uma fração unitária? Justifique sua resposta!

\item  Existe uma fração de denominador ímpar que seja igual à fração unitária $\dfrac{1}{2}$? Justifique sua resposta!
\end{enumerate}

\ifdefined\prof
\begin{solucao}

\begin{enumerate}
\item       Pelo item b) da Atividade \textit{Numerador da 1\textsuperscript{a} com o denominador da 2\textsuperscript{a}}      , se uma dada fração é
igual a alguma fração unitária, então o produto do numerador da fração dada pelo
denominador da fração unitária tem que ser igual ao denominador da fração dada,
isto é, o denominador da fração dada tem que ser um múltiplo inteiro do seu
numerador. Isto só acontece para as frações       $\dfrac{4}{20}$       e
$\dfrac{6}{18}$.
    \item       Não, pois frações unitárias são sempre menores ou iguais a 1,
enquanto que uma fração com numerador maior do que o denominador é sempre maior
do que 1.

    \item       Não, pois pelo item b) da Atividade \textit{Numerador da 1\textsuperscript{a} com o denominador da 2\textsuperscript{a}}      , se
existisse uma fração com denominador ímpar que fosse igual à fração
$\dfrac{1}{2}$, então o numerador da fração dada multiplicado por       $2$, um
número par, teria que ser igual ao denominador da fração dada multiplicado por
1, o que dá um número ímpar. Portanto, um número par teria que ser igual a um
número ímpar, o que não é possível.
\end{enumerate}

\end{solucao}
\fi

\end{document}