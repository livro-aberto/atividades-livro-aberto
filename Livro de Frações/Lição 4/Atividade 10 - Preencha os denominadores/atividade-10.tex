
\documentclass[10 pt,usenames,dvipsnames, oneside]{article}
\usepackage{../../modelo-fracoes}
\graphicspath{{../../../Figuras/licao04/}}


\begin{document}

\begin{center}
  \begin{minipage}[l]{3cm}
\includegraphics[width=2cm]{../../../Figuras/logo}       
\end{minipage}\hfill
\begin{minipage}[r]{.8\textwidth}
 {\Large \scshape Atividade: Preencha os denominadores}  
\end{minipage}
\end{center}
\vspace{.2cm}

\ifdefined\prof
%Caixa do Para o Professor
\begin{goals}
%Objetivos específicos
\begin{enumerate}
\item       Determinar uma fração igual a uma dada fração com numerador ou
denominador especificados.
\end{enumerate}

\tcblower

%Orientações e sugestões
\begin{itemize}
\item       Recomenda-se que, nesta atividade, os alunos trabalhem
individualmente. No entanto, é fundamental que os alunos sejam estimulados a
explicar o raciocínio realizado.
\item       Espera-se que, neste estágio, os alunos consigam obter as
respostas usando a propriedade que       $\dfrac{m \times a}{m \times b}$       é
equivalente a       $\dfrac{a}{b}$       e sem recorrer a desenhos de modelos de
área de frações.
\item       Observe que, no item (e), não existe um número natural       $n$
tal que       $6 \times n = 9$. Para resolver o item, o aluno pode usar o
resultado do item (d) e substituir       $\dfrac{9}{12}$       por
$\dfrac{3}{4}$       e proceder com o exercício. A mesma observação aplica-se ao
item (f).
\item       Observe para seus alunos que os Itens (e) e (f) são exemplos de
frações iguais para os quais não é possível obter uma fração multiplicando-se o
numerador e o denominador da outra por um mesmo número natural.
\end{itemize}
\end{goals}

\bigskip
\begin{center}
{\large \scshape Atividade}
\end{center}
\fi

\textit{(Van de Walle, 2009)}

Preencha os $\square$ com números de forma a tornar as igualdades verdadeiras.

\noindent\begin{tabular}{m{.14\textwidth}m{.14\textwidth}m{.14\textwidth}m{.14\textwidth}m{.14\textwidth}m{.14\textwidth}}
a)  $\dfrac{5}{3} = \dfrac{\square}{6}$ & b) $\dfrac{2}{3} = \dfrac{6}{\square}$ & c) $\dfrac{8}{12} = \dfrac{\square}{3}$ & d) $\dfrac{9}{12} = \dfrac{3}{\square}$& e) $\dfrac{9}{12} = \dfrac{6}{\square}$& f)  $\dfrac{6}{8} = \dfrac{\square}{12}$
\end{tabular}

\ifdefined\prof
\begin{solucao}

\begin{enumerate}
\item Uma vez que       $6 = 2 \times 3$, então       $\dfrac{5}{3} =
\dfrac{2 \times 5}{2 \times 3} = \dfrac{10}{6}$. Logo,       $\square$       deve
ser preenchido com       $10$.
\item       Uma vez que       $6 = 3 \times 2$, então       $\dfrac{2}{3} =
\dfrac{3 \times 2}{3 \times 3} = \dfrac{6}{9}$. Logo,       $\square$       deve
ser preenchido com       $9$.
\item       Uma vez que       $12 = 4 \times 3$       e       $8 = 4 \times
2$, então       $\dfrac{8}{12} = \dfrac{4 \times 2}{4 \times 3} = \dfrac{2}{3}$.
Logo,       $\square$       deve ser preenchido com       $2$.
\item       Uma vez que       $9 = 3 \times 3$       e       $12 = 3 \times
4$, então       $\dfrac{9}{12} = \dfrac{3 \times 3}{3 \times 4} = \dfrac{3}{4}$.
Logo,       $\square$       deve ser preenchido com       $4$.
\item       Pelo item d,       $\dfrac{9}{12} = \dfrac{3}{4}$. Uma vez que
$6 = 2 \times 3$, então       $\dfrac{3}{4} = \dfrac{2 \times 3}{2 \times 4} =
\dfrac{6}{8}$. Logo,       $\square$       deve ser preenchido com       $8$.
\item       Pela solução do item e,       $\square$       deve ser
preenchido com       $9$.
\end{enumerate}

\end{solucao}
\fi

\end{document}