
\documentclass[10 pt,usenames,dvipsnames, oneside]{article}
\usepackage{../../modelo-fracoes}
\graphicspath{{../../../Figuras/licao04/}}


\begin{document}

\begin{center}
  \begin{minipage}[l]{3cm}
\includegraphics[width=2cm]{../../../Figuras/logo}       
\end{minipage}\hfill
\begin{minipage}[r]{.8\textwidth}
 {\Large \scshape Atividade: Qual é a maior e qual é a menor fração?}  
\end{minipage}
\end{center}
\vspace{.2cm}

\ifdefined\prof
%Caixa do Para o Professor
\begin{goals}
%Objetivos específicos
\begin{enumerate}
\item       Comparar mais do que duas frações (no caso, três) usando frações
equivalentes.
\end{enumerate}

\tcblower

%Orientações e sugestões
\begin{itemize}
\item       Esta é uma atividade que pode ser desenvolvida individualmente.
Contudo, é fundamental que os alunos sejam estimulados a explicar o raciocínio
realizado.
\item       Sugere-se que seja observado para os alunos que o procedimento
descrito nesta atividade para ordenar três frações pode ser aplicado para um
número arbitrário de frações.
\item       Esta atividade foi concebida para ser resolvida usando a notação
de fração, sem o uso do recurso de modelos de frações uma vez que, neste
estágio, espera-se que o aluno já tenha o domínio desta técnica de manipulação
aritmética.
\item       Observe que a ordenação poderia ser feita comparando-se duas
frações por vez. A solução indicada reduz a ordenação à ordenação de números
naturais (os numeradores das frações iguais às frações dadas e todas de mesmo
denominador).
\end{itemize}
\end{goals}

\bigskip
\begin{center}
{\large \scshape Atividade}
\end{center}
\fi

O objetivo desta atividade é determinar qual é a maior e qual é a menor fração entre
as frações $\dfrac{11}{6}$, $\dfrac{28}{15}$ e $\dfrac{37}{20}$.

\begin{enumerate} %s
  \item     Determine três frações de mesmo denominador que sejam iguais às frações     $\dfrac{11}{6}$,     $\dfrac{28}{15}$ e     $\dfrac{37}{20}$.
  \item     Usando as frações do item a), determine qual é a maior e qual é a menor fração entre as frações     $\dfrac{11}{6}$,     $\dfrac{28}{15}$ e     $\dfrac{37}{20}$.
\end{enumerate} %s


\ifdefined\prof
\begin{solucao}

\begin{enumerate}
\item             $60$       é um múltiplo comum de       $6$,       $20$
   e       $15$      :       $60 = 10 \times 6$,       $60 = 4 \times 15$
e       $60 = 3 \times 20$. Portanto,       

\begin{align*}
\dfrac{11}{6} &= \dfrac{10 \times11}{10 \times 6} = \dfrac{110}{60},\\
\dfrac{28}{15} &= \dfrac{4\times 28}{4 \times 15} = \dfrac{112}{60} \quad {\rm  e}\\
\dfrac{37}{20} &= \dfrac{3 \times 37}{3 \times 20} = \dfrac{111}{60}.\\
\end{align*}
\mbox{} \newline
    \item       Tem-se que       $\dfrac{110}{60} < \dfrac{111}{60} <
\dfrac{112}{60}$. Logo,

  $$\dfrac{11}{6} < \dfrac{37}{20} < \dfrac{28}{15}.$$
\end{enumerate}

\end{solucao}
\fi

\end{document}