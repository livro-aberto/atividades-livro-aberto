
\documentclass[10 pt,usenames,dvipsnames, oneside]{article}
\usepackage{../../modelo-fracoes}
\graphicspath{{../../../Figuras/licao04/}}


\begin{document}

\begin{center}
  \begin{minipage}[l]{3cm}
\includegraphics[width=2cm]{../../../Figuras/logo}       
\end{minipage}\hfill
\begin{minipage}[r]{.8\textwidth}
 {\Large \scshape Atividade: Existe fração entre $\frac{3}{5}$ e $\frac{4}{5}$?}  
\end{minipage}
\end{center}
\vspace{.2cm}

\ifdefined\prof
%Caixa do Para o Professor
\begin{goals}
%Objetivos específicos
\begin{enumerate}
\item       Perceber a propriedade de densidßade das frações ao obter frações
que estão entre duas frações diferentes quaisquer, mesmo no caso de numeradores
consecutivos e denominadores iguais. Isto é, que dadas duas frações
$\dfrac{a}{b}$       e       $\dfrac{c}{d}$       diferentes (suponha
$\dfrac{a}{b}<\dfrac{c}{d}$), sempre é possível determinar uma terceira fração
  $\dfrac{p}{q}$       tal que       $\dfrac{a}{b}<\dfrac{p}{q}<\dfrac{c}{d}$.
\end{enumerate}

\tcblower

%Orientações e sugestões
\begin{itemize}
\item       Recomenda-se que, nesta atividade, os alunos trabalhem
individualmente ou em duplas. No entanto, é fundamental que os alunos sejam
estimulados a explicar o raciocínio realizado.
    \item       Caso seja viável, recomenda-se, na discussão da atividade, o uso
de um software (o GeoGebra, por exemplo) para marcar na reta numérica as
sucessivas frações dadas pelos alunos.
\end{itemize}
\end{goals}

\bigskip
\begin{center}
{\large \scshape Atividade}
\end{center}
\fi

Fabrício acredita que não existem frações entre $\dfrac{3}{5}$ e $\dfrac{4}{5}$ (isto é, maiores de que $\dfrac{3}{5}$ e menores do que $\dfrac{4}{5}$) porque $3 < 4$ e não existe número natural entre $3$ e $4$. Fabrício continua: ``Pelo mesmo motivo, não existem frações entre $\dfrac{11}{10}$ e $\dfrac{12}{10}$ e entre $\dfrac{19}{20}$ e $\dfrac{20}{20}$!''. Você concorda com as afirmações e argumentos de Fabrício? Se você acha que Fabrício está errado, determine:

\begin{enumerate}
\item  Uma fração entre $\dfrac{3}{5}$ e $\dfrac{4}{5}$;

\item  Duas frações entre $\dfrac{11}{10}$ e $\dfrac{12}{10}$;

\item  Uma fração entre $\dfrac{19}{20}$ e $\dfrac{20}{20}$.
\end{enumerate}

\ifdefined\prof
\begin{solucao}

\begin{enumerate}
    \item       Note que       $\dfrac{3}{5} = \dfrac{2 \times 3}{2 \times 5} =
\dfrac{6}{10}$       e       $\dfrac{4}{5} = \dfrac{2 \times 4}{2 \times 5} =
\dfrac{8}{10}$. Portanto,       $\dfrac{7}{10}$       é tal que       $\dfrac{3}{5}
< \dfrac{7}{10} < \dfrac{4}{5}$.
    \item       Note que       $\dfrac{11}{10} = \dfrac{3 \times 11}{3 \times 10}
= \dfrac{33}{30}$       e       $\dfrac{12}{10} = \dfrac{3 \times 12}{3 \times 10}
= \dfrac{36}{30}$. Portanto,       $\dfrac{34}{30}$       e       $\dfrac{35}{30}$
     são tais que       $\dfrac{11}{10} < \dfrac{34}{30} < \dfrac{35}{30} <
\dfrac{36}{30}$.
    \item       Note que       $\dfrac{19}{20} = \dfrac{4 \times 19}{4 \times 20}
= \dfrac{76}{80}$       e       $\dfrac{20}{20} = \dfrac{4 \times 20}{4 \times 20}
= \dfrac{80}{80}$. Portanto,       $\dfrac{77}{80}$,       $\dfrac{78}{80}$       e
      $\dfrac{79}{80}$       são tais que       $\dfrac{19}{20} < \dfrac{77}{80} <
\dfrac{78}{80} < \dfrac{79}{80} < \dfrac{80}{80}$.
\end{enumerate}

\end{solucao}
\fi

\end{document}