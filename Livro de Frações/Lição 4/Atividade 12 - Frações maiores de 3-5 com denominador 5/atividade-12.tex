
\documentclass[10 pt,usenames,dvipsnames, oneside]{article}
\usepackage{../../modelo-fracoes}
\graphicspath{{../../../Figuras/licao04/}}


\begin{document}

\begin{center}
  \begin{minipage}[l]{3cm}
\includegraphics[width=2cm]{../../../Figuras/logo}       
\end{minipage}\hfill
\begin{minipage}[r]{.8\textwidth}
 {\Large \scshape Atividade: Frações maiores que $\frac{3}{5}$ com denominador $5$}  
\end{minipage}
\end{center}
\vspace{.2cm}

\ifdefined\prof
%Caixa do Para o Professor
\begin{goals}
%Objetivos específicos
\begin{enumerate}
\item       Reconhecer que, dada uma fração       $p = \frac{n}{d}$, existe
um quantidade finita de frações da forma       $\frac{k}{d}$       que são
menores do que       $p$       e uma quantidade infinita de frações da forma
$\frac{k}{d}$       que são maiores do que       $p$.
\end{enumerate}

\tcblower

%Orientações e sugestões
\begin{itemize}
\item       Recomenda-se que, nesta atividade, os alunos trabalhem
individualmente ou em duplas. No entanto, é fundamental que os alunos sejam
estimulados a explicar o raciocínio realizado.
\item       Alguns alunos podem ainda necessitar de apoio de material
concreto para responder à questão.
\item       Recomenda-se que, na discussão da atividade, uma reta numérica
com quintos marcados seja usada como uma contrapartida visual para as respostas
das perguntas.
\end{itemize}

\begin{center}\begin{tikzpicture}[x=23mm,y=23mm]
\draw[->] (-0.1,0) -- (3.1,0);
\foreach \x in {0,...,3}{ \draw (\x,-3pt) -- (\x, 3pt);  \node[above] at (\x,3pt) {\x}; }

\foreach \x in {0,1,...,15}{
\draw (\x/5,-2pt) -- (\x/5, 2pt);
\node[below] at (\x/5,-2 pt) {$\frac{\x}{5}$};}

\fill[common] (3/5,0) circle (2pt);
\end{tikzpicture}
\end{center}
\end{goals}

\bigskip
\begin{center}
{\large \scshape Atividade}
\end{center}
\fi

\begin{enumerate} %s
  \item     Quantas são as frações com denominador igual a 5 que são menores do que $\frac{3}{5}$? Explique como você chegou à sua resposta.
  \item     Quantas são as frações com denominador igual a 5 que são maiores do que $\frac{3}{5}$? Explique como você chegou à sua resposta.
\end{enumerate} %s

\ifdefined\prof
\begin{solucao}

\begin{enumerate}
\item       Três frações:       $\frac{0}{5}$,       $\frac{1}{5}$       e
$\frac{2}{5}$. Justificativa:       $\frac{3}{5}$       são três
``cópias''       de       $\frac{1}{5}$. Qualquer outra fração de denominador
$5$       também é composta por uma quantidade inteira de       ``cópias''
de       $\frac{1}{5}$, quantidade esta determinada pelo numerador de fração.
Para se ter então uma fração de denominador       $5$       menor do que
$\frac{3}{5}$, devemos ter menos do que       $3$             ``cópias''
de       $\frac{1}{5}$      :       $2$             ``cópias''      ,       $1$
   ``cópia''       ou       $0$             ``cópia''. Assim, as
frações de denominador       $5$       menor do que       $\frac{3}{5}$
são       $\frac{0}{5}$,       $\frac{1}{5}$       e       $\frac{2}{5}$.
\item       Infinitas frações       $\frac{4}{5}$,       $\frac{5}{5}$,
$\frac{6}{5}$,       $\frac{7}{5}$, etc. Justificativa:       $\frac{3}{5}$
são três       ``cópias''       de       $\frac{1}{5}$. Qualquer outra fração
de denominador       $5$       também é composta por uma quantidade inteira de
``cópias''       de       $\frac{1}{5}$, quantidade esta determinada pelo
numerador de fração. Para se ter então uma fração de denominador       $5$
maior do que       $\frac{3}{5}$, devemos ter mais do que       $3$
``cópias''       de       $\frac{1}{5}$      :       $4$             ``cópias''
,       $5$             ``cópias''      ,       $6$             ``cópias''
,       $7$             ``cópias''      , etc. Assim, as frações de
denominador       $5$       maiores do que       $\frac{3}{5}$       são
$\frac{4}{5}$,       $\frac{5}{5}$,       $\frac{6}{5}$,       $\frac{7}{5}$,
etc.

\end{enumerate}

\end{solucao}
\fi

\end{document}