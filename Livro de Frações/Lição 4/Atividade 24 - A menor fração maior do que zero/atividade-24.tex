
\documentclass[10 pt,usenames,dvipsnames, oneside]{article}
\usepackage{../../modelo-fracoes}
\graphicspath{{../../../Figuras/licao04/}}


\begin{document}

\begin{center}
  \begin{minipage}[l]{3cm}
\includegraphics[width=2cm]{../../../Figuras/logo}       
\end{minipage}\hfill
\begin{minipage}[r]{.8\textwidth}
 {\Large \scshape Atividade: A menor fração maior do que zero}  
\end{minipage}
\end{center}
\vspace{.2cm}

\ifdefined\prof
%Caixa do Para o Professor
\begin{goals}
%Objetivos específicos
\begin{enumerate}
\item       Perceber a propriedade de densidade das frações ao obter frações
arbitrariamente próximas de       $0$       e arbitrariamente próximas de
$1$.
\end{enumerate}

\tcblower

%Orientações e sugestões
\begin{itemize}
\item       Recomenda-se que, para facilitar a logística de condução desta
atividade, que ela seja feita com as perguntas sendo propostas uma a uma por
você para a turma, de modo que a resposta de uma pergunta dada por um aluno seja
então usada como referência para a pergunta subsequente. Outra possibilidade é
dividir a turma em grupos de 3 a 5 alunos. Cada grupo responde a primeira
pergunta e então passa sua resposta para um outro grupo que deve então responder
a próxima questão tendo como referência a resposta recebida e assim
sucessivamente.
    \item       Caso seja viável, recomenda-se, na discussão da atividade, o uso
de um software (o GeoGebra, por exemplo) para marcar na reta numérica as
sucessivas frações dadas pelos alunos. O recurso de ampliação e redução pode ser
usado visualizar as frações quando estas se acumulam em       $0$       e em
  $1$.
\end{itemize}
\end{goals}

\bigskip
\begin{center}
{\large \scshape Atividade}
\end{center}
\fi

\begin{enumerate}
 \item Escreva uma fração que seja menor do que 1 e maior do que 0.
 \item Existe uma fração maior do que 0 e menor do que a fração que você escreveu no item a)? Em caso afirmativo, escreva uma tal fração.
 \item Existe uma fração menor do que a fração que você escreveu no item b)? Em caso afirmativo, escreva uma tal fração.
 \item Dada uma fração menor do que 1 e maior do que 0, é sempre possível determinar uma outra fração menor ainda? Em caso afirmativo, explique como tal fração pode ser obtida.
 \item Existe uma fração menor do que 1 e que seja maior do que a fração que você escreveu no item a)? Em caso afirmativo, escreva uma tal fração.
 \item Existe uma fração menor do que 1 e que seja maior do que a fração que você escreveu no item e)? Em caso afirmativo, escreva uma tal fração.
 \item Dada uma fração menor do que 1, é sempre possível determinar uma outra fração menor do que 1 e que seja maior do que a fração dada? Em caso afirmativo, explique como tal fração pode ser obtida.
\end{enumerate}

\ifdefined\prof
\begin{solucao}

\begin{enumerate} %s
    \item             $\dfrac{1}{2}$, por exemplo.
    \item       Sim,       $\dfrac{1}{3}$.
    \item       Sim,       $\dfrac{1}{4}$.
    \item       Sim:       $\dfrac{1}{5} < \dfrac{1}{4}$,       $\dfrac{1}{6} <
\dfrac{1}{5}$,       $\dfrac{1}{7} < \dfrac{1}{6}$, etc. Mais geralmente, dada uma
fração, basta considerar a fração de mesmo numerador e denominador maior do que
o denominador da fração dada. Esta segunda fração será sempre menor do que a
fração dada.
    \item       Sim,       $\dfrac{2}{3}$. Enquanto que para       $\dfrac{1}{2}$,
é necessário       $\dfrac{1}{2}$       para completar a unidade, para
$\dfrac{2}{3}$       é necessário       $\dfrac{1}{3}$       que é menor que
$\dfrac{1}{2}$, logo       $\dfrac{3}{4} > \dfrac{1}{2}$.
    \item       Sim,       $\dfrac{3}{4}$. Enquanto que para       $\dfrac{2}{3}$,
é necessário       $\dfrac{1}{3}$       para completar a unidade, para
$\dfrac{3}{4}$       é necessário       $\dfrac{1}{4}$       que é menor que
$\dfrac{1}{3}$, logo       $\dfrac{3}{4} > \dfrac{2}{3}$.
    \item       Sim,       $\dfrac{4}{5} > \dfrac{3}{4}$,       $\dfrac{5}{6} >
\dfrac{4}{5}$,       $\dfrac{6}{7} > \dfrac{5}{6}$, etc.
\end{enumerate} %s

  Mais geralmente, as frações cujo numerador é um número natural e o denominador
é o sucessor do numerador formam uma sucessão crescente de frações menores do
que   $1$.


\end{solucao}
\fi

\end{document}