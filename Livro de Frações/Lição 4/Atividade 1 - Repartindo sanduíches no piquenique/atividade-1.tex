
\documentclass[10 pt,usenames,dvipsnames, oneside]{article}
\usepackage{../../modelo-fracoes}
\graphicspath{{../../../Figuras/licao04/}}


\begin{document}

\begin{center}
  \begin{minipage}[l]{3cm}
\includegraphics[width=2cm]{../../../Figuras/logo}       
\end{minipage}\hfill
\begin{minipage}[r]{.8\textwidth}
 {\Large \scshape Atividade: Repartindo sanduíches no piquenique}  
\end{minipage}
\end{center}
\vspace{.2cm}

\ifdefined\prof
%Caixa do Para o Professor
\begin{goals}
%Objetivos específicos
\begin{enumerate}
\item Reconhecer que existem frações de denominadores diferentes que representam uma mesma quantidade da unidade.
\item       Reconhecer que as frações       $\frac{1}{2}$       e
$\frac{2}{4}$       são iguais a partir da observação das representações destas
frações em modelos de área retangulares.
\item       Reconhecer que, em uma equipartição de uma região retangular, só
é possível escolher uma quantidade de partes que corresponda à metade desta
região se a quantidade total de partes for um número par.
\end{enumerate}

\tcblower

%Orientações e sugestões
\begin{itemize}
\item       Recomenda-se que, nesta atividade, os alunos trabalhem
individualmente ou em duplas. No entanto, é fundamental que os alunos sejam
estimulados a explicar o raciocínio realizado.
\item       Reforce para seus alunos que o item b) deve ser respondido com a
partição apresentada, isto é, sem gerar novas partições.
\item       Observe que o item c) pode ser respondido apenas pela fração
$\frac{1}{2}$. No entanto, é importante estimular os alunos a pereceberem que
a metade do sanduíche pode ser obtida por       $\frac{2}{4}$       do
sanduíche.
\item       É importante, ao final da atividade, observar para os alunos que
uma mesma parte do retângulo (metade do retângulo) está sendo descrita por
frações com numeradores e denominadores diferentes (isto é, por frações
equivalentes) que, por expressarem uma mesma quantidade, são frações iguais. Assim,       $\frac{1}{2}$,       $\frac{2}{4}$,
$\frac{4}{8}$, etc. são respostas válidas para o item b) desta atividade.
\end{itemize}
\end{goals}

\bigskip
\begin{center}
{\large \scshape Atividade}
\end{center}
\fi

A turma de Rita vai fazer um piquenique. A professora comprou pães para a turma preparar sanduíches. Cada colega de Rita preparou um sanduíche e partiu-o em partes iguais. Veja como alguns dos colegas repartiram o seu sanduiche:


\begin{center}
\begin{tabular}{cccc}
\begin{tikzpicture}
\draw[fill=common, fill opacity=.3] (0,0) rectangle (20,20);
\draw (0,0) -- (20,20);
\node[below] at (10,0){(A)};
\end{tikzpicture}
&
\begin{tikzpicture}
\draw[fill=common, fill opacity=.3] (0,0) rectangle (20,20);
\draw (0,20/3) -- (20,20/3);
\draw (0,40/3) -- (20,40/3);
\node[below] at (10,0){(B)};
\end{tikzpicture}
&
\begin{tikzpicture}
\draw[fill=common, fill opacity=.3] (0,0) rectangle (20,20);
\draw (0,0) -- (20,20);
\draw (20,0) -- (0,20);
\node[below] at (10,0){(C)};
\end{tikzpicture}
&
\begin{tikzpicture}
\draw[fill=common, fill opacity=.3] (0,0) rectangle (20,20);
\draw (10,0) -- (10,20);
\draw (0,10) -- (20,10);
\node[below] at (10,0){(D)};
\end{tikzpicture}
\end{tabular}
\end{center}


\begin{enumerate} %s
  \item     Nessas repartições, que fração do sanduíche pode representar cada uma das partes em que o sanduíche foi repartido?
  \item     Em quais dessas repartições é possível comer metade do sanduíche repartido sem parti-lo ainda mais? Justifique sua resposta!
  \item     Para cada uma das repartições que você deu como resposta no item b), expresse, por meio de frações, a metade do sanduíche.
\end{enumerate} %s

\ifdefined\prof
\begin{solucao}

\begin{enumerate}
\item       Em a):       $\frac{1}{2}$. Em b):       $\frac{1}{3}$. Em c):
$\frac{1}{4}$. Em d):       $\frac{1}{4}$.
\item       É possível comer metade do sanduíche apenas nas repartições a),
c) e d) pois, para elas, a quantidade de partes iguais em que o sanduíche foi
dividido é um número par.
\item       Em a):       $\frac{1}{2}$. Em c):       $\frac{2}{4}$. Em d):
$\frac{2}{4}$.
\end{enumerate}

\end{solucao}
\fi

\end{document}