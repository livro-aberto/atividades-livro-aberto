
\documentclass[10 pt,usenames,dvipsnames, oneside]{article}
\usepackage{../../modelo-fracoes}
\graphicspath{{../../../Figuras/licao04/}}


\begin{document}

\begin{center}
  \begin{minipage}[l]{3cm}
\includegraphics[width=2cm]{../../../Figuras/logo}       
\end{minipage}\hfill
\begin{minipage}[r]{.8\textwidth}
 {\Large \scshape Atividade: A fração aumenta ou diminui?}  
\end{minipage}
\end{center}
\vspace{.2cm}

\ifdefined\prof
%Caixa do Para o Professor
\begin{goals}
%Objetivos específicos
\begin{enumerate}
\item       Comparar uma fração com uma outra fração determinada a partir da
alteração dos termos (numerador ou denominador) da primeira fração a partir de
somas e multiplicações por números naturais.
\end{enumerate}

\tcblower

%Orientações e sugestões
\begin{itemize}
\item       Recomenda-se que, nesta atividade, os alunos trabalhem
individualmente ou em duplas. No entanto, é fundamental que os alunos sejam
estimulados a explicar o raciocínio realizado.
\item       Enquanto que esta atividade usa a fração       $\dfrac{4}{7}$
como referência, a discussão da atividade com os alunos pode incluir a questão
se as conclusões obtidas para       $\dfrac{4}{7}$       mudam se a fração de
referência mudar. Neste contexto, o item (D) é especialmente interessante pois,
neste caso, a conclusão (se a fração ficará menor, maior ou igual a fração
original) de fato dependerá se a fração original é maior, menor ou igual a
$1$.
\end{itemize}
\end{goals}

\bigskip
\begin{center}
{\large \scshape Atividade}
\end{center}
\fi

Responda às seguintes questões:

\begin{enumerate} %d
\item     A fração determinada pela adição de 1 ao numerador da fração     $\dfrac{4}{7}$ é maior, menor ou igual a     $\dfrac{4}{7}$? Explique como chegou a essa conclusão.
\item     A fração determinada pela adição de 1 ao denominador da fração     $\dfrac{4}{7}$ é maior, menor ou igual a     $\dfrac{4}{7}$? Explique como chegou a essa conclusão.
\item     A fração determinada pela subtração de 1 ao denominador da fração     $\dfrac{4}{7}$ é maior, menor ou igual a     $\dfrac{4}{7}$? Explique como chegou a essa conclusão.
\item     A fração determinada pela adição de 2 ao numerador e ao denominador da fração     $\dfrac{4}{7}$ é maior, menor ou igual a     $\dfrac{4}{7}$? Explique como chegou a essa conclusão.
\item     A fração determinada pela multiplicação por 2 do numerador e do denominador da fração     $\dfrac{4}{7}$ é maior, menor ou igual a     $\dfrac{4}{7}$? Explique como chegou a essa conclusão.
\item     A fração determinada pela adição de 1 ao numerador e subtração de 1 ao denominador da fração     $\dfrac{4}{7}$ é maior, menor ou igual a     $\dfrac{4}{7}$? Explique como chegou a essa conclusão.
\end{enumerate} %d

\ifdefined\prof
\begin{solucao}

\begin{enumerate} %s
\item       A fração determinada pela adição de 1 ao numerador da fração
$\dfrac{4}{7}$       é a fração       $\dfrac{5}{7}$       que é maior do que
$\dfrac{4}{7}$, pois em cinco sétimos temos um sétimo a mais do que em quatro
sétimos.

\item       A fração determinada pela adição de 1 ao denominador da fração
$\dfrac{4}{7}$       é a fração       $\dfrac{4}{8}$       que é menor do que
$\dfrac{4}{7}$, pois como um oitavo é menor do um sétimo, quatro oitavos
também será menor do que quatro sétimos.
\item       A fração determinada pela subtração de 1 ao denominador da
fração       $\dfrac{4}{7}$       é a fração       $\dfrac{3}{7}$       que é
menor do que       $\dfrac{4}{7}$, pois em três sétimos temos um sétimo a menos
do que em quatro sétimos.
\item       A fração determinada pela adição de 2 ao numerador e ao
denominador da fração       $\dfrac{4}{7}$       é a fração       $\dfrac{6}{9}$
que é maior do que       $\dfrac{4}{7}$, pois       $\dfrac{6}{9} =
\dfrac{2}{3} = \dfrac{7 \times 2}{7 \times 3} = \dfrac{14}{21}$,       $\dfrac{4}{7}
= \dfrac{3 \times 4}{3 \times 7} = \dfrac{12}{21}$       e       $14 > 12$.
\item       A fração determinada pela multiplicação por 2 do numerador e do
denominador da fração       $\dfrac{4}{7}$       é a fração       $\dfrac{8}{14}$
que é igual a       $\dfrac{4}{7}$.
\item       A fração determinada pela adição de 1 ao numerador e subtração
de 1 ao denominador da fração       $\dfrac{4}{7}$       é a fração
$\dfrac{5}{6}$       que é maior do que       $\dfrac{4}{7}$, pois
$\dfrac{5}{6} > \dfrac{4}{6} > \dfrac{4}{7}$.
\end{enumerate} %s

\end{solucao}
\fi

\end{document}