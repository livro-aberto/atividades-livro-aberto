
\documentclass[10 pt,usenames,dvipsnames, oneside]{article}
\usepackage{../../modelo-fracoes}
\graphicspath{{../../../Figuras/licao04/}}


\begin{document}

\begin{center}
  \begin{minipage}[l]{3cm}
\includegraphics[width=2cm]{../../../Figuras/logo}       
\end{minipage}\hfill
\begin{minipage}[r]{.8\textwidth}
 {\Large \scshape Atividade: Denominadores iguais para decidir a ordem}  
\end{minipage}
\end{center}
\vspace{.2cm}

\ifdefined\prof
%Caixa do Para o Professor
\begin{goals}
%Objetivos específicos
\begin{enumerate}
\item       Comparar frações por meio de igualdade de frações.
\end{enumerate}

\tcblower

%Orientações e sugestões
\begin{itemize}
    \item       Esta é uma atividade que pode ser desenvolvida individualmente.
Contudo, é fundamental que os alunos sejam estimulados a explicar o raciocínio
realizado.
    \item       A discussão da atividade pode incluir o uso de outras
estratégias, que não a igualdade de frações, para se estabelecer a comparação
das frações apresentadas.
\end{itemize}
\end{goals}

\bigskip
\begin{center}
{\large \scshape Atividade}
\end{center}
\fi

Para cada par de frações na mesma linha da tabela a seguir, determine frações de mesmo denominador que sejam iguais a cada uma das frações dadas. Em seguida, compare essas frações, preenchendo a coluna em branco, com um dos símbolos ``$>$'', ``$<$'' ou ``$=$'', de forma que, em cada linha da tabela, a comparação estabelecida seja verdadeira. Vamos fazer o Item a) juntos: observe que

$$\dfrac{5}{6} = \dfrac{5 \times 5}{5 \times 6} = \dfrac{25}{30} \text{ e } \dfrac{4}{5} = \dfrac{6 \times 4}{6 \times 5} = \dfrac{24}{30}.$$

Como $25 > 24$, segue-se que $\dfrac{25}{30} > \dfrac{24}{30}$. Portanto, preenchemos a coluna em branco da primeira linha com $>$.


\begin{center}
\begin{longtable}{|	m{.1\textwidth}|m{.25\textwidth}|m{.25\textwidth}|m{.25\textwidth}|}
\hline
Item &  Fração &  $>$, $<$ ou $=$ &  Fração \\
\hline \hline
a) & \parbox[b][1.2cm][c]{3cm}{ $\dfrac{5}{6} = \dfrac{25}{30}$ } &   $>$&  $\dfrac{24}{30} = \dfrac{4}{5}$ \\
\hline
b) & \parbox[b][1.2cm][c]{3cm}{ $\dfrac{3}{4} = \dfrac{\square}{\square}$} &   &  $\dfrac{\square}{\square} = \dfrac{2}{3}$ \\
\hline
c) &  \parbox[b][1.2cm][c]{3cm}{$\dfrac{2}{10} = \dfrac{\square}{\square}$} &   &  $\dfrac{\square}{\square} = \dfrac{3}{15}$ \\
\hline
d) & \parbox[b][1.2cm][c]{3cm}{ $\dfrac{1}{4} = \dfrac{\square}{\square}$} &   &  $\dfrac{\square}{\square} = \dfrac{6}{25}$ \\
\hline
e) & \parbox[b][1.2cm][c]{3cm}{ $\dfrac{22}{7} = \dfrac{\square}{\square}$} &  &  $\dfrac{\square}{\square} = \dfrac{31}{10}$ \\
\hline
f) & \parbox[b][1.2cm][c]{3cm}{ $\dfrac{22}{33} = \dfrac{\square}{\square}$} &   &  $\dfrac{\square}{\square} = \dfrac{24}{36}$ \\
\hline
g) & \parbox[b][1.2cm][c]{3cm}{ $\dfrac{5}{10} = \dfrac{\square}{\square}$} &   &  $\dfrac{\square}{\square} = \dfrac{50}{100}$ \\
\hline
h) & \parbox[b][1.2cm][c]{3cm}{ $\dfrac{7}{5} = \dfrac{\square}{\square}$} &  &  $\dfrac{\square}{\square} = \dfrac{17}{12}$ \\
\hline
i) & \parbox[b][1.2cm][c]{3cm}{ $\dfrac{7}{12} = \dfrac{\square}{\square}$} &  &  $\dfrac{\square}{\square} = \dfrac{9}{20}$ \\
\hline
\end{longtable}
\end{center}

\ifdefined\prof
\begin{solucao}

\renewcommand\arraystretch{2}
\noindent
    \begin{tabular}{lrcl}

       item &  Fração &  ``$>$'', ``$<$'' ou ``$=$'' &  Fração \\
      \hline
       a) &  $\dfrac{5}{6} = \dfrac{25}{30}$ &   $>$  &  $\dfrac{24}{30} =
\dfrac{4}{5}$ \\

       b) &  $\dfrac{3}{4} = \dfrac{9}{12}$ &   $>$  &  $\dfrac{8}{12} =
\dfrac{2}{3}$ \\

       c) &  $\dfrac{2}{10} = \dfrac{1}{5}$ &   $=$  &  $\dfrac{1}{5} =
\dfrac{3}{15}$ \\

       d) &  $\dfrac{6}{25} = \dfrac{24}{100}$ &   $<$  &  $\dfrac{25}{100} =
\dfrac{1}{4}$ \\

       e) &  $\dfrac{22}{7} = \dfrac{220}{70}$ &   $>$  &  $\dfrac{217}{70} =
\dfrac{31}{10}$ \\

       f) &  $\dfrac{22}{33} = \dfrac{2}{3}$ &   $=$  &  $\dfrac{2}{3} =
\dfrac{24}{36}$ \\

       g) &  $\dfrac{5}{10} = \dfrac{50}{100}$ &   $=$  &  $\dfrac{50}{100} =
\dfrac{50}{100}$ \\

       h) &  $\dfrac{7}{5} = \dfrac{84}{60}$ &   $<$  &  $\dfrac{85}{60} =
\dfrac{17}{12}$ \\

       i) &  $\dfrac{12}{6} = \dfrac{2}{1}$ &   $<$  &  $\dfrac{3}{1} =
\dfrac{9}{3}$ \\

    \end{tabular}

\end{solucao}
\fi

\end{document}