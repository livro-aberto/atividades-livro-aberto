
\documentclass[10 pt,usenames,dvipsnames, oneside]{article}
\usepackage{../../modelo-fracoes}
\graphicspath{{../../../Figuras/licao04/}}


\begin{document}

\begin{center}
  \begin{minipage}[l]{3cm}
\includegraphics[width=2cm]{../../../Figuras/logo}       
\end{minipage}\hfill
\begin{minipage}[r]{.8\textwidth}
 {\Large \scshape Atividade: Quadradinhos de numeradores}  
\end{minipage}
\end{center}
\vspace{.2cm}

\ifdefined\prof
%Caixa do Para o Professor
\begin{goals}
%Objetivos específicos
\begin{enumerate}
\item       Reconhecer que, para cada       $0 \leq i \leq 3$, as frações
$\dfrac{i}{3}$,       $\dfrac{2 \times i}{2 \times 3 }$,       $\dfrac{3 \times
i}{3 \times 3}$       e       $\dfrac{4 \times i}{4 \times 3}$       são iguais a
partir da observação das representações destas frações na reta numérica.\end{enumerate}

\tcblower

%Orientações e sugestões
\begin{itemize}
\item       Recomenda-se que, nesta atividade, os alunos trabalhem
individualmente ou em duplas. No entanto, é fundamental que os alunos sejam
estimulados a explicar o raciocínio realizado.
\item       Nas retas numéricas apresentadas, as origens estão alinhadas e
as unidades correspondem a segmentos unitários congruentes, o que garante que
uma fração associada a um determinado ponto em uma reta seja a mesma fração nos
pontos correspondentes nas demais retas.
\item       Caso seus alunos não percebam, aponte para o fato de que as
segunda, terceira e quarta retas numéricas são obtidas por meio de subdivisões
dos terços da primeira reta numérica em duas, três e quatro partes iguais,
respectivamente. Para resolver o item c) desta atividade, se faz necessário
dividir cada terço em cinco partes iguais.
\item       É importante, ao final da atividade, observar para os alunos
que, nesta atividade, cada ponto marcado na reta numérica está sendo descrito
por frações com numeradores e denominadores diferentes (isto é, por frações
equivalentes) mas que, não obstante, por corresponderem ao mesmo ponto da reta
numérica, estas frações são iguais.
\end{itemize}
\end{goals}

\bigskip
\begin{center}
{\large \scshape Atividade}
\end{center}
\fi

\begin{enumerate} %d
\item     Preencha os quadradinhos     $\square$ com numeradores adequados de modo que cada fração corresponda a sua respectiva marca em cada reta numérica.    \mbox{} \newline

\begin{center}
\begin{tikzpicture}[x=45mm,y=45mm]
\draw (0,0) -- (3,0);
\foreach \x in {0,...,3}{
\draw (\x,-3pt) -- (\x, 3pt);
\node[above] at (\x,3pt) {\x};}

\foreach \x in {0,.333,...,3}{
\draw (\x,-2pt) -- (\x, 2pt);
\node[below] at (\x,-2 pt) {$\dfrac{\square}{3}$};
\draw[dotted] (\x,-30pt) -- (\x, -60pt);
}

\fill[common] (1+1/3,0) circle (2pt);

\begin{scope}[shift={(0,-75pt)}]
\draw (0,0) -- (3,0);
\foreach \x in {0,...,3}{
\draw (\x,-3pt) -- (\x, 3pt);
\node[above] at (\x,3pt) {\x};}

\foreach \x in {0,.333,...,3}{
\draw (\x,-2pt) -- (\x, 2pt);
\node[below] at (\x,-2 pt) {$\dfrac{\square}{6}$};
\draw[dotted] (\x,-30pt) -- (\x, -60pt);
}

\foreach \x in {0,.333,...,6}{
\draw (\x/2,-2pt) -- (\x/2, 2pt);}


\fill[common] (1+1/3,0) circle (2pt);

\end{scope}


\begin{scope}[shift={(0,-150pt)}]
\draw (0,0) -- (3,0);
\foreach \x in {0,...,3}{
\draw (\x,-3pt) -- (\x, 3pt);
\node[above] at (\x,3pt) {\x};}

\foreach \x in {0,.333,...,3}{
\draw (\x,-2pt) -- (\x, 2pt);
\node[below] at (\x,-2 pt) {$\dfrac{\square}{9}$};
\draw[dotted] (\x,-30pt) -- (\x, -60pt);
}

\foreach \x in {0,.333,...,9}{
\draw (\x/3,-2pt) -- (\x/3, 2pt);}



\fill[common] (1+1/3,0) circle (2pt);

\end{scope}

\begin{scope}[shift={(0,-225pt)}]
\draw (0,0) -- (3,0);
\foreach \x in {0,...,3}{
\draw (\x,-3pt) -- (\x, 3pt);
\node[above] at (\x,3pt) {\x};}

\foreach \x in {0,.333,...,3}{
\draw (\x,-2pt) -- (\x, 2pt);
\node[below] at (\x,-2 pt) {$\dfrac{\square}{12}$};
%\draw[dotted] (\x,-30pt) -- (\x, -60pt);
}

\foreach \x in {0,.333,...,12}{
\draw (\x/4,-2pt) -- (\x/4, 2pt);}

\fill[common] (1+1/3,0) circle (2pt);

\end{scope}
\end{tikzpicture}
\end{center}
\item     Escreva quatro frações com numeradores diferentes (consequentemente com denominadores também diferentes) que correspondam ao ponto azul em destaque na figura.
\item     Determine uma fração de denominador     $15$ que corresponda ao ponto azul em destaque. Justifique sua resposta usando uma reta numérica!
\end{enumerate} %d

\ifdefined\prof
\clearpage
\begin{solucao}

\noindent a)

 \noindent 

 \begin{center}\begin{tikzpicture}[x=45mm,y=45mm]
  \draw (0,0) -- (3,0);
  \foreach \x in {0,...,3}{ \draw (\x,-3pt) -- (\x, 3pt);  \node[above] at (\x,3pt) {\x}; }

  \foreach \x in {0,1,...,9}{
  \draw (\x/3,-2pt) -- (\x/3, 2pt);
  \node[below] at (\x/3,-2 pt) {$\dfrac{\x}{3}$};
  \draw[dotted] (\x/3,-35pt) -- (\x/3, -60pt);
  }

  \fill[common] (1+1/3,0) circle (2pt);

  \begin{scope}[shift={(0,-80pt)}]
  \draw (0,0) -- (3,0);
  \foreach \x in {0,...,3}{
  \draw (\x,-3pt) -- (\x, 3pt);
  \node[above] at (\x,3pt) {\x};
  }

  \foreach \x in {0,2,...,18}{
  \draw (\x/6,-2pt) -- (\x/6, 2pt);
  \node[below] at (\x/6,-2 pt) { $\dfrac{\x}{6}$ };
  \draw[dotted] (\x/6,-35pt) -- (\x/6, -60pt);
  }

  \foreach \x in {0,.333,...,6}{
  \draw (\x/2,-2pt) -- (\x/2, 2pt);}


  \fill[common] (1+1/3,0) circle (2pt);

  \end{scope}


  \begin{scope}[shift={(0,-160pt)}]
  \draw (0,0) -- (3,0);
  \foreach \x in {0,...,3}{
  \draw (\x,-3pt) -- (\x, 3pt);
  \node[above] at (\x,3pt) {\x};}

  \foreach \x in {0,3,...,27}{
  \draw (\x/9,-2pt) -- (\x/9, 2pt);
  \node[below] at (\x/9,-2 pt) {$\dfrac{\x}{9}$};
  \draw[dotted] (\x/9,-35pt) -- (\x/9, -60pt);
  }

  \foreach \x in {0,.333,...,9}{
  \draw (\x/3,-2pt) -- (\x/3, 2pt);}



  \fill[common] (1+1/3,0) circle (2pt);

  \end{scope}

  \begin{scope}[shift={(0,-240pt)}]
  \draw (0,0) -- (3,0);
  \foreach \x in {0,...,3}{
  \draw (\x,-3pt) -- (\x, 3pt);
  \node[above] at (\x,3pt) {\x};}

  \foreach \x in {0,4,...,36}{
  \draw (\x/12,-2pt) -- (\x/12, 2pt);
  \node[below] at (\x/12,-2 pt) {$\dfrac{\x}{12}$};
  %\draw[dotted] (\x,-30pt) -- (\x, -60pt);
  }

  \foreach \x in {0,.333,...,12}{
  \draw (\x/4,-2pt) -- (\x/4, 2pt);}

  \fill[common] (1+1/3,0) circle (2pt);

  \end{scope}
 \end{tikzpicture}

 \end{center}

\noindent b) $\dfrac{4}{3} = \dfrac{8}{6} = \dfrac{12}{9} = \dfrac{16}{12}$.

\noindent c) No item b) foi estabelecido que o ponto azul corresponde a fração
$\dfrac{4}{3}$       pois, ao se justapor 4 segmentos que são       $\dfrac{1}{3}$
      do segmento unitário (que está, aqui, servindo como unidade) a partir da
origem       $0$, este ponto é a outra extremidade desta justaposição. Agora, ao
se subdividir estes 4 segmentos que são       $\dfrac{1}{3}$       do segmento
unitário em 5 partes iguais, obtêm-se 20 segmentos justapostos que são
$\dfrac{1}{15}$       do segmento unitário. Sendo o ponto azul extremo desta
justaposição, segue-se que ele corresponde a fração       $\dfrac{20}{15}$.



\noindent 
\begin{center}
\begin{tikzpicture}[x=45mm,y=45mm]
  \draw (0,0) -- (3,0);
  \foreach \x in {0,...,3}{ \draw (\x,-3pt) -- (\x, 3pt);  \node[above] at (\x,3pt) {\x}; }

  \foreach \x in {0,1,...,9}{
  \draw (\x/3,-2pt) -- (\x/3, 2pt);
  \node[below] at (\x/3,-2 pt) {$\dfrac{\x}{3}$};
  \draw[dotted] (\x/3,-35pt) -- (\x/3, -60pt);
  }

  \fill[common] (1+1/3,0) circle (2pt);

  \begin{scope}[shift={(0,-80pt)}]
  \draw (0,0) -- (3,0);
  \foreach \x in {0,...,3}{
  \draw (\x,-3pt) -- (\x, 3pt);
  \node[above] at (\x,3pt) {\x};
  }

  \foreach \x in {0,5,...,45}{
  \node[below] at (\x/15,-2 pt) { $\dfrac{\x}{15}$ };
  %\draw[dotted] (\x/15,-35pt) -- (\x/15, -60pt);
  }

  \foreach \x in {0,1,...,45}{
  \draw (\x/15,-2pt) -- (\x/15, 2pt);}


  \fill[common] (1+1/3,0) circle (2pt);

  \end{scope}
\end{tikzpicture}
\end{center}

\end{solucao}
\fi

\end{document}