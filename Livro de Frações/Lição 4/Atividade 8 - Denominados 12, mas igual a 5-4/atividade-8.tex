
\documentclass[10 pt,usenames,dvipsnames, oneside]{article}
\usepackage{../../modelo-fracoes}
\graphicspath{{../../../Figuras/licao04/}}


\begin{document}

\begin{center}
  \begin{minipage}[l]{3cm}
\includegraphics[width=2cm]{../../../Figuras/logo}       
\end{minipage}\hfill
\begin{minipage}[r]{.8\textwidth}
 {\Large \scshape Atividade: Denominador 12, mas igual a $\frac{5}{4}$}  
\end{minipage}
\end{center}
\vspace{.2cm}

\ifdefined\prof
%Caixa do Para o Professor
\begin{goals}
%Objetivos específicos
\begin{enumerate}
\item       Determinar uma fração igual a uma dada fração com denominador
especificado a partir da observação das representações destas frações em
diversos modelos de frações, incluindo a reta numérica.
\end{enumerate}

\tcblower

%Orientações e sugestões
\begin{itemize}
\item       Recomenda-se que esta atividade seja desenvolvida em grupos de 3
alunos (cada aluno do grupo poderá usar um modelo diferente para obter a fração
solicitada).
    \item       É importante, ao final da atividade, observar para os alunos que
uma mesma parte em cada modelo de área e um mesmo ponto na reta numérica estão
sendo descritos por frações com numeradores e denominadores diferentes (isto é,
por frações equivalentes) mas que, não obstante, estas frações são iguais por
expressarem uma mesma quantidade ou por serem representadas por um mesmo ponto
na reta numérica.
\end{itemize}
\end{goals}

\bigskip
\begin{center}
{\large \scshape Atividade}
\end{center}
\fi

O objetivo desta atividade é determinar uma fração de denominador $12$ que seja igual à fração $\frac{5}{4}$.

\begin{enumerate}
  \item     Tomando um círculo como unidade:
\begin{enumerate}
  \item Faça um desenho que represente     $\frac{5}{4}$ da unidade.
  \item Usando o desenho feito, represente uma fração de denominador     $12$ que seja igual a     $\frac{5}{4}$.
\end{enumerate}
  \item     Tomando um quadrado como unidade:
  \begin{enumerate}
  \item Faça um desenho que represente     $\frac{5}{4}$ da unidade.
  \item Usando o desenho feito, represente uma fração de denominador     $12$ que seja igual a     $\frac{5}{4}$.
  \end{enumerate}
\item     Desenhe uma reta numérica e, em seguida, marque os números     $0$,     $1$ e     $\frac{5}{4}$. A partir deste desenho, represente uma fração de denominador     $12$ que seja igual a     $\frac{5}{4}$.
\end{enumerate} %s

\ifdefined\prof
\begin{solucao}

\begin{enumerate} %s
    \item       Tomando um círculo como unidade, o dividimos em       $4$
partes iguais e tomamos       $5$       cópias de uma parte para obter
$\frac{5}{4}$       da unidade. Dividindo cada uma das       $5$       cópias em
      $3$       partes iguais, obtemos então       $15$       cópias de
$\frac{1}{12}$       da unidade. Portanto,       $\frac{5}{4} = \frac{15}{12}$.



\noindent 
\begin{center}\begin{tikzpicture}[x=1mm,y=1mm, scale=1.45]
 \draw[fill=common, fill opacity=.3] (0,0) circle (5);
 \node at (0,7) {Unidade};


 \begin{scope}[xshift=45]
 \draw[->] (-10,0) -- (-6,0);
 \node[text width=23 mm] at (-8,-11) {Divide-se a unidade em 4 partes iguais.};


 \draw[fill=common, fill opacity=.3] (0,0) circle (5);
 \draw (90:5) -- (-90:5);
 \draw (0:5) -- (180:5);

 \draw[->] (6,0) -- (10,0);
 \node[text width=22 mm] at (10,-11) {Uma parte corresponde a um quarto.};

 \draw[fill=common, fill opacity=.3, xshift=32, yshift=-5] (0:5) arc (0:90:5) -- (0,0)--cycle;

 \end{scope}

 \begin{scope}[ yshift=-26mm]

 \node[text width=30 mm] at (3,-12) {Junta-se 5 cópias de uma parte para obter cinco quartos.};

 \draw[fill=common, fill opacity=.3] (0,0) circle (5);
 \draw (90:5) -- (-90:5);
 \draw (0:5) -- (180:5);

 \draw[fill=common, fill opacity=.3, xshift=6mm, yshift=-5] (0:5) arc (0:90:5) -- (0,0)--cycle;


 \begin{scope}[xshift=65]
  \draw[->] (-10,0) -- (-6,0);
 \node[text width=33 mm] at (4,-14) {Divide-se cada cópia em 3 partes iguais obtendo 15 cópias de um doze avos.};

 \draw[fill=common, fill opacity=.3] (0,0) circle (5);
 \foreach \x in {0,30,...,150}{ \draw (\x:5) -- (\x:-5);}

 \draw[fill=common, fill opacity=.3, xshift=6mm, yshift=-5] (0:5) arc (0:90:5) -- (0,0)--cycle;
 \foreach \x in {30,60}{ \draw[xshift=6mm, yshift=-5] (\x:5) -- (0,0);}
 \end{scope}
 \end{scope}

 \end{tikzpicture}
 \end{center}

    \item       Tomando um quadrado como unidade, o dividimos em       $4$
partes iguais e tomamos       $5$       cópias de uma parte para obter
$\frac{5}{4}$       da unidade. Dividindo cada uma das       $5$       cópias em
      $3$       partes iguais, obtemos então       $15$       cópias de
$\frac{1}{12}$       da unidade. Portanto,       $\frac{5}{4} = \frac{15}{12}$.

\noindent 
\begin{center}
\begin{tikzpicture}[x=1mm,y=1mm,scale=1.45]
           \draw[fill=common, fill opacity=.3] (0,0) rectangle (10,10);
	   \node at (5,12) {Unidade};


 \begin{scope}[xshift=45]
 \draw[->] (-5,5) -- (-1,5);
 \node[text width=23 mm] at (-4,-8) {Divide-se a unidade em 4 partes iguais.};

 \draw[fill=common, fill opacity=.3] (0,0) rectangle (10,10);
 \draw (0,5) -- (10,5);
 \draw (5,0) -- (5,10);

 \draw[->] (11,5) -- (15,5);
 \node[text width=22 mm] at (14,-8) {Uma parte corresponde a um quarto.};

 \draw[fill=common, fill opacity=.3, xshift=16mm, yshift=2.5mm] (0,0) rectangle (5,5);

 \end{scope}

 \begin{scope}[ yshift=-27mm]

 \node[text width=30 mm] at (9,-8) {Junta-se 5 cópias de uma parte para obter cinco quartos.};

 \draw[fill=common, fill opacity=.3] (0,0) rectangle (10,10);
 \draw (0,5) -- (10,5);
 \draw (5,0) -- (5,10);

 \draw[fill=common, fill opacity=.3, xshift=11mm, yshift=5mm] (0,0) rectangle (5,5);

 \draw[->] (17,5) -- (21,5);
 \begin{scope}[xshift=62]

  \node[text width=32 mm] at (10,-9) {Divide-se cada cópia em 3 partes iguais obtendo 15 cópias de um doze avos.};

 \draw[fill=common, fill opacity=.3] (0,0) rectangle (10,10);
 \foreach \x in {10,20,...,50} \draw (\x/6,0) -- (\x/6,10);
 \draw (0,5) -- (10,5);

 \draw[fill=common, fill opacity=.3, xshift=11mm, yshift=5mm] (0,0) rectangle (5,5);
 \foreach \x in {10,20} \draw[xshift=11mm, yshift=5mm] (\x/6,0) -- (\x/6,5);
 \end{scope}
 \end{scope}

          \end{tikzpicture}
          \end{center}


\end{enumerate} %s
% \end{resposta*}
%
% \pagebreak
% \end{multicols}
\begin{enumerate}

    \item[c)] Após marcar os números       $0$       e       $1$       na reta
numérica, dividimos o segmento unitário (aquele de extremidades em       $0$
  e       $1$) em       $4$       partes iguais. Cada parte é um segmento que
corresponde a       $\frac{1}{4}$       da unidade. Ao se justapor       $5$
  segmentos que são       $\frac{1}{4}$       da unidade a partir da origem 0, a
fração       $\frac{5}{4}$       corresponde ao ponto que é a outra extremidade
desta justaposição. Agora, ao se subdividir estes 5 segmentos que são
$\frac{1}{4}$       da unidade em 5 partes iguais, obtêm-se       $15$
segmentos justapostos que são       $\frac{1}{12}$       da unidade. O ponto que
corresponde a       $\frac{5}{4}$       é ainda extremo desta justaposição e,
portanto, que ele corresponde também a fração       $\frac{15}{12}$.



\begin{center}

 \begin{tikzpicture}[x=21mm,y=21mm]
  \draw (0,0) -- (3,0);
  \foreach \x in {0,...,3}{ \draw (\x,-3pt) -- (\x, 3pt);  \node[above] at (\x,3pt) {\x}; }

  \foreach \x in {0,1,...,5}{
  \draw (\x/4,-2pt) -- (\x/4, 2pt);
  \node[below] at (\x/4,-2 pt) {$\frac{\x}{4}$};}
  \draw[dotted] (5/4,-.3) -- (5/4, -.6);


  \fill[common] (5/4,0) circle (2pt);

  \begin{scope}[shift={(0,-.7)}]
  \draw (0,0) -- (3,0);
  \foreach \x in {0,...,3}{
  \draw (\x,-3pt) -- (\x, 3pt);
  \node[above] at (\x,3pt) {\x};
  }

  \foreach \x in {0,3,...,15}{
  \node[below] at (\x/12,-2 pt) { $\frac{\x}{12}$ };
  %\draw[dotted] (\x/15,-35pt) -- (\x/15, -60pt);
  }

  \foreach \x in {0,1,...,15}{
  \draw (\x/12,-2pt) -- (\x/12, 2pt);}


  \fill[common] (1+1/4,0) circle (2pt);

  \end{scope}
\end{tikzpicture}

\end{center}
\end{enumerate}

\end{solucao}
\fi

\end{document}