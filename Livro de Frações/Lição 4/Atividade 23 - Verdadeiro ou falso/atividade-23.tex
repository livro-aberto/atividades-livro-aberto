
\documentclass[10 pt,usenames,dvipsnames, oneside]{article}
\usepackage{../../modelo-fracoes}
\graphicspath{{../../../Figuras/licao04/}}


\begin{document}

\begin{center}
  \begin{minipage}[l]{3cm}
\includegraphics[width=2cm]{../../../Figuras/logo}       
\end{minipage}\hfill
\begin{minipage}[r]{.8\textwidth}
 {\Large \scshape Atividade: Verdadeiro ou falso?}  
\end{minipage}
\end{center}
\vspace{.2cm}

\ifdefined\prof
%Caixa do Para o Professor
\begin{goals}
%Objetivos específicos
\begin{enumerate}
\item     Estabelecer criticamente uma avaliação sobre a comparação entre
frações a partir da observação dos termos dessas fraçoes, incluindo a questão da
recíproca da seguinte propriedade:     ``se existe número natural $n$ tal que
$\dfrac{a}{b} = \dfrac{n \times c}{n \times d}$, então $\dfrac{a}{b} =
\dfrac{c}{d}$''    .
\end{enumerate}

\tcblower

%Orientações e sugestões
\begin{itemize}
\item     Recomenda-se que, nesta atividade, os alunos trabalhem
individualmente ou em duplas. No entanto, é fundamental que os alunos sejam
estimulados a explicar o raciocínio realizado.
  \item     Note que o item d) é falso porque estamos dando a liberdade de a
escolha envolver frações que não são irredutíveis nem unitárias, por isso
existem contraexemplos. Avalie a discussão sobre a veracidade da afirmação do
item d) quando acrescentamos a informação ``uma das frações é irredutível'' ou
``uma das frações é unitária''. Neste caso, as novas afirmações são verdadeiras,
e as justificativas para elas são generalizações de questões já propostas.
\end{itemize}
\end{goals}

\bigskip
\begin{center}
{\large \scshape Atividade}
\end{center}
\fi

Diga se cada uma das sentenças a seguir é verdadeira ou falsa. Explique a sua resposta com exemplos, desenhos ou palavras.
\begin{enumerate}
 \item  Se duas frações têm numeradores e denominadores diferentes, então elas representam quantidades diferentes.
 \item Se duas frações têm denominadores iguais, mas numeradores diferentes, então elas representam quantidades diferentes.
 \item Se duas frações têm numeradores iguais e maiores do que zero, mas denominadores diferentes, então elas representam quantidades diferentes.
 \item Se duas frações representam quantidades iguais, então o numerador e o denominador de uma são obtidos multiplicando-se o numerador e o denominador da outra por um mesmo número natural.
\end{enumerate}


\ifdefined\prof
\begin{solucao}

\begin{enumerate}
  \item A sentença é falsa. Por exemplo, as frações $\dfrac{1}{2}$ e
$\dfrac{3}{6}$ têm numeradores e denominadores diferentes, mas elas são iguais,
uma vez que $\dfrac{1}{2} = \dfrac{3 \times 1}{3 \times 2} = \dfrac{3}{6}$.
    \item A sentença é verdadeira: se duas frações têm denominadores iguais, é
maior a fração que tem o maior numerador e, em particular, elas são diferentes.
De fato: lembrando que o denominador de fração especifica o número de partes em
que a unidade foi dividida e o numerador especifica quantas cópias desta parte
foram tomadas, para um mesmo denominador, quanto maior o numerador, mais cópias
são tomadas e, portanto, maior é a quantidade representada pela fração.
    \item  A sentença é verdadeira: se duas frações têm numeradores iguais, é
maior a fração que tem o menor denominador e, em particular, elas são
diferentes. De fato: considerando que o numerador especifica o número de cópias
da unidade que está sendo dividida por um número de pessoas, número este
especificado pelo denominador da fração, para um mesmo numerador, quanto menor o
denominador, maior a porção que cada pessoa vai receber, quantidade esta
representada pela fração, pois o mesmo número de cópias da unidade está sendo
divivido por um número menor de pessoas.
    \item A sentença é falsa. Por exemplo, $\dfrac{2}{4}$ e $\dfrac{3}{6}$ são
frações iguais, pois $\dfrac{2}{4}$ é igual a $\dfrac{1}{2}$ e $\dfrac{3}{6}$
também é igual a $\dfrac{1}{2}$, mas não existe um número natural que
multiplicado por $2$ dê igual a $3$, bem como não existe número natural que
multiplicado por $3$ dê $2$.
\end{enumerate}

\end{solucao}
\fi

\end{document}