
\documentclass[10 pt,usenames,dvipsnames, oneside]{article}
\usepackage{../../modelo-fracoes}
\graphicspath{{../../../Figuras/licao04/}}


\begin{document}

\begin{center}
  \begin{minipage}[l]{3cm}
\includegraphics[width=2cm]{../../../Figuras/logo}       
\end{minipage}\hfill
\begin{minipage}[r]{.8\textwidth}
 {\Large \scshape Atividade: Ajude Ana a mostrar que Jorge está errado}  
\end{minipage}
\end{center}
\vspace{.2cm}

\ifdefined\prof
%Caixa do Para o Professor
\begin{goals}
%Objetivos específicos
\begin{enumerate}
\item       Perceber que mesmo se       $n < p$       e       $m < q$, pode
ocorrer que       $\dfrac{n}{m} \geq \dfrac{p}{q}$.
\end{enumerate}

\tcblower

%Orientações e sugestões
\begin{itemize}
\item       Esta é uma atividade que pode ser desenvolvida individualmente.
Contudo, é fundamental que os alunos sejam estimulados a explicar o raciocínio
realizado.
\item       O tipo de situação descrita na atividade é um equívoco comum
entre os alunos, isto é, eles equivocamente acham que se       $n < p$       e
$m < q$, então necessariamente       $\dfrac{n}{m} < \dfrac{p}{q}$.
\end{itemize}
\end{goals}

\bigskip
\begin{center}
{\large \scshape Atividade}
\end{center}
\fi

Jorge e Ana estão comparando as frações $\dfrac{2}{3}$ e $\dfrac{6}{10}$. Jorge afirma que
$\dfrac{2}{3} < \dfrac{6}{10}$ porque $2 < 6$ e $3 < 10$. Ana diz que $\dfrac{2}{3} > \dfrac{6}{10}$. Use desenhos, palavras ou apenas números para ajudar Ana a explicar a Jorge porque ele está errado.

\ifdefined\prof
\begin{solucao}

Tem-se que   $\dfrac{2}{3} = \dfrac{10 \times 2}{10 \times 3} = \dfrac{20}{30}$
e
$\dfrac{6}{10} = \dfrac{3 \times 6}{3 \times 10} = \dfrac{18}{30}$.
Como   $\dfrac{20}{30} > \dfrac{18}{30}$  , segue-se que   $\dfrac{2}{3} >
\dfrac{6}{10}$.
\end{solucao}
\fi

\end{document}