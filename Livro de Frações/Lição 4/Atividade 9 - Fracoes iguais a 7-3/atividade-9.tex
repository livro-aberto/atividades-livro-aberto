

\documentclass[10 pt,usenames,dvipsnames, oneside]{article}
\usepackage{../../modelo-fracoes}
\graphicspath{{../../../Figuras/licao04/}}


\begin{document}

\begin{center}
  \begin{minipage}[l]{3cm}
\includegraphics[width=2cm]{../../../Figuras/logo}       
\end{minipage}\hfill
\begin{minipage}[r]{.8\textwidth}
 {\Large \scshape Atividade: Frações iguais a $\frac{7}{3}$}  
\end{minipage}
\end{center}
\vspace{.2cm}

\ifdefined\prof
%Caixa do Para o Professor
\begin{goals}
%Objetivos específicos
\begin{enumerate}
\item       Determinar uma fração igual a uma dada fração irredutível com
denominador especificado.
\end{enumerate}

\tcblower

%Orientações e sugestões
\begin{itemize}
\item       Recomenda-se que, nesta atividade, os alunos trabalhem
individualmente ou em duplas. No entanto, é fundamental que os alunos sejam
estimulados a explicar o raciocínio realizado.
\item       Espera-se, principalmente nos Itens c e d, que os alunos
consigam obter a fração solicitada usando a propriedade que       $\dfrac{m
\times a}{m \times b}$       é equivalente a       $\dfrac{a}{b}$       e sem
recorrer a desenhos de modelos de área de frações. 
\end{itemize}
\end{goals}

\bigskip
\begin{center}
{\large \scshape Atividade}
\end{center}
\fi

Determine uma fração que seja igual a $\dfrac{7}{3}$ e que tenha denominador

\begin{tabular}{m{.2\textwidth}m{.2\textwidth}m{.2\textwidth}m{.2\textwidth}}
 a) $6$ & b) $21$ & c) $123$ & d) $210$
\end{tabular}

\ifdefined\prof
\begin{solucao}

\begin{enumerate}
\item       Como       $6 = 2 \times 3$, segue-se que       $\dfrac{7}{3} =
\dfrac{2 \times 7}{2 \times 3} = \dfrac{14}{6}$.
\item       Como       $21 = 7 \times 3$, segue-se que       $\dfrac{7}{3} =
\dfrac{7 \times 7}{7 \times 3} = \dfrac{49}{21}$.
\item       Como       $123 = 41 \times 3$, segue-se que       $\dfrac{7}{3}
= \dfrac{41 \times 7}{41 \times 3} = \dfrac{287}{123}$.
\item       Como       $210 = 70 \times 3$, segue-se que       $\dfrac{7}{3}
= \dfrac{70 \times 7}{70 \times 3} = \dfrac{490}{210}$.
\end{enumerate}

\end{solucao}
\fi

\end{document}