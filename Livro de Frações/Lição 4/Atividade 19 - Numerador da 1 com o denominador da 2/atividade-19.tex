
\documentclass[10 pt,usenames,dvipsnames, oneside]{article}
\usepackage{../../modelo-fracoes}
\graphicspath{{../../../Figuras/licao04/}}


\begin{document}

\begin{center}
  \begin{minipage}[l]{3cm}
\includegraphics[width=2cm]{../../../Figuras/logo}       
\end{minipage}\hfill
\begin{minipage}[r]{.8\textwidth}
 {\Large \scshape Atividade: Numerador da 1\textsuperscript{a} com o denominador da 2\textsuperscript{a}}  
\end{minipage}
\end{center}
\vspace{.2cm}

\ifdefined\prof
%Caixa do Para o Professor
\begin{goals}
%Objetivos específicos
\begin{enumerate}
    \item       Verificar que se       $a \cdot d = b \cdot c$, com       $b
\not = 0$       e       $d \not = 0$, então as frações       $\dfrac{a}{b}$
e       $\dfrac{c}{d}$       são iguais (equivalentes).
\end{enumerate}

\tcblower

%Orientações e sugestões
\begin{itemize}
    \item       Esta é uma atividade que pode ser desenvolvida individualmente.
Contudo, é fundamental que os alunos sejam estimulados a explicar o raciocínio
realizado.
    \item       Note que para as frações usadas no exemplo e no item a), os
numeradores e denominadores de uma fração não são múltiplos inteiros dos
numeradores e denominadores da outra fração.
\end{itemize}
\end{goals}

\bigskip
\begin{center}
{\large \scshape Atividade}
\end{center}
\fi

Dadas duas frações, se o produto do numerador da primeira fração pelo denominador da segunda fração for igual ao produto do denominador da primeira fração pelo numerador da segunda fração, então as frações são iguais.

Vamos ver um exemplo: para as frações $\dfrac{14}{6}$ e $\dfrac{21}{9}$, note que  $14 \times 9 = 126 = 6 \times 21$. Vamos agora usar este fato de que $14 \times 9 = 6 \times 21$
para concluir que $\dfrac{14}{6} = \dfrac{21}{9}$:

$$\dfrac{14}{6} = \dfrac{9 \times 14}{9 \times 6} = \dfrac{14 \times 9}{9 \times 6} = \dfrac{6 \times 21}{9 \times 6} = \dfrac{6 \times 21}{6 \times 9} = \dfrac{21}{9}.$$

\begin{enumerate}
 \item Use o procedimento do exemplo para mostrar que $\dfrac{2}{8} = \dfrac{5}{20}$.
 \item Verdadeirou ou falso? Se duas frações são iguais, então o produto do numerador da primeira fração pelo denominador da segunda fração é igual ao produto do denominador da primeira fração pelo numerador da segunda fração. Justifique sua resposta.
\end{enumerate}

\ifdefined\prof
\begin{solucao}

\begin{enumerate}
\item       Para as frações       $\dfrac{2}{8}$       e
$\dfrac{5}{20}$, tem-se que        $2 \times 20 = 40 = 8 \times 5$. Agora,
$$\dfrac{2}{8} = \dfrac{20 \times 2}{20 \times 8} = \dfrac{2 \times 20}{20 \times
8} = \dfrac{8 \times 5}{20 \times 8} = \dfrac{8 \times 5}{8 \times 20} =
\dfrac{5}{20}.$$
\item       A afirmação é verdadeira.

Ao se multiplicar o numerador e o denominador da primeira fração pelo
denominador da segunda fração obtém-se uma fração de igual valor cujo numerador
é o produto do numerador da primeira fração pelo denominador da segunda fração e
cujo denominador é o produto do numerador da primeira fração pelo denominador da
segunda fração.

Do mesmo modo, ao se multiplicar o numerador e o denominador da segunda fração
pelo denominador da primeira fração obtém-se uma fração de igual valor cujo
numerador é igual ao denominador da primeira fração multiplicado pelo numerador
da segunda fração e cujo denominador é o produto do denominadora da primeira
fração pelo denominador da segunda fração.

Como as frações iniciais são iguais, estas novas frações também são iguais e
têm o mesmo denominador. Portanto, seus numeradores devem ser iguais, isto é, o
produto do numerador da primeira fração pelo denominador da segunda fração é
igual ao produto do denominador da primeira fração pelo numerador da segunda
fração.
\end{enumerate}

\end{solucao}
\fi

\end{document}