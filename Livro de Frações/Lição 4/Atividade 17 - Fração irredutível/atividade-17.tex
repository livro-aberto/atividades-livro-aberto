
\documentclass[10 pt,usenames,dvipsnames, oneside]{article}
\usepackage{../../modelo-fracoes}
\graphicspath{{../../../Figuras/licao04/}}


\begin{document}

\begin{center}
  \begin{minipage}[l]{3cm}
\includegraphics[width=2cm]{../../../Figuras/logo}       
\end{minipage}\hfill
\begin{minipage}[r]{.8\textwidth}
 {\Large \scshape Atividade: Fração irredutível}  
\end{minipage}
\end{center}
\vspace{.2cm}

\ifdefined\prof
%Caixa do Para o Professor
\begin{goals}
%Objetivos específicos
\begin{enumerate}
\item       Simplificar frações de modo a obter uma fração igual
irredutível.
\end{enumerate}

\tcblower

%Orientações e sugestões
\begin{itemize}
\item       Recomenda-se que, nesta atividade, os alunos trabalhem
individualmente ou em duplas. No entanto, é fundamental que os alunos sejam
estimulados a explicar o raciocínio realizado.
\item       Um pré-requisito desta atividade é o conceito de máximo divisor
comum. Assim, avalie a necessidade de uma revisão deste conceito com seus
alunos. Os alunos devem perceber que se dois números são divididos pelo maior
divior comum entre eles, os dois novos números obtidos são agora primos entre
si, isto é, o máximo divisor comum entre eles é 1. Este fato vai apoiar o
``assim''       das respostas.
\item       A discussão desta atividade pode incluir o uso de materiais
concretos na linha da proposta da Atividade \textit{Oito panquecas para 24 amigos}, isto é, relacionar
frações equivalentes com a minimização de cortes em uma equipartição.
\end{itemize}
\end{goals}

\bigskip
\begin{center}
{\large \scshape Atividade}
\end{center}
\fi

Dizemos que uma fração é {\bf irredutível} se o máximo divisor comum entre o seu numerador e o seu denominador é igual a $1$. Para cada fração indicada a seguir, determine uma fração igual, mas que seja irredutível.
\vspace{.2cm}

\begin{tabular}{m{.18\textwidth}m{.18\textwidth}m{.18\textwidth}m{.18\textwidth}m{.18\textwidth}}
a) $\dfrac{2}{4}$, & b) $\dfrac{6}{9}$, & c) $\dfrac{4}{2}$, & d) $\dfrac{5}{35}$, & e) $\dfrac{50}{100}$.
\end{tabular}

\ifdefined\prof
\begin{solucao}

\begin{enumerate}
\item Note que o máximo divisor comum de   $2$   e   $4$   é 2. Assim,
$\dfrac{2}{4} = \dfrac{2 \times 1}{2 \times 2} = \dfrac{1}{2}$. Resposta:
$\dfrac{1}{2}$.
 \item Note que o máximo divisor comum de   $6$   e   $3$   é 3. Assim,
$\dfrac{6}{9} = \dfrac{3 \times 2}{3 \times 3} = \dfrac{2}{3}$. Resposta:
$\dfrac{2}{3}$.
 \item Note que o máximo divisor comum de   $2$   e   $4$   é 2. Assim,
$\dfrac{4}{2} = \dfrac{2 \times 2}{2 \times 1} = \dfrac{2}{1}$. Resposta:
$\dfrac{2}{1}$.
 \item Note que o máximo divisor comum de   $5$   e   $35$   é 5. Assim,
$\dfrac{5}{35} = \dfrac{5 \times 1}{5 \times 7} = \dfrac{1}{7}$. Resposta:
$\dfrac{1}{7}$.
 \item Note que o máximo divisor comum de   $50$   e   $100$   é 50. Assim,
$\dfrac{50}{100} = \dfrac{50 \times 1}{50 \times 2} = \dfrac{1}{2}$. Resposta:
$\dfrac{1}{2}$.
\end{enumerate}

\end{solucao}
\fi

\end{document}