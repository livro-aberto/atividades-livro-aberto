\documentclass[10 pt,usenames,dvipsnames, oneside]{article}
\usepackage{../../../modelo-ensino-medio}



\begin{document}

\begin{center}
  \begin{minipage}[l]{3cm}
\includegraphics[width=2cm]{logo}    
\end{minipage}\hfill
\begin{minipage}[r]{.8\textwidth}
 {\Large \scshape Atividade: Classificação de variáveis}  
\end{minipage}
\end{center}
\vspace{.2cm}

\ifdefined\prof
\begin{objetivos}
\item 
\end{objetivos}

\begin{goals}
\begin{enumerate}
\item Diferenciar variável qualitativa e variável quantitativa.
\item Identificar variáveis qualitativas binárias.
\end{enumerate}

\end{goals}

\bigskip
\begin{center}
{\large \scshape Atividade}
\end{center}
\fi

Suponha que cada uma das variáveis a seguir foi observada para todos os alunos de sua turma. Indique se cada uma delas é uma variável qualitativa ou quantitativa. Se for uma variável qualitativa, indique se ela é binária (apenas duas respostas possíveis) ou não.
\begin{enumerate}
\item {} 
Altura (em metros).

\item {} 
Peso (em quilos).

\item {} 
Índice de massa corporal (IMC) dado pelo quaociente entre o peso (em quilos) e o quadrado da medida da altura (em metros).

\item {} 
Tempo de sono na noite anterior.

\item {} 
Se foi dormir na noite anterior antes ou depois da meia-noite.

\item {} 
Mês de nascimento.

\item {} 
Número de irmãos.

\item {} 
Nota obtida na última avaliação de Matemática.

\item {} 
Se tirou nota maior ou igual a 6,0 ou menor do que 6,0 na última avaliação de Matemática.

\item {} 
Distância da casa à escola.

\item {} 
Se o indivíduo possui cartão de crédito ou não.

\item {} 
Modo de locomoção para a escola.

\end{enumerate}


\ifdefined\prof
\begin{solucao}
\begin{enumerate}
\item quantitativa.
\item quantitativa.
\item quantitativa.
\item quantitativa. 
\item qualitativa binária.
\item qualitativa.
\item quantitativa. 
\item quantitativa.
\item qualitativa binária.
\item quantitativa.
\item qualitativa binária.
\item qualitativa.
\end{enumerate}
\end{solucao}
\fi

\end{document}