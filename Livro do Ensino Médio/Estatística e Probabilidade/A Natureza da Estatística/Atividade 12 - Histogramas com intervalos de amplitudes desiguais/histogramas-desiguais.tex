\documentclass[10 pt,usenames,dvipsnames, oneside]{article}
\usepackage{../../../modelo-ensino-medio}



\begin{document}

\begin{center}
  \begin{minipage}[l]{3cm}
\includegraphics[width=2cm]{logo}    
\end{minipage}\hfill
\begin{minipage}[r]{.8\textwidth}
 {\Large \scshape Atividade: Histogramas com intervalos de amplitudes desiguais}  
\end{minipage}
\end{center}
\vspace{.2cm}

\ifdefined\prof
\begin{objetivos}
\item 
\end{objetivos}

\begin{goals}
\begin{enumerate}
\item Construir histogramas nos casos em que os intervalos apresentam amplitudes desiguais.

\item Definir densidade de frequência absoluta e relativa.
\end{enumerate}

\tcblower

Nessa atividade o histograma deve ser construído usando a escala de densidade de frequência (absoluta ou relativa).
\end{goals}

\bigskip
\begin{center}
{\large \scshape Atividade}
\end{center}
\fi

Suponha a seguinte distribuição de frequências de salários medidos em salários mínimos para 200 funcionários de uma empresa.


\begin{table}[H]
\centering
\begin{tabu} to \linewidth {|c|c|c|}
\hline
\thead
Intervalo de classe & Frequência absoluta & Frequência relativa \\
\hline
{[} 2,0 ; 3,0 {[} & 12 & 0,06 \\ 
\hline
{[} 3,0 ; 5,0 {[} & 40 & 0,20 \\
\hline
{[} 5,0 ; 7,0 {[} & 80 & 0,40 \\
\hline
{[} 7,0 ; 10,0 {[} & 48 & 0,24 \\
\hline
{[} 10,0 ; 15,0 {[} & 20 & 0,10 \\
\hline
\end{tabu}
\end{table}

\begin{enumerate}
\item {} 
Determine as amplitudes de cada intervalo considerado na tabela.

\item {} 
Construa um histograma adequado para esses dados.

\end{enumerate}

\ifdefined\prof
\begin{solucao}
\begin{enumerate}

\item \adjustbox{valign=t}
{
\begin{tabu} to \linewidth {|c|c|c|c|}
\hline
\thead
Intervalo de classe & Frequência absoluta & Amplitude &  Dens. de freq. absoluta \\
\hline
{[} 2,0 ; 3,0 {[} & 12 & 1 & 12 \\ 
\hline
{[} 3,0 ; 5,0 {[} & 40 & 2 & 20 \\
\hline
{[} 5,0 ; 7,0 {[} & 80 & 2 & 40 \\
\hline
{[} 7,0 ; 10,0 {[} & 48 & 3 & 16 \\
\hline
{[} 10,0 ; 15,0 {[} & 20 & 5 & 4 \\
\hline
\end{tabu}
}

\clearpage
\item \adjustbox{valign=t}
{
\begin{tikzpicture}[yscale=0.75, scale=0.5, every node/.style={scale=.75}]

\draw (0,0) -- (0,20) node [rotate=90, above, yshift=20, midway, scale=1.4] {frequencia absoluta};

\draw (1,-0.2) -- (16,-0.2);

\node [above,scale=1.8] at (8.5,21.5) {Histograma errado};

\foreach \y/\z in {0/0,5/20,10/40,15/60,20/80} \draw (0,\y) -- (-0.2,\y) node [above, rotate=90 ,scale=1.2] {\z}; 
\foreach \x/\z in {0/0,2/2,3/3,5/5,7/7,10/10,15/15} \draw (\x+1,-0.2) -- (\x+1,-0.4) node [below, scale=1.2] {\z }; 

\foreach \x/\y/\z in {3/4/12,4/6/40,6/8/80,8/11/48,11/16/20}
\draw [fill=primario] (\x,0) rectangle (\y,\z/4) node [above, scale=1.2] at (\x/2+\y/2,\z/4)  {(\z)}
;

\foreach \x/\y/\z in {4/6/0.161,6/8/0.323,8/11/0.290,11/16/0.202}
\node [above, , scale=1.3] at (\x/2+\y/2,0) {{\z}}
;
\node [scale=1.4, align=center,  below right, yshift = 1cm] at (8.5,18.5) {Area total = $1 \times 12+2 \times 40+$ \\ $2 \times 80+ 3 \times 48+5 \times 20=496$ \\ \\ Area relativa do primeiro retangulo $= \frac{12}{496} \approx 0,0242$ \\ \\ No entanto, a frequencia relativa desse intervalo e \\ $\frac{12}{200} = 0,06$};

\end{tikzpicture}
}

\vspace{2em}
\begin{tikzpicture}[yscale=0.75,scale=.5, every node/.style={scale=.75}]

\draw (0,0) -- (0,20) node [rotate=90, above, yshift=20, midway, scale=1.4] {densidade de frequencia absoluta};

\draw [very thin] (1,-0.2) -- (16,-0.2) ;

\node [above,scale=1.8] at (8.5,21.5) {Histograma correto};


\foreach \y/\z in {0/0,2/4,6/12,8/16,10/20,20/40} \draw (0,\y) -- (-0.2,\y) node [above, rotate=90 ,scale=1.2] {\z}; 
\foreach \x/\z in {0/0,2/2,3/3,5/5,7/7,10/10,15/15} \draw (\x+1,-0.2) -- (\x+1,-0.4) node [below, scale=1.2] {\z }; 


\foreach \x/\y/\z in {3/4/12,4/6/20,6/8/40,8/11/16,11/16/4}
\draw [fill=primario] (\x,0) rectangle (\y,\z/2);
\end{tikzpicture}



\end{enumerate}
\end{solucao}
\fi

\end{document}