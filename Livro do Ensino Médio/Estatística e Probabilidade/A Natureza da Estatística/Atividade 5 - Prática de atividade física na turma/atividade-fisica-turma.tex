\documentclass[10 pt,usenames,dvipsnames, oneside]{article}
\usepackage{../../../modelo-ensino-medio}



\begin{document}

\begin{center}
  \begin{minipage}[l]{3cm}
\includegraphics[width=2cm]{logo}    
\end{minipage}\hfill
\begin{minipage}[r]{.8\textwidth}
 {\Large \scshape Atividade: Prática de atividade física na turma}  
\end{minipage}
\end{center}
\vspace{.2cm}

\ifdefined\prof
\begin{objetivos}
\item \textbf{EM13MAT102} Analisar tabelas, gráficos e amostras de pesquisas estatísticas em relatórios divulgados por diferentes meios de comunicação, identificando, quando for o caso, inadequações que possam induzir a erros de interpretação, como escalas e amostras inapropriadas.

\item \textbf{EM13MAT406}: Construir e interpretar tabelas e gráficos de frequências com base em dados obtidos em pesquisas por amostras estatísticas, incluindo ou não o uso de softwares que interrelacionem estatística, geometria e álgebra.
\end{objetivos}

\begin{goals}
\begin{enumerate}
\item Conduzir uma coleta de dados sobre a turma envolvendo as informações do suplemento “Prática de Esporte e Atividade Física” para comparar os resultados dessa “amostra” com os da PNAD/2015.

\end{enumerate}

\tcblower
\begin{enumerate}
\item Preparar uma tabela a ser preenchida pela turma com as informações: sexo, idade, prática ou não de atividade física em seu tempo livre, e a modalidade, de maneira a viabilizar a comparação dos dados obtidos com os resultados da PNAD/2015. A tabela poderá conter outras variáveis se forem julgadas de interesse pela turma como por exemplo, local da prática, duração da prática entre outras. Mas, para efeito de comparação com os infográficos, sexo e idade serão as variáveis necessárias nesse levantamento. Comente com os alunos que essa será uma amostra de conveniência, pois o interesse é estudar o perfil da turma quanto à prática de atividades físicas e por isso, as respostas da turma podem não ser similares às da pesquisa.

\item Com base nas respostas obtidas, resumir a informação em tabelas de frequências, contar quantas respostas foram sim, calcular a porcentagem da turma que pratica atividade física e comparar com o resultado geral das pessoas de 15 anos ou mais, o percentual correspondente a essa faixa etária e o percentual correspondente a esse grau de instrução. Construir uma tabela de frequências com as modalidades esportivas incluindo as categorias apresentadas no infográfico do IBGE. Construir gráficos para representar as distribuições de frequências das variáveis investigadas nessa pesquisa. Construir gráficos de barras múltiplas, isto é, gráficos de barras separados por grupos diferentes, como por exemplo, sexo.
\end{enumerate}

\end{goals}

\bigskip
\begin{center}
{\large \scshape Atividade}
\end{center}
\fi

Deseja-se comparar os hábitos de atividade física em tempo livre dos alunos da turma com os dados obtidos da PNAD/2015. Para isso preencha o formulário de dados fornecido pelo professor. Construa tabelas e gráficos resumindo a informação obtida.

\end{document}