\documentclass[10 pt,usenames,dvipsnames, oneside]{article}
\usepackage{../../../modelo-ensino-medio}


\begin{document}

\begin{center}
  \begin{minipage}[l]{3cm}
\includegraphics[width=2cm]{logo}    
\end{minipage}\hfill
\begin{minipage}[r]{.8\textwidth}
 {\Large \scshape Atividade: Pesquisa sobre a prática de esportes e atividade física}  
\end{minipage}
\end{center}
\vspace{.2cm}

\ifdefined\prof
\begin{objetivos}
\item \textbf{EM13MAT102} Analisar tabelas, gráficos e amostras de pesquisas estatísticas em relatórios divulgados por diferentes meios de comunicação, identificando, quando for o caso, inadequações que possam induzir a erros de interpretação, como escalas e amostras inapropriadas.

\item \textbf{EM13MAT406} Construir e interpretar tabelas e gráficos de frequências com base em dados obtidos em pesquisas por amostras estatísticas, incluindo ou não o uso de softwares que interrelacionem estatística, geometria e álgebra.
\end{objetivos}

\begin{goals}
\begin{enumerate}

\item Apresentar os conceitos de população e amostra.

\item Comparar os diferentes tipos de variáveis analisados em uma pesquisa para adiante identificar variáveis qualitativas e quantitativas.

\end{enumerate}

\tcblower

\begin{itemize}
\item No item a), espera-se que sejam indicadas algumas entre as seguintes variáveis: idade, sexo, educação, trabalho, rendimento, se pratica ou não atividade física, modalidade da atividade para quem pratica, motivação para a prática de atividade física, local da prática, frequência da prática, duração da atividade, participação em competições, etc.

\item No item b) deve-se informar as variáveis que assumem atributos (respostas não-numéricas) tais como sexo, prática de atividade física (sim ou não), modalidade da atividade física praticada, etc.

\item No item c) deve-se informar as variáveis que assumem valores numéricos tais como idade, rendimento, duração da atividade física, etc.
\end{itemize}

\end{goals}

\bigskip
\begin{center}
{\large \scshape Atividade}
\end{center}
\fi

\textit{Fonte: IBGE, Suplemento da PNAD/2015}
\vspace{.5em}

A Pesquisa Nacional por \index{Amostra}Amostra de Domicílios (PNAD), realizada pelo \href{https://www.ibge.gov.br/estatisticas-novoportal/sociais/populacao/9127-pesquisa-nacional-por-amostra-de-domicilios.html}{IBGE}, obtém informações anuais sobre características demográficas e socioeconômicas da população, como sexo, idade, educação, trabalho e rendimento, e características dos domicílios. Com periodicidade variável, a PNAD obtém informações sobre migração, fecundidade, entre outras, tendo os domicílios como unidade de coleta da informação. Temas específicos abrangendo aspectos demográficos, sociais e econômicos também são investigados.

Um aspecto fundamental da Estatística praticado nessa pesquisa é a forma na qual a amostra, subconjunto da população, é selecionada. Essa seleção é cuidadosamente planejada de modo que seja adequado estender os resultados obtidos na amostra para a população.

Para que os resultados de uma amostra possam ser estendidos para a população, é necessário planejar com cuidado como a amostra será selecionada, pois o critério de seleção da amostra depende da estrutura da população. Por exemplo, para saber se o feijão cozinhando na panela está bem temperado, basta provar uma pequena colherada. Por quê?  Partimos do pressuposto de que todos os ingredientes foram bem misturados e, assim, a mistura é homogênea.

Quando dispomos de dados provenientes de um subconjunto da população sempre podemos descrever os dados nos restringindo apenas ao subconjunto. Se quisermos estender nossas conclusões para a população, será necessário o uso de outras tecnologias que permitam calcular as incertezas associadas a essas extensões.

Na PNAD 2015 foi realizada a investigação de um tema específico chamado "Suplemento de Práticas de Esporte e Atividade Física" no qual foram investigadas as pessoas moradoras de 15 anos ou mais de idade, \textbf{em seu tempo livre}, no período de referência de 365 dias, com o objetivo de quantificar aquelas que praticaram algum esporte ou atividade física no período considerado bem como a sua percepção quanto a isso. As informações levantadas nessa pesquisa foram obtidas por meio de um questionário no qual se perguntou:
\begin{itemize}
\item {} 
Se a pessoa moradora havia praticado esporte, e em caso afirmativo, a respectiva modalidade.

\item {} 
Independente da resposta anterior, também se perguntou se a pessoa praticava alguma atividade física que não considerava como esporte, informando, em caso positivo, também a modalidade.

\item {} 
Outras informações levantadas nessa pesquisa foram: motivação para a prática da atividade física, local onde é praticada a atividade, frequência na qual a atividade é praticada, duração da atividade; e a participação em competições.

\item {} 
Também foram levantadas informações sobre as pessoas que responderam que não praticavam atividade física. Perguntou-se o motivo de não o fazerem e se haviam praticado anteriormente, caso em que se perguntou a modalidade praticada, a idade em que parou de praticar e a causa da interrupção.

\item {} 
Além dessas informações, a pesquisa investigou também a avaliação da população sobre a opção de o poder público investir no desenvolvimento de atividades físicas e esportivas ou em outra área (saúde, educação, etc.) na vizinhança de seu domicílio.

\end{itemize}
\begin{enumerate}
\item {} 
Liste pelo menos oito variáveis investigadas na PNAD e no "Suplemento de Práticas de Esporte e Atividade Física" da PNAD 2015, baseando-se no texto apresentado.

\item {} 
Das variáveis citadas no item anterior, quais delas apresentam respostas não numéricas?

\item {} 
Das variáveis citadas no item a), quais delas apresentam respostas numéricas?

\end{enumerate}

Cada uma das unidades investigadas em um estudo estatístico é denominada um elemento.  Assim, cada parafuso investigado é um elemento na atividade "Escolha do fornecedor"; cada paciente observado é um elemento na atividade "Comparação de medicamentos"; e cada domicílio e seus residentes são elementos na atividade da PNAD.

Cada característica observada de um elemento é uma variável estatística. Assim, a medida do diâmetro do parafuso é uma variável na atividade "Escolha do fornecedor", o tempo de cura da dor de cabeça é uma variável na atividade "Comparação de medicamentos" e, na atividade da PNAD, estão presentes várias variáveis estatísticas de interesse do domicílio e de seus residentes tais como local, número de cômodos, número de residentes; sexo, idade e rendimento dos residentes, etc.


\ifdefined\prof
\clearpage
\begin{solucao}
\begin{enumerate}
\item Sexo. Idade. Educação. Trabalho. Rendimento. Prática de Atividade Física(AF). Modalidade da AF para quem pratica. Motivação para a AF. Local da Prática da AF. Duração da Prática da AF etc.

\item Sexo. Educação. Trabalho. Prática de AF. Modalidade de AF. Motivação da Prática de AF. Local da Prática da AF.

\item Idade. Rendimento. Duração da Prática de AF.
\end{enumerate}
\end{solucao}
\fi

\end{document}