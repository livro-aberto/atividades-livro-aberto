\documentclass[10 pt,usenames,dvipsnames, oneside]{article}
\usepackage{../../../modelo-ensino-medio}



\begin{document}

\begin{center}
  \begin{minipage}[l]{3cm}
\includegraphics[width=2cm]{logo}    
\end{minipage}\hfill
\begin{minipage}[r]{.8\textwidth}
 {\Large \scshape Atividade: Análise de Gráfico}  
\end{minipage}
\end{center}
\vspace{.2cm}

\ifdefined\prof
\begin{objetivos}
\item \textbf{EM13MAT406}: Construir e interpretar tabelas e gráficos de frequências com base em dados obtidos em pesquisas por amostras estatísticas, incluindo ou não o uso de softwares que interrelacionem estatística, geometria e álgebra.
\end{objetivos}

\begin{goals}
\begin{enumerate}
\item Mostrar que podem existir diversas formas de usar barras para representar algum tipo de dado, mas que nem todos os gráficos que usam barras são gráficos de barras no sentido da representação de uma distribuição de frequências.
\end{enumerate}

\tcblower

O gráfico desse exemplo é “um gráfico de barras”, mas as barras representam o valor da inflação da alimentação acumulado nos últimos 12 meses em função do tempo: de agosto de 2016 até agosto de 2017. Na seção “Explorando 2”, veremos que, para esse tipo de informação - valores de uma variável quantitativa ao longo do tempo -, é mais comum usar um gráfico de linhas unindo por segmentos os pontos consecutivos dados (tempo, valor da variável).

\end{goals}

\bigskip
\begin{center}
{\large \scshape Atividade}
\end{center}
\fi

Observe o gráfico a seguir publicado em um jornal.

\begin{figure}[H]
\centering
\begin{tikzpicture}[scale=0.7, every node/.style={scale=0.9}]
\begin{scope}[x=20, y = 10]
\draw (0,0) -- (28,0);
\foreach \ye in {-5,5,10,15}{
\draw [help lines, lightgray] (0,\ye) -- (28,\ye);
}
\foreach \x/ \y/\z in {1/16.79/AGO,2/16.11/SET,3/14.85/OUT,4 /11.57/NOV,5/9.36/DEZ,6/6.47/JAN,7/4.34/FEV,8/3/MAR,9 /2.54/ABR,10/1.08/MAI,11/-0.56/JUN,12/-3.07/JUL,13/-5.19 /AGO}{
   \draw [fill=\currentcolor!80] (2*\x-0.5,0) rectangle  (2*\x+0.5,\y);
   \node at (2*\x,-8) {\z};
}

\foreach \x/\y/\z in {1/16/79,2/16/11,3/14/85,4/11/57,5/9/36,6/6/47,7/4/34,8/3/0,9/2/52,10/1/08} 
\node [above] at (2*\x,\y.\z) {\y,\z};

\foreach \x/\y/\z in {11/-0/56,12/-3/07,13/-5/19} 
\node [below] at (2*\x,\y.\z) {\y,\z};

\node at (2,-10) {2016};
\node at (12,-10) {2017};
\node [right] at (0,21.5) {INFLAÇÃO DA ALIMENTAÇÃO NO   DOMICÍLIO};
\node [right] at (0,20) {(acumulado em 12 meses, em   $\%$)};
\end{scope}
\end{tikzpicture}
\caption{Inflação da alimentação acumulada nos últimos 12 meses (Fonte: IBGE)}
\label{est1-fig-8}
\end{figure}
\begin{enumerate}
\item {} 
Como você classificaria esse gráfico?

\item {} 
Qual é a informação representada pelo comprimento da barra nesse gráfico?

\item {} 
Que tipo(s) de variável(is) ele está representando?

\item {} 
Construa um gráfico diferente para representar a mesma informação, marcando num plano Cartesiano os pontos $(x,y)$ em que $x$ corresponde ao tempo e $y$ corresponde à inflação acumulada no domicílio, unindo os pontos consecutivos por segmentos. É possível perceber a partir desse gráfico algum tipo de comportamento no período observado?

\end{enumerate}



\ifdefined\prof
\begin{solucao}
\begin{enumerate}
\item É um gráfico que usa barras, mas nesse gráfico o comprimento das barras não é frequência.

\item Valor da inflação da alimentação acumulada nos últimos 12 meses. Esses valores são apresentados em função do período de tempo: agosto de 2016 até agosto de 2017.

\item O valor da inflação da alimentação acumulada nos últimos 12 meses é uma variável quantitativa, o período de tempo representado em mês/ano é uma variável qualitativa ordinal.

\item (gráfico) Como evoluiu a inflação da alimentação acumulada em 12 meses no período investigado, a saber, agosto de 2016 até agosto de 2017. Mais precisamente, percebe-se que a inflação da alimentação acumulada em 12 meses apresentou no período analisado uma forte tendência de queda.

\begin{figure}[H]
\centering


\begin{tikzpicture}
\begin{scope}[x=20, y=10, scale=.7]
      \draw (0,0) -- (28,0);
      \draw (0,-7) -- (0,20);

      \foreach \ye in {-5,0,5,10,15,20}{
         \draw [help lines, gray] (0,\ye) -- (28,\ye);
         \node [left, overlay] at (0,\ye) {\ye};
      };
      \foreach \x/\y/\z in {1/16.79/AGO,2/16.11/SET,3/14.85/OUT,4/11.57/NOV,5/9.36/DEZ,6/6.47/JAN,7/4.34/FEV,8/3/MAR,9/2.54/ABR,10/1.08/MAI,11/-0.56/JUN,12/-3.07/JUL,13/-5.19/AGO}{
      \draw [help lines, gray] (2*\x,-7) -- (2*\x,20);
      \node at (2*\x,-8) {\z};
      }
	  \draw [help lines] (2*14,-7) -- (2*14,20);
      \draw [color=primario, very thick] (2,16.79) node[ponto] {}
      \foreach \x/\y/\z in {1/16.79/AGO,2/16.11/SET,3/14.85/OUT,4/11.57/NOV,5/9.36/DEZ,6/6.47/JAN,7/4.34/FEV,8/3/MAR,9/2.54/ABR,10/1.08/MAI,11/-0.56/JUN,12/-3.07/JUL,13/-5.19/AGO}{
         -- (2*\x,\y) node[ponto] {}
      };
      \node at (2,-9.5) {2016};
      \node at (12,-9.5) {2017};
      \node [right] at (0,23.5) {INFLAÇÃO DA ALIMENTAÇÃOO NO DOMICÍLIO};
      \node [right] at (0,22) {(acumulado em 12 meses, em $\%$)};
      \end{scope}
\end{tikzpicture}
\end{figure}
\end{enumerate}
\end{solucao}
\fi

\end{document}