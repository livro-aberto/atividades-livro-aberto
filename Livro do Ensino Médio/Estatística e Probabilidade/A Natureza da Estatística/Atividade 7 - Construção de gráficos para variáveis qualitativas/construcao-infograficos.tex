\documentclass[10 pt,usenames,dvipsnames, oneside]{article}
\usepackage{../../../modelo-ensino-medio}



\begin{document}

\begin{center}
  \begin{minipage}[l]{3cm}
\includegraphics[width=2cm]{logo}    
\end{minipage}\hfill
\begin{minipage}[r]{.8\textwidth}
 {\Large \scshape Atividade: Construção de gráficos para variávei quantitativas}  
\end{minipage}
\end{center}
\vspace{.2cm}

\ifdefined\prof
\begin{objetivos}
\item \textbf{EM13MAT406} Construir e interpretar tabelas e gráficos de frequências com base em dados obtidos em pesquisas por amostras estatísticas, incluindo ou não o uso de softwares que interrelacionem estatística, geometria e álgebra.
\end{objetivos}

\begin{goals}
\begin{enumerate}
\item Construir gráficos de distribuições de frequências para variáveis qualitativas.
\end{enumerate}

\tcblower

Embora os gráficos solicitados nesta atividade sejam simples, recomenda-se sugerir aos alunos usar algum recurso tecnológico para a construção dos mesmos, tais como, uma planilha ou o GeoGebra.

\end{goals}

\bigskip
\begin{center}
{\large \scshape Atividade}
\end{center}
\fi

Considerando o infográfico 4 da atividade \textit{Análise de infográficos}, transforme o gráfico de setores em gráfico de retângulos e os gráficos de retângulos em gráficos de setores.

\ifdefined\prof
\begin{solucao}

\begin{figure}[H]
\centering

\begin{tikzpicture}
\begin{scope}[x=19,y=19, scale=.75]
\node at (0,11) {O poder público deve investir em atividades fí­sicas?};
\draw [fill=session2!80] (-1,10) rectangle (1,2.67);
\draw [fill=session3!80] (-1,2.67) rectangle (1,1.2);
\draw [fill=secundario!80] (-1,1.2) rectangle (1,0);
\node [color=white] at (0,6.335) {\textbf{Sim}};
\node [color=white] at (0,1.935) {\textbf{Não}};
\draw [thick,->] (-1.5,9.5) -- (-7,9.5) -- (-7,8.5); 
\draw [thick,->] (1.5,1.935) -- (7,1.935) -- (7,3); 
\draw[fill=session2!80] (7,7) -- (7,10) arc (90:-118.08:3);
\draw [fill=primario!80] (7,7) -- + (-118.08:3) arc (-118.08:-194.76:3);
\draw [fill=session1!80] (7,7) -- + (-194.76:3) arc (-194.76:-254.16:3);
\draw [fill=secundario!80] (7,7) -- + (-254.16:3) arc (-254.16:-270:3);
\draw (7,7) -- (7,10);
\draw[fill=session4!80] (-7,4.5) -- (-7,7.5) arc (90:-237.96:3);
\draw [fill=box3!80] (-7,4.5) -- + (-237.96:3) arc (-237.96:-266.76:3);
\draw [fill=secundario!80] (-7,4.5) -- + (-266.76:3) arc (-266.76:-270:3);
\draw (-7,4.5) -- (-7,7.5);
\draw [fill=session4!80, line width=0] (-10,-1) rectangle (-9,-2);
\node [right] at (-9,-1.5) {Atividades para};
\node [right] at (-9,-2.5) {pessoas em geral};
\draw [fill=box3!80, line width=0] (-10,-3.5) rectangle (-9,-4.5);
\node [right] at (-9,-4) {Formação de atletas};
\draw [fill=secundario!80, line width=0] (-10,-5) rectangle (-9,-6);
\node [right] at (-9,-5.5) {Outra};
\draw [fill=session1!80, line width=0] (4,-1) rectangle (5,-2);
\node [right] at (5,-1.5) {Saúde};
\draw [fill=primario!80, line width=0] (4,-2.5) rectangle (5,-3.5);
\node [right] at (5,-3) {Segurança};
\draw [fill=session2!80, line width=0] (4,-4) rectangle (5,-5);
\node [right] at (5,-4.5) {Educação};
\draw [fill=secundario!80, line width=0] (4,-5.5) rectangle (5,-6.5);
\node [right] at (5,-6) {Outra};
\end{scope}
\end{tikzpicture}
\end{figure}

\end{solucao}
\fi

\end{document}