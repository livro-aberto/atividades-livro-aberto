\documentclass[10 pt,usenames,dvipsnames, oneside]{article}
\usepackage{../../../modelo-ensino-medio}



\begin{document}

\begin{center}
  \begin{minipage}[l]{3cm}
\includegraphics[width=2cm]{logo}    
\end{minipage}\hfill
\begin{minipage}[r]{.8\textwidth}
 {\Large \scshape Atividade: Estratégia de investimento}  
\end{minipage}
\end{center}
\vspace{.2cm}

\ifdefined\prof
%Habilidades da BNCC
\begin{objetivos}
\item \textbf{EM13MAT316} Estudar o efeito de uma transformação simples numa
distribuição de dados: adição (posição) e multiplicação (escala).
\item \textbf{EM13MAT409} Interpretar e comparar conjuntos de dados estatísticos por meio de diferentes diagramas e gráficos, como o histograma, o de caixa (box-plot), o de ramos e folhas, reconhecendo os mais eficientes para sua análise.
\end{objetivos}

%Caixa do Para o Professor
\begin{goals}
%Objetivos específicos
\begin{enumerate}
\item Definir medidas que caracterizam a dispersão de um conjunto de dados.
\end{enumerate}

\tcblower

%Orientações e sugestões
Nessa atividade são apresentados dois conjuntos de dados temporais que apresentam mesma média, mesma mediana e mesma moda e, no entanto, seus gráficos de linha são distintos. O objetivo principal é mostrar que as medidas de posição podem ser insuficientes para caracterizar a distribuição dos dados, levando à necessidade de usar medidas de dispersão. Lembre que é esperado, do Ensino Fundamental, que os estudantes já tenham a noção de amplitude amostral, uma medida bruta de dispersão, pois só leva em conta o mínimo e o máximo observados.

Esta atividade pode ser vinculada às disciplinas de História ou Geografia, por exemplo, no estudo do período da Crise Econômica de 1929 ou outros temas relacionados com o PIB e crescimento econômico. 
\end{goals}

\bigskip
\begin{center}
{\large \scshape Atividade}
\end{center}
\fi

Para investir na bolsa de valores compramos ações de empresas por intermédio de uma corretora a um certo preço e depois de um período de tempo vendemos estas ações na expectativa de que seus preços tenham aumentado. No entanto, também podemos perder com o investimento, caso o preço da ação diminua no período de investimento. Uma ação é a menor parte do capital de uma empresa. Veja na figura a seguir um esquema simplificado do investimento na bolsa de valores.

\begin{figure}[H]
\centering

\noindent\includegraphics[width=300bp]{{resized001}.png}
\caption{Esquema simplificado de investimento na bolsa de valores}\label{\detokenize{PE104-3:fig-ativ-bolsa-de-valores}}\label{\detokenize{PE104-3:id1}}\end{figure}

Suponha que você tenha a oportunidade de investir um capital, comprando ações de uma de duas Companhias \(A\) ou \(B\) e para escolher uma das duas, disponha de duas amostras de preços do valor destas ações (em reais) registrados no fechamento da bolsa de valores em dez sextas-feiras consecutivas. Veja na figura e na tabela a seguir a cotação das ações ao longo das últimas 10 semanas.

\begin{figure}[H]
\centering

\resizebox{.75\linewidth}{!}
{
\begin{tikzpicture}[xscale=1.5,scale=.5, every node/.style={scale=.9}]

\draw [help lines] (0,0) grid (11,7);
\tikzstyle{quad}=[fill=destacado!80,rectangle, minimum height=3pt,minimum width=3pt]
\tikzstyle{los}=[fill=\currentcolor!80,rectangle, minimum height=3pt, minimum width=3pt, rotate=45]

\foreach \x/\y/\z in {1/61/a,2/56/b,3/65/c,4/57/d,5/67/e,6/63/f,7/67/g,8/58/h,9/67/i,10/56/j} \node (\z) [los] at (\x,\y/10-2.5) {};
\draw [very thick, \currentcolor=!80] (a) -- (b) -- (c) -- (d) -- (e) -- (f) -- (g) -- (h) -- (i) -- (j);

\foreach \x/\y/\z in {1/67/A,2/48/B,3/52/C,4/82/D,5/77/E,6/33/F,7/67/G,8/42/H,9/90/I,10/57/J} \node (\z) [quad]  at (\x,\y/10-2.5) {};
\draw [very thick, destacado!80] (A) -- (B) -- (C) -- (D) -- (E) -- (F) -- (G) -- (H) -- (I) -- (J);

\foreach \x in {0,...,11} \node [below] at (\x,-.2) {\x};

\foreach \x/\y in {0/25,1/35,2/45,3/55,4/65,5/75,6/85,7/95} \node [left] at (-.2,\x) {\y};

\node [above,font=\bfseries] at (5.5,7.5) {Cotação das Ações};

\node [below,font=\bfseries] at (5.5,-1) {Semana};
\node [above, rotate=90,font=\bfseries] at (-1,3.5) {Cotação};

\node [los, label=below right:Ações da Companhia A] at (1.5,-2.5) {};

\node [quad, label=right:Ações da Companhia B] at (7,-2.5) {};

\end{tikzpicture}
}
\caption{Gráficos de linha da cotação das ações}\label{\detokenize{PE104-3:fig-coloque-aqui-o-nome}}\label{\detokenize{PE104-3:id2}}\end{figure}

\begin{table}[H]
\centering

\begin{tabular}{|c|*{10}{f|}c|}
\hline
\tcolor{Semana} & 1 & 2 & 3 & 4 & 5 & 6 & 7 & 8 & 9 & 10 & Total \\
\hline
$\tmat{A}$ & 61 & 56 & 63 & 57 & 67 & 63 & 67 & 58 & 67 & 56 & $615$ \\
\hline
$\tmat{B}$ & 67 & 48 & 52 & 82 & 77 & 33 & 67 & 42 & 90 & 57 & $615$ \\
\hline
\end{tabular}
\end{table}

\begin{enumerate}
\item Observando o gráfico, qual das duas companhias você escolheria para investir? Por quê?

\item {} 
Obtenha as médias das cotações das ações das companhias $A$ e $B$ nas semanas observadas e compare-as.

\item {} 
Obtenha as medianas das cotações das ações das companhias $A$ e $B$ nas semanas observadas e compare-as, lembrando que os dados da tabela estão apresentados na ordem temporal.

\item {} 
Obtenha as modas das cotações das ações das companhias $A$ e $B$ nas semanas observadas e compare-as.

\item {} 
Analisando apenas as medidas de posição obtidas em \titem{a)}, \titem{b)} e \titem{c)}, pode-se dizer que as duas companhias diferem uma da outra? Por quê?

\item {} 
Um investimento que apresenta grandes ganhos e perdas pode ser chamado de alto risco, já investimentos cujos valores flutuam pouco são considerados de baixo risco. Se você é um investidor da bolsa de valores avesso ao risco, isto é, você gostaria de escolher o investimento com menores flutuações, em qual das companhias você investiria o seu dinheiro? Por quê?

\end{enumerate}

\ifdefined\prof
\begin{solucao}

\begin{enumerate}
\item A escolha pode ser tanto pela $A$ como pela $B$, mas deve vir acompanhada de uma justificativa. Por exemplo, “eu escolheria a companhia A porque os preços oscilam menos”, “escolheria a companhia $B$ porque os preços oscilam mais”, “escolheria a companhia $B$ porque foi a que apresentou maior valor de cotação entre os dias observados”{} etc.

\item Dado que são $10$ observações em cada um dos conjuntos e que as somas das $10$, resultam em $615$, segue que a média das cotações na companhia $A$ é R\$ $61{,}50$, que também é a média das cotações na companhia $B$.

\item Para obter as medianas é necessário antes ordenar os valores. Na tabela a seguir os valores das cotações foram ordenados para cada companhia.


\begin{table}[H]
\centering

\begin{tabular}{|*{11}{f|}}
\hline
\tmat{A} & 56 & 56 & 57 & 58 & 61 & 63 & 63 & 67 & 67 & 67 \\
\hline
\tmat{B} & 33 & 43 & 48 & 52 & 57 & 67 & 67 & 77 & 82 & 90 \\
\hline
\end{tabular}
\end{table}

Como são $10$ observações em cada conjunto e $10$ é um número par, temos que a mediana será dada pela média das duas posições centrais, a saber, posições 5 e 6: Mediana$=\dfrac{x_{(5)}+x_{(6)}}{2}$.

Na companhia $A$ teremos Mediana=$\dfrac{61+63}{2}=62$ reais e, na companhia B, Mediana=$\dfrac{57+67}{2}=62$ reais.

\item Na companhia $A$ o valor mais frequente foi $67$, ocorrendo $3$ vezes. Na companhia $B$, o valor mais frequente foi $67$, ocorrendo duas vezes. Logo, tanto em $A$ como em $B$ o valor da moda foi $67$ reais.

\item Não, pois tais medidas são idênticas nas duas companhias.

\item Analisando os gráficos de linha da figura $57$, percebe-se que as cotações da companhia $B$ variam mais do que as da companhia $A$ e, portanto, como menor risco envolve menos variação, escolheria a companhia $A$. Observe que as amplitudes (diferença entre o maior e menor valores) observadas nas companhias $A$ e $B$ são $67-56=11$ e $90-33=57$, respectivamente, confirmando que na companhia $A$ a variação das cotações é menor.
\end{enumerate}

\end{solucao}
\fi

\end{document}