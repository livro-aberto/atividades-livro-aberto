\documentclass[10 pt,usenames,dvipsnames, oneside]{article}
\usepackage{../../../modelo-ensino-medio}



\begin{document}

\begin{center}
  \begin{minipage}[l]{3cm}
\includegraphics[width=2cm]{logo}    
\end{minipage}\hfill
\begin{minipage}[r]{.8\textwidth}
 {\Large \scshape Atividade: Inflação anual}  
\end{minipage}
\end{center}
\vspace{.2cm}

\ifdefined\prof
%Habilidades da BNCC
\begin{objetivos}
\item \textbf{EM13MAT316} Estudar o efeito de uma transformação simples numa
distribuição de dados: adição (posição) e multiplicação (escala).
\end{objetivos}

%Caixa do Para o Professor
\begin{goals}
%Objetivos específicos
\begin{itemize}
\item Comparar diferentes conjuntos de dados que apresentam a mesma variância, mas suas médias são diferentes.

\item Perceber a necessidade de definir uma medida que avalie a magnitude da variância (desvio padrão) em relação à média.
\end{itemize}

\tcblower

%Orientações e sugestões
Nesta atividade são apresentados dois conjuntos de dados cujas variâncias são iguais, mas cujas médias são distintas. Pretende-se na discussão, levar à definição de coeficiente de variação, uma medida útil para avaliar a magnitude da variância. Como o dado observado é a inflação anual de um país, a atividade começa com um pequeno texto introdutório sobre inflação.
\end{goals}

\bigskip
\begin{center}
{\large \scshape Atividade}
\end{center}
\fi

A seguir são apresentados dados sobre as inflações anuais em dois países. Antes de trabalhar com os dados, vamos tentar explicar o que é \index{inflação}inflação. De uma maneira bem simples, pode-se dizer que a inflação é o aumento contínuo nos preços de produtos e serviços. Esse aumento costuma ser avaliado de forma mensal, gerando os índices de inflação, que refletem a variação nos preços.

A inflação pode ser medida de várias formas. O índice oficial de inflação no Brasil é o IPCA (Índice de Preços ao Consumidor Amplo), que mede a variação mensal de preços de produtos considerando o consumo de famílias com renda mensal entre 1 e 40 salários mínimos. O IBGE (Instituto Brasileiro de Geografia e Estatística) é o orgão responsável pela medição e divulgação do IPCA. Veja neste
\href{https://www.youtube.com/watch?v=JVcDZOlIMBk}{link}, um vídeo produzido pelo IBGE, explicando o IPCA.

Foram observadas as inflações anuais de dois países $A$ e $B$ para os anos de 2011 a 2015, conforme tabela a seguir.

\begin{table}[H]
\centering
\caption{Inflação anual}
\begin{tabular}{|c|c|c|c|c|c|c|}
\hline
\tcolor{País} & \tcolor{2011} & \tcolor{2012} & \tcolor{2013} & \tcolor{2014} & \tcolor{2015} & \tcolor{Soma} \\
\hline
$A$ & $2{,}00\%$ & $1{,}80\%$ & $2{,}10\%$ & $2{,}20\%$ & $1{,}90\%$ & $10{,}00\%$ \\
\hline
$B$ & $0{,}01$\% & $-0{,}19\%$ & $-0{,}09\%$ & $0{,}21\%$ & $0{,}11\%$ & $0{,},05\%$ \\
\hline
\end{tabular}
\end{table}

\begin{enumerate}
\item {} 
Calcule as médias das inflações anuais dos dois países. Há diferenças entre elas?

\item {} 
Calcule as variâncias das inflações anuais dos dois países, sabendo que para o país $A$, \(\displaystyle{\sum^5_{i=1}}x^2_i=20,1\)  (\% \(^2\) ) e para o país $B$,  \(\displaystyle{\sum^5_{i=1}}x^2_i=0,1005\)  (\% \(^2\) ). Há diferença entre elas?

\item {} 
Qual dos países apresenta maior variação inflacionária quando comparada à média inflacionária?

\end{enumerate}

\ifdefined\prof
\begin{solucao}

\begin{enumerate}
\item No país $A$, a inflação média anual, considerando estes 5 anos, é $\bar{x}=\frac{10}{5}=2{,}00\%$. No país $B$, a inflação média anual, considerando estes 5 anos, é $\bar{x}=0{,}055=0{,}01\%$. Logo, as inflações anuais médias dos dois países são bem diferentes.

\item Usando a fórmula simplificada para o cálculo da variância, temos, para o país $A$, $s^2=\frac{1}{5-1}(20,1-5\cdot2^2)=0{,}025(\%^2)$. Para o país $B$, temos $s^2=\frac{1}{5-1}(0{,}1005-5\cdot0{,}012)=0{,}025(\%^2)$. Logo, as variâncias destes dois conjuntos de inflações anuais são iguais e, consequentemente, os desvios padrões também são iguais.

\item Verifique que os cinco desvios da média produzidos pelos dados dos dois países são idênticos, levando à mesma variância (mesmo desvio padrão). No entanto, a média no país $A$ ($2\%$) é bem maior do que no país $B$, indicando uma variação relativa à média menos forte do que no país $B$. A seguir, será definido o coeficiente de variação, que avalia essa propriedade de dispersão relativa à média. Observe que o desvio padrão para os dois países é $\sqrt{0{,}025}\approx0{,}16\%$ de modo que no país A o desvio padrão corresponde a $8\%$ da média $(\frac{0{,}16}{2}=0{,}08)$, enquanto que no país B, corresponde a $1.600\%$ da média $(\frac{0{,}16}{0{,}01}=16)$, ou seja, a flutuação em torno da média é muito mais forte no país $B$.
\end{enumerate}

\end{solucao}
\fi

\end{document}