\documentclass[10 pt,usenames,dvipsnames, oneside]{article}
\usepackage{../../../modelo-ensino-medio}



\begin{document}

\begin{center}
  \begin{minipage}[l]{3cm}
\includegraphics[width=2cm]{logo}    
\end{minipage}\hfill
\begin{minipage}[r]{.8\textwidth}
 {\Large \scshape Atividade: Aproximação para o valor do desvio padrão amostral}  
\end{minipage}
\end{center}
\vspace{.2cm}

\ifdefined\prof
%Habilidades da BNCC
\begin{objetivos}
\item \textbf{EM13MAT316} Resolver e elaborar problemas, em diferentes contextos, que envolvem cálculo e
interpretação das medidas de tendência central (média, moda, mediana) e das de dispersão
(amplitude, variância e desvio padrão).
\end{objetivos}

%Caixa do Para o Professor
\begin{goals}
%Objetivos específicos
\begin{enumerate}
\item Calcular uma aproximação grosseira do desvio padrão amostral em função da amplitude amostral.

\item Comparar os resultados obtidos pela fórmula de aproximação com os valores exatos do desvio padrão amostral.

\item Avaliar o valor obtido do desvio padrão, comparando-o com a aproximação.
\end{enumerate}

\tcblower

%Orientações e sugestões
Nesta atividade pretende-se apresentar interpretações para o desvio padrão, evitando que ele torne-se apenas uma medida a mais sem muito sentido para o aluno. Além disso, esta atividade pode ser útil para o aluno avaliar se ele calculou corretamente um desvio padrão. É muito comum, mesmo informando-se somatórios e permitindo-se o uso de calculadoras, a produção de resultados incorretos para a variância e, consequentemente, para o desvio padrão. Uma ferramenta útil pode ser comparar o valor obtido do desvio padrão com a razão $\frac{R}{4}$. Se a diferença for grande (mais de $50\%$ do valor obtido de s) recomenda-se verificar novamente o cálculo de $s$.
\end{goals}

\bigskip
\begin{center}
{\large \scshape Atividade}
\end{center}
\fi

Nos conjuntos de dados, quando não há valores atípicos (valores muito altos ou muito baixos em relação à maior parte dos valores no conjunto), a maior parte dos valores se situará no intervalo centrado na média distando 2 desvios padrões à esquerda e à direita da média. A partir desta suposição, pode-se obter uma fórmula para estimar o valor do desvio padrão amostral \(s\) .
\begin{equation*}
\begin{split}\left \{ \begin{array}{l} \text{Max}=x_{(n)}\approx \bar{x}+2\cdot s \\ \text{Min}=x_{(1)}\approx \bar{x}-2\cdot s\end{array}\right.\end{split}
\end{equation*}
Tomando a diferença das primeiras expressões apresentadas, obtemos
\begin{equation*}
\begin{split}R= \text{Max-Min} \approx 4\cdot s\end{split}
\end{equation*}
tal que
\begin{equation*}
\begin{split}s\approx \frac{R}{4}\end{split}
\end{equation*}\begin{enumerate}
\item {} 
Use esta fórmula para estimar o valor do desvio padrão amostral dos dados da atividade \hyperref[\detokenize{PE104-0:ativ-notas-de-artes}]{Notas de Arte} e compare o valor obtido com o desvio padrão amostral \(s\). Use os dados na tabela a seguir, produzidos pelo GeoGebra.

\end{enumerate}

\begin{table}[H]
\centering
\begin{tabular}{|l|l|}
\hline
\multicolumn{2}{|c|}{\cellcolor{\currentcolor!80}{\textcolor{white}{\textbf{Estatística}}}}\\
\hline
$n$ & 35 \\
\hline
Média & 5,9286 \\
\hline
$\sigma$ & 1,9362 \\
\hline
$s$ & 1,9645 \\
\hline
$\Sigma x$ & 207,5 \\
\hline
$\Sigma x^2$ & 1361,39 \\
\hline
Min & 0,8 \\
\hline
$Q_1$ & 5,4 \\
\hline
Mediana & 6,5 \\
\hline
$Q_3$ & 7,5 \\
\hline
Max & 8 \\
\hline
\end{tabular}
\caption{Estatísticas resumo das Notas de Artes}\label{\detokenize{PE104-5:fig-resumonartes}}\label{\detokenize{PE104-5:id2}}
\end{table}
% \begin{figure}[H]
% \centering
% \capstart

% \noindent\includegraphics[width=100bp]{{summary_NArtes}.png}
% \caption{Estatísticas resumo das Notas de Artes}\label{\detokenize{PE104-5:fig-resumonartes}}\label{\detokenize{PE104-5:id2}}\end{figure}
\begin{enumerate}
\setcounter{enumi}{1}
\item {} 
Idem para estimar o valor do desvio padrão amostral dos dados da atividade \hyperref[\detokenize{PE104-0:ativ-maratona-de-ny}]{A Maratona} e compare o valor obtido com o desvio padrão amostral \(s\). Use os dados na figura a seguir, produzidos pelo GeoGebra.

\end{enumerate}

\begin{table}[H]
\centering
\begin{tabular}{|l|f|f|}
\hline
\tcolor{} & $\tcolor{Homens}$ & $\tcolor{Mulheres}$ \\
\hline
$n$ & 100 & 100 \\
\hline
Média & 150{,}6942 & 171{,}9166 \\
\hline
$\sigma$ & 7{,}6617 & 11{,}075 \\
\hline
$s$ &  7{,}7003 & 11{,}1308 \\
\hline
Min &  130{,}88 & 146{,}88 \\
\hline
$Q_1$ &  148{,}37 & 166{,}31 \\
\hline
Mediana & 152{,}995 & 175{,}625 \\
\hline
$Q_3$ & 156{,}66 & 158{,}33 \\
\hline
Max & 158{,}33 & 185{,}15 \\
\hline
\end{tabular}
\caption{Estatísticas resumo dos 100 melhores tempos para homens e mulheres - Maratona de Nova Iorque/2017}\label{\detokenize{PE104-5:fig-summarymaratonamulheres}}\label{\detokenize{PE104-5:id3}}
\end{table}

\begin{enumerate}
\setcounter{enumi}{2}
\item {} 
Idem para estimar o valor de desvio padrão amostral dos dados da atividade \hyperref[\detokenize{PE104-3:ativ-estrategia-de-investimento}]{Estratégia de Investimento}. Use os dados na figura a seguir, produzidos pelo GeoGebra.

\end{enumerate}

\begin{table}[H]
\centering
\begin{tabular}{|l|l|l|l|}
\hline
\multicolumn{2}{|c|}{\cellcolor{\currentcolor!80}\textcolor{white}{\textbf{Companhia A}}} & \multicolumn{2}{c|}{\cellcolor{\currentcolor!80}\textcolor{white}{\textbf{Companhia A}}} \\
\hline
$n$ & 10 & $n$ & 10 \\
\hline
Média & 61.5 & Média & 61.5 \\
\hline
$\sigma$ & 4.3417 & $\sigma$ & 17.3738 \\
\hline
$s$ & 4.5765 & $s$ & 18.3136 \\
\hline
$\Sigma x$ & 615 & $\Sigma x$ & 615 \\
\hline
$\Sigma x^2$ & 28011 & $\Sigma x^2$ & 40841 \\
\hline
Min & 56 & Min & 33 \\
\hline
$Q_1$ & 57 & $Q_1$ & 48 \\
\hline
Mediana & 62 & Mediana & 62 \\
\hline
$Q_3$ & 67 & $Q_3$ & 77 \\
\hline
Max & 67 & Max & 90 \\
\hline
\end{tabular}
\caption{Estatísticas resumo das cotações das ação nas Companhias A e B.}\label{\detokenize{PE104-5:fig-estrategia}}\label{\detokenize{PE104-5:id4}}
\end{table}

\ifdefined\prof
\begin{solucao}

\begin{enumerate}
\item Da \hyperref[\detokenize{PE104-5:fig-resumonartes}]{tabela \ref{\detokenize{PE104-5:fig-resumonartes}}} vemos que $s\approx1{,}96$ e que $R=8-0{,}8=7{,}2$. Pela fórmula apresentada temos $s\approx7{,}24=1{,}8$. Comparando o valor aproximado de $s (1{,}8)$ com o valor calculado de $s (1{,}96)$ vemos que a aproximação é um pouco menor do que o valor de $s$. O erro percentual cometido por esta aproximação corresponde a $8\%$ do valor de $s$, pois $\frac{|1{,}8-1{,}96|}{1{,}96}\approx0{,}08$.

\item Da \hyperref[\detokenize{PE104-5:fig-summarymaratonamulheres}]{tabela \ref{\detokenize{PE104-5:fig-summarymaratonamulheres}}}, para a categoria homens, vemos que $s\approx7,70$ minutos e que $R=158{,}33-130{,}88=27{,}45$. Pela fórmula apresentada temos $s\approx\frac{27{,}45}{4}\approx6{,}86$ minutos. Comparando o valor aproximado de $s (6{,}86)$ com o valor calculado de $s (7{,}70)$ vemos que a aproximação é um pouco menor do que o valor de $s$. O erro percentual cometido por esta aproximação corresponde a cerca de $11\%$ do valor de $s$, pois $\frac{|6{,}86-7{,}70|}{7,70}\approx0{,}11$.

Da \hyperref[\detokenize{PE104-5:fig-summarymaratonamulheres}]{tabela \ref{\detokenize{PE104-5:fig-summarymaratonamulheres}}}, para a categoria mulheres, vemos que $s\approx11{,}13$ minutos e que $R=185{,}15-146{,}88=38{,}27$. Pela fórmula apresentada temos $s\approx38{,}274=9{,}57$ minutos. Comparando o valor aproximado de $s(9{,}57)$ com o valor calculado de $s(11{,}13)$ vemos que a aproximação é um pouco menor do que o valor de s. O erro percentual cometido por esta aproximação corresponde a cerca de $14\%$ do valor de $s$, pois $\frac{|9{,}57-11{,}13|}{11{,}13}\approx0,14$.

\item Da \hyperref[\detokenize{PE104-5:fig-estrategia}]{tabela \ref{\detokenize{PE104-5:fig-estrategia}}} vemos que, para a companhia $A$, $s\approx4{,}5765$ e que $R=67-56=11$. Pela fórmula apresentada temos $s\approx\frac{11}{4}=2{,}75$. Comparando o valor aproximado de $s(2{,}75)$ com o valor calculado de $s(4{,}5765)$ vemos que a aproximação é menor do que o valor de $s$. O erro percentual cometido por esta aproximação corresponde a $40\%$ do valor de $s$, pois $\frac{|2{,}75-4{,}5765|}{4{,}5765}\approx0{,}4$.

Da \hyperref[\detokenize{PE104-5:fig-estrategia}]{tabela \ref{\detokenize{PE104-5:fig-estrategia}}} vemos que, para a companhia $B$, $s\approx17{,}3738$ e que $R=90-33=57$. Pela fórmula apresentada temos $s\approx\frac{57}{4}=14{,}25$. Comparando o valor aproximado de $s(14{,}25)$ com o valor calculado de $s(17{,}3738)$ vemos que a aproximação é menor do que o valor de s. O erro percentual cometido por esta aproximação corresponde a $18\%$ do valor de $s$, pois $\frac{|14{,}25-17{,}3738|}{17{,}3738}\approx0{,}18$.
\end{enumerate}

\end{solucao}
\fi

\end{document}