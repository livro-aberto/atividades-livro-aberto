\begin{tikzpicture}

\end{tikzpicture}\documentclass[10 pt,usenames,dvipsnames, oneside]{article}
\usepackage{../../../modelo-ensino-medio}



\begin{document}

\begin{center}
  \begin{minipage}[l]{3cm}
\includegraphics[width=2cm]{logo}    
\end{minipage}\hfill
\begin{minipage}[r]{.8\textwidth}
 {\Large \scshape Atividade: Comparação de conjuntos de dados}  
\end{minipage}
\end{center}
\vspace{.2cm}

\ifdefined\prof
%Habilidades da BNCC
\begin{objetivos}
\item \textbf{EM13MAT409} Interpretar e comparar conjuntos de dados estatísticos por meio de diferentes diagramas e gráficos, como o histograma, o de caixa (box-plot), o de ramos e folhas, reconhecendo os mais eficientes para sua análise.
\item \textbf{EM13MAT202} Planejar e executar pesquisa amostral usando dados coletados ou de diferentes fontes sobre questões relevantes atuais, incluindo ou não, apoio de recursos tecnológicos, e comunicar os resultados por meio de relatório contendo gráficos e interpretação das medidas de tendência central e das de dispersão.
\end{objetivos}

%Caixa do Para o Professor
\begin{goals}
%Objetivos específicos
\begin{enumerate}
\item Comparar diferentes distribuições de uma mesma variável quando separada por grupos.
\end{enumerate}

\tcblower

%Orientações e sugestões
Nesta atividade serão coletados dados de uma mesma variável que possa ser separada em grupos, com o intuito de comparar as suas medidas de posição e dispersão. Sugerem-se algumas opções, dependendo do tamanho da turma e do contexto escolar, podem até ser escolhidas variáveis distintas para grupos pequenos de alunos, por exemplo, um grupo trabalha com as médias de Matemática, outro grupo trabalha com alturas, etc.

Uma vez coletados os dados, serão calculadas suas medidas de posição e dispersão e comparadas, tentando orientar os estudantes a comentar as observações e não apenas fazer os cálculos. Para a realização dos cálculos deve ser usado suporte tecnológico: calculadoras, aplicativos, etc.

O intuito é dar uma perspectiva para os estudantes da forma em que a estatística é utilizada na ciência para responder perguntas como:

\begin{itemize}
\item Uma determinada espécie vegetal cresce melhor perto de uma fonte de água ou longe da mesmo? Na sombra de uma árvore ou recebendo luz direta do sol?
\item As meninas são mais altas que os meninos numa certa idade? Acontece o mesmo em todas as idades?
\end{itemize}

De forma ideal, pode ser formulada primeiro a pergunta, e depois coletados os dados, apelando a informações encontradas num artigo científico ou numa publicação de jornal, com o intuito de tentar contrastar uma afirmação dada num texto com dados coletados diretamente.
\end{goals}

\bigskip
\begin{center}
{\large \scshape Atividade}
\end{center}
\fi

Para realizar esta atividade será necessário coletar dois conjuntos de dados da mesma natureza, correspondentes a grupos distintos, os quais queremos comparar. Por exemplo:
\begin{itemize}
\item {} 
alturas de homens e mulheres;

\item {} 
alturas de alunos de 1º e de 9º ano do Ensino Fundamental;

\item {} 
notas de disciplinas distintas;

\item {} 
notas de turmas distintas na mesma disciplina;

\item {} 
medições de produtos naturais: comprimento das folhas de vegetais (alface, rúcula, etc) comprados em lojas distintas, altura de árvores ou plantas similares locais da cidade distintos;

\end{itemize}

entre outros que podem ser escolhidos dependendo da região e dos recursos disponíveis na escola.

No seu caderno ou em uma planilha eletrônica, registre os dados coletados, como indicado no modelo de tabela a seguir, lembrando que quanto mais dados você coletar com os critérios definidos, os resultados do experimento terão maior chance de refletir a realidade.

\begin{table}[H]
\centering

\begin{tabular}{|c|c|}
\hline
\tmcol{2}{|c|}{Variável: altura em cm} \\
\hline
\tcolor{Turma A} & \tcolor{Turma B} \\
\hline
$155$ & $165$ \\
\hline
$168$ & $159$ \\
\hline
$\vdots$ & $\vdots$ \\
\hline
\end{tabular}
\end{table}

Para calcular as medidas de posição e dispersão, utilize de forma cuidadosa as  fórmulas apresentadas. De forma alternativa, você pode digitar os dados no \href{https://ggbm.at/KbYqnQ6Q}{Aplicativo de medidas de posição e dispersão do Livro Aberto} e obter as medidas resumo dos dados.

\begin{table}[H]
\centering
\caption{Registre os seus resultados}
\begin{tabular}{|l|c|c|}
\hline
\tcolor{Nome da categoria} & \tcolor{Grupo A} & \tcolor{Grupo B} \\
\hline
Mínimo (\(x_{(1)}\)) & & \\
\hline
Máximo  (\(x_{(n)}\)) & & \\
\hline
Média & & \\
\hline
$Q_1 $& & \\
\hline
Mediana & & \\
\hline
$Q_3$ & & \\
\hline
Amplitude amostral ($R$) & & \\
\hline
Dist. entre quartis ($DQ$) & & \\
\hline
Desvio médio absoluto ($DM$) & & \\
\hline
Variância amostral (\(s^2\)) & & \\
\hline
Desvio padrão amostral (\(s\)) & & \\
\hline
\end{tabular}
\end{table}

Sugere-se a construção dos histogramas para comparar os dois grupos. Você pode usar o GeoGebra para esta construção.
\begin{enumerate}
\item {} 
Discuta as suas observações com a turma. Lembre-se de interpretar as medidas de dispersão e não apenas as de posição, que informação adicional oferecem?

\item {} 
Analisando os dois conjuntos de dados obtidos, que medida de posição você julga mais adequada para resumir a informação do conjunto? Por quê?

\item Os resultados que você obteve parecem refletir a realidade? Existe algum resultado científico que suporte estas observações? Consulte professores de outras áreas sobre suas conclusões.

\end{enumerate}

\ifdefined\prof
% \begin{solucao}

% \begin{enumerate}
% \item
% \end{enumerate}

% \end{solucao}
\fi

\end{document}