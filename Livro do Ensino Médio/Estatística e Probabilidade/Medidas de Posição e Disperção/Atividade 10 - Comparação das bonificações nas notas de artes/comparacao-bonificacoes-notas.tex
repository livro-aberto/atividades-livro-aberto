\documentclass[10 pt,usenames,dvipsnames, oneside]{article}
\usepackage{../../../modelo-ensino-medio}



\begin{document}

\begin{center}
  \begin{minipage}[l]{3cm}
\includegraphics[width=2cm]{logo}    
\end{minipage}\hfill
\begin{minipage}[r]{.8\textwidth}
 {\Large \scshape Atividade: Comparação das bonificações nas Notas de Artes}  
\end{minipage}
\end{center}
\vspace{.2cm}

\ifdefined\prof
%Habilidades da BNCC
\begin{objetivos}
\item \textbf{EM13MAT316} Resolver e elaborar problemas, em diferentes contextos, que envolvem cálculo e interpretação das medidas de tendência central (média, moda, mediana) e das de dispersão (amplitude, variância e desvio padrão).
\item \textbf{EM13MAT409} Interpretar e comparar conjuntos de dados estatísticos por meio de diferentes diagramas e gráficos, como o histograma, o de caixa (box-plot), o de ramos e folhas, reconhecendo os mais eficientes para sua análise.
\end{objetivos}

%Caixa do Para o Professor
\begin{goals}
%Objetivos específicos
\begin{enumerate}
\item Avaliar o efeito no coeficiente de variação de um conjunto de dados quando realizamos transformações de adição de uma constante e de multiplicação por uma constante.
\end{enumerate}

\tcblower

%Orientações e sugestões
Nesta atividade pretende-se retornar ao item \titem{e)} da \hyperref[\detokenize{PE104-0:ativ-notas-de-artes}]{Notas de Arte} quando foi perguntado ao estudante o que ele achava melhor: ganhar um ponto ou um acréscimo de $20\%$ em sua nota. A ideia será propor a mesma pergunta de um ponto de vista do professor, que prefere que a distribuição das notas apresente o menor coeficiente de variação.
\end{goals}

\bigskip
\begin{center}
{\large \scshape Atividade}
\end{center}
\fi

Vamos retornar à atividade \hyperref[\detokenize{PE104-0:ativ-notas-de-artes}]{Notas de Arte} e às duas possibilidades de bonificação das notas: acrescentar um ponto a todos os alunos ou aumentar em 20\% a nota de cada aluno. Suponha, que o professor deseja que o resultado geral de sua turma apresente o menor coeficiente de variação. Partindo deste ponto de vista, qual das duas possibilidades é mais interessante para o professor adotar?

Para facilitar, use as informações a seguir.

\begin{table}[H]
\centering

\begin{tabular}{|l|c|c|c|}
\hline
$\tmat{n=35}$ & \tcolor{Antes} & \tcolor{1 pt} & $\tmat{20\%}$ \\
\hline
\(\sum x\) & 207,5 & 242,5 & 249,0 \\
\hline
\(\sum x^2\) & 1361,39 & 1811,39 & 1960,402 \\
\hline
\end{tabular}

\caption{Dados sobre as somas simples e somas de quadrados das notas antes da bonificação (antes), após serem acrescidas de um ponto (1 pt) e após serem aumentadas em 20\% (20\%)}
\end{table}

\ifdefined\prof
\begin{solucao}

O professor deverá escolher o aumento de um ponto para cada estudante, pois esta bonificação acarretará num coeficiente de variação menor, implicando em maior homogeneidade da turma em relação à média, conforme os cálculos a seguir.

Considerando o acréscimo de um ponto a todos os alunos temos que a média passa a ser $\bar{x}=\dfrac{242{,}5}{35}\approx6{,}93$. A variância, calculada por $s^2$ é dada por $\dfrac{1811{,}39-35\cdot6{,}93^2}{35-1}\approx3{,}84$ e, o desvio padrão, s≈1,96. Assim, o coeficiente de variação da turma, resultante desta bonificação será dado por $CV=\dfrac{1{,}96}{6{,}93}\cdot100\approx28\%$.

Considerando um aumento de $20\%$ para cada nota temos que a média passa a ser $\bar{x}=\dfrac{249{,}0}{35}\approx7{,}11$. A variância, calculada por $s^2$ é dada por $\dfrac{1960{,}402-35\cdot7{,}11^2}{35-1}\approx5{,}56$ e, o desvio padrão, $s\approx2{,}36$. Assim, o coeficiente de variação da turma, resultante desta bonificação será dado por $CV=\dfrac{2{,}36}{7{,}11}\cdot100\approx33$\%.

\end{solucao}
\fi

\end{document}