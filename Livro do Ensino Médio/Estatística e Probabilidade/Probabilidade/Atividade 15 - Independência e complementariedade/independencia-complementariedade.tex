\documentclass[10 pt,usenames,dvipsnames, oneside]{article}
\usepackage{../../../modelo-ensino-medio}



\begin{document}

\begin{center}
  \begin{minipage}[l]{3cm}
\includegraphics[width=2cm]{logo}    
\end{minipage}\hfill
\begin{minipage}[r]{.8\textwidth}
 {\Large \scshape Atividade: Independência e complementariedade}  
\end{minipage}
\end{center}
\vspace{.2cm}

\ifdefined\prof
%Habilidades da BNCC
\begin{objetivos}
\item a
\end{objetivos}

%Caixa do Para o Professor
\begin{goals}
%Objetivos específicos
\begin{enumerate}
\item Aplicar o conceito de independência entre dois eventos para entender que o respectivos eventos complementares herdam essa a condição de eventos independentes.
\end{enumerate}

\tcblower

%Orientações e sugestões
Essa atividade é um exercício teórico de dedução muito simples que revela uma propriedade importante entre eventos independentes: dada uma coleção de eventos independentes, se para alguns eventos (ou todos) considerarmos os seus complementares em vez do próprio, a coleção continua independente. Com fins de simplificação e não tornar o processo complicado, consideraremos nesta atividade apenas o caso para dois eventos independentes. Mas, de fato, o resultado vale para quaisquer coleções de eventos independentes.
\end{goals}

\bigskip
\begin{center}
{\large \scshape Atividade}
\end{center}
\fi

Sejam \(A\) e \(B\) dois eventos independentes.  Mostre que os pares de eventos a seguir também são independentes:
\begin{enumerate}
\item {} 
\(\overline{A}\) e \(\overline{B}\);

\item {} 
\(A\) e \(\overline{B}\); e

\item {} 
\(\overline{A}\) e \(B\).

\end{enumerate}

\ifdefined\prof
\begin{solucao}

\begin{enumerate}
\item Por hipótese temos que $P(A\cap B)=P(A)\cdot P(B)$, pois $A$ e $B$ são independentes. Queremos provar que $\overline{A}$ e $\overline{B}$ também são independentes, ou equivalentemente, que $P(\overline{A}\cap \overline{B})=P(\overline{A})\cdot P(\overline{B})$.
Pelas Leis de De Morgan trabalhadas em atividade anterior, sabemos que $\overline{A}\cap \overline{B}=(\overline{A\cup B})$. Portanto, usando a propriedade do evento complementar, podemos escrever $P(\overline{A}\cap \overline{B})=1−P(A\cup B)$.

Mas, 
\begin{equation*}
P(A\cup B)=P(A)+P(B)-P(A\cap B)=P(A)+P(B)-P(A)\cdot P(B)
\end{equation*}
ela independência entre $A$ e $B$. Assim,
\begin{align*}
P(\overline{A}\cap \overline{B})&=1-P(A)-P(B)+P(A)\cdot P(B)\\ 
&=1-P(A)-P(B)\cdot(1−P(A))\\
&=(1−P(A))\cdot(1-P(B))\\
&=P(\overline{A})\cdot P(\overline{B})
\end{align*}.
Portanto, se $A$ e $B$ são independentes, então $\overline{A}$ e $\overline{B}$ também são independentes.

\item Por hipótese temos que $P(A\cap B)=P(A)\cdot P(B)$, pois $A$ e $B$ são independentes. Queremos provar que $A$ e $\overline{B}$ também são independentes, ou equivalentemente, que $P(A\cap \overline{B})=P(A)\cdot P(\overline{B})$. Observe que podemos escrever $A=(A\cap B)\cup (A\cap\overline{B})$ com os dois eventos do lado direito sendo disjuntos. Assim, $P(A)=P(A\cap B)+P(A\cap \overline{B})$ implicando que $P(A\cap \overline{B})=P(A)-P(A\cap B)=P(A)-P(A)\cdot P(B)$ pela independedência de $A$ e $B$. Logo, $P(A\cap \overline{B})=P(A)\cdot(1−P(B))=P(A)\cdot P(\overline{B})$ e, portanto, se $A$ e $B$ são independentes, então $A$ e $\overline{B}$ também são.
\item Idem ao item anterior, bastando trocar de posição as letras $A$ e $B$.
\end{enumerate}

\end{solucao}
\fi

\end{document}