\documentclass[10 pt,usenames,dvipsnames, oneside]{article}
\usepackage{../../../modelo-ensino-medio}



\begin{document}

\begin{center}
  \begin{minipage}[l]{3cm}
\includegraphics[width=2cm]{logo}    
\end{minipage}\hfill
\begin{minipage}[r]{.8\textwidth}
 {\Large \scshape Atividade: Avaliando probabilidades a partir de um histograma}  
\end{minipage}
\end{center}
\vspace{.2cm}

\ifdefined\prof
%Habilidades da BNCC
\begin{objetivos}
\item a
\end{objetivos}

%Caixa do Para o Professor
\begin{goals}
%Objetivos específicos
\begin{enumerate}
\item Avaliar probabilidades de eventos usando dados quantitativos coletados, representados em um histograma e em uma tabela de frequências.
\end{enumerate}

\tcblower

%Orientações e sugestões
\begin{itemize}
\item A partir de dados coletados e representados por meio de um histograma, os alunos deverão responder perguntas referentes a probabilidades de certos eventos e para isso deverão usar a noção frequentista de probabilidade.
\end{itemize}
\end{goals}

\bigskip
\begin{center}
{\large \scshape Atividade}
\end{center}
\fi

No capítulo \textbf{Medidas de posição e dispersão} foram trabalhados os dados sobre os 100 melhores tempos atingidos na Maratona de Nova Iorque (2017) para as categorias homens e mulheres. Na \hyperref[maratona-homens-prob]{figura \ref{maratona-homens-prob}}, apresenta-se um histograma construído para os 100 melhores tempos na maratona de Nova Iorque (2017) para a categoria homens, após a conversão destes tempos para minutos.

\begin{figure}[H]
\centering

\begin{tikzpicture}[xscale=0.5,yscale=.8, scale=0.3]

\draw (-0.2,0) -- (33.5,0);
\draw (0,0) -- (0,25);

\foreach \x in {0,5,10,15,20,25}  \draw (0,\x) -- (-0.5,\x) node [above, rotate=90] at (-0.3,\x) {\x}  
;


\foreach \x/\y in {3/7,6/4,9/1,12/2,15/4,18/4,21/14,24/21,27/24,30/19}{ \draw [fill=\currentcolor!80] (\x,0) rectangle (\x+3,\y);
\node [above, align=center, scale=.9] at (\x+1.5,\y) {$(\y)$};}

\foreach \x/\y in {3/130,6/\quad,9/136,12/\quad,15/142,18/\quad,21/148,24/\quad,27/153,30/\quad,33/160} \draw (\x,0) -- (\x,-0.25) node [below] {\y};

\foreach \x in {0,5,10,15,20,25}  \draw [dashed] (0,\x) -- (33.5,\x)
;

\node [rotate=90] at (-4,12.5) {frequência absoluta};
\node  at (16.5,-3) {tempos em minutos};
\node [align=center] at (15.6, 27.5) {Histograma dos 100 melhores tempos na categoria \\ Maratona de Nova Iorque - 2017};


\end{tikzpicture}
\caption{Histograma dos 100 melhores tempos para homens na Maratona de Nova Iorque (2017), destacando a frequência absoluta de cada intervalo de classe.}
\label{maratona-homens-prob}
\end{figure}


Na \hyperref[maratonatabela]{
tabela \ref{maratonatabela}} são apresentados os intervalos de classe e suas respectivas frequências, usados na construção do histograma da \hyperref[maratona-homens-prob]{figura \ref{maratona-homens-prob}}. Os intervalos considerados são fechados à esquerda e abertos à direita.

\begin{table}[H]
\centering
\begin{tabu} to \textwidth{|c|c|c|}
\hline
\thead
\parbox[c][1cm]{3.5cm}{\centering Intervalo de classe} & \parbox[c][1cm]{3.5cm}{\centering Frequência Absoluta} & \parbox[c][1cm]{3.5cm}{\centering Frequência Relativa} \\
\hline\relax
$[130,0;133,0[$ & $7$ & $0{,}07$ \\
\hline
$[133,0;136,0[$ & $4$ & $0{,}04$ \\
\hline
$[136,0;139,0[$ & $1$ & $0{,}01$ \\
\hline
$[139,0;142,0[$ & $2$ & $0{,}02$ \\
\hline
$[142,0;145,0[$ & $4$ & $0{,}04$ \\
\hline
$[145,0;148,0[$ & $4$ & $0{,}04$ \\
\hline
$[148,0;151,0[$ & $14$ & $0{,}14$ \\
\hline
$[151,0;154,0[$ & $21$ & $0{,}21$ \\
\hline
$[154,0;157,0[$ & $24$ & $0{,}24$ \\
\hline
$[157,0;160,0[$ & $19$ & $0{,}19$ \\
\hline
\end{tabu}
\caption{Distribuição de frequências dos 100 melhores tempos na categoria homens da maratona de Nova Iorque (2017)}
\label{maratonatabela}
\end{table}

Suponha que o comportamento dos 100 melhores tempos para homens na Maratona de Nova Iorque (2017) represente bem os 100 melhores tempos para homens em qualquer Maratona de Nova Iorque.

Com base nessa suposição, estime a probabilidade de que na próxima maratona de Nova Iorque o tempo de conclusão da corrida, entre os 100 melhores na categoria homens,
\begin{enumerate}
\item {} 
ocorra entre $157{,}0$ e $160{,}0$ minutos;

\item {} 
seja inferior a $154{,}0$ minutos;

\item {} 
seja superior a $152{,}5$ minutos;

\item {} 
caia entre $152{,}5$ e $158$ minutos.

Observação: Para responder os dois últimos itens, suponha, em cada intervalo, que as frequências obervadas são proporcionais aos comprimentos dos intervalos, para poder avaliar frequências em subintervalos.

\end{enumerate}

\ifdefined\prof
\begin{solucao}

Para responder as questões desta atividade será usada a interpretação frequentista de probabilidade.

\begin{enumerate}
\item Pela tabela de frequências observa-se que $19$ tempos de chegada dos $100$ melhores estão no intervalo dado, portanto, a probabilidade será $\frac{19}{100}=0{,}19$.

\item Pela tabela de frequências observa-se que há $100−24−19=57$ tempos de chegada entre os $100$ melhores que são inferiores a $154{,}0$ minutos. Portanto a probabilidade será $\frac{57}{100}=0{,}57$. A resposta poderia ser obtida também, somando-se as frequências dos intervalos de classe do início até o intervalo $[151{,}0 ; 154{,}0[$: $7+4+1+2+4+4+14+21=57$, obtendo-se $0{,}57$ como probabilidade.

\item Pela tabela, vemos que $152{,}5$ minutos não é extremo de intervalo na construção apresentada, desse modo podemos concluir que a probabilidade solicitada será um número no intervalo $]0{,}43 ; 0{,}64[$. Observe que $152{,}5$ está no intervalo $[151{,}0 ; 154{,}0[$ de comprimento 3, cuja frequência absoluta é $21$. Observe também que o subintervalo $[152{,}5; 154{,}0[$ de $[151{,}0 ; 154{,}0[$ tem comprimento $1{,}5$. Usando a suposição de proporcionalidade da frequência em cada intervalo por unidade de comprimento do intervalo, podemos a aproximar a frequência do subintervalo por $21\times\frac{1{,}5}{3}=10{,}5$. Logo, a probabilidade será obtida por $\frac{10{,}5+24+19}{100}=\frac{53{,}5}{100}=0{,}535$ que de fato é um número no intervalo inicialmente obtido.

\item Pela tabela, nem $152,5$ nem $158,0$ são extremos de intervalo, porém podemos obter um limite superior para a probabilidade solicitada, que neste caso será um número menor do que $0,64$ $\big(\frac{21+24+19}{100}\big)$. Para um valor pontual, serão necessárias duas aproximações de frequências. A primeira no intervalo $[152{,}5; 154{,}0[$ já realizada no item anterior que foi aproximada para $10{,}5$. A frequência do intervalo $[154{,}0 ; 157{,}0 [$ é $24$. Para o intervalo $[157{,}0 ; 158{,}0[$ calcularemos uma frequência aproximada, usando o argumento da proporcionalidade. O intervalo $[157{,}0 ; 160{,}0[$ tem comprimento $3$ com frequência $10$, então, uma aproximação para o subintervalo $[157{,}0 ; 158{,}0[$ de $[157{,}0 ; 160{,}0[$ de comprimento $1$ é dada por $19\times\frac{1}{3}\approx6{,}3$. Assim, a probabilidade será $\frac{10{,}5+24+6{,}3}{100}=\frac{40{,}8}{100}=0{,}408$.
\end{enumerate}

\end{solucao}
\fi

\end{document}