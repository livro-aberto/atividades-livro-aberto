\documentclass[10 pt,usenames,dvipsnames, oneside]{article}
\usepackage{../../../modelo-ensino-medio}



\begin{document}

\begin{center}
  \begin{minipage}[l]{3cm}
\includegraphics[width=2cm]{logo}    
\end{minipage}\hfill
\begin{minipage}[r]{.8\textwidth}
 {\Large \scshape Atividade: Não determinístico (aleatório) ou determinístico?}  
\end{minipage}
\end{center}
\vspace{.2cm}

\ifdefined\prof
%Habilidades da BNCC
\begin{objetivos}
\item a
\end{objetivos}

%Caixa do Para o Professor
\begin{goals}
%Objetivos específicos
\begin{enumerate}
\item Reconhecer fenômenos aleatórios, distinguindo-os de fenômenos determinísticos.
\end{enumerate}

\end{goals}

\bigskip
\begin{center}
{\large \scshape Atividade}
\end{center}
\fi

Classifique cada experimento a seguir em aleatório ou determinístico.

Deseja-se observar:
\begin{enumerate}[rightmargin=3mm]
\item 
O valor constante de cada prestação quando se financia um eletrodoméstico, estabelecendo-se a taxa de juros efetiva ao mês, a quantidade de meses do financiamento e o pagamento da primeira prestação no ato da compra.

\item 
A quantidade de metros cúbicos de água consumida em sua residência no primeiro semestre do próximo ano.


\item 
A distância percorrida por um objeto em movimento, conhecendo-se a velocidade e o tempo transcorrido.

\item 
O valor a ser pago na conta de luz da sua residência no próximo mês.

\item 
A sua média final em Matemática desse ano.

\end{enumerate}

\ifdefined\prof
\begin{solucao}

\begin{enumerate}
\item determinísitco
\item Dleatório
\item Determinísitico
\item Aleatório
\item Aleatório
\end{enumerate}

\end{solucao}
\fi

\end{document}