\documentclass[10 pt,usenames,dvipsnames, oneside]{article}
\usepackage{../../../modelo-ensino-medio}



\begin{document}

\begin{center}
  \begin{minipage}[l]{3cm}
\includegraphics[width=2cm]{logo}    
\end{minipage}\hfill
\begin{minipage}[r]{.8\textwidth}
 {\Large \scshape Atividade: Simulação do lançamento de um dado honesto}  
\end{minipage}
\end{center}
\vspace{.2cm}

\ifdefined\prof
%Habilidades da BNCC
\begin{objetivos}
\item a
\end{objetivos}

%Caixa do Para o Professor
\begin{goals}
%Objetivos específicos
\begin{enumerate}
\item Aplicar o modelo probabilístico equiprovável e usar tecnologia para realizar simulações de um fenômeno aleatório.
\end{enumerate}

\tcblower

%Orientações e sugestões
Essa atividade demanda o uso de tenologia. Sugere-se realizá-la em laboratório de informática. Sugere-se também que os alunos trabalhem em pequenos grupos de dois ou três alunos para cada computador disponível. As atividades poderão ser adaptadas de acordo com o conhecimento prévio dos estudantes. Nesssa atividade poderá ser usado exemplo dado anteriormente, gerando-se um número aleatório entre $1$ e $6$ e, depois, contando as quantidades obtidas de cada face.
\end{goals}

\bigskip
\begin{center}
{\large \scshape Atividade}
\end{center}
\fi

Deseja-se simular o lançamento de um dado honesto uma grande quantidade de vezes e comparar a frequência relativa de faces “6”{} com a probabilidade teórica \(\frac{1}{6}\approx 0,167\) de obter uma face “6”{} quando o dado é honesto.

\begin{figure}[H]
\centering

\noindent\includegraphics[width=225bp]{{lancamento_dado}.png}
\end{figure}
\begin{enumerate}
\item {} 
Usando o GeoGebra ou algum outro recurso tecnológico, simule 30 lançamentos do dado e observe a quantidade de faces “6”{} obtidas, calculando a frequência relativa.

\item {} 
Repita a simulação para 60, 120, 300 e 1500 lançamentos do dado.

\item {} 
Complete o quadro a seguir e comente sobre os resultados que você obteve.

\end{enumerate}
\begin{table}[H]
\centering
\begin{tabu} to \textwidth{|c|c|}
\hline
\thead
\centering
Número de Observações & Frequência relativa de 6 \\
\hline
30 &\\
\hline
60 &\\
\hline
120 &\\
\hline
300 &\\
\hline
1500 &\\
\hline
\end{tabu}
\end{table}

\ifdefined\prof
\begin{solucao}

\begin{enumerate}
\item Você pode realizar a simulação usando a planilha do GeoGebra e a função \textit{=NúmeroAleatório(1,6)} na célula A1. Essa função retornará, com probabilidades iguais, um entre os números 1,2,3,4,5 e 6. Depois, arraste, copiando esta função para mais 29 células A2 até A30, obtendo as 30 simulações. Veja exemplo no início dessa seção. Depois use a função \textit{=ContarSe(5<x<7,A1:A30)}. É claro que as respostas irão variar, dependendo da simulação. Mas, espera-se que o número de faces “6” obtidas oscile em torno de 5, pois a função gera o número $6$ com probabilidade $\frac{1}{6}$ e assim, em média espera-se obter $30\cdot16=5$ dígitos $6$.

\item Idem ao item anterior, só que agora o número de células a ser considerado na planilha será $60$, $120$, $300$ e $1500$.

\item No preenchimento da tabela você deverá perceber que a medida que o número de simulações é maior, a frequência relativa de faces 6 se aproxima da probabilidade teórica de obter uma face 6 ($\approx0{,}167$). Se de fato o gerador de números aleatórios do programa que você está usando é bom, esse é o resultado esperado. Por exemplo, em uma simulação com o GeoGebra foram observadas as seguintes frequências relativas de faces 6 conforme o número de lançamentos:

\begin{table}[H]
\centering

\begin{tabular}{|f|f|}
\hline
$\tcolor{Número de Observações}$ & $\tcolor{Frequência relativa de 6's}$ \\
\hline
30 & 0{,}17 \\
\hline
60 & 0{,}18 \\
\hline
120 & 0{,}18 \\
\hline
300 & 0{,}16 \\
\hline
1500 & 0{,}17 \\
\hline
\end{tabular}
\end{table}
\end{enumerate}

\end{solucao}
\fi

\end{document}