\documentclass[10 pt,usenames,dvipsnames, oneside]{article}
\usepackage{../../../modelo-ensino-medio}



\begin{document}

\begin{center}
  \begin{minipage}[l]{3cm}
\includegraphics[width=2cm]{logo}    
\end{minipage}\hfill
\begin{minipage}[r]{.8\textwidth}
 {\Large \scshape Atividade: Desempenho de exames diagnósticos}  
\end{minipage}
\end{center}
\vspace{.2cm}

\ifdefined\prof
%Habilidades da BNCC
\begin{objetivos}
\item a
\end{objetivos}

%Caixa do Para o Professor
\begin{goals}
%Objetivos específicos
\begin{enumerate}
\item Reconhecer uma probabilidade condicional.
\end{enumerate}

\tcblower

%Orientações e sugestões
\begin{itemize}
\item Essa atividade envolve uma adaptação de uma questão do ENEM (2014).
\end{itemize}
\end{goals}

\bigskip
\begin{center}
{\large \scshape Atividade}
\end{center}
\fi

Testes diagnósticos para detectar uma doença não são infalíveis. Para analisar o desempenho de um desses testes, realizam-se estudos em populações, contendo pessoas sãs e portadoras da doença.

\begin{figure}[H]
\centering

\noindent\includegraphics[width=200bp]{{amostras_sangue}.png}


\caption{Amostras de sangue para realização de exame}
\end{figure}


Quatro situações distintas podem ocorrer
\begin{enumerate}
\item {} 
a pessoa TEM a doença e o resultado do teste é POSITIVO.

\item {} 
a pessoa TEM a doença e o resultado do teste é NEGATIVO.

\item {} 
a pessoa NÃO TEM a doença e o resultado do teste é POSITIVO.

\item {} 
a pessoa NÃO TEM a doença e o resultado do teste é NEGATIVO.

\end{enumerate}

Observe que nas situações \titem{b)} e \titem{c)}, o teste falha, pois deveria ser positivo quando a pessoa tem a doença e, negativo, quando a pessoa não tem a doença. Já nas situações \titem{a)} e \titem{d)} o teste acerta o diagnóstico.

Dois índices de desempenho para avaliação de um teste diagnóstico costumam ser usados: a sensibilidade e especificidade.

A \textbf{sensibilidade} é definida como a probabilidade de o resultado do teste ser POSITIVO, dado que a pessoa examinada tem a doença. Já a \textbf{especificidade} é a probabilidade do teste ser NEGATIVO, dado que a pessoa examinada não tem a doença.


O quadro a seguir refere-se a um teste diagnóstico para a doença \(X\), aplicado em uma amostra composta por duzentas pessoas, sendo 100 sadias e 100 portadoras da doença \(X\).

\begin{table}[H]
\centering
\begin{tabu} to \textwidth{|l|l|l|}
\hline
\thead
Resultado do teste & doente & sadia \\
\hline
Positivo & 95 & 15 \\
\hline
Negativo & 5 & 85 \\
\hline
\end{tabu}
\end{table}


Uma pessoa entre as duzentas dessa amostra será sorteada.
\begin{enumerate}
\item {} 
Qual a probabilidade de ela tenha a doença \(X\)?

\item {} 
Qual a probabilidade de que ela NÃO tenha a doença \(X\)?

\item {} 
Se o resultado do teste da pessoa sorteada foi positivo, calcule a probabilidade de que ela tenha a doença.

\item {} 
Se o resultado do teste da pessoa sorteada foi negativo, calcule a probabilidade de que ela tenha a doença.

\item {} 
Sabendo que a pessoa sorteada tem a doença, qual a probabilidade de seu teste ter resultado positivo?

\item {} 
Sabendo que a pessoa sorteada  NÃO tem a doença, qual a probabilidade de seu teste ter resultado negativo?

\item {} 
Determine uma estimativa da sensibilidade e da especificidade desse teste, usando a informação do quadro acima.

\end{enumerate}

\ifdefined\prof
\begin{solucao}

Defina 
\begin{itemize}
\item $A$: o evento "a pessoa sorteada tem a doença $X$"{} e 
\item $B$: o evento "o resultado do teste da pessoa sorteada foi positivo".
\end{itemize}

\begin{enumerate}
\item Entre os duzentos indivíduos, $100$ têm a doença $X$, logo $P(A)=\frac{100}{200}=0{,}5$.
\item $P(\overline{A})=1-0,5=0,5$.
\item $P(A|B)=\frac{P(A\cap B)}{P(B)}=\frac{95}{110}\approx0{,}864$.
\item $P(A|\overline{B})=\frac{5}{90}\approx0,056$.
\item Entre as $100$ pessoas doentes, $95$ resultados foram positivos, logo $P(B|A)=0{,}95$.
\item Entre as $100$ pessoas sadias, 85 resultados foram negativos, logo $P(\overline{B}|\overline{A})=0,85$.
\item $0{,}95$ e $0{,}85$, respectivamente, conforme os itens \titem{c)} e \titem{d)}.
\end{enumerate}

\end{solucao}
\fi

\end{document}