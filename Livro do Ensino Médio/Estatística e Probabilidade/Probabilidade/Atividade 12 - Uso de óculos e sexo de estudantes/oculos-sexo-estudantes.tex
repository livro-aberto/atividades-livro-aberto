\documentclass[10 pt,usenames,dvipsnames, oneside]{article}
\usepackage{../../../modelo-ensino-medio}



\begin{document}

\begin{center}
  \begin{minipage}[l]{3cm}
\includegraphics[width=2cm]{logo}    
\end{minipage}\hfill
\begin{minipage}[r]{.8\textwidth}
 {\Large \scshape Atividade: Uso de óculos e sexo de estudandes}  
\end{minipage}
\end{center}
\vspace{.2cm}

\ifdefined\prof
%Habilidades da BNCC
\begin{objetivos}
\item a
\end{objetivos}

%Caixa do Para o Professor
\begin{goals}
%Objetivos específicos
\begin{enumerate}
\item Reconhecer que a probabilidade de um evento pode se alterar, conhecendo-se uma informação parcial do fenômeno sob investigação.
\item Aplicar a definição de probabilidade condicional para reconhecer eventos independentes e eventos dependentes.
\end{enumerate}

\tcblower

%Orientações e sugestões
Nesta atividade uma tabela de dupla entrada será fornecida para verificar uma possível relação entre usar óculos e sexo de um estudante do Ensino Médio. Recomenda-se construir esta mesma tabela com os dados dos alunos de sua turma e responder aos itens, usando esses dados.

Como sugestão de discussão, sugere-se uma pesquisa na internet para investigar a proporção de jovens que usa óculos. Por exemplo, em 21 de maio de 2018, foi publicada a seguinte reportagem "\href{https://www.dn.pt/sociedade/interior/miopia-aumenta-nos-jovens-e-a-culpa-e-da-falta-de-sol-e-dos-computadores-5656709.html}{Miopia aumenta nos jovens e a culpa é da falta de sol e dos computadores}”
\end{goals}

\bigskip
\begin{center}
{\large \scshape Atividade}
\end{center}
\fi

Na tabela a seguir estão os dados de uma turma de segundo ano do Ensino Médio com 40 alunos quanto ao gênero e se ele usa ou não óculos.

\begin{table}[H]
\centering
\begin{tabu} to \textwidth{|l|c|c|c|}
\hline
\thead
gênero & usa óculos & não usa óculos & total \\
\hline
feminino & $6$ & $16$ & $22$ \\ 
\hline
masculino & $5$ & $13$ & $18$ \\
\hline
total & $11$ & $29$ & $40$ \\
\hline
\end{tabu}
\end{table}

Se um estudante desta turma é sorteado, pede-se determinar a probabilidade de que ele
\begin{enumerate}
\item {} 
use óculos;

\item {} 
use óculos, sabendo que é do gênero feminino;

\item {} 
use óculos, sabendo que é do gênero masculino;

\item {} 
seja do gênero feminino;

\item {} 
seja do gênero feminino, sabendo que usa óculos;

\item {} 
seja do gênero feminino, sabendo que não usa óculos.

\item {} 
Analisando os dados da tabela e as respostas obtidas, há razões para supor que gênero é independente de uso de óculos ou não? Por quê?

\end{enumerate}

\ifdefined\prof
\begin{solucao}

Defina os seguintes eventos

\begin{itemize}
\item $M$: “estudante do gênero masculino”, 
\item $F$: “estudante do sexo feminino”,
\item $O$: “estudante usa óculos”{} e 
\item $\overline{O}$: “estudante não usa óculos”.
\end{itemize}

\begin{enumerate}
\item $P(O)=\frac{11}{40}=0{,}275$
\item Há $22$ alunos do gênero feminino e 6 usam óculos. Assim, a probabilidade é $\frac{6}{22}\approx0{,}273$.
\item Há $18$ alunos do gênero masculino e $5$ usam óculos. Assim, a probabilidade é $\frac{5}{18}\approx0{,}278$.
\item $P(F)=\frac{22}{40}=0{,}55$
\item Há $11$ alunos que usam óculos e 6 são do gênero feminino. Assim, a probabilidade é $\frac{6}{11}\approx0{,}545$.
\item Há $29$ alunos que não usam óculos e $16$ são do gênero feminino. Assim, a probabilidade é $\frac{16}{29}\approx0{,}552$.
\item Sem discriminar por gênero, a probabilidade de usar óculos é $0{,}275$. Discriminando por sexo, observa-se que a probabilidade de uma menina usar óculos é $0{,}272$ e de um menino usar óculos é $0{,}278$. Sem discriminar por uso de óculos, a probabilidade de ser uma menina é $0{,}55$. Discriminando por uso de óculos, observa-se que a probabilidade de ser uma menina entre os alunos que usam óculos é $0{,}545$ e a probabilidade de ser uma menina entre os que não usam óculos é $0{,}552$, ou seja, ambas aproximadamente iguais a probabilidade de ser uma menina sem levar em conta o uso de óculos. Portanto, independentemente, de sexo, a probabilidade de usar óculos é aproximadamente $0{,}275$ e independentemente de uso de óculos, a probabilidade de ser uma menina é aproximadamente $0{,}55$. Como estamos lidando com uma amostra da população, probabilidades gerais, por exemplo “uso de óculos”{} e probabilidades parciais “uso de óculos segundo o sexo”{} aproximadamente iguais revelam a independência no sentido estatístico dos eventos considerados.
\end{enumerate}

\end{solucao}
\fi

\end{document}