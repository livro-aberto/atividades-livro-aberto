\documentclass[10 pt,usenames,dvipsnames, oneside]{article}
\usepackage{../../../modelo-ensino-medio}



\begin{document}

\begin{center}
  \begin{minipage}[l]{3cm}
\includegraphics[width=2cm]{logo}    
\end{minipage}\hfill
\begin{minipage}[r]{.8\textwidth}
 {\Large \scshape Atividade: Avaliando probabilidades}  
\end{minipage}
\end{center}
\vspace{.2cm}

\ifdefined\prof
%Habilidades da BNCC
\begin{objetivos}
\item a
\end{objetivos}

%Caixa do Para o Professor
\begin{goals}
%Objetivos específicos
\begin{enumerate}
\item Reconhecer diferentes interpretações da probabilidade (clássica, frequentista e subjetiva).
\end{enumerate}

\tcblower

%Orientações e sugestões

Nesta atividade serão apresentados três blocos de três itens cada. Em cada item será solicitada a probabilidade de um determinado evento. No bloco I será adequado adotar a interpretação clássica da probabilidade, de modo que as respostas deverão surgir de forma natural e sem problemas. É importante discutir com os alunos como foram obtidas as respostas do bloco I. No bloco II, a interpretação adequada de probabilidade será a frequentista. Neste bloco, não existe a resposta certa. O objetivo neste caso é fazer o aluno pensar, pois tratam-se de situações aleatórias para as quais faz sentido atribuir uma probabilidade (chance). No bloco III, a interpretação adequada de probabilidade é a subjetiva e, portanto, também não haverá a resposta certa. O objetivo principal é levar o aluno a pensar em como atribuir probabilidades para eventos aleatórios. Esta atividade serve como estímulo à discussão do conceito de probabilidade. No item b, do bloco II, discuta com seus alunos sobre como investigar a proporção de nascimentos de meninos e meninas.

\end{goals}

\bigskip
\begin{center}
{\large \scshape Atividade}
\end{center}
\fi

Responda aos itens a seguir.
\begin{enumerate}
\item {} 
A probabilidade de ocorrer cara quando lançamos uma moeda honesta é $0{,}5$. Isso significa que toda vez que lançarmos essa moeda $100$ vezes, ocorrerão $50$ caras? Por quê?

\item {} 
Foi publicada a previsão do tempo, indicando que a probabilidade de chover amanhã na região onde você mora e estuda é de $30\%$. Que decisão você tomaria com base nessa previsão: levar ou não um guarda-chuva para a escola? Por quê? Como você interpreta essa previsão?

\item {} 
Um estudo na área de Saúde indicou que a probabilidade de uma pessoa vir a ter o Diabetes é $10\%$. Isso significa que ao acompanhar um grupo de $500$ pessoas, $50$ delas terão Diabetes? Por quê?

\end{enumerate}

\ifdefined\prof
\begin{solucao}

No primeiro bloco de itens pode-se pensar que cada resultado possível tenha a mesma chance de ocorrer. Assim temos,
\begin{enumerate}
\item $\frac{1}{4}=0{,}4-40\%$, pois são $10$ cartões e quatro deles apresentam "triângulos"{} que representam meninas.
\item Novamente,
\begin{enumerate}
\item $\frac{2}{10}=0{,}2=20\%$, pois são $10$ casas e duas delas têm exatamente $4$ moradores.
\item $\frac{4}{10}=0{,}4=40\%$, pois são $10$ casas e quatro delas têm mais de $4$ moreadores
\end{enumerate}
\item $\frac{1}{4}=0{,}25=40\%$, pois a área da região pintada corresponde a um quarto da área do círculo, isto é, a um setor circular de ângulo reto.
\end{enumerate}

\end{solucao}
\fi

\end{document}