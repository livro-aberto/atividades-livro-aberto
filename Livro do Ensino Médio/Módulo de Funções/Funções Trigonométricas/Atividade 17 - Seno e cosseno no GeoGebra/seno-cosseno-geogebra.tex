\documentclass[10 pt,usenames,dvipsnames, oneside]{article}
\usepackage{../../../modelo-ensino-medio}



\begin{document}

\begin{center}
  \begin{minipage}[l]{3cm}
\includegraphics[width=2cm]{logo}    
\end{minipage}\hfill
\begin{minipage}[r]{.8\textwidth}
 {\Large \scshape Atividade: Seno e Cosseno no GeoGebra}  
\end{minipage}
\end{center}
\vspace{.2cm}

\ifdefined\prof
%Habilidades da BNCC
% \begin{objetivos}
% \item 
% \end{objetivos}

%Caixa do Para o Professor
\begin{goals}
%Objetivos específicos
\begin{enumerate}
\item Construir trechos do gráfico das funções seno e cosseno, correspondentes a uma volta completa no círculo trigonométrico, em uma perspectiva dinâmica que correlacione a posição do ponto $(x,f(x))$ do gráfico com a posição do número x situado no círculo trigonométrico.
\end{enumerate}

\end{goals}

\bigskip
\begin{center}
{\large \scshape Atividade}
\end{center}
\fi

Abra uma tela no GeoGebra (versão smartphone Classic ou versão computador Classic). Siga os passos orientados a seguir.
\begin{enumerate}
\item Construa os pontos $A(0,0)$ e $B(1,0)$ e a circunferência de centro A que passa por B.
\item Tome um ponto $C$ qualquer (ponto em objeto) na circunferência.
\item Trace as perpendiculares por $C$ a $Ox$ e a $Oy$, que encontrarão os eixos respectivamente nos pontos $D$ e $E$ (automaticamente denominados pelo GeoGebra).
\item Construa os segmentos $AD$ e $AE$ --- o GeoGebra os nomeará automaticamente por $h$ e $i$.
\item Construa o arco circular de centro A e extremidades B e C, nessa ordem. O GeoGebra o denominará como d.
\item Construa os pontos $F=(d,x(C))$ e $G=(d,y(C))$.
\item Movimente o ponto C e observe o caminho percorrido por $F$ e por $G$.
\item Habilite o rastro de $F$ e $G$ e movimente $C$.
\item Descreva o caminho percorrido por $F$ e $G$.
\end{enumerate}

Obs.: a construção descrita está disponível no link \url{https://www.geogebra.org/m/ze5k6ur9}.

\ifdefined\prof
\begin{solucao}

O gabarito aqui consiste exatamente na construção do gráfico das funções cosseno e seno, que serão os caminhos percorridos pelos pontos $F$ e $G$ respectivamente no item \titem{i)}
\end{solucao}
\fi

\end{document}