\documentclass[10 pt,usenames,dvipsnames, oneside]{article}
\usepackage{../../../modelo-ensino-medio}



\begin{document}

\begin{center}
  \begin{minipage}[l]{3cm}
\includegraphics[width=2cm]{logo}    
\end{minipage}\hfill
\begin{minipage}[r]{.8\textwidth}
 {\Large \scshape Atividade: Marcando ângulos no círculo trigonométrico}  
\end{minipage}
\end{center}
\vspace{.2cm}

\ifdefined\prof
%Habilidades da BNCC
% \begin{objetivos}
% \item 
% \end{objetivos}

%Caixa do Para o Professor
\begin{sugestions}
\vspace{1em}
Caro professor, para esse item, vale muito a pena estimular os alunos a usar uma calculadora, preferencialmente científica, que pode ser acessada pelo próprio smartphone.

\end{sugestions}

\bigskip
\begin{center}
{\large \scshape Atividade}
\end{center}
\fi

Dê a menor determinação positiva dos arcos a seguir e indique sua expressão geral dos arcos.
\begin{enumerate}
\item $1047^{\circ}$
\item $327\pi\rad$
\item $247\rad$
\item $247^{\circ}$
\item $247\pi\rad$
\item $-1032\rad$
\end{enumerate}

\ifdefined\prof
\begin{solucao}

\begin{enumerate}
\item $1047^{\circ} = 2\cdot360^{\circ} + 327^{\circ}$, o que indica que esse arco deu duas voltas inteiras no círculo trigonométrico e que tem como menor determinação positiva  o arco de $327^{\circ}$. A expressão geral é $327$ + $k\cdot360^{\circ}, k\in \Z$.

\item $327\pi=326\pi+\pi=162\cdot2\pi+\pi\rad$, o que indica que esse arco deu $163$ voltas completas no círculo trigonométrico e que tem como menor determinação positiva o arco $\pi\rad$. Sua expressão geral é $2k\pi+\pi\rad, k\in\Z$.
\item $247\rad\cong39\cdot2\pi+(247-39\cdot\pi)$, o que indica que esse arco deu $39$ voltas inteiras no círculo trigonométrico e que tem como menor determinação positiva o arco $(247-39\cdot2\pi)$. Sua expressão geral é $(247-39\cdot2\pi)+2k\pi, k\in\Z$.
\item $247^{\circ}$ é menor que $360^{\circ}$, logo já é a menor determinação positiva. A expressão geral dos arcos é $247^{\circ}+360^{\circ}\cdot k, k\in\Z$.
\item $247\pi\rad=123\cdot2\pi+\pi$, o que indica que $\pi$ é a menor determinação positiva desse arco e que, portanto, sua expressão geral é $2k\pi+\pi\rad,k \in\Z$.
\item $-1032\rad=-164\cdot2\pi-(-1032+164\cdot\pi)$, o que indica que $(-1032+164\cdot2\pi)$ é a menor determinação negativa. Somando $2\pi$, obtemos $(-1032+164\cdot2\pi)$, que será a menor determinação positiva. Logo, a expressão geral dos arcos é $(-1032+164\cdot2\pi)+2k\pi,k\in\Z$.
\end{enumerate}

\end{solucao}
\fi

\end{document}