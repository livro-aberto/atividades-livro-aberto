\documentclass[10 pt,usenames,dvipsnames, oneside]{article}
\usepackage{../../../modelo-ensino-medio}



\begin{document}

\begin{center}
  \begin{minipage}[l]{3cm}
\includegraphics[width=2cm]{logo}    
\end{minipage}\hfill
\begin{minipage}[r]{.8\textwidth}
 {\Large \scshape Atividade: Pressão Arterial}  
\end{minipage}
\end{center}
\vspace{.2cm}

% \ifdefined\prof
% %Habilidades da BNCC
% % \begin{objetivos}
% % \item 
% % \end{objetivos}

% %Caixa do Para o Professor
% \begin{goals}
% %Objetivos específicos
% \begin{enumerate}
% \item
% \end{enumerate}

% \tcblower

% %Orientações e sugestões
% \begin{itemize}
% \item 
% \end{itemize}
% \end{goals}

% \bigskip
% \begin{center}
% {\large \scshape Atividade}
% \end{center}
% \fi
\textit{(ENEM 2017)}

Um cientista, em seus estudos para modelar a pressão arterial de uma pessoa, utiliza uma função do tipo $P(t) = A + B\cos(Kt)$ em que $A$, $B$ e $K$ são constantes reais positivas e $t$ representa a variável tempo, medida em segundos. Considere que um batimento cardíaco representa o intervalo de tempo entre duas sucessivas pressões máximas. Ao analisar um caso específico, o cientista obteve os dados:

\begin{table}[H]
\centering

\begin{tabular}{|c|f|}
\hline
Pressão mínima & 78 \\
\hline
Pressão máxima & 120 \\
\hline
Número de batimentos cardíacos por minuto & 90 \\
\hline
\end{tabular}
\end{table}

A função P(t) obtida, por este cientista, ao analisar o caso específico foi:

\begin{enumerate}
\item $P(t) = 99 + 21\cos(3\pi t)$
\item $P(t) = 78 + 42\cos(3\pi t)$
\item $P(t) = 99 + 21\cos(2\pi t)$
\item $P(t) = 99 + 21\cos(t)$
\item $P(t) = 78 + 42\cos(t)$
\end{enumerate}

\ifdefined\prof
\begin{solucao}

O período da função é dado pelo tempo entre um batimento cardíaco e outro. Como em $1$ minuto ($60$ s) o coração faz $90$ batimentos, o tempo necessário para um batimento é $\frac{2}{3}$s. 

Utilizando o fato de que o período é $\frac{2\pi}{K}$, segue que $K = 3\pi$. Por outro lado, o parâmetro $B$ fornece a amplitude da função, que tem $120$ como valor máximo e $78$ como valor mínimo. Portanto, $B =\frac{(120-78)}{2} = 21$. Segue que $P(t) = A + 21\cos(3\pi t)$.

Para determinar o parâmetro $A$, note que $P(t)$ é máximo quando o cosseno for máximo, isto é, quanto $t = 0$. Logo, $120 = P(0) = A + 21\cos(0) = A + 21$, donde concluímos que $A = 99$.

\end{solucao}
\fi

\end{document}