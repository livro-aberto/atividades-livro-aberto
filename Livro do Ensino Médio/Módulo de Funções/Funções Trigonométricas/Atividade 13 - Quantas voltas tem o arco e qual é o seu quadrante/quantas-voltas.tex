\documentclass[10 pt,usenames,dvipsnames, oneside]{article}
\usepackage{../../../modelo-ensino-medio}



\begin{document}

\begin{center}
  \begin{minipage}[l]{3cm}
\includegraphics[width=2cm]{logo}    
\end{minipage}\hfill
\begin{minipage}[r]{.8\textwidth}
 {\Large \scshape Atividade: Quantas voltas tem o arco e qual é o seu quadrante?}  
\end{minipage}
\end{center}
\vspace{.2cm}

\ifdefined\prof
%Habilidades da BNCC
% \begin{objetivos}
% \item 
% \end{objetivos}

%Caixa do Para o Professor
\begin{sugestions}
\vspace{1em}
Caro professor, para esse item, vale muito a pena estimular os alunos a usar uma calculadora, preferencialmente científica, que pode ser acessada pelo próprio smartphone.

\end{sugestions}

\bigskip
\begin{center}
{\large \scshape Atividade}
\end{center}
\fi

Considerando os quadrantes no círculo trigonométrico, indique em qual quadrante se localiza a extremidade de cada arco indicado a seguir. Informe também qual o número de voltas completas em torno do círculo é possível se dar com cada um desses arcos.

\begin{enumerate}
\item $8{,}5\rad$
\item $-1{,}37\rad$
\item $\dfrac{\sqrt{2}}{5}\rad$
\item $-0{,}03\rad$
\item $17\dfrac{3}{5}\rad$
\item $-20{,}42\rad$
\end{enumerate}

\ifdefined\prof
\begin{solucao}

\begin{enumerate}
\item Tomando as aproximações decimais para as extremidades dos quadrantes no círculo trigonométrico, podemos ver que $8{,}5\rad - 6{,}28\rad = 2{,}22\rad$, o que indica que o arco de $8{,}5\rad$ deu uma volta inteira no círculo trigonométrico e percorreu ainda um arco de aproximadamente $2{,}22\rad$. Esse é um arco do segundo quadrante, pois o valor $2{,}22$ está entre as aproximações decimais de $\frac{\pi}{2}\cong1{,}57$ e $\pi\cong3{,}14$
\item Tornando o arco positivo côngruo a $-1,{,}37\rad$ e trabalhando com as aproximações decimais para as extremidades do círculo trigonométrico, encontramos $-1{,}37\rad+2\pi\rad \cong 4{,}91\rad$, ou seja, é um arco do $4$\super{o} quadrante. Esse arco não dá nenhuma volta no círculo trigonométrico.
\item $\frac{\sqrt{2}}{6}\rad\cong0{,}28\rad$ e é um arco do $1$\super{o} quadrante. Esse arco não dá nenhuma volta no círculo trigonométrico.
\item $0>-0{,}03>-1{,}57\cong-\frac{\pi}{2}$, assim, é um arco do $4$\super{o} quadrante e não dá nenhuma volta no círculo trigonométrico.
\item $17\frac{3}{5}\rad=17{,}6\rad\cong2\cdot2\pi+5{,}03\rad$. Isso indica que esse arco deu duas voltas inteiras e tem extremidade aproximadamente em $5{,}03\rad$, que é um arco do $4$\super{o} quadrante.
\item 
\begin{align*}
-20{,}42\rad&\cong-3\cdot2\pi-1{,}57044407846\rad \\
-1{,}57044407846\rad+1\pi&\cong4{,}71274122\rad\\
\frac{3\pi}{2}\rad<4{,}71274122\rad&<2\pi\rad
\end{align*}
\end{enumerate}
É um arco do $4$\super{o} quadrante, dando $3$ voltas no círculo trigonométrico.
\end{solucao}
\fi

\end{document}