\documentclass[10 pt,usenames,dvipsnames, oneside]{article}
\usepackage{../../../modelo-ensino-medio}



\begin{document}

\begin{center}
  \begin{minipage}[l]{3cm}
\includegraphics[width=2cm]{logo}    
\end{minipage}\hfill
\begin{minipage}[r]{.8\textwidth}
 {\Large \scshape Atividade: Prevendo a altura da cadeira na Roda Gigante}  
\end{minipage}
\end{center}
\vspace{.2cm}

% \ifdefined\prof
% %Habilidades da BNCC
% % \begin{objetivos}
% % \item 
% % \end{objetivos}

% %Caixa do Para o Professor
% \begin{goals}
% %Objetivos específicos
% \begin{enumerate}
% \item
% \end{enumerate}

% \tcblower

% %Orientações e sugestões
% \begin{itemize}
% \item 
% \end{itemize}
% \end{goals}

% \bigskip
% \begin{center}
% {\large \scshape Atividade}
% \end{center}
% \fi

Na atividade \hyperref[trig-ativ15]{\textit{Retornando à Rio Star}} foi pedido que você escrevesse a altura da cabine em função do tempo e você chegou na seguinte expressão:
\begin{equation*}
h(t)=45{,}5-42{,}5\cdot\cos\bigg(\frac{t\pi}{9}\bigg),
\end{equation*}
onde $t$ representa o tempo, em minutos, após o início da observação, quando a cabine estava na posição mais baixa. Baseado nela, diga em qual altura a cabine estará nos seguintes instantes:

\begin{enumerate}
\item $1$ min e $30$ seg após o início da observação:
\item $2$ min e $15$ seg após o início da observação;
\item $3$ min após o início da observação;
\item $7$ min e $30$ seg após o início da observação;
\item $12$ min após o início da observação;
\item $15$ min e $45$ seg após o início da observação;
\item $21$ min após o início da observação;
\item $7$ min e $30$ seg antes do início da observação;
\end{enumerate}

\ifdefined\prof
\begin{solucao}

\begin{enumerate}
\item $45{,}5-21,25\sqrt{3}$ mts
\item $45{,}5-21,25\sqrt{2}$ mts
\item $24{,}25$ mts
\item $45{,}5+21,25\sqrt{3}$ mts
\item $66{,}75$ mts
\item $45{,}5-21,25\sqrt{3}$ mts
\item $24{,}25$ mts
\item $66{,}75$ mt
\end{enumerate}

\end{solucao}
\fi

\end{document}