\documentclass[10 pt,usenames,dvipsnames, oneside]{article}
\usepackage{../../../modelo-ensino-medio}



\begin{document}

\begin{center}
  \begin{minipage}[l]{3cm}
\includegraphics[width=2cm]{logo}    
\end{minipage}\hfill
\begin{minipage}[r]{.8\textwidth}
 {\Large \scshape Atividade: Intervalo de tempo}  
\end{minipage}
\end{center}
\vspace{.2cm}

\ifdefined\prof
%Habilidades da BNCC
\begin{objetivos}
\item \textbf{EM13MAT304} Resolver e elaborar problemas com funções exponenciais nos quais é necessário compreender e interpretar a variação das grandezas envolvidas, em contextos como o da Matemática Financeira e o do crescimento de seres vivos microscópicos, entre outros. 
\end{objetivos}

%Caixa do Para o Professor
\begin{goals}
%Objetivos específicos
\begin{enumerate}
\item Concluir que o fator de crescimento de uma função exponencial depende apenas do comprimento do intervalo de tempo considerado;
\end{enumerate}

\tcblower

%Orientações e sugestões
\begin{itemize}
\item Nos itens em que se espera uma generalização (c, d, e) certifique-se de que os estudantes foram além dos exemplos, que escreveram intervalos do tipo $[t,t+h]$.
\end{itemize}
\end{goals}

\bigskip
\begin{center}
{\large \scshape Atividade}
\end{center}
\fi

Um biólogo observou que a área ocupada por uma cultura de bactérias em uma placa de Petri estava crescendo a uma taxa de $44\%$ a cada hora. No início da observação a área ocupada pela cultura era de $100 mm^{2}$. A função exponencial que modela essa situação é, portanto, dada por $A(t)=100\cdot(1,45)^{t}$.

\begin{enumerate}

\item{}
Determine o fator de crescimento na primeira meia hora.

\item{}
Determine o fator de crescimento na segunda, na terceira e na décima meia hora, ou seja, os valores de $\dfrac{A(1)}{A(0{,}5)}, \dfrac{A(1{,}5)}{A(1)}$ e $\dfrac{A(5)}{A(4{,}5)}$.

\item{}
Mostre que o fator de crescimento é o mesmo em qualquer intervalo de meia hora.

\item{}
Mostre que, para qualquer intervalo de $\dfrac{1}{4}$ de hora, o fator de crescimento é o mesmo.

\item{}
Justifique a seguinte afirmação: "O fator de crescimento da área na cultura de bactérias em um dado intervalo de tempo depende apenas do tamanho desse intervalo.''

\end{enumerate}

\ifdefined\prof
\begin{solucao}

\begin{enumerate}
\item $\sqrt{1,44}=1,2$, ou $20\%$ a cada meia hora.

\item Todos iguais a $1,2$.

\item $\dfrac{A(t+0,5)}{A(t)}=\dfrac{100\cdot 1,44^{t+0,5}}{100\cdot 1,44^t}=\dfrac{1,44^t \cdot 1,44^{0,5}}{1,44^t}=\sqrt{1,44}$.

\item $\dfrac{A(t+0,25)}{A(t)}=\dfrac{100\cdot 1,44^{t+0,25}}{100\cdot 1,44^t}=\dfrac{1,44^t \cdot 1,44^{0,25}}{1,44^t}=\sqrt[4]{1,44}$.

\item $\dfrac{A(t+h)}{A(t)}=\dfrac{100\cdot 1,44^{t+h}}{100\cdot 1,44^t}=\dfrac{1,44^t \cdot 1,44^{h}}{1,44^t}=1,44^h$.

\end{enumerate}

\end{solucao}
\fi

\end{document}