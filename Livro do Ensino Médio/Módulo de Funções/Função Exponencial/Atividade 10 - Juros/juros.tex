\documentclass[10 pt,usenames,dvipsnames, oneside]{article}
\usepackage{../../../modelo-ensino-medio}



\begin{document}

\begin{center}
  \begin{minipage}[l]{3cm}
\includegraphics[width=2cm]{logo}    
\end{minipage}\hfill
\begin{minipage}[r]{.8\textwidth}
 {\Large \scshape Atividade: Juros}  
\end{minipage}
\end{center}
\vspace{.2cm}

\ifdefined\prof
%Habilidades da BNCC
\begin{objetivos}
\item \textbf{EM13MAT304} Resolver e elaborar problemas com funções exponenciais nos quais é necessário compreender e interpretar a variação das grandezas envolvidas, em contextos como o da Matemática Financeira e o do crescimento de seres vivos microscópicos, entre outros. 
\end{objetivos}

%Caixa do Para o Professor
\begin{goals}
%Objetivos específicos
\begin{enumerate}
\item Investigar o crescimento de uma movimentação financeira com uma dada taxa de capitalização composta. Relacionar taxa percentual $r$ e fator de crescimento $(1+r)$ / decaimento $(1 - r)$.
\end{enumerate}

\tcblower

%Orientações e sugestões
\begin{itemize}
\item Reforce com os estudantes a relação entre a taxa percentual e o fator de multiplicação. 
\end{itemize}
\end{goals}

\bigskip
\begin{center}
{\large \scshape Atividade}
\end{center}
\fi

Você acabou de adquirir um produto de R\$ $200$ e o pagamento proposto pela loja é da seguinte maneira: uma entrada de R\$ $100$ paga no ato da compra e mais uma parcela de  R\$ $110{,}25$ após 2 meses.

\begin{enumerate}

\item{}
Considerando que a taxa percentual mensal de acréscimo será a mesma nos dois meses, qual é o valor dessa taxa na transação proposta?

\item{}
Com base na taxa percentual que você encontrou no item anterior, que valor deveria ser pago se a segunda parcela fosse apenas $1$ mês depois da compra?

\item{}
E se a segunda parcela fosse paga com $3$ meses de intervalo, qual seria o valor a pagar? 

\end{enumerate}

\ifdefined\prof
\begin{solucao}

\begin{enumerate}

\item{}
$100(1+r)^2 =110,25 \iff r=0,05 = 5\%$ 

\item{}
$100(1+0,05)=105$

\item{}
$100(1+0,05)^3=115,76$

\end{enumerate}
\end{solucao}
\fi

\end{document}