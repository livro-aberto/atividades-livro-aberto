\documentclass[10 pt,usenames,dvipsnames, oneside]{article}
\usepackage{../../../modelo-ensino-medio}



\begin{document}

\begin{center}
  \begin{minipage}[l]{3cm}
\includegraphics[width=2cm]{logo}    
\end{minipage}\hfill
\begin{minipage}[r]{.8\textwidth}
 {\Large \scshape Atividade: Na Geometria}  
\end{minipage}
\end{center}
\vspace{.2cm}

\ifdefined\prof
%Habilidades da BNCC
\begin{objetivos}
\item \textbf{EM13MAT508} Identificar e associar progressões geométricas (PG) a funções exponenciais de domínios discretos, para análise de propriedades, dedução de algumas fórmulas e resolução de problemas.
\end{objetivos}

%Caixa do Para o Professor
\begin{goals}
%Objetivos específicos
\begin{enumerate}
\item Identificar progressões geométricas em construções geométricas
\end{enumerate}

\tcblower

%Orientações e sugestões
\begin{itemize}
\item Esta atividade depende de conhecimentos sobre semelhanças de triângulos, arco capaz, relações métricas no triângulo retângulo.
\end{itemize}
\end{goals}

\bigskip
\begin{center}
{\large \scshape Atividade}
\end{center}
\fi

\begin{enumerate}

\item{}
Na figura abaixo, os ângulos indicados são retos e o segmento $AB$ mede $1$. Justifique a seguinte afirmação: “A sequência formada pelas medidas dos segmentos $AB$, $BC$, $CD$, $DE$, $EF$ e $FG$, nesta ordem, é uma progressão geométrica crescente.”

\begin{figure}[H]
\centering
\includegraphics[width=250bp]{imagem_na_geometria.png}
\end{figure}

\item{}
Na figura abaixo, o arco $AC$ é uma semicircunferência e o segmento $BD$ é perpendicular ao diâmetro $AC$. Podemos afirmar que as medidas dos segmentos $AB$, $BD$ e $BC$ formam, nesta ordem, uma progressão geométrica? Porquê?

\begin{figure}[H]
\centering
\includegraphics[width=200bp]{imagem2_na_geometria.png}
\end{figure}

\end{enumerate}

\ifdefined\prof
\begin{solucao}

\begin{enumerate}
\item {} Os ângulos $A\hat{B}C$, $B\hat{C}D$, $C\hat{D}E$, $D\hat{E}F$ e $E\hat{F}G$. Isso implica que os triângulos retângulos $ABC$, $BCD$, $CDE$, $DEF$ e $EFG$ são todos semelhantes. Assim,
\[
\dfrac{BC}{AB}=\dfrac{CD}{BC}=\dfrac{DE}{CD}=\dfrac{EF}{DE}=\dfrac{FG}{EF}
\]
E como $AB=1$ todas essas razões são iguais ao comprimento da hipotenusa $BC$ que é maior que $1$. Logo a PG é crescente.
\item Considere o triângulo $ACD$. Por se tratar de um triângulo inscrito em uma semicircunferência, o ângulo $A\hat{D}C$ é reto, isto é, o triângulo é retângulo. Assim, $BD$ é altura e $AB$ e $BC$ são projeções dos catetos sobre a hipotenusa. Pela semelhança entre os triângulos $ABD$ e $BCD$ podemos afirmar que
\[
\dfrac{BD}{AB}=\dfrac{BC}{BD}
\]
isto é, que as medidas dos segmentos, $AB, BD$ e $BC$.
\end{enumerate}

\end{solucao}
\fi

\end{document}