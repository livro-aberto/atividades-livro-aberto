\documentclass[10 pt,usenames,dvipsnames, oneside]{article}
\usepackage{../../../modelo-ensino-medio}



\begin{document}

\begin{center}
  \begin{minipage}[l]{3cm}
\includegraphics[width=2cm]{logo}    
\end{minipage}\hfill
\begin{minipage}[r]{.8\textwidth}
 {\Large \scshape Atividade: Juros sobre juros}  
\end{minipage}
\end{center}
\vspace{.2cm}

\ifdefined\prof
%Habilidades da BNCC
\begin{objetivos}
\item \textbf{EM13MAT304} Resolver e elaborar problemas com funções exponenciais nos quais é necessário compreender e interpretar a variação das grandezas envolvidas, em contextos como o da Matemática Financeira e o do crescimento de seres vivos microscópicos, entre outros. 
\end{objetivos}

%Caixa do Para o Professor
\begin{goals}
%Objetivos específicos
\begin{enumerate}
\item Praticar em situações concretas os conceitos de juro, taxa e período.
\end{enumerate}

\tcblower

%Orientações e sugestões
\begin{itemize}
\item A ideia aqui não é fazer uma exposição exaustiva sobre juros compostos, mas sim se apoiar nas ideias construídas até aqui para resolver os problemas propostos.

\item  Sugerimos dividir a turma em duplas/trios para a resolução dos problemas e posterior apresentação para os colegas. Caso julgue necessário aumente a lista de problemas para deixar a atividade mais interessante.
\end{itemize}
\end{goals}

\bigskip
\begin{center}
{\large \scshape Atividade}
\end{center}
\fi

Resolva os seguintes problemas:

\begin{enumerate}

\item{}
Se uma pessoa deseja obter R\$ $27.500{,}00$ em um ano, quanto deverá depositar hoje em uma alternativa de poupança que rende $1{,}7\%$ de juros compostos ao mês?

\item{}
Qual a taxa percentual mensal de juros de uma aplicação de R\$ $40.000{,}00$ que produz um total de R\$ $43.894{,}63$ ao final de um quadrimestre.

\item{}
Determinar o juro total a ser pago em um empréstimo de R\$ $88.000{,}00$ pelo prazo de $5$ meses à taxa composta de $4{,}5\%$ ao mês.

\item{}
Qual das opções gera um valor maior ao final de 1 ano: aplicar um capital de $R\$60.000{,}00$ à taxa de juros de $9{,}9\%$ ao semestre ou à taxa de $20{,}78\%$ ao ano.

\end{enumerate}

\ifdefined\prof
\begin{solucao}

\begin{enumerate}

\item{}
$\dfrac{27.500}{1{,}017^{12}}=22.463{,}70$

\item{}
$40.000(1+r)^4=43.894{,}63 \iff r=0,0235=2{,}35\%$

\item
$88.000(1{,}045)^5-88.000=21.664{,}01$

\item
$(1{,}099)^2=1{,}2078$, portanto geram o mesmo valor.

\end{enumerate}

\end{solucao}
\fi

\end{document}