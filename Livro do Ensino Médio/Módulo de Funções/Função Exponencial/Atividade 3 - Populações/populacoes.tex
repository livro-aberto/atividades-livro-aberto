\documentclass[10 pt,usenames,dvipsnames, oneside]{article}
\usepackage{../../../modelo-ensino-medio}



\begin{document}

\begin{center}
  \begin{minipage}[l]{3cm}
\includegraphics[width=2cm]{logo}    
\end{minipage}\hfill
\begin{minipage}[r]{.8\textwidth}
 {\Large \scshape Atividade: Populações}  
\end{minipage}
\end{center}
\vspace{.2cm}

\ifdefined\prof
%Habilidades da BNCC
\begin{objetivos}
\item \textbf{EM13MAT403} Comparar e analisar as representações, em plano cartesiano, das funções exponencial e logarítmica para identificar as características fundamentais (domínio, imagem, crescimento) de cada uma, com ou sem apoio de tecnologias digitais, estabelecendo relações entre elas.
\end{objetivos}

%Caixa do Para o Professor
\begin{goals}
%Objetivos específicos
\begin{enumerate}
\item Deduzir uma equação exponencial a partir da identificação do valor inicial e do fator de crescimento/decaimento;

\end{enumerate}

\tcblower

%Orientações e sugestões
\begin{itemize}
	\item O site do Instituto Chico Mendes de Conservação da Biodiversidade (ICMBio) apresenta uma classificação de acordo com o grau do risco de extinção de uma espécie. Ela pode ser acessada neste endereço: \url{https://www.icmbio.gov.br/ran/images/Arquivos/especies_ameacadas/categorias_criterios_iucn_2012.pdf}.\\ Pode ser um tema interessante a ser discutido também com o professor de Biologia.
	\end{itemize}
\end{goals}

\bigskip
\begin{center}
{\large \scshape Atividade}
\end{center}
\fi

Um grupo de biólogos está estudando uma espécie animal cuja população vem diminuindo ao longo dos anos. Depois de reunirem os dados percebem que a cada ano a quantidade de indivíduos reduz para aproximadamente $\dfrac{1}{3}$ da quantidade do ano anterior.

\begin{enumerate}

\item {}
Escreva uma expressão matemática que relaciona o número de indivíduos dessa população ao longo dos anos, sabendo que no início das medições os cientistas tenham encontrado $300$ mil indivíduos.

\item {}
Admitindo que esse padrão se repita ao longo dos anos, em quanto tempo a população entrará em extinção?

\item {}
Como consequência, a população da espécie que é a principal presa da espécie estudada apresentou um crescimento que duplicava a cada 6 meses. Escreva uma expressão matemática que represente a variação anual do número de indivíduos dessa população de presas, que no início das medições contava com $5\times10^5$ indivíduos.

\end{enumerate}

\ifdefined\prof
\clearpage
\begin{solucao}

\begin{enumerate}

	\item{}
	$P(n)=300000 \times \left(\dfrac{1}{3}\right)^{n}$.

	\item{}
	Ao final do quinto ano.

	\item{}
	$p(n)=5 \times10^{5} \times 4^{n}$.

	\end{enumerate}

\end{solucao}
\fi

\end{document}