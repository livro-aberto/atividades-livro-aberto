\documentclass[10 pt,usenames,dvipsnames, oneside]{article}
\usepackage{../../../modelo-ensino-medio}



\begin{document}

\begin{center}
  \begin{minipage}[l]{3cm}
\includegraphics[width=2cm]{logo}    
\end{minipage}\hfill
\begin{minipage}[r]{.8\textwidth}
 {\Large \scshape Atividade: As colônias}  
\end{minipage}
\end{center}
\vspace{.2cm}

\ifdefined\prof
%Habilidades da BNCC
\begin{objetivos}
\item \textbf{EM13MAT304} Resolver e elaborar problemas com funções exponenciais nos quais é necessário compreender e interpretar a variação das grandezas envolvidas, em contextos como o da Matemática Financeira e o do crescimento de seres vivos microscópicos, entre outros. 
\end{objetivos}

%Caixa do Para o Professor
\begin{goals}
%Objetivos específicos
\begin{enumerate}
	\item Reconhecer situações em que uma quantidade aumenta ou diminui a uma taxa percentual constante por unidade da outra.
	\item Expressar o crescimento/decaimento exponencial em termos de variação percentual e vice-versa.
\end{enumerate}

\tcblower

%Orientações e sugestões
\begin{itemize}
	\item Discuta com os estudantes a possibilidade exibir tabelas e gráficos com valores não necessariamente iguais a 6\% ou 10\% dos valores anteriores, mas que possam ser aproximados tentando retratar alguma coleta de dados real e que tenha levado os cientistas a escolher o modelo exponencial.
	\item No item (g), se achar conveniente, apresente as expressões $(1+r)$ para crescimento e $(1-r)$ para decaimento, nas quais a taxa percentual $r$ é sempre considerada um número positivo.
\end{itemize}
\end{goals}

\bigskip
\begin{center}
{\large \scshape Atividade}
\end{center}
\fi

Analise cada uma das situações abaixo e em seguida responda as perguntas para cada uma delas.

 \textbf{(A)} Uma população de 100 coelhos é introduzida em uma reserva ecológica. Após um período de observação de 12 meses, os biólogos concluíram que essa colônia cresceu ao longo do ano seguindo uma taxa percentual aproximada de $6\%$ ao mês, isto é, a cada mês a população de coelhos na colônia estava aproximadamente $6\%$ maior em relação ao registro do mês anterior.


 \textbf{(B)} Um laboratório está pesquisando a eficácia de um antibiótico e uma equipe de biomédicos o adiciona em uma colônia de bactérias com uma população de $950.000$ indivíduos. As células então começam a morrer de maneira que ao final de 12 horas, os pesquisadores afirmam que população da colônia diminuiu a uma taxa percentual de $10\%$ a cada hora.

\begin{enumerate}

\item{}
Elabore uma tabela com os possíveis dados observados pelos pesquisadores em cada uma das situações.
\item{}
Descreva como você obteve os dados das tabelas anteriores.
\item{}
Pode-se afirmar que os dados tabelas apresentam crescimento e decaimento exponenciais? Em caso afirmativo, quais são os fatores em cada situação?
\item{}
Qual a relação do fator de crescimento/decaimento com a taxa percentual?
\item{}
Escreva uma expressão matemática para cada uma das situações que relacione o números de indivíduos nas colônias com o número de meses (ou horas) decorridos desde o início das observações.
\item{}
Com a ajuda de uma calculadora compare os valores gerados pela expressão matemática com as que você calculou no item \titem{a)}.
\item{}
Denotando por $P(t)$ a população no tempo $t$, $P_0$ seu valor inicial e $r$ a taxa percentual observada, generalize as expressões obtidas no item anterior.
\end{enumerate}

\ifdefined\prof
\begin{solucao}

\begin{enumerate}
	
	\item{}
	Os valores abaixo não são a única resposta correta, e podem variar dependendo da maneira como se procede os cálculos.

	\begin{table}[H]
	\centering
	
	\begin{tabu} to \textwidth{|>{$}c<{$}|>{$}c<{$}|}
	\hline
	\tnumber
	\bm{t}\text{ \textbf{(meses)}} & \bm{C}\text{ \textbf{coelhos}} \\
	\hline
	0 & 100 \\
	\hline
	1 & 106 \\
	\hline
	2 & 112 \\
	\hline
	3 & 119 \\
	\hline
	4 & 126 \\
	\hline
	5 & 133 \\
	\hline 
	6 & 141 \\
	\hline
	7 & 150 \\
	\hline
	8 & 159 \\
	\hline
	9 & 169 \\
	\hline
	10 & 179 \\
	\hline
	11 & 190 \\
	\hline
	12 & 201 \\
	\hline
	\end{tabu}
	\hspace{1em}	
	\begin{tabu} to \textwidth{|>{$}c<{$}|>{$}c<{$}|}
	\hline
	\tnumber
	\bm{t}\text{ \textbf{(meses)}} & \bm{B}\text{ \textbf{bactérias}} \\
	\hline
	0 & 950.000 \\
	\hline
	1 & 855.000 \\
	\hline
	2 & 769.500 \\
	\hline
	3 & 692.550 \\
	\hline
	4 & 623.295 \\
	\hline 
	5 & 560.965 \\
	\hline
	6 & 504.868 \\
	\hline
	7 & 454.382 \\
	\hline
	8 & 408.943 \\
	\hline
	9 & 368.049 \\
	\hline
	10 & 331.244 \\
	\hline
	11 & 298.120 \\
	\hline
	12 & 268.308 \\
	\hline
	\end{tabu}
	\end{table}

	\item{}
	$100+6\%100 = 106$, $106+6\%106=112$, e assim sucessivamente.

	$950.000-10\%950.000 = 855.000$, $855.000-10\%855.000 = 769.500$, e assim sucessivamente. 

	\item{}
	Sim. Basta fazer as divisões de cada valor pelo anterior e observar que os valores ficam próximos de constantes, em cada caso: $1,06$ no primeiro caso e $0,9$ no segundo caso.

	\item{}
	O fator será dado por $1$ mais a taxa percentual considerada positiva no caso de crescimento ($1+6\%$) e negativa para o decaimento ($1-10\%$).

	\item{}
	$C(t)=100\cdot1,06^{t}$, e $B(t)=950.000\cdot0,9^t$.

	\item{}
	---
	\item{}
	$P(t)=P_0\cdot(1+r)^t$.

	\end{enumerate}

\end{solucao}
\fi

\end{document}