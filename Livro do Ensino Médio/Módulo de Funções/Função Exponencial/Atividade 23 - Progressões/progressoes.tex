\documentclass[10 pt,usenames,dvipsnames, oneside]{article}
\usepackage{../../../modelo-ensino-medio}



\begin{document}

\begin{center}
  \begin{minipage}[l]{3cm}
\includegraphics[width=2cm]{logo}    
\end{minipage}\hfill
\begin{minipage}[r]{.8\textwidth}
 {\Large \scshape Atividade: Progressões}  
\end{minipage}
\end{center}
\vspace{.2cm}

\ifdefined\prof
%Habilidades da BNCC
\begin{objetivos}
\item \textbf{EM13MAT508} Identificar e associar progressões geométricas (PG) a funções exponenciais de domínios discretos, para análise de propriedades, dedução de algumas fórmulas e resolução de problemas.
\end{objetivos}

%Caixa do Para o Professor
\begin{goals}
%Objetivos específicos
\begin{enumerate}
\item Identificar progressões geométricas de razão positiva definidas recursivamente como funções exponenciais de domínio discreto (natural);
\end{enumerate}

\tcblower

%Orientações e sugestões
\begin{itemize}
\item A ideia é apresentar PGs definidas recursivamente, incluindo as que têm primeiro termo ou razão negativos. O estudante deve identificar as que têm primeiro termo e razão positivos com o crescimento/decaimento exponenciais.
\end{itemize}
\end{goals}

\bigskip
\begin{center}
{\large \scshape Atividade}
\end{center}
\fi

Sequências de números reais são usualmente denotadas por $[a_{1}, a_{2}, a_{3}, a_{4}, ... ]$ e cada elemento é chamado de termo e está identificado de acordo com a sua posição na sequência. Por exemplo, o símbolo $a_{37}$ representa o trigésimo sétimo número real na sequência considerada.

\begin{enumerate}

\item{}
Em cada item a seguir determine quais são os números que compõem a sequência:

\begin{enumerate}
\item O primeiro termo é $10$ e cada termo é igual ao anterior multiplicado por $3$;

\item $a_{1}=5$ e cada termo é $100\%$ maior que o anterior;

\item  $a_{1}=4$, $a_{2}= -a_{1}$, $a_{3}= -a_{2}$, … , $a_{n+1}= -a_{n}$ para todo $n$ natural;

\item  $a_{5}=-100$ e $a_{n+1}=\dfrac{a_{n}}{2}$ para todo $n$ natural;

\item  $a_{1}=4$, $a_{3}=36$ e cada termo é igual ao anterior multiplicado por um valor constante;

\item  $a_{1}=8$ e $\dfrac{a_{n+1}}{a_{n}}=0,4$ para todo $n$ natural; 

\item  $a_{2}=1000$ e cada termo é $20\%$ menor que o anterior; 

\item  $a_{1}=1$ e $a_{n+1}=-3a_{n}$ para todo $n$ naturtal;

\item  $a_{1}=\sqrt{7}$ e $a_{n+1}=a_{n}$ para todo $n$ natural.
\end{enumerate}

\item{}
Que características têm em comum todas as sequências do item anterior?

\item{}
Quais delas apresentam crescimento ou decaimento exponencial? O que as diferencia das outras?

\end{enumerate}

\ifdefined\prof
\begin{solucao}

\begin{enumerate}

\item 
\begin{enumerate}
\item $(10, 30, 90, 270, 810…)$

\item  $(5, 10, 20, 40, 80, …)$

\item  $(4, -4, 4, -4, 4,...)$

\item  $(-100; -50; -25; -12,5; …)$

\item  $(4,12,36,108, 324,...)$

\item  $(8; 3,2; 1,28; 0,512;...)$

\item  $(1000; 800; 640; 512; 409,6;...)$

\item  $(1, -3, 9, -27, 81, …)$

\item  $(\sqrt 7, \sqrt 7, \sqrt 7, \sqrt 7,...)$
\end{enumerate}

\item Os termos, a partir do segundo, podem ser obtidos multiplicando-se um fator constante pelo termo anterior.

\item Crescimento (1), (2), (5). Decaimento (6) e (7). Todas têm primeiro termo positivo e fator de multiplicação positivo e diferente de 1.


\end{enumerate}

\end{solucao}
\fi

\end{document}