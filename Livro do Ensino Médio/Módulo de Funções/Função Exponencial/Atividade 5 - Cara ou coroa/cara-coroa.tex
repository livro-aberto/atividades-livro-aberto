\documentclass[10 pt,usenames,dvipsnames, oneside]{article}
\usepackage{../../../modelo-ensino-medio}



\begin{document}

\begin{center}
  \begin{minipage}[l]{3cm}
\includegraphics[width=2cm]{logo}    
\end{minipage}\hfill
\begin{minipage}[r]{.8\textwidth}
 {\Large \scshape Atividade: Cara ou coroa?}  
\end{minipage}
\end{center}
\vspace{.2cm}

\ifdefined\prof
%Habilidades da BNCC
\begin{objetivos}
\item \textbf{EM13MAT403} Comparar e analisar as representações, em plano cartesiano, das funções exponencial e logarítmica para identificar as características fundamentais (domínio, imagem, crescimento) de cada uma, com ou sem apoio de tecnologias digitais, estabelecendo relações entre elas.
\end{objetivos}

%Caixa do Para o Professor
\begin{goals}
%Objetivos específicos
	\begin{enumerate}
	\item Reconhecer o padrão de crescimento exponencial que aparece no experimento descrito;
	\item Identificar o fator de crescimento e reconhecer o papel que ele desempenha na situação descrita.

	\end{enumerate}

\tcblower

%Orientações e sugestões
	\begin{itemize}
	\item Estimule os estudantes a realizarem o experimento.
	\end{itemize}
	
\end{goals}

\bigskip
\begin{center}
{\large \scshape Atividade}
\end{center}
\fi

Na teoria das probabilidades, ao analisar as chances de um determinado evento acontecer, é comum considerarmos todas as possibilidades para assim podermos quantificar a probabilidade. O conjunto formado por todas essas possibilidades é chamado de \textbf{espaço amostral}.
Você possui uma moeda, vai girá-la no ar e analisar qual face ficou voltada para cima quando ela cair. O espaço amostral desse experimento contém dois elementos: {cara (C), coroa (K)}. Responda.

\begin{enumerate}

\item{}
Qual o espaço amostral para o experimento de lançar a mesma moeda 2 vezes?

\item{}
E três vezes?

\item{}
Explique a seguinte afirmação:

\textit{“O tamanho do espaço amostral do lançamento de uma moeda várias vezes ao ar aumenta exponencialmente em relação a quantidade de lançamentos”}

\end{enumerate}

\ifdefined\prof
\begin{solucao}

\begin{enumerate}

	\item{}
	$\{CC, CK, KC, KK\}$.

	\item{}
	$\{CCC, CCK, CKK, CKC, KKK, KKC, KCK, KCC\}$.

	\item{}
	Deve ser notado que o tamanho do espaço amostral do lançamento de uma moeda $n$ vezes ao ar é dado por $2^{n}$.

	\end{enumerate}

\end{solucao}
\fi

\end{document}