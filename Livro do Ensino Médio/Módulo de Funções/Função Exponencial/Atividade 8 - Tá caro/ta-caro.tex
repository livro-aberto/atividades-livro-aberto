\documentclass[10 pt,usenames,dvipsnames, oneside]{article}
\usepackage{../../../modelo-ensino-medio}



\begin{document}

\begin{center}
  \begin{minipage}[l]{3cm}
\includegraphics[width=2cm]{logo}    
\end{minipage}\hfill
\begin{minipage}[r]{.8\textwidth}
 {\Large \scshape Atividade: Tá caro!}  
\end{minipage}
\end{center}
\vspace{.2cm}

\ifdefined\prof
%Habilidades da BNCC
\begin{objetivos}
\item \textbf{EM13MAT304} Resolver e elaborar problemas com funções exponenciais nos quais é necessário compreender e interpretar a variação das grandezas envolvidas, em contextos como o da Matemática Financeira e o do crescimento de seres vivos microscópicos, entre outros. 
\end{objetivos}

%Caixa do Para o Professor
\begin{goals}
%Objetivos específicos
\begin{enumerate}
	\item Revisar o conceito de porcentagem.
\end{enumerate}

\tcblower

%Orientações e sugestões
\begin{itemize}
	\item Avalie a necessidade de realizar com a turma uma revisão maior sobre porcentagem.
\end{itemize}
\end{goals}

\bigskip
\begin{center}
{\large \scshape Atividade}
\end{center}
\fi

Você está precisando comprar um tênis novo e vem pesquisando os preços. Na sua última consulta o modelo que você deseja estava custando $160$ reais. Ao fazer uma nova consulta, você verifica que agora ele custa $240$ reais, e resolve não comprar até o preço baixar um pouco.

\begin{enumerate}

\item{}
Qual foi o valor do aumento de preço observado e que valor ele representa em termos percentuais? Explique seu cálculo.

\item{}
Levando em consideração que houve um aumento geral nos preços, você estava esperando um aumento de no máximo $10\%$ no valor do tênis. Que valor máximo deveria estar custando o tênis para que você o comprasse.

\item{}
Construa uma narrativa de outra situação em que o mesmo aumento de preço de $80$ reais esteja dentro da margem de $10\%$ esperada.

\end{enumerate}

\ifdefined\prof
\begin{solucao}

	\begin{enumerate}

	\item{}
	Aumento de $R\$ 80,00$, que representa $50\%$ do valor original do tênis.

	\item{}
	$R\$ 176,00$.

	\item{}
	Narrativa apresentando um produto ou serviço cujo valor seja maior ou igual a $R\$ 800,00$.

	\end{enumerate}

\end{solucao}
\fi

\end{document}