\documentclass[10 pt,usenames,dvipsnames, oneside]{article}
\usepackage{../../../modelo-ensino-medio}



\begin{document}

\begin{center}
  \begin{minipage}[l]{3cm}
\includegraphics[width=2cm]{logo}    
\end{minipage}\hfill
\begin{minipage}[r]{.8\textwidth}
 {\Large \scshape Atividade: Fatores e taxas}  
\end{minipage}
\end{center}
\vspace{.2cm}

\ifdefined\prof
%Habilidades da BNCC
\begin{objetivos}
\item \textbf{EM13MAT304} Resolver e elaborar problemas com funções exponenciais nos quais é necessário compreender e interpretar a variação das grandezas envolvidas, em contextos como o da Matemática Financeira e o do crescimento de seres vivos microscópicos, entre outros. 
\end{objetivos}

%Caixa do Para o Professor
\begin{goals}
%Objetivos específicos
\begin{enumerate}
\item Identificar, a partir da expressão exponencial seus principais elementos e reconhecer nela o crescimento ou decaimento.

\item Esboçar gráficos de exponenciais discretas.
\end{enumerate}

\tcblower

%Orientações e sugestões
\begin{itemize}
\item Até esse momento estamos considerando apenas domínio natural, portanto os gráficos deverão ser esboçados apenas com pontos.

\item Estimule os estudantes a escrever a variação percentual na representação decimal e com o símbolo de porcentagem.
\end{itemize}
\end{goals}

\bigskip
\begin{center}
{\large \scshape Atividade}
\end{center}
\fi

Classifique dentre os modelos dados abaixo os que apresentam crescimento exponencial ou decaimento exponencial. Identifique o que se pede nas colunas da tabela.

\begin{table}[H]
\centering

\begin{tabu} to \textwidth{|>{\centering}m{.2\textwidth}|c|c|>{\centering}m{.1\textwidth}|>{\centering}m{.2\textwidth}|>{\centering}m{.15\textwidth}|}
\hline
\thead
Modelo & Valor inicial & Fator & Taxa percentual & Crescimento ou decaimento? & Esboço do gráfico 
\tabularnewline 
\hline
$y(z)=3(1{,}7)^{z}$ & & & & & \tikz\draw [<->] (0,1cm) -- (0,0) -- (1cm,0); 
\tabularnewline 
\hline
$p(r)=10(0{,}2)^{r}$ & & & &  & \tikz\draw [<->] (0,1cm) -- (0,0) -- (1cm,0); 
\tabularnewline 
\hline
$s(t)=2\cdot \left(\dfrac{1}{2}\right)^{t}$ & & & & & \tikz\draw [<->] (0,1cm) -- (0,0) -- (1cm,0); 
\tabularnewline 
\hline
$M(x)=5\cdot 3^{x}$ & & & & & \tikz\draw [<->] (0,1cm) -- (0,0) -- (1cm,0); 
\tabularnewline 
\hline
$F(y)=0,2\cdot 7^{y} $ & & & & & \tikz\draw [<->] (0,1cm) -- (0,0) -- (1cm,0); 
\tabularnewline 
\hline
\end{tabu}
\end{table}

\ifdefined\prof
\begin{solucao}

\begin{table}[H]
\centering
\setlength\tabulinesep{3.5pt}

\begin{tabu} to \linewidth{|c|c|c|>{\centering}m{.2\linewidth}|>{\centering}e{.2\textwidth}|}
\hline
\thead
Valor inicial & Fator & Taxa percentual & Crescimento ou decaimento? & Esboço do gráfico 
\tabularnewline 
\hline
$3$ & $1{,}7$ & $0{,}7=70\%$ & Crescimento & 

%Figura 1

\adjustbox{valign=c}
{
\resizebox{.2\textwidth}{!}
{
\begin{tikzpicture}[scale=1/3]
\draw [thick] (0,6cm) -- (0,0) -- (6cm,0);
\draw [domain=0:1.3, densely dashed, \currentcolor!80] 
plot (\x,{3*(1.7)^(\x)}) 

node at (0,3) [fill, circle, inner sep=1pt, label=left:{$3$}, overlay] {}; 
\end{tikzpicture}
}
}

\tabularnewline 
\hline
$10$ & $0{,}2$ & $-0{,}8=-80\%$ & Decaimento & 

%Figura 2

\adjustbox{valign=c}
{
\resizebox{.2\textwidth}{!}
{
\begin{tikzpicture}[scale=1/5]
\draw [thick] (0,11cm) -- (0,0) -- (11cm,0);
\draw [very thick, domain=0:11, densely dashed, \currentcolor!80] 
plot (\x,{10*(0.2)^(\x)}) 

node at (0,10) [fill, circle, inner sep=1pt, label=left:{$10$}, overlay] {}; 
\end{tikzpicture}
}
}

\tabularnewline 
\hline
$2$ & $0{,}5$ & $-0{,}5=-50\%$ & Decaimento & 

%Figura 3

\adjustbox{valign=c}
{
\resizebox{.2\textwidth}{!}
{
\begin{tikzpicture}[scale=1/2]
\draw [thick] (0,5cm) -- (0,0) -- (5cm,0);
\draw [very thick, domain=0:5, \currentcolor!80, densely dashed] 
plot (\x,{2*(1/2)^(\x)}) 

node at (0,2) [fill, circle, inner sep=1pt, label=left:{$2$}, overlay] {};
\end{tikzpicture}
}
}

\tabularnewline 
\hline
$5$ & $3$ & $2=200\%$ & Crescimento & 

%Figura 4

\adjustbox{valign=c}
{
\resizebox{.2\textwidth}{!}
{
\begin{tikzpicture}[scale=1/5]
\draw [thick] (0,11cm) -- (0,0) -- (11cm,0);
\draw [very thick, domain=0:.715, densely dashed,\currentcolor!80] 
plot (\x,{5*(3)^(\x)}) 

node at (0,5) [fill, circle, inner sep=1pt, label=left:{$5$}, overlay] {}; 
\end{tikzpicture}
}
}

\tabularnewline 
\hline
$0{,}2$ & $7$ & $6=600\%$ & Crescimento & 

%Figura 5

\adjustbox{valign=c}
{
\resizebox{.2\textwidth}{!}
{
\begin{tikzpicture}[scale=1/2]
\draw [thick] (0,5cm) -- (0,0) -- (5cm,0);
\draw [very thick, domain=0:1.65, densely dashed,\currentcolor!80] 
plot (\x,{0.2*(7)^(\x)}) 

node at (0,.2) [fill, circle, inner sep=1pt, label=left:{$0{,}2$}, overlay] {};  
\end{tikzpicture}
}
}

\tabularnewline 
\hline
\end{tabu}
\end{table}

\end{solucao}
\fi

\end{document}