\documentclass[10 pt,usenames,dvipsnames, oneside]{article}
\usepackage{../../../modelo-ensino-medio}



\begin{document}

\begin{center}
  \begin{minipage}[l]{3cm}
\includegraphics[width=2cm]{logo}    
\end{minipage}\hfill
\begin{minipage}[r]{.8\textwidth}
 {\Large \scshape Atividade: Antílopes}  
\end{minipage}
\end{center}
\vspace{.2cm}

\ifdefined\prof
%Habilidades da BNCC
\begin{objetivos}
\item \textbf{EM13MAT304} Resolver e elaborar problemas com funções exponenciais nos quais é necessário compreender e interpretar a variação das grandezas envolvidas, em contextos como o da Matemática Financeira e o do crescimento de seres vivos microscópicos, entre outros. 
\end{objetivos}

%Caixa do Para o Professor
\begin{goals}
%Objetivos específicos
\begin{enumerate}
\item Comparar o crescimento exponencial e o crescimento de variação linear por meio de tabelas.
\end{enumerate}

\tcblower

%Orientações e sugestões
\begin{itemize}
\item Atividade retirada de \textit{Sawalha, Yamamah, "The Effects Of Teaching Exponential Functions Using Authentic Problem Solving On Students’ Achievement
And Attitude" (2018).Wayne State University Dissertations. 1959.}
\url{https://digitalcommons.wayne.edu/oa_dissertations/1959}

\item Para identificar os crescimentos exponenciais é necessário fazer alguns arredondamentos. A ideia é que o estudante identifique as tabelas que mais se aproximam dos modelos teóricos.
\end{itemize}
\end{goals}

\bigskip
\begin{center}
{\large \scshape Atividade}
\end{center}
\fi

As tabelas abaixo mostram os dados de três possíveis variações nas populações de antílopes em uma reserva. Baseando-se nos dados, responda às questões a seguir.
\begin{multicols}{3}

\begin{table}[H]
\centering

\begin{tabu} to \textwidth{|c|c|}
\hline
\thead
Ano & Antílopes \\
\hline
2017 & $1.000$ \\
\hline
2018 & $1.030$ \\
\hline
2019 & $1.061$ \\
\hline
2020 & $1.093$ \\
\hline
\end{tabu}

\caption*{Tabela 1}
\end{table}

\begin{table}[H]
\centering

\begin{tabu} to \textwidth{|c|c|}
\hline
\thead
Ano & Antílopes \\
\hline
2017 & $1.000$ \\
\hline
2018 & $1.030$ \\
\hline
2019 & $1.060$ \\
\hline
2020 & $1.090$ \\
\hline
\end{tabu}

\caption*{Tabela 2}
\end{table}


\begin{table}[H]
\centering

\begin{tabu} to \textwidth{|c|c|}
\hline
\thead
Ano & Antílopes \\
\hline
2017 & $1.000$ \\
\hline
2018 & $1.003$ \\
\hline
2019 & $1.006$ \\
\hline
2020 & $1.009$ \\
\hline
\end{tabu}

\caption*{Tabela 3}
\end{table}
\end{multicols}



\begin{enumerate}

\item{}
Que tabela mostra a população de antílopes crescendo a uma taxa de 3\% ao ano?

\item{}
Que tabela mostra a população de antílopes crescendo a uma taxa de 30 antílopes por ano?

\item{}
Descreva o crescimento da tabela remanescente.

\item{}
Alguma dessas variações é linear? E exponencial? Explique.

\end{enumerate}

\ifdefined\prof
\clearpage
\begin{solucao}

\begin{enumerate}

\item{}
Tabela $1$.

\item{}
Tabela $2$.

\item{}
$0,3\%$ ao ano ou $3$ antílopes por ano.

\item{}
Tabela $1$ é exponencial com fator $1,03$, tabela $2$ tem variação linear com taxa variação $30$ antílopes. A tabela $3$ pode ser considerada exponencial com fator $1,003$ ou linear com taxa de variação $3$ antílopes.

\end{enumerate}

\end{solucao}
\fi

\end{document}