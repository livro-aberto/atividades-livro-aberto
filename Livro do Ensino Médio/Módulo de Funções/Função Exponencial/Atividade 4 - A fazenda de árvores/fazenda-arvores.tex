\documentclass[10 pt,usenames,dvipsnames, oneside]{article}
\usepackage{../../../modelo-ensino-medio}



\begin{document}

\begin{center}
  \begin{minipage}[l]{3cm}
\includegraphics[width=2cm]{logo}    
\end{minipage}\hfill
\begin{minipage}[r]{.8\textwidth}
 {\Large \scshape Atividade: A fazenda de árvores}  
\end{minipage}
\end{center}
\vspace{.2cm}

\ifdefined\prof
%Habilidades da BNCC
\begin{objetivos}
\item \textbf{EM13MAT403} Comparar e analisar as representações, em plano cartesiano, das funções exponencial e logarítmica para identificar as características fundamentais (domínio, imagem, crescimento) de cada uma, com ou sem apoio de tecnologias digitais, estabelecendo relações entre elas.
\end{objetivos}

%Caixa do Para o Professor
\begin{goals}
%Objetivos específicos
\begin{enumerate}
\item Deduzir uma fórmula para o decaimento exponencial aproximado a partir dos dados de uma tabela;
\item Fazer previsões a partir da expressão encontrada tanto para o problema direto quanto para o inverso.
\end{enumerate}

\tcblower

%Orientações e sugestões
\begin{itemize}
	\item Os dados dessa atividade ficam próximos da exponencial com fator $0{,}95$. Caso seja possível, use ferramentas tecnológicas para análise dos dados.
	\item No item \titem{b)} os valores são pedidos de $5$ em $5$ anos. Verifique se algum estudante percebeu que basta multiplicar $(0,95)^{5}$, e caso positivo peça que ele compartilhe com a turma. Se não, conduza as discussões até essa conclusão.
	\item No item \titem{c)} a pergunta é do tipo problema inverso. Não devemos esperar o uso de logaritmos na solução, mas que o estudante analise os resultados a partir dos dados que ele mesmo gerou no item anterior, fazendo, se necessário, novos cálculos. Algo como: $35$ anos $\rightarrow$ $16,61\%$, $40$ anos $\rightarrow$ $12,85\%$, então os $15\%$ deve estar em algum lugar no meio do caminho, etc.
\end{itemize}
\end{goals}

\bigskip
\begin{center}
{\large \scshape Atividade}
\end{center}
\fi

Uma fazenda de árvores começou a colher um lote que havia sido plantada anos atrás. A tabela abaixo mostra o número de árvores remanescentes para cada um dos 8 anos de colheita 


\begin{table}[H]
\centering

\begin{tabu} to \textwidth{|d{.15\textwidth}|*{9}{c|}}
\hline
\tcolor{Ano} & 0 & 1 & 2 & 3 & 4 & 5 & 6 & 7 & 8 \\
\hline
\tcolor{Árvores remanescentes} & 10.000 & 9.502 & 9.026 & 8.574 & 8.145 & 7.737 & 7.350 & 6.982 & 6.634 \\
\hline
\end{tabu}
\end{table}

\begin{enumerate}

\item {}
Escreva uma expressão para função que relaciona as árvores remanescentes com o tempo decorrido.

\item{}
Analise os dados da tabela e faça uma previsão para os próximos anos mostrados na tabela a seguir.

\item{}
Os donos da fazenda querem parar a colheita quando sobrarem 15\% do número inicial de árvores do lote. Quando devem parar? 

\end{enumerate}

\begin{table}[H]
\centering

\begin{tabu} to \textwidth{|d{.15\textwidth}|*{7}{>{\centering}m{3em}|}}
\hline
\tcolor{Ano} & 10 & 15 & 20 & 25 & 30 & 35 & 40 \\
\hline
\tcolor{Árvores remanescentes} & & & & & & & \\
\hline
\end{tabu}
\end{table}

\ifdefined\prof
\begin{solucao}

\begin{enumerate}

	\item{}
	$f(n)=10.000 \times 0,95^{n}$.

	\item{}
	\adjustbox{valign=t}
	{
	\begin{tabu} to \textwidth{|>{\centering}d{.15\textwidth}|*{7}{c|}}
	\hline
	\tcolor{Ano} & 10 & 15 & 20 & 25 & 30 & 35 & 40 \\
	\hline
	\tcolor{Árvores remanescentes} & 5.987 & 4.633 & 3.585 & 2.774 & 2.146 & 1.661 & 1.285 \\
	\hline
	\end{tabu}
	}

	\item{}
	No trigésimo sexto ano.

	\end{enumerate}

\end{solucao}
\fi

\end{document}