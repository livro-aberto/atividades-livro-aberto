\documentclass[10 pt,usenames,dvipsnames, oneside]{article}
\usepackage{../../../modelo-ensino-medio}



\begin{document}

\begin{center}
  \begin{minipage}[l]{3cm}
\includegraphics[width=2cm]{logo}    
\end{minipage}\hfill
\begin{minipage}[r]{.8\textwidth}
 {\Large \scshape Atividade: Multiplicação manual}  
\end{minipage}
\end{center}
\vspace{.2cm}

\ifdefined\prof
%Habilidades da BNCC
\begin{objetivos}
\item \textbf{EM13MAT305} Resolver e elaborar problemas com funções logarítmicas nos quais seja necessário compreender e interpretar a variação das grandezas envolvidas, em contextos como os de abalos sísmicos, pH, radioatividade, Matemática Financeira, entre outros.
\end{objetivos}

%Caixa do Para o Professor
\begin{goals}
%Objetivos específicos
\begin{enumerate}
\item Observar as dificuldades inerentes à realização de operações de multiplicação
\end{enumerate}

\end{goals}

\bigskip
\begin{center}
{\large \scshape Atividade}
\end{center}
\fi
Calcule, sem a utilização de equipamentos eletrônicos,
$$
19683 \times 2187.
$$
Verifique, com a ajuda de uma calculadora, se obteve a resposta correta


\ifdefined\prof
\begin{solucao}

Recomenda-se perguntar os resultados obtidos à turma antes de apresentar a resposta correta e espera-se obter diversos resultados distintos. \textbf{O resultado é $43046721$}, mas a experiência mostra que dificilmente algum estudante chega ao resultado correto e recomenda-se que o/a professor/a diga que tais erros eram esperados, pois é muito fácil cometer erros ao multiplicar grandes números.

\end{solucao}
\fi

\end{document}