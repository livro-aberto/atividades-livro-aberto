\documentclass[10 pt,usenames,dvipsnames, oneside]{article}
\usepackage{../../../modelo-ensino-medio}



\begin{document}

\begin{center}
  \begin{minipage}[l]{3cm}
\includegraphics[width=2cm]{logo}    
\end{minipage}\hfill
\begin{minipage}[r]{.8\textwidth}
 {\Large \scshape Atividade: Funções logarítmicas distintas}  
\end{minipage}
\end{center}
\vspace{.2cm}

\ifdefined\prof
%Habilidades da BNCC
\begin{objetivos}
\item \textbf{EM13MAT403} Analisar e estabelecer relações, com ou sem apoio de tecnologias digitais, entre as representações de funções exponencial e logarítmica expressas em tabelas e em plano cartesiano, para identificar as características fundamentais (domínio, imagem, crescimento) de cada função.
\end{objetivos}

%Caixa do Para o Professor
\begin{goals}
%Objetivos específicos
\begin{enumerate}
\item Analisar e estabelecer relações entre as representações de funções exponencial e logarítmica.
\end{enumerate}

\tcblower

%Orientações e sugestões
\begin{itemize}
\item Por fim é proposta outra atividade no GeoGebra, que pretende mostrar que trocar de base é o mesmo que multiplicar a função logarítmica por uma constante, que é ressaltada na observação seguinte. Não havendo a possibilidade de usar o \textit{applet}, essa atividade poderia ser deixada de lado e a conclusão poderia ser reforçada como interpretação da equação $\log_a x = \log_a b \log_b x$.
\end{itemize}
\end{goals}

\bigskip
\begin{center}
{\large \scshape Atividade}
\end{center}
\fi

No \textit{applet} do GeoGebra, disponível através do atalho abaixo, podemos mover a barra com o valor $a$, alterando o gráfico da função $h(x) = a \log_2 x$. Qual o valor de $a$ para que o gráfico de $h(x)$ coincida com o gráfico de $f(x) = \log_{1{,}2} x$?


\begin{figure}[H]
\centering

\includegraphics[width=.3\linewidth]{QRcode_mult_constante.png}

\url{https://www.geogebra.org/calculator/qyuzhbmp}
\end{figure}

\textbf{Desafio:} aplique o Teorema da mudança de base em $g(x)= \log_2 x$, utilizando a base $1{,}2$ para encontrar algebricamente o valor de $a$.

\ifdefined\prof
\begin{solucao}

$a = 3{,}8$, que é uma aproximação muito boa para $$\log_{1{,}2}2 \approx 3{,}8017840169239,$$ que é a resposta do desafio.

\end{solucao}
\fi

\end{document}