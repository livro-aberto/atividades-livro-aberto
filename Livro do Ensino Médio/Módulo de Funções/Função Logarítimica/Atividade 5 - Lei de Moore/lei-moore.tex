\documentclass[10 pt,usenames,dvipsnames, oneside]{article}
\usepackage{../../../modelo-ensino-medio}



\begin{document}

\begin{center}
  \begin{minipage}[l]{3cm}
\includegraphics[width=2cm]{logo}    
\end{minipage}\hfill
\begin{minipage}[r]{.8\textwidth}
 {\Large \scshape Atividade: Lei de Moore}  
\end{minipage}
\end{center}
\vspace{.2cm}

\ifdefined\prof
%Habilidades da BNCC
\begin{objetivos}
\item \textbf{EM13MAT305} Resolver e elaborar problemas com funções logarítmicas nos quais seja necessário compreender e interpretar a variação das grandezas envolvidas, em contextos como os de abalos sísmicos, pH, radioatividade, Matemática Financeira, entre outros.
\end{objetivos}

%Caixa do Para o Professor
\begin{goals}
%Objetivos específicos
\begin{enumerate}
\item Resolver um problema contextualizado aplicando a noção de logaritmo.
\end{enumerate}

\tcblower

Em seguida indica-se aos alunos o exercício sobre a Lei de Moore. Ao apresentar a solução deve-se aproveitar a oportunidade para utilizar a notação de base 10 omitindo a base.

\end{goals}

\bigskip
\begin{center}
{\large \scshape Atividade}
\end{center}
\fi

Na computação, a Lei de Moore estima como os computadores progridem. Ela prevê que a capacidade computacional dobra a cada 1,5 ano. Para simplificar nossos cálculos, vamos aproximá-la\footnote{Em 5 anos há $3,\overline{3}$ períodos de $1,5$ anos, então o crescimento seria de $2^{3,\overline{3}} = 10,07$, com duas casas decimais.} por um crescimento de 10 vezes a cada 5 anos. Supondo que as previsões de Moore estejam corretas, utilize a notação de logaritmo para representar quantos quinquênios serão necessários para:
\begin{enumerate}
\item a capacidade computacional aumentar 100 vezes,
\item a capacidade computacional aumentar 1000 vezes.
\end{enumerate}
Qual seria a sua estimativa para a quantidade de quinquênios necessários para a que capacidade computacional aumente 1100 vezes?

\ifdefined\prof
\begin{solucao}

Suponhamos que hoje a capacidade computacional seja representada pela constante c. Após 5 anos, teremos $10c$. Após 10 anos, teremos $10 \times (10c) = c \times 10^2$. Após $t$ quinquênios , teremos $c \times 10^t$
\begin{enumerate} 
\item Queremos chegar a 100 vezes o que é hoje, ou seja, queremos
$$
t = \log 100 = \log_{10} 100  = 2.
$$
\item Queremos chegar a 1000 vezes o que é hoje, ou seja, queremos
$$
t = \log 1000 = \log_{10} 1000  = 3.
$$
\end{enumerate}

\end{solucao}
\fi

\end{document}