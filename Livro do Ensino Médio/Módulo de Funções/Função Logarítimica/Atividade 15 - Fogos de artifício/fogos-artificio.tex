\documentclass[10 pt,usenames,dvipsnames, oneside]{article}
\usepackage{../../../modelo-ensino-medio}



\begin{document}

\begin{center}
  \begin{minipage}[l]{3cm}
\includegraphics[width=2cm]{logo}    
\end{minipage}\hfill
\begin{minipage}[r]{.8\textwidth}
 {\Large \scshape Atividade: Fogos de artifício}  
\end{minipage}
\end{center}
\vspace{.2cm}

\ifdefined\prof
%Habilidades da BNCC
\begin{objetivos}
\item \textbf{EM13MAT403} Analisar e estabelecer relações, com ou sem apoio de tecnologias digitais, entre as representações de funções exponencial e logarítmica expressas em tabelas e em plano cartesiano, para identificar as características fundamentais (domínio, imagem, crescimento) de cada função.
\end{objetivos}

%Caixa do Para o Professor
\begin{goals}
%Objetivos específicos
\begin{enumerate}
\item Resolver questões envolvendo logaritmos

e a unidade de decibel.
\end{enumerate}

\tcblower

%Orientações e sugestões
\begin{itemize}
\item 	Recomenda-se a resolução da atividade em sala para oportunizar o surgimento de dúvidas, tendo em vista que a definição de decibel pode parecer complicada à primeira vista.
\end{itemize}
\end{goals}

\bigskip
\begin{center}
{\large \scshape Atividade}
\end{center}
\fi

Algumas cidades brasileiras estão limitando o nível sonoro dos fogos de artifício utilizados em celebrações, permitindo apenas fogos de até $100dB$ a 100 metros de distância. A classificação do som como forte ou fraco está relacionada à intensidade sonora, medida em watts por metro quadrado $W/m^2$. A menor intensidade sonora audível ou limiar de audibilidade possui intensidade $I_0 = 10^{-12} W/m^2$. O nível sonoro pode ser calculado a partir da intensidade da onda pela expressão $I_{dB} = 10 \times log(I/I_0)$, onde $I_0$ é o limiar de audibilidade. Um técnico mede a intensidade sonora de um foguete como sendo de $3{,}7 W/m^2$ a 100 metros de distância. Esse foguete poderia ser utilizado de acordo com a norma determinada por essas cidades?

\ifdefined\prof
\begin{solucao}


	Calculamos o nível de intensidade
\begin{align*}
	I_{dB} =& 10 \log \left(\frac{3{,}7}{10^{-12}}\right)\\
	=& 10 \log (3{,}7\times 10^{12})\\
	=& 10 (\log 3{,}7+12)\\
	=& 10\times 12{,}5682 = 125{,}682
	\end{align*}
	e verificamos que o artefato estaria em desacordo com a norma.


\end{solucao}
\fi

\end{document}