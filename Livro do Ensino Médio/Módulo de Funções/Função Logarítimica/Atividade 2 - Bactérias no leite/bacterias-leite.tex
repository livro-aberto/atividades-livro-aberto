\documentclass[10 pt,usenames,dvipsnames, oneside]{article}
\usepackage{../../../modelo-ensino-medio}



\begin{document}

\begin{center}
  \begin{minipage}[l]{3cm}
\includegraphics[width=2cm]{logo}    
\end{minipage}\hfill
\begin{minipage}[r]{.8\textwidth}
 {\Large \scshape Atividade: Bactérias no leite}  
\end{minipage}
\end{center}
\vspace{.2cm}

\ifdefined\prof
%Habilidades da BNCC
\begin{objetivos}
\item \textbf{EM13MAT305} Resolver e elaborar problemas com funções logarítmicas nos quais seja necessário compreender e interpretar a variação das grandezas envolvidas, em contextos como os de abalos sísmicos, pH, radioatividade, Matemática Financeira, entre outros.
\end{objetivos}

%Caixa do Para o Professor
\begin{goals}
%Objetivos específicos
\begin{enumerate}
\item Desenvolver raciocínios informais sobre o número de vezes que uma operação de multiplicação é aplicada.
\item Reconhecer questões em que as incógnitas encontram-se em expoentes.
\end{enumerate}

\end{goals}

\bigskip
\begin{center}
{\large \scshape Atividade}
\end{center}
\fi

Uma população de uma determinada bactéria contamina um copo de leite à temperatura ambiente. Suponha que o leite seja considerado impróprio para o consumo se a população do copo for maior ou igual a $16.200$ indivíduos. A população de bactérias tem, inicialmente, $200$ indivíduos e sua população triplica a cada hora até atingir $40.000$ indivíduos. Em quanto tempo a população atingirá $16.200$ indivíduos?

\ifdefined\prof
\begin{solucao}

A evolução do número de bactérias pode ser tabelado
\begin{table}[H]
\centering

\begin{tabu} to \textwidth{|c|l|l|}
\hline
\thead
Tempo (horas) & Evolução das Bactérias & População \\
\hline
$0$ & $200$ & $200$ \\
\hline
$1$ & $200\times3$ & $600$ \\
\hline
$2$ & $200\times3^2$ & $1.800$ \\
\hline
$3$ & $200\times3^3$ & $5.400$ \\
\hline
$4$ & $200\times3^4$ & $16.200$ \\
\hline
\end{tabu}
\end{table}
A população atinge 16.200 indivíduos após 4 horas.

\end{solucao}
\fi

\end{document}