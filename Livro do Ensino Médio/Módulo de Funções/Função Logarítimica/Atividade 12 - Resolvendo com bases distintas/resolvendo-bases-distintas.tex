\documentclass[10 pt,usenames,dvipsnames, oneside]{article}
\usepackage{../../../modelo-ensino-medio}



\begin{document}

\begin{center}
  \begin{minipage}[l]{3cm}
\includegraphics[width=2cm]{logo}    
\end{minipage}\hfill
\begin{minipage}[r]{.8\textwidth}
 {\Large \scshape Atividade: Resolvendo com bases distintas}  
\end{minipage}
\end{center}
\vspace{.2cm}

\ifdefined\prof
%Habilidades da BNCC
\begin{objetivos}
\item \textbf{EM13MAT403} Analisar e estabelecer relações, com ou sem apoio de tecnologias digitais, entre as representações de funções exponencial e logarítmica expressas em tabelas e em plano cartesiano, para identificar as características fundamentais (domínio, imagem, crescimento) de cada função.
\end{objetivos}

%Caixa do Para o Professor
\begin{goals}
%Objetivos específicos
\begin{enumerate}
\item Investigar a aplicação de logaritmos em diferentes bases na solução de problemas.
\item Observar que a propriedade da mudança de base surge naturalmente dessa investigação.
\end{enumerate}

\tcblower

%Orientações e sugestões
\begin{itemize}
\item Sugere-se a que os estudantes desenvolvam a atividade para que experimentem o surgimento da propriedade e reforcem a técnica utilizada na solução do exemplo.
\end{itemize}
\end{goals}

\bigskip
\begin{center}
{\large \scshape Atividade}
\end{center}
\fi

Vamos supor que um grupo de $10$ pessoas tenha chegado à cidade de São Paulo portando uma nova variedade de uma doença extremamente infecciosa. Suponha que a taxa de crescimento da doença é de $100\%$ ao dia sem que nenhuma medida contenção seja aplicada. Vamos calcular quantos dias levará para que a doença ultrapasse $2000$ pessoas utilizando logaritmos em base $10$ e em base $2$. Utilize as aproximações $\log 2 = 0{,}3$ e $\log_2 10 = 3{,}33$.

\ifdefined\prof
\begin{solucao}

	Calculando com base $10$:
	\begin{align*} &10 \times 2^t = 2000\\
	\Rightarrow& t\log 2 = \log 200\\
	\Rightarrow& t = \frac{\log 200}{\log 2}\\
	& = \frac{\log 2 + 2}{\log 2}= \frac{2{,}3}{0{,}3} \approx 7{,}\overline{6}.
	\end{align*}
	Calculando com base $2$:
	\begin{align*} &10 \times 2^t = 2000\\
	\Rightarrow& t\log_{2} 2 = \log_{2} 200\\
	\Rightarrow& t = \log_{2} 200  = 1+2\log_2 10 \approx 7{,}66.
	\end{align*}

\end{solucao}
\fi

\end{document}