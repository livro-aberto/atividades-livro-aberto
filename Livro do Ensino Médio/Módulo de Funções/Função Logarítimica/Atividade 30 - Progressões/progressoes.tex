\documentclass[10 pt,usenames,dvipsnames, oneside]{article}
\usepackage{../../../modelo-ensino-medio}



\begin{document}

\begin{center}
  \begin{minipage}[l]{3cm}
\includegraphics[width=2cm]{logo}    
\end{minipage}\hfill
\begin{minipage}[r]{.8\textwidth}
 {\Large \scshape Atividade: Progressões}  
\end{minipage}
\end{center}
\vspace{.2cm}

\ifdefined\prof
%Habilidades da BNCC
\begin{objetivos}
\item \textbf{EM13MAT508} Identificar e associar progressões geométricas (PG) a funções exponenciais de domínios discretos, para análise de propriedades, dedução de algumas fórmulas e resolução de problemas.
\end{objetivos}

%Caixa do Para o Professor
\begin{goals}
%Objetivos específicos
\begin{enumerate}
\item Conhecer o desenvolvimento histórico das ideias envolvendo logaritmos.
\item Entender a relação dos logaritmos aos conteúdos de P.A. e P.G.
\end{enumerate}

\end{goals}

\bigskip
\begin{center}
{\large \scshape Atividade}
\end{center}
\fi

Vamos pensar que o objeto $P$ na primeira reta move-se, a cada segundo, uma unidade para a direita e que inicia na origem. Já o objeto $Q$ inicia a duas unidades da origem e move-se, a cada segundo, uma quantidade de unidades igual à distância que está da origem. Escreva os 10 primeiros termos da sequência de distâncias à origem de cada um dos objetos.
\begin{itemize}
\item Você reconhece as duas sequências que estão se formando? 
\item Quais os valores dos logaritmos em base 2 dos valores encontrados na segunda sequência?
\item Você consegue reconhecer esses logaritmos na primeira sequência?
\end{itemize}

\ifdefined\prof
\begin{solucao}

\begin{itemize}
	\item $\{1,2,3,4,5,6,7,8,\cdots\};$
	\item[] $\{2,4,8,16,32,64,\cdots\};$
	\item $\{1,2,3,4,5,6,7,8,\cdots\};$
	\item são os valores da primeira sequência.
	\end{itemize}

\end{solucao}
\fi

\end{document}