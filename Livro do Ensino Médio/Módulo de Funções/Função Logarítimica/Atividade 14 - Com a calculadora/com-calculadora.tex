\documentclass[10 pt,usenames,dvipsnames, oneside]{article}
\usepackage{../../../modelo-ensino-medio}



\begin{document}

\begin{center}
  \begin{minipage}[l]{3cm}
\includegraphics[width=2cm]{logo}    
\end{minipage}\hfill
\begin{minipage}[r]{.8\textwidth}
 {\Large \scshape Atividade: Com a calculadora}  
\end{minipage}
\end{center}
\vspace{.2cm}

\ifdefined\prof
%Habilidades da BNCC
\begin{objetivos}
\item \textbf{EM13MAT403} Analisar e estabelecer relações, com ou sem apoio de tecnologias digitais, entre as representações de funções exponencial e logarítmica expressas em tabelas e em plano cartesiano, para identificar as características fundamentais (domínio, imagem, crescimento) de cada função.
\end{objetivos}

%Caixa do Para o Professor
\begin{goals}
%Objetivos específicos
\begin{enumerate}
\item Aplicar a propriedade da mudança de base em cálculos com a calculadora.
\end{enumerate}

\end{goals}

\bigskip
\begin{center}
{\large \scshape Atividade}
\end{center}
\fi

Utilize uma calculadora ou aplicativo de celular para calcular:
\begin{enumerate}
\item $\log_2 375$;
\item $\log_7 698$;
\item $\log_{2{,}71} 166$.
\end{enumerate}

\ifdefined\prof
\begin{solucao}

	\begin{enumerate}
	\item $\log_2 375=\dfrac{\log 375}{\log 2} = \dfrac{2{,}57403126772}{0{,}47712125471} \approx 5{,}39492056215;$
	\item $\log_7 698 =\dfrac{\log 698}{\log 7} = \dfrac{2{,}84385542262}{0{,}84509804001} \approx 3{,}36511894238;$
	\item $\log_{2{,}71} 166 = \dfrac{\log 166}{\log 2{,}71} = \dfrac{2{,}22010808804}{0{,}43296929087} \approx 5{,}12763407205.$
	\end{enumerate}

\end{solucao}
\fi

\end{document}