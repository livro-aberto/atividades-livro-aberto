\documentclass[10 pt,usenames,dvipsnames, oneside]{article}
\usepackage{../../../modelo-ensino-medio}



\begin{document}

\begin{center}
  \begin{minipage}[l]{3cm}
\includegraphics[width=2cm]{logo}    
\end{minipage}\hfill
\begin{minipage}[r]{.8\textwidth}
 {\Large \scshape Atividade: Base $e$}  
\end{minipage}
\end{center}
\vspace{.2cm}

\ifdefined\prof
%Habilidades da BNCC
\begin{objetivos}
\item \textbf{EM13MAT403} Analisar e estabelecer relações, com ou sem apoio de tecnologias digitais, entre as representações de funções exponencial e logarítmica expressas em tabelas e em plano cartesiano, para identificar as características fundamentais (domínio, imagem, crescimento) de cada função.
\end{objetivos}

%Caixa do Para o Professor
\begin{goals}
%Objetivos específicos
\begin{enumerate}
\item Utilizar logaritmos em base $e$.

\item Aplicar a função logarítmica natural para reescrever funções exponenciais.
\end{enumerate}

\end{goals}

\bigskip
\begin{center}
{\large \scshape Atividade}
\end{center}
\fi

Escreva as seguintes funções exponenciais utilizando a base $e$, como no exemplo acima, utilizando aproximações com três casas decimais de precisão para os logaritmos naturais.
\begin{enumerate}
\item $f(x)=3^x$;
\item $g(x)=10^x$;
\item $f(x)=75^x$.
\end{enumerate}

\ifdefined\prof
\begin{solucao}

	\begin{enumerate}
\item $3^x = e^{\ln 3^x} = e^{x\ln 3} \approx e^{1{,}098 x}$;
\item $10^x = e^{\ln 10^x} = e^{x\ln 10} \approx e^{2{,}302 x}$;
\item $75^x = e^{\ln 75^x} = e^{x\ln 75} \approx e^{4{,}317 x}$.
\end{enumerate}

\end{solucao}
\fi

\end{document}