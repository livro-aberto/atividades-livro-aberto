\documentclass[10 pt,usenames,dvipsnames, oneside]{article}
\usepackage{../../../modelo-ensino-medio}



\begin{document}

\begin{center}
  \begin{minipage}[l]{3cm}
\includegraphics[width=2cm]{logo}    
\end{minipage}\hfill
\begin{minipage}[r]{.8\textwidth}
 {\Large \scshape Atividade: Cálculos sem a calculadora}  
\end{minipage}
\end{center}
\vspace{.2cm}

\ifdefined\prof
%Habilidades da BNCC
\begin{objetivos}
\item \textbf{EM13MAT305} Resolver e elaborar problemas com funções logarítmicas nos quais seja necessário compreender e interpretar a variação das grandezas envolvidas, em contextos como os de abalos sísmicos, pH, radioatividade, Matemática Financeira, entre outros.
\end{objetivos}

%Caixa do Para o Professor
\begin{goals}
%Objetivos específicos
\begin{enumerate}
\item Praticar a definição de logaritmo e o seu cálculo através do conceito propriamente dito.
\end{enumerate}

\end{goals}

\bigskip
\begin{center}
{\large \scshape Atividade}
\end{center}
\fi

Calcule exatamente ou encontre um número natural que aproxime, com erro menor do que 1, os seguintes logaritmos (justificando as suas respostas):
\begin{multicols}{2}
\begin{enumerate}
\item $\log_2 8$;
\item $\log_{1/2} 32$;
\item $\log 1000$;
\item $\log_2 10$;
\item $\log 113$;
\item $\log_3 30 $.
\end{enumerate}
\end{multicols}

\ifdefined\prof
\begin{solucao}

\begin{enumerate}
	\item $\log_2 8 = 3$ pois $2^3 = 8$;
	\item $\log_{1/2} 32 = -5$ pois $(1/2)^{-5} = 2^5 = 32$;
	\item $\log 1000= 3$ pois $10^3 = 1000$;
	\item Como $2^3=8 <10 <2^4 =16$, temos $\log_2 10$ pode ser aproximado por 3 ou 4 com erro menor do que 1;
	\item Como $10 ^2 = 100 < 113 < 10^3 = 1000$, temos $\log 113$ pode ser aproximado por 2 ou 3 com erro menor do que 1;
	\item Como $3^3= 27 <30 <3^4 =81$, temos $\log_3 30$ pode ser aproximado por 3 ou 4 com erro menor do que 1.
	\end{enumerate}

\end{solucao}
\fi

\end{document}