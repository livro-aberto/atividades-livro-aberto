\documentclass[10 pt,usenames,dvipsnames, oneside]{article}
\usepackage{../../../modelo-ensino-medio}



\begin{document}

\begin{center}
  \begin{minipage}[l]{3cm}
\includegraphics[width=2cm]{logo}    
\end{minipage}\hfill
\begin{minipage}[r]{.8\textwidth}
 {\Large \scshape Atividade: Desenvolvimento bacteriano}  
\end{minipage}
\end{center}
\vspace{.2cm}

\ifdefined\prof
%Habilidades da BNCC
\begin{objetivos}
\item \textbf{EM13MAT305} Resolver e elaborar problemas com funções logarítmicas nos quais seja necessário compreender e interpretar a variação das grandezas envolvidas, em contextos como os de abalos sísmicos, pH, radioatividade, Matemática Financeira, entre outros.
\end{objetivos}

%Caixa do Para o Professor
\begin{goals}
%Objetivos específicos
\begin{enumerate}
\item Observar padrões nos valores dos logaritmos quando os logaritmandos são multiplicados por um número natural.
\end{enumerate}

\tcblower

%Orientações e sugestões
\begin{itemize}
\item  A atividade \textit{"Desenvolvimento bacteriano"} busca destacar a propriedade do produto. Recomenda-se que, ao resolver os exercícios com os estudantes, seja ressaltado que no item b) dobramos o logaritmando e a resposta aumentou em 1, no item c) o logaritmando é multiplicado por 32 e a resposta aumenta em 5, que é $\log_2 32$; e no item d) o logaritmando é multiplicado novamente por 32 e a resposta aumenta em 5 (que é o logaritmo de 32 em base 2).
\end{itemize}
\end{goals}

\bigskip
\begin{center}
{\large \scshape Atividade}
\end{center}
\fi

Uma espécie de bactéria dobra a população a cada dia e uma cultura tem 100 indivíduos inicialmente. Sabendo que $2^5=32$ e $2^6=64$, vamos estimar quantos dias levará para que a população: 
\begin{enumerate}
\item Chegue a 3200?
\item Chegue a 6400?
\item Chegue a 204800?
\item Chegue a 6553600?
\end{enumerate}

\ifdefined\prof
\begin{solucao}

	A população após $n$ meses é $P(n) = 100 \times 2^n$, assim,
	\begin{enumerate}[label = \alph*)]
	\item $3200 = P(n) = 100 \times 2^n \Rightarrow 2^n = 32 \Rightarrow n=5$.
	\item $6400 = P(n) = 100 \times 2^n \Rightarrow 2^n = 64 \Rightarrow n=6$.
	\item $204800 = P(n) = 100 \times 2^n \Rightarrow 2^n = 2048 = 32 \times 64 \Rightarrow n=5+6=11$.
	\item $6553600 = P(n) = 100 \times 2^n \Rightarrow 2^n = 65536 = 2048 \times 32 \Rightarrow n=11+5=16$.
	\end{enumerate}

\end{solucao}
\fi

\end{document}