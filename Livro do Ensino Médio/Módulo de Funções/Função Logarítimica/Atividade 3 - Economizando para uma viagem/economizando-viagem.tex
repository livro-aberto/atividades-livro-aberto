\documentclass[10 pt,usenames,dvipsnames, oneside]{article}
\usepackage{../../../modelo-ensino-medio}



\begin{document}

\begin{center}
  \begin{minipage}[l]{3cm}
\includegraphics[width=2cm]{logo}    
\end{minipage}\hfill
\begin{minipage}[r]{.8\textwidth}
 {\Large \scshape Atividade: Economizando para uma viagem}  
\end{minipage}
\end{center}
\vspace{.2cm}

\ifdefined\prof
%Habilidades da BNCC
\begin{objetivos}
\item \textbf{EM13MAT305} Resolver e elaborar problemas com funções logarítmicas nos quais seja necessário compreender e interpretar a variação das grandezas envolvidas, em contextos como os de abalos sísmicos, pH, radioatividade, Matemática Financeira, entre outros.

\item \textbf{EM13MAT303} Interpretar e comparar situações que envolvam juros simples com as que envolvem juros compostos, por meio de representações gráficas ou de análise de planilhas, destacando o crescimento linear ou exponencial de cada caso.
\end{objetivos}

%Caixa do Para o Professor
\begin{goals}
%Objetivos específicos
\begin{enumerate}
\item Reconhecer questões em que as incógnitas encontram-se em expoentes.
\item Desenvolver a interpretação e a comparação de situações que envolvam juros simples com as que envolvem juros compostos.
\end{enumerate}

\end{goals}

\bigskip
\begin{center}
{\large \scshape Atividade}
\end{center}
\fi

Rebeca recebeu de sua avó paterna uma herança em dinheiro no valor de R\$ $40.000{,}00$. Ela pretende utilizar esse dinheiro para conhecer a Inglaterra e o pacote de viagens que ela deseja custa R\$ $44.000{,}00$. Aplicando o valor da herança a juro de $1\%$ ao mês, quantos meses levará Rebeca para conseguir o valor desejado e realizar seu sonho?

\ifdefined\prof
\begin{solucao}

Calculando os montantes mês a mês, vemos que Rebeca precisará esperar 10 meses.
	\begin{table}[H]
	\centering

	\begin{tabu} to \textwidth{|c|l|l|}
	\hline
	\thead
	$0$ & $40.000$ & $40.000$ \\
	\hline
	$1$ & $40.000\times1{,}01$ & $40.400$ \\
	\hline
	$\vdots$ & $\vdots$ & $\vdots$ \\
	\hline
	$9$ & $40.000\times(1{,}01)^9$ & $43.720$ \\
	\hline
	$10$ & $40.000\times(1{,}01)^{10}$ & $44.160$ \\
	\hline
	\end{tabu}
	\end{table}

\end{solucao}
\fi

\end{document}