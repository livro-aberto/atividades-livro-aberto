\documentclass[10 pt,usenames,dvipsnames, oneside]{article}
\usepackage{../../../modelo-ensino-medio}



\begin{document}

\begin{center}
  \begin{minipage}[l]{3cm}
\includegraphics[width=2cm]{logo}    
\end{minipage}\hfill
\begin{minipage}[r]{.8\textwidth}
 {\Large \scshape Atividade: Além da tabela}  
\end{minipage}
\end{center}
\vspace{.2cm}

\ifdefined\prof
%Habilidades da BNCC
\begin{objetivos}
\item \textbf{EM13MAT305} Resolver e elaborar problemas com funções logarítmicas nos quais seja necessário compreender e interpretar a variação das grandezas envolvidas, em contextos como os de abalos sísmicos, pH, radioatividade, Matemática Financeira, entre outros.
\end{objetivos}

%Caixa do Para o Professor
\begin{goals}
%Objetivos específicos
\begin{enumerate}
\item Sintetizar a propriedade do logaritmo a partir da sua validade nos exemplos específicos.
\end{enumerate}

\tcblower

%Orientações e sugestões
\begin{itemize}
\item Ainda desenvolvendo a investigação das propriedades do logaritmo, chegamos a uma aplicação na matemática, que é o cálculo dos logaritmos de números além daqueles nas tabelas.

Posteriormente, será realizada a exploração dessas propriedades na resolução de problemas práticos, mas, conforme indicado pelos resultados do pisa, a exploração da matemática pura também é importante, inclusive para a resolução de problemas aplicados. Assim, essa etapa do desenvolvimento do conteúdo parece adequada para a exploração dessas propriedades.
\end{itemize}
\end{goals}

\bigskip
\begin{center}
{\large \scshape Atividade}
\end{center}
\fi

Utilize a \hyperref[potencias2_log]{tabela \ref{potencias2_log}} para encontrar os valores dos seguintes logaritmos:
\begin{enumerate}
\item $\log_2 64$;
\item $\log_2 48$;
\item $\log_2 60$;
\item $\log_2 (3/2)$;
\item $\log_2 (600/1024)$.
\end{enumerate}

\begin{table}[H]
\centering
\setlength\tabulinesep{1.5pt}
\begin{tabu} to \textwidth{|>{$}l<{$}|>{$}c<{$}|}
\hline
\rowfont{\color{white}}\rowcolor{\currentcolor!80}
\bm{2^{\log_2n}} & \bm{n} \tabularnewline
\hline
2^0 & 1 \tabularnewline
\hline
2^1 & 2 \tabularnewline
\hline
2^{1{,}58} & 3 \tabularnewline
\hline
2^{2} & 4 \tabularnewline
\hline
2^{2{,}32} & 5 \tabularnewline
\hline
2^{2{,}58} & 6 \tabularnewline
\hline
2^{2{,}81} & 7 \tabularnewline
\hline
2^{3} & 8 \tabularnewline
\hline
2^{3{,}16} & 9 \tabularnewline
\hline
2^{3{,}32} & 10 \tabularnewline
\hline
\end{tabu}\hspace{2em}
\begin{tabu} to \textwidth{|>{$}l<{$}|>{$}c<{$}|}
\hline
\rowfont{\color{white}}\rowcolor{\currentcolor!80}
\bm{2^{\log_2n}} & \bm{n} \tabularnewline
\hline
2^{3{,}46} & 11 \tabularnewline
\hline
2^{3{,}58} & 12 \tabularnewline
\hline
2^{3{,}7} & 13 \tabularnewline
\hline
2^{3{,}81} & 14 \tabularnewline
\hline
2^{3{,}9} & 15 \tabularnewline
\hline
2^{4} & 16 \tabularnewline
\hline
2^{4{,}08} & 17 \tabularnewline
\hline
2^{4{,}16} & 18 \tabularnewline
\hline
2^{4{,}24} & 19 \tabularnewline
\hline
2^{4{,}32} & 20 \tabularnewline
\hline
\end{tabu}\hspace{2em}
\begin{tabu} to \textwidth{|>{$}l<{$}|>{$}c<{$}|}
\hline
\rowfont{\color{white}}\rowcolor{\currentcolor!80}
\bm{2^{\log_2n}} & \bm{n} \tabularnewline
\hline
2^{4{,}39} & 21 \tabularnewline
\hline
2^{4{,}46} & 22 \tabularnewline
\hline
2^{4{,}52} & 23 \tabularnewline
\hline
2^{4{,}58}& 24 \tabularnewline
\hline
2^{4{,}64} & 25 \tabularnewline
\hline
2^{4{,}7} & 26 \tabularnewline
\hline
2^{4{,}75} & 27 \tabularnewline
\hline
2^{4{,}81} & 28 \tabularnewline
\hline
2^{4{,}85} & 29 \tabularnewline
\hline
2^{4{,}9} & 30 \tabularnewline
\hline
\end{tabu}

\caption{Expoentes de $2$ aproximando os naturais de $1$ à $30$}\label{potencias2_log}
\end{table}

\ifdefined\prof
\begin{solucao}

	\begin{enumerate}
	\item $\log_2 64 = \log_2 16 \times 4= \log_2 16 + \log_2 4 \approx 4+2 =6$;
	\item $\log_2 48 = \log_2 24 \times 2= \log_2 24 + \log_2 2 \approx 4{,}58+1 =5{,}58;$
	\item $\log_2 60 = \log_2 30 \times 2= \log_2 30 + \log_2 2 \approx 4{,}9+1 =5{,}9;$
	\item $\log_2 3/2 = \log_2 3 - \log_2 2 \approx 1{,}58-1 =0{,}58;$
	\item $\log_2 600/1024 = \log_2 75/128= \log_2 3 \times 25- \log_2 8 \times 16= \log_2 3 + \log_2 25- (\log_2 8 +\log_2 16) \approx 1{,}58 + 4{,}64-(3+4)= 6{,}22-7 = -0{,}78.$
	\end{enumerate}

\end{solucao}
\fi

\end{document}