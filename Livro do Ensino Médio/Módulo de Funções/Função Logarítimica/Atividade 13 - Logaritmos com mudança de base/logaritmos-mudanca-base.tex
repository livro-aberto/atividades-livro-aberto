\documentclass[10 pt,usenames,dvipsnames, oneside]{article}
\usepackage{../../../modelo-ensino-medio}



\begin{document}

\begin{center}
  \begin{minipage}[l]{3cm}
\includegraphics[width=2cm]{logo}    
\end{minipage}\hfill
\begin{minipage}[r]{.8\textwidth}
 {\Large \scshape Atividade: Logaritmos com mudança de base}  
\end{minipage}
\end{center}
\vspace{.2cm}

\ifdefined\prof
%Habilidades da BNCC
\begin{objetivos}
\item \textbf{EM13MAT403} Analisar e estabelecer relações, com ou sem apoio de tecnologias digitais, entre as representações de funções exponencial e logarítmica expressas em tabelas e em plano cartesiano, para identificar as características fundamentais (domínio, imagem, crescimento) de cada função.
\end{objetivos}

%Caixa do Para o Professor
\begin{goals}
%Objetivos específicos
\begin{enumerate}
\item Aplicar a propriedade da mudança de base em aproximações manuais.
\end{enumerate}

\end{goals}

\bigskip
\begin{center}
{\large \scshape Atividade}
\end{center}
\fi

Determine o valor dos logaritmos abaixo, sabendo que $\log 2 = 0{,}301$, $\log 3 = 0{,}477$ e $\log 7 = 0{,}845$:
\begin{enumerate}
\item $\log_3 7$;
\item $\log_7 21$;
\item $\log_9 16$.
\end{enumerate}

\ifdefined\prof
\begin{solucao}

\begin{enumerate}
	\item  $\log_3 7=\dfrac{\log 7}{\log 3} = \dfrac{0{,}845}{0{,}477} \approx 1{,}7715;$
	\item  $\log_7 21=\dfrac{\log 3 + \log 7}{\log 7} = \dfrac{0{,}477 + 0{,}845}{0{,}845} \approx 1{,}5645;$
	\item  $\log_9 16=\dfrac{\log 16}{\log 9} =\dfrac{4\log 2}{2\log 3} = \frac{2*0{,}301}{0{,}477} \approx 1{,}262.$
	\end{enumerate}

\end{solucao}
\fi

\end{document}