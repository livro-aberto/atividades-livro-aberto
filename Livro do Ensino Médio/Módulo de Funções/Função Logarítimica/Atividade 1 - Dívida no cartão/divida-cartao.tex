\documentclass[10 pt,usenames,dvipsnames, oneside]{article}
\usepackage{../../../modelo-ensino-medio}



\begin{document}

\begin{center}
  \begin{minipage}[l]{3cm}
\includegraphics[width=2cm]{logo}    
\end{minipage}\hfill
\begin{minipage}[r]{.8\textwidth}
 {\Large \scshape Atividade: Dívida no cartão}  
\end{minipage}
\end{center}
\vspace{.2cm}

\ifdefined\prof
%Habilidades da BNCC
\begin{objetivos}
\item \textbf{EM13MAT305} Resolver e elaborar problemas com funções logarítmicas nos quais seja necessário compreender e interpretar a variação das grandezas envolvidas, em contextos como os de abalos sísmicos, pH, radioatividade, Matemática Financeira, entre outros.
\end{objetivos}

%Caixa do Para o Professor
\begin{goals}
%Objetivos específicos
\begin{enumerate}
\item Observar situações práticas em que a incógnita está no expoente.
\item Desenvolver raciocínios informais sobre o número de vezes que uma operação de multiplicação é aplicada.
\end{enumerate}

%Orientações e sugestões
\end{goals}

\bigskip
\begin{center}
{\large \scshape Atividade}
\end{center}
\fi

Um pessoa tem uma dívida de R\$ $10.000{,}00$ no cartão de crédito, que cobra juros de 10\% ao mês. Se essa dívida não for amortizada, em quantos meses ela ultrapassará R\$ $15.000{,}00$? E R\$ $20.000{,}00$? E R\$ $25.000{,}00$?

\ifdefined\prof
\begin{solucao}

Podemos utilizar a tabela para calcular o montante em cada mês a partir de R$\$10.000{,}00$.

\begin{table}[H]
\centering

\begin{tabu} to \textwidth{|c|c|}
\hline
\thead
Mês & Montante (R\$) \\
\hline
0 & $10.000$\\
\hline
1 & $11.000$\\
\hline
2 & $12.100$\\
\hline 
3 & $13.310$\\
\hline
4 & $14.640$\\
\hline
5 & $16.100$\\
\hline
6 & $17.710$\\
\hline
7 & $19.480$\\
\hline
8 & $21.430$\\
\hline
9 & $23.570$\\
\hline
10 & $25.930$\\
\hline
\end{tabu}
\end{table}

Assim, atingimos R$\$15.000{,}00$, R$\$20.000{,}00$ e R$\$25.000{,}00$ em, respectivamente 5, 8 e 10 meses.

\end{solucao}
\fi

\end{document}