\documentclass[10 pt,usenames,dvipsnames, oneside]{article}
\usepackage{../../../modelo-ensino-medio}



\begin{document}

\begin{center}
  \begin{minipage}[l]{3cm}
\includegraphics[width=2cm]{logo}    
\end{minipage}\hfill
\begin{minipage}[r]{.8\textwidth}
 {\Large \scshape Atividade: Regressão linear}  
\end{minipage}
\end{center}
\vspace{.2cm}

\ifdefined\prof
%Habilidades da BNCC
\begin{objetivos}
\item \textbf{EM13MAT104} Interpretar taxas e índices de natureza socioeconômica (índice de desenvolvimento humano, taxas de inflação, entre outros), investigando os processos de cálculo desses números, para analisar criticamente a realidade e produzir argumentos.

\item \textbf{EM13MAT102} Analisar tabelas, gráficos e amostras de pesquisas estatísticas apresentadas em relatórios divulgados por diferentes meios de comunicação, identificando, quando for o caso, inadequações que possam induzir a erros de interpretação, como escalas e amostras não apropriadas.
\end{objetivos}

%Caixa do Para o Professor
\begin{goals}
%Objetivos específicos
\begin{enumerate}
\item Analisar uma situação real a partir de dados concretos, com auxilio da tecnologia, e fazer previsões sobre valores futuros.
\item Avaliar criticamente os resultados obtidos.
\end{enumerate}

\end{goals}

\bigskip
\begin{center}
{\large \scshape Atividade}
\end{center}
\fi

Utilizando os dados reais do número de casos nos Estados Unidos (fonte: \textit{Our World in Data}) entre 22/01/2020 e 21/03/2020, aplicamos regressão linear com uma ferramenta computacional para estimar o número de casos pela fórmula $f(t) = 0{,}000275279 \times 1{,}36451^t$, onde $t=0$ representa o dia 22/01 e $t= 59$ o dia 21/03. %(Ref: https://towardsdatascience.com/modeling-exponential-growth-49a2b6f22e1f)

Utilizando a fórmula, quantos dias seriam necessários para que a pandemia ultrapassasse 100.000 infectados naquele país?

\ifdefined\prof
\begin{solucao}

Utilizando os dados reais do número de casos nos Estados Unidos (fonte: Our World in Data) entre 22/01/2020 e 21/03/2020, aplicamos regressão linear com uma ferramenta computacional para estimar o número de casos pela fórmula $f(t) = 0{,}000275279 \times 1{,}36451^t$, onde $t=0$ representa o dia 22/01 e $t= 59$ o dia 21/03. %(Ref: https://towardsdatascience.com/modeling-exponential-growth-49a2b6f22e1f)

	Utilizando a fórmula, quantos dias seriam necessários para que a pandemia ultrapassasse 100.000 infectados naquele país?

\end{solucao}
\fi

\end{document}