\documentclass[10 pt,usenames,dvipsnames, oneside]{article}
\usepackage{../../../modelo-ensino-medio}



\begin{document}

\begin{center}
  \begin{minipage}[l]{3cm}
\includegraphics[width=2cm]{logo}    
\end{minipage}\hfill
\begin{minipage}[r]{.8\textwidth}
 {\Large \scshape Atividade: Funções inversas na prática}  
\end{minipage}
\end{center}
\vspace{.2cm}

\ifdefined\prof
%Habilidades da BNCC
\begin{objetivos}
\item \textbf{EM13MAT403} Analisar e estabelecer relações, com ou sem apoio de tecnologias digitais, entre as representações de funções exponencial e logarítmica expressas em tabelas e em plano cartesiano, para identificar as características fundamentais (domínio, imagem, crescimento) de cada função.
\end{objetivos}

%Caixa do Para o Professor
\begin{goals}
%Objetivos específicos
\begin{enumerate}
\item Analisar e estabelecer relações entre as
representações de funções exponencial e
logarítmica.
\end{enumerate}

\tcblower

%Orientações e sugestões
\begin{itemize}
\item A troca nos papeis de $x$ e $y$ como argumentos das funções pode causar dificuldade e pode ser necessário relembrar que uma função age sobre seu argumento, independente das variáveis que sejam utilizadas.

	A atividade propõem colocar a "mão na massa" na exponencial e no logaritmo, observando a ação da inversão explicitada na observação anterior.
\end{itemize}
\end{goals}

\bigskip
\begin{center}
{\large \scshape Atividade}
\end{center}
\fi

Complete a tabela abaixo calculando a exponencial do logaritmo e o logaritmo da exponencial.

\begin{table}[H]
\centering

\begin{tabu} to \textwidth{|c|c|c|}
\hline
\thead
$\bm{x}$ & $\bm{y=2^x}$ & $\bm{{\log_2 x}}$ \\
\hline
$1$ & & \\
\hline
$2$ & & \\
\hline
$3$ & & \\
\hline
$4$ & & \\
\hline
\end{tabu}
\hspace{2em}
\begin{tabu} to \textwidth{|c|c|c|}
\hline
\thead
$\bm{y}$ & $\bm{x=3^y}$ & $\bm{{\log_3 x}}$ \tabularnewline
\hline
$-1$ & & \\
\hline
$1$ & & \\
\hline
$0$ & & \\
\hline
$2$ & & \\
\hline
\end{tabu}

\end{table}


\ifdefined\prof
\begin{solucao}


\begin{table}[H]
	\centering
	
	\begin{tabu} to \textwidth{|c|c|c|}
	\hline
	\thead
	$\bm{x}$ & $\bm{y=2^x}$ & $\bm{{\log_2 x}}$ \\
	\hline
	$1$ & $2$ & $1$ \\
	\hline
	$2$ & $4$ & $2$ \\
	\hline
	$3$ & $8$ & $3$ \\
	\hline
	$4$ & $16$ & $4$ \\
	\hline
	\end{tabu}
	\hspace{2em}
	\begin{tabu} to \textwidth{|c|c|c|}
	\hline
	\thead
	$\bm{y}$ & $\bm{x=3^y}$ & $\bm{{\log_3 x}}$ \tabularnewline
	\hline
	$-1$ & $1/3$ & $-1$ \\
	\hline
	$1$ & $3$ & $1$ \\
	\hline
	$0$ & $1$ & $0$ \\
	\hline
	$2$ & $9$ & $2$ \\
	\hline
	\end{tabu}
	\end{table}

\end{solucao}
\fi

\end{document}