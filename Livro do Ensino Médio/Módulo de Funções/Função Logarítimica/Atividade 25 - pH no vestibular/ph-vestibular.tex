\documentclass[10 pt,usenames,dvipsnames, oneside]{article}
\usepackage{../../../modelo-ensino-medio}



\begin{document}

\begin{center}
  \begin{minipage}[l]{3cm}
\includegraphics[width=2cm]{logo}    
\end{minipage}\hfill
\begin{minipage}[r]{.8\textwidth}
 {\Large \scshape Atividade: pH no vestibular}  
\end{minipage}
\end{center}
\vspace{.2cm}

\ifdefined\prof
%Habilidades da BNCC
\begin{objetivos}
\item \textbf{EM13MAT305} Resolver e elaborar problemas com funções logarítmicas nos quais seja necessário compreender e interpretar a variação das grandezas envolvidas, em contextos como os de abalos sísmicos, pH, radioatividade, Matemática Financeira, entre outros.
\end{objetivos}

%Caixa do Para o Professor
\begin{goals}
%Objetivos específicos
\begin{enumerate}
\item Resolver questões contextualizadas envolvendo o contextos de escalas logarítmicas.
\end{enumerate}

\tcblower

%Orientações e sugestões
\begin{itemize}
\item O estudo do pH abre a oportunidade para o estudo interdisciplinar do assunto na química. Isso é particularmente recomendado pois o cálculo pode ser realizado de forma diferente em outros contextos.
\end{itemize}
\end{goals}

\bigskip
\begin{center}
{\large \scshape Atividade}
\end{center}
\fi
\textit{(Adaptado de (UDESC 2009))}

“Chuva ácida” é um termo que se refere à precipitação, a partir da
atmosfera, de chuva com quantidades de ácidos nítricos e sulfúrico maiores que o
normal. Os precursores da chuva ácida vem tanto de fontes naturais, tais como vulcões
e vegetação em decomposição, quanto de processos industriais, principalmente
emissões de dióxido de enxofre e óxidos de nitrogênio resultantes da queima de
combustíveis fósseis. O pH da água da chuva considerado normal é de 5,5 (devido
à presença de ácido carbônico proveniente da solubilização de dióxido de carbono).
Um químico monitorando uma região altamente industrializada observou que o pH da
água da chuva era igual a 4,5. Considerando que a acidez está relacionada com a
concentração de $H_3O^+$, é correto afirmar que a água com pH 4,5 era:
\begin{enumerate}[itemsep=0pt]
\item duas vezes mais básica que o normal.
\item duas vezes mais ácida que o normal.
\item dez vezes mais básica que o normal.
\item dez vezes mais ácida que o normal.
\item cem vezes mais ácida que o normal.
\end{enumerate}

\ifdefined\prof
\begin{solucao}

A chuva ácida tem pH 1 unidade menor do que o pH da chuva comum. Isso indica que a atividade de íons de hidrônio é 10 vezes maior. Assim a chuva ácida é dez vezes mais ácida.

\end{solucao}
\fi

\end{document}