\documentclass[10 pt,usenames,dvipsnames, oneside]{article}
\usepackage{../../../modelo-ensino-medio}



\begin{document}

\begin{center}
  \begin{minipage}[l]{3cm}
\includegraphics[width=2cm]{logo}    
\end{minipage}\hfill
\begin{minipage}[r]{.8\textwidth}
 {\Large \scshape Atividade: Ordens de grandeza}  
\end{minipage}
\end{center}
\vspace{.2cm}

\ifdefined\prof
%Habilidades da BNCC
\begin{objetivos}
\item \textbf{EM13MAT313}  Resolver e elaborar problemas que envolvem edições em que se discuta o emprego de algarismos significativos e algarismos duvidosos, utilizando, quando necessário, a notação científica. 
\end{objetivos}

%Caixa do Para o Professor
\begin{goals}
%Objetivos específicos
\begin{enumerate}
\item Aplicar a propriedade do logaritmo da potência para encontrar aproximações inteiras de logaritmos.
\item Explorar o conceito de erro nas aproximações dos logaritmos.
\end{enumerate}

\end{goals}

\bigskip
\begin{center}
{\large \scshape Atividade}
\end{center}
\fi

Encontre potências de $10$ consecutivas que limitem os logaritmandos abaixo superiormente e inferiormente e use-as para encontrar a parte inteira de:
\begin{enumerate}
\item $\log 6{,}7$;
\item $\log 23$;
\item $\log 179{,}28$;
\item $\log 8341$.
\end{enumerate}

\ifdefined\prof
\begin{solucao}

\begin{enumerate}
	\item Como $10^0=1 < 6{,}7 <10 =10^1$ e $f(t)=10^t$ é crescente, temos $0 < \log 6{,}7 < 1$ e sua parte inteira (e sua ordem de grandeza) é $0$.
	\item Como $10^1=10 < 23 <100 =10^2$ e $f(t)=10^t$ é crescente, temos $1 < \log 23 < 2$ e sua parte inteira (e sua ordem de grandeza) é $1$.
	\item Como $10^2=100 < 179{,}28 <1000 =10^3$ e $f(t)=10^t$ é crescente, temos $2 < \log 179{,}28 < 3$ e sua parte inteira (e sua ordem de grandeza) é $2$.
	\item Como $10^3=1000 < 8341 <10000 =10^4$ e $f(t)=10^t$ é crescente, temos $3 < \log 8341 < 4$ e sua parte inteira (e sua ordem de grandeza) é $3$.
	\end{enumerate}

\end{solucao}
\fi

\end{document}