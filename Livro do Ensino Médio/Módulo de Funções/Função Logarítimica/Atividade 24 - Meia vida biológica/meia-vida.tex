\documentclass[10 pt,usenames,dvipsnames, oneside]{article}
\usepackage{../../../modelo-ensino-medio}



\begin{document}

\begin{center}
  \begin{minipage}[l]{3cm}
\includegraphics[width=2cm]{logo}    
\end{minipage}\hfill
\begin{minipage}[r]{.8\textwidth}
 {\Large \scshape Atividade: Meia vida biológica}  
\end{minipage}
\end{center}
\vspace{.2cm}

\ifdefined\prof
%Habilidades da BNCC
\begin{objetivos}
\item \textbf{EM13MAT305} Resolver e elaborar problemas com funções logarítmicas nos quais seja necessário compreender e interpretar a variação das grandezas envolvidas, em contextos como os de abalos sísmicos, pH, radioatividade, Matemática Financeira, entre outros.
\end{objetivos}

%Caixa do Para o Professor
\begin{goals}
%Objetivos específicos
\begin{enumerate}
\item Resolver questões contextualizadas envolvendo o contexto de decaimento.
\end{enumerate}

\tcblower

%Orientações e sugestões
\begin{itemize}
\item Na farmácia e na medicina utiliza-se a meia vida como meio para aproximar a concentração plasmática de um fármaco em um organismo após a administração. A resolução da questão é análoga às resoluções envolvendo o decaimento radioativo e a adaptação das técnicas conhecidas entre contextos diferentes é fundamental, uma vez que a exploração de todos os contextos específicos possíveis não é viável.
\end{itemize}
\end{goals}

\bigskip
\begin{center}
{\large \scshape Atividade}
\end{center}
\fi

A meia vida de um fármaco é o tempo necessário para a concentração plasmática dele cair pela metade em um organismo. Esse conceito serve para estimarmos qual a concentração de determinado medicamento no organismo. O antibiótico Amoxicilina tem meia vida biológica de $61{,}3$ minutos. Vejamos a seguinte situação: um paciente de $75$ Kg recebe uma dose do medicamento de $500$ mg. Uma pessoa com esse peso tem cerca de 5 litros de sangue (a quantidade exata pode variar de pessoa para pessoa). Supondo que o medicamento foi completamente dissolvido e distribuído uniformemente no sangue, teríamos a concentração plasmática do medicamento igual a $\frac{500}{5000}= 0{,}1$ mg/ml. Escreva uma função exponencial de base $e$ para descrever concentração plasmática (em mg/ml) após $t$ minutos. Utilize a função encontrada para determinar quanto tempo levará para a concentração cair abaixo de $0{,}005$ mg/ml. 



\ifdefined\prof
\begin{solucao}

Como a concentração cai pela metade a cada $61{,}3$ minutos, poderíamos escrever a seguinte função para a concentração
$$
Q(t)=0{,}1(1/2)^{t/61{,}3},
$$
que pode ser escrita na base $e$ como
\begin{align*}
Q(t)=&0{,}1e^{(t/61{,}3)\ln{(1/2)}}\\
\approx &0{,}1e^{-t\frac{0{,}693}{61{,}3}} = 0{,}1e^{-0{,}0113t}.
\end{align*}
Assim o tempo necessário para a concentração cair abaixo de 0,005 pode ser calculado resolvendo:
\begin{align*}
&0{,}005 = 0{,}1e^{-0{,}0113t}\\
\Rightarrow &e^{-0{,}0113t} = 0{,}05\\
\Rightarrow &t = \frac{-\ln 0{,}05}{0{,}0113} \approx 265{,}1.
\end{align*}
E seria necessário mais de 265min e 10s para a concentração cair abaixo de $0{,}005$ mg/ml.

\end{solucao}
\fi

\end{document}