\documentclass[10 pt,usenames,dvipsnames, oneside]{article}
\usepackage{../../../modelo-ensino-medio}



\begin{document}

\begin{center}
  \begin{minipage}[l]{3cm}
\includegraphics[width=2cm]{logo}    
\end{minipage}\hfill
\begin{minipage}[r]{.8\textwidth}
 {\Large \scshape Atividade: Cálculos com ferramentas computacionais}  
\end{minipage}
\end{center}
\vspace{.2cm}

\ifdefined\prof
%Habilidades da BNCC
\begin{objetivos}
\item \textbf{EM13MAT305} Resolver e elaborar problemas com funções logarítmicas nos quais seja necessário compreender e interpretar a variação das grandezas envolvidas, em contextos como os de abalos sísmicos, pH, radioatividade, Matemática Financeira, entre outros.
\end{objetivos}

%Caixa do Para o Professor
\begin{goals}
%Objetivos específicos
\begin{enumerate}
\item Encontrar o valor dos logaritmos em bases distintas através de tabelas e/ou de aplicativos.
\end{enumerate}

\end{goals}

\bigskip
\begin{center}
{\large \scshape Atividade}
\end{center}
\fi

No applet do geogebra, mova os controles deslizantes das variáveis “a” e “b” e encontre um valor aproximado (com até duas casas decimais) para:
\begin{multicols}{2}
\begin{enumerate}
\item $\log_{1,41} 4$;
\item $\log_{2,71} 4$;
\item $\log_2 5$;
\item $\log_4 7$;
\item $\log_{2,71} 9$;
\item $\log_3 7$.
\end{enumerate}
\end{multicols}

\ifdefined\prof
\begin{solucao}

	\begin{enumerate}
	\item $\log_{1,41} 4 \approx 4,03$;
	\item $\log_{2,71} 4 \approx 1,39$;
	\item $\log_{2} 5 \approx 2,32$;
	\item $\log_{4} 7 \approx 1,4$;
	\item $\log_{2,71} 9 \approx 2,2$;
	\item $\log_{3} 7 \approx 1,77$.
	\end{enumerate}

\end{solucao}
\fi

\end{document}