\documentclass[10 pt,usenames,dvipsnames, oneside]{article}
\usepackage{../../../modelo-ensino-medio}



\begin{document}

\begin{center}
  \begin{minipage}[l]{3cm}
\includegraphics[width=2cm]{logo}    
\end{minipage}\hfill
\begin{minipage}[r]{.8\textwidth}
 {\Large \scshape Atividade: Escrevendo os logaritmos}  
\end{minipage}
\end{center}
\vspace{.2cm}

\ifdefined\prof
%Habilidades da BNCC
\begin{objetivos}
\item \textbf{EM13MAT305} Resolver e elaborar problemas com funções logarítmicas nos quais seja necessário compreender e interpretar a variação das grandezas envolvidas, em contextos como os de abalos sísmicos, pH, radioatividade, Matemática Financeira, entre outros.
\end{objetivos}

%Caixa do Para o Professor
\begin{goals}
%Objetivos específicos
\begin{enumerate}
\item Compreender o logaritmo como o expoente ao qual a base deve ser elevada para obter determinado número e reconhecer que essa definição corresponde aos valores procurados nas atividades anteriores.
\end{enumerate}

\tcblower

%Orientações e sugestões
Indica-se que seja reforçado com os estudantes que o logaritmo é o expoente, como na observação do primeiro quadro. Então os estudantes podem praticar individualmente a notação na atividade "Escrevendo os logaritmos", que propõem a identificação dos valores obtidos nas atividades anteriores através dos logaritmos.
\end{goals}

\bigskip
\begin{center}
{\large \scshape Atividade}
\end{center}
\fi

Identifique nas atividades "Bactérias no leite"\, e "Economizando para uma viagem"\, os logaritmandos, as bases e as aproximações de logaritmos encontradas na tabela como respostas e escreva-os utilizando a notação simbólica $\log_b a \approx c$.

% \ifdefined\prof
% \begin{solucao}

% \begin{enumerate}
% \item
% \end{enumerate}

% \end{solucao}
% \fi

\end{document}