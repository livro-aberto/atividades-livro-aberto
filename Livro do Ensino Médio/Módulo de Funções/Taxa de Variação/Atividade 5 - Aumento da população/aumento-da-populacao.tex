\documentclass[10 pt,usenames,dvipsnames, oneside]{article}
\usepackage{../../../modelo-ensino-medio}


\begin{document}

\begin{center}
  \begin{minipage}[l]{3cm}
\includegraphics[width=2cm]{logo}    
\end{minipage}\hfill
\begin{minipage}[r]{.8\textwidth}
 {\Large \scshape Atividade: Aumento da população}  
\end{minipage}
\end{center}
\vspace{.2cm}

\ifdefined\prof
\begin{objetivos}
\item \textbf{LAF2} Compreender a taxa de variação como uma medida de covariação entre grandezas e utilizá-la para interpretar situações reais.
\end{objetivos}

\begin{goals}
\begin{enumerate}

\item [OE1] Perceber a importância do intervalo no cálculo da taxa de variação média.

\item [OE2] Diferenciar “variação da função” de “taxa de variação média”.

\end{enumerate}

\tcblower


\begin{itemize}
\item Podemos cair no equívoco de achar que a taxa de variação média total é a média
aritmética das taxas intermediárias. Nesse caso, achar que $1{,}67$ deveria ser a média
aritmética entre $2$ e $1{,}43$. Contudo, um cálculo simples mostra que não é o caso. Isso se
deve ao fato de estarmos considerando intervalos diferentes de tempo. Para funcionar
devemos fazer a média aritmética ponderada, nesse caso.
\end{itemize}

\end{goals}


\bigskip
\begin{center}
{\large \scshape Atividade}
\end{center}
\fi

A população brasileira alcançou os 210,1 milhões de habitantes em 2019. Segundo dados do IBGE houve um aumento de $0{,}79\%$ em relação a 2018 quando o instituto estimou a população do país em 208,5 milhões de habitantes. A tabela abaixo mostra a população aproximada de brasileiros nos anos de 2007, 2012 e 2019.

\begin{table}[H]
\centering
\begin{tabu} to \textwidth{|c|c|}
\hline
\thead
População aproximada & Ano \\
\hline
190 milhões & 2007 \\
\hline
200 milhões & 2012 \\
\hline
210 milhões & 2019 \\
\hline
\end{tabu}
\end{table}
\begin{enumerate}
\item Qual foi a variação da população entre 2007 e 2012? E entre 2012 e 2019?
\item Quando a população aumentou mais rápido? Explique.
\item De quanto foi o aumento médio anual entre 2007 e 2012? E entre 2012 e 2019? E entre 2007 e 2019?
\end{enumerate}

\ifdefined\prof
\clearpage
\begin{solucao}

\begin{enumerate}
\item Aumento de 10 milhões de pessoas. Idem.
\item Entre 2007 e 2012, pois foi um intervalo menor de tempo (5 anos) comparado aos 7 anos entre 2012 e 2019.
\item 2 milhões de pessoas a mais por ano entre 2007 e 2012. Aproximadamente 1,42 milhão de pessoas a mais por ano entre 2012 e 2019. No período total, tivemos aumento de aproximadamente $1{,}67$ milhão de pessoas por ano.
\end{enumerate}

\end{solucao}
\fi

\end{document}