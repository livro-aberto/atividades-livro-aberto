\documentclass[10 pt,usenames,dvipsnames, oneside]{article}
\usepackage{../../../modelo-ensino-medio}


\begin{document}

\begin{center}
  \begin{minipage}[l]{3cm}
\includegraphics[width=2cm]{logo}    
\end{minipage}\hfill
\begin{minipage}[r]{.8\textwidth}
 {\Large \scshape Atividade: Decodificando a mensagem}  
\end{minipage}
\end{center}
\vspace{.2cm}

\ifdefined\prof
\begin{objetivos}
\item \textbf{LAF1} Compreender função como uma relação de dependência entre duas variáveis, as ideias de domínio, contradomínio e imagem, e suas representações algébricas e gráficas e utilizá-las para analisar, interpretar e resolver problemas em contextos diversos, inclusive fenômenos naturais, sociais e de outras áreas.
\end{objetivos}

\begin{goals}
\begin{enumerate}

\item[OE1] Estabelecer modelo matemático a partir de funções, mais especificamente, em uma situação que envolve codificação de mensagens.

\item[OE2] Compreender intuitivamente as condições necessárias para a existência da inversa de uma função. (injetividade e sobrejetividade)

\end{enumerate}

\tcblower

\begin{itemize}
\item Na solução do item (d) estimule seus estudantes a descrever com palavras de maneira precisa o que acontece com os números maiores que $26$ caso ele use a expressão $f(x)=x+14$.
\end{itemize}

\end{goals}

\bigskip
\begin{center}
{\large \scshape Atividade}
\end{center}
\fi

Um dos conceitos mais importantes para a segurança na \emph{internet} nos dias de de hoje é o que chamamos de \textbf{criptografia} (do grego \emph{criptos} = escondido, \emph{grafia} = escrita). Segundo o site \emph{wikipedia} ela é o estudo dos princípios e técnicas pelas quais a informação pode ser transformada da sua forma original para outra codificada, de forma que possa ser conhecida apenas por seu destinatário (detentor da “chave secreta”), o que a torna difícil de ser decifrada por alguém não autorizado. Em outras palavras, cria-se um código que pode ser facilmente desfeito (decodificado) mas apenas por aqueles que conhecem a codificação.

Considere a seguinte maneira de codificar o alfabeto

\begin{table}[H]
\centering
\setlength\tabcolsep{3pt}
\begin{tabu} to \textwidth{|c|c|c|c|c|c|c|c|c|c|c|c|c|c|c|c|c|c|c|c|c|c|c|c|c|c|c|}
\hline
\cellcolor{primario!80}{\textcolor{white}{\textbf{Original}}} & A & B & C & D & E & F & G & H & I & J & K & L & M & N & O & P & Q & R & S & T & U & V & W & X & Y & Z \\
\hline
\cellcolor{primario!80}{\textcolor{white}{\textbf{Código}}} & P & Q & R & S & T & U & V & W & X & Y & Z & A & B & C & D & E & F & G & H & I & J & K & L & M & N & O \\
\hline
\end{tabu}
\end{table}

\begin{enumerate}
\item {} 
Use o código acima para codificar a palavra IMAGEM.

\item {} 
Se você recebesse uma mensagem com a expressão RGXEIDVGPUPG, como faria para decodificá-la?

A codificação acima pode também ser representada em um gráfico em que no eixo horizontal estão as letras originais e no vertical os seus respectivos códigos.

\begin{figure}[H]
\centering

\begin{tikzpicture}
\draw[scale=.5](0,0)grid(26,26);
\foreach \i [count=\x from 0] in{{A}, {B},{C}, {D}, {E}, {F}, {G}, {H},{I},{J},{K},{L},{M},{N},{O},{P},{Q},{R},{S},{T},{U},{V},{W},{X},{Y},{Z}}
\draw (.2+.5*\x,-.4) node {\i};
\foreach \i [count=\x from 0] in{{A}, {B},{C}, {D}, {E}, {F}, {G}, {H},{I},{J},{K},{L},{M},{N},{O},{P},{Q},{R},{S},{T},{U},{V},{W},{X},{Y},{Z}}
\draw (-.4,.2+.5*\x) node {\i};
\fill[color=primario!80](5.5,0)--(6,0)--(6,.5)--(5.5,.5);
\fill[color=primario!80](6,.5)--(6.5,.5)--(6.5,1)--(6,1);
\fill[color=primario!80](6.5,1)--(7,1)--(7,1.5)--(6.5,1.5);
\fill[color=primario!80](7,1.5)--(7.5,1.5)--(7.5,2)--(7,2);
\fill[color=primario!80] ( 7.5 , 2.0 )--( 8.0 , 2.0 )--( 8.0 , 2.5 )--( 7.5 , 2.5 );
\fill[color=primario!80] ( 8.0 , 2.5 )--( 8.5 , 2.5 )--( 8.5 , 3.0 )--( 8.0 , 3.0 );
\fill [color=primario!80]( 8.5 , 3.0 )--( 9.0 , 3.0 )--( 9.0 , 3.5 )--( 8.5 , 3.5 );
\fill [color=primario!80]( 9.0 , 3.5 )--( 9.5 , 3.5 )--( 9.5 , 4.0 )--( 9.0 , 4.0 );
\fill [color=primario!80]( 9.5 , 4.0 )--( 10.0 , 4.0 )--( 10.0 , 4.5 )--( 9.5 , 4.5 );
\fill [color=primario!80]( 10.0 , 4.5 )--( 10.5 , 4.5 )--( 10.5 , 5.0 )--( 10.0 , 5.0 );
\fill[color=primario!80] ( 10.5 , 5.0 )--( 11.0 , 5.0 )--( 11.0 , 5.5 )--( 10.5 , 5.5 );
\fill[color=primario!80] ( 11.0 , 5.5 )--( 11.5 , 5.5 )--( 11.5 , 6.0 )--( 11.0 , 6.0 );
\fill[color=primario!80] ( 11.5 , 6.0 )--( 12.0 , 6.0 )--( 12.0 , 6.5 )--( 11.5 , 6.5 );
\fill[color=primario!80] ( 12.0 , 6.5 )--( 12.5 , 6.5 )--( 12.5 , 7.0 )--( 12.0 , 7.0 );
\fill[color=primario!80] ( 12.5 , 7.0 )--( 13.0 , 7.0 )--( 13.0 , 7.5 )--( 12.5 , 7.5 );
\fill[color=primario!80] ( 0.0 , 7.5 )--( 0.5 , 7.5 )--( 0.5 , 8.0 )--( 0.0 , 8.0 );
\fill[color=primario!80] ( 0.5 , 8.0 )--( 1.0 , 8.0 )--( 1.0 , 8.5 )--( 0.5 , 8.5 );
\fill[color=primario!80] ( 1.0 , 8.5 )--( 1.5 , 8.5 )--( 1.5 , 9.0 )--( 1.0 , 9.0 );
\fill[color=primario!80] ( 1.5 , 9.0 )--( 2.0 , 9.0 )--( 2.0 , 9.5 )--( 1.5 , 9.5 );
\fill[color=primario!80] ( 2.0 , 9.5 )--( 2.5 , 9.5 )--( 2.5 , 10.0 )--( 2.0 , 10.0 );
\fill[color=primario!80] ( 2.5 , 10.0 )--( 3.0 , 10.0 )--( 3.0 , 10.5 )--( 2.5 , 10.5 );
\fill[color=primario!80] ( 3.0 , 10.5 )--( 3.5 , 10.5 )--( 3.5 , 11.0 )--( 3.0 , 11.0 );
\fill[color=primario!80] ( 3.5 , 11.0 )--( 4.0 , 11.0 )--( 4.0 , 11.5 )--( 3.5 , 11.5 );
\fill[color=primario!80] ( 4.0 , 11.5 )--( 4.5 , 11.5 )--( 4.5 , 12.0 )--( 4.0 , 12.0 );
\fill[color=primario!80] ( 4.5 , 12.0 )--( 5.0 , 12.0 )--( 5.0 , 12.5 )--( 4.5 , 12.5 );
\fill[color=primario!80] ( 5.0 , 12.5 )--( 5.5 , 12.5 )--( 5.5 , 13.0 )--( 5.0 , 13.0 );
\draw(12.3,-1)node{alfabeto};
\draw(-1,12.3) node[rotate=90.]{C\'{o}digo};
\end{tikzpicture}
\end{figure}
\item {} 
Usando ainda o código acima escreva uma mensagem codificada com duas ou três palavras e troque com algum colega seu de classe. Decodifique a mensagem que recebeu.

Você deve ter percebido que a codificação é uma função do conjunto das letras do alfabeto em si mesmo: todas as letras precisam ter um código e uma mesma letra não pode ter mais de um código associada a si.

\item {} 
Seja \(X\) o conjunto dos números naturais de \(1\) a \(26\). Fazendo a correspondência, \(A \mapsto 1, B \mapsto 2, C \mapsto 3\), e assim por diante até \(Z \mapsto 26\), determine uma função \(f:X\to X\) que corresponda ao código acima. Observe que por exemplo, \(f(1)=16\).

\item {} 
Usando a expressão \(f(x)=x^2\) crie um novo código entre as letras, representando-o no gráfico. O que devemos fazer quando os valores são  maiores que 26?

\item {} 
Considerando o código do gráfico abaixo, tente decodificar a palavra APQGJXV.

\begin{figure}[H]
\centering

\begin{tikzpicture}
\draw[scale=.5](0,0)grid(26,26);
\foreach \i [count=\x from 0] in{{A}, {B},{C}, {D}, {E}, {F}, {G}, {H},{I},{J},{K},{L},{M},{N},{O},{P},{Q},{R},{S},{T},{U},{V},{W},{X},{Y},{Z}}
\draw (.2+.5*\x,-.4) node {\i};
\foreach \i [count=\x from 0] in{{A}, {B},{C}, {D}, {E}, {F}, {G}, {H},{I},{J},{K},{L},{M},{N},{O},{P},{Q},{R},{S},{T},{U},{V},{W},{X},{Y},{Z}}
\draw (-.4,.2+.5*\x) node {\i};
\fill[color=primario!80] ( 0.0 , 1.5 )--( 0.5 , 1.5 )--( 0.5 , 2.0 )--( 0.0 , 2.0 );
\fill[color=primario!80] ( 0.5 , 2.0 )--( 1.0 , 2.0 )--( 1.0 , 2.5 )--( 0.5 , 2.5 );
\fill[color=primario!80] ( 1.0 , 2.5 )--( 1.5 , 2.5 )--( 1.5 , 3.0 )--( 1.0 , 3.0 );
\fill[color=primario!80] ( 1.5 , 3.0 )--( 2.0 , 3.0 )--( 2.0 , 3.5 )--( 1.5 , 3.5 );
\fill[color=primario!80] ( 2.0 , 3.5 )--( 2.5 , 3.5 )--( 2.5 , 4.0 )--( 2.0 , 4.0 );
\fill[color=primario!80] ( 2.5 , 4.0 )--( 3.0 , 4.0 )--( 3.0 , 4.5 )--( 2.5 , 4.5 );
\fill[color=primario!80] ( 3.0 , 4.5 )--( 3.5 , 4.5 )--( 3.5 , 5.0 )--( 3.0 , 5.0 );
\fill[color=primario!80] ( 3.5 , 5.0 )--( 4.0 , 5.0 )--( 4.0 , 5.5 )--( 3.5 , 5.5 );
\fill[color=primario!80] ( 4.0 , 0.0 )--( 4.5 , 0.0 )--( 4.5 , 0.5 )--( 4.0 , 0.5 );
\fill[color=primario!80] ( 4.5 , 0.5 )--( 5.0 , 0.5 )--( 5.0 , 1.0 )--( 4.5 , 1.0 );
\fill[color=primario!80] ( 5.0 , 1.0 )--( 5.5 , 1.0 )--( 5.5 , 1.5 )--( 5.0 , 1.5 );
\fill[color=primario!80] ( 5.5 , 1.5 )--( 6.0 , 1.5 )--( 6.0 , 2.0 )--( 5.5 , 2.0 );
\fill[color=primario!80] ( 6.0 , 2.0 )--( 6.5 , 2.0 )--( 6.5 , 2.5 )--( 6.0 , 2.5 );
\fill[color=primario!80] ( 6.5 , 2.5 )--( 7.0 , 2.5 )--( 7.0 , 3.0 )--( 6.5 , 3.0 );
\fill[color=primario!80] ( 7.0 , 3.0 )--( 7.5 , 3.0 )--( 7.5 , 3.5 )--( 7.0 , 3.5 );
\fill[color=primario!80] ( 7.5 , 3.5 )--( 8.0 , 3.5 )--( 8.0 , 4.0 )--( 7.5 , 4.0 );
\fill[color=primario!80] ( 8.0 , 4.0 )--( 8.5 , 4.0 )--( 8.5 , 4.5 )--( 8.0 , 4.5 );
\fill[color=primario!80] ( 8.5 , 4.5 )--( 9.0 , 4.5 )--( 9.0 , 5.0 )--( 8.5 , 5.0 );
\fill[color=primario!80] ( 9.0 , 5.0 )--( 9.5 , 5.0 )--( 9.5 , 5.5 )--( 9.0 , 5.5 );
\fill[color=primario!80] ( 9.5 , 5.5 )--( 10.0 , 5.5 )--( 10.0 , 6.0 )--( 9.5 , 6.0 );
\fill[color=primario!80] ( 10.0 , 6.0 )--( 10.5 , 6.0 )--( 10.5 , 6.5 )--( 10.0 , 6.5 );
\fill[color=primario!80] ( 10.5 , 6.5 )--( 11.0 , 6.5 )--( 11.0 , 7.0 )--( 10.5 , 7.0 );
\fill[color=primario!80] ( 11.0 , 7.0 )--( 11.5 , 7.0 )--( 11.5 , 7.5 )--( 11.0 , 7.5 );
\fill[color=primario!80] ( 11.5 , 10.0 )--( 12.0 , 10.0 )--( 12.0 , 10.5 )--( 11.5 , 10.5 );
\fill[color=primario!80] ( 12.0 , 10.5 )--( 12.5 , 10.5 )--( 12.5 , 11.0 )--( 12.0 , 11.0 );
\fill[color=primario!80] ( 12.5 , 11.0 )--( 13.0 , 11.0 )--( 13.0 , 11.5 )--( 12.5 , 11.5 );
\draw(12.3,-1)node{alfabeto};
\draw(-1,12.3) node[rotate=90.]{C\'{o}digo};
\end{tikzpicture}
\end{figure}

\item {} 
Quais letras do código acima são impossíveis de decodificar e por quê?

\item {} 
Que propriedades deve ter um código para que seja possível decodificá-lo?

\end{enumerate}


\ifdefined\prof
\begin{solucao}
\begin{enumerate}
\item XBPVTB

\item Usaria a linha debaixo para descobrir a letra original correspondente: CRIPTOGRAFAR.

\item Resposta pessoal

\item Uma resposta possível seria:

\begin{table}[H]
\centering
\setlength\tabcolsep{2.5pt}
\begin{tabu} to \textwidth{|c|*{26}{>{$}c<{$}|}}
\hline
\cellcolor{primario!80}{\textcolor{white}{\textbf{Original}}} & 1 & 2 & 3 & 4 & 5 & 6 & 7 & 8 & 9 & 10 & 11 & 12 & 13 & 14 & 15 & 16 & 17 & 18 & 19 & 20 & 21 & 22 & 23 & 24 & 25 & 26 \\
\hline
\cellcolor{primario!80}{\textcolor{white}{\textbf{Código}}} & 16 & 17 & 18 & 19 & 20 & 21 & 22 & 23 & 24 & 25 & 26 & 1 & 2 & 3 & 4 & 5 & 6 & 7 & 8 & 9 & 10 & 11 & 12 & 13 & 14 & 15 \\
\hline
\end{tabu}
\end{table}

Outra possibilidade é escrever $f(x)=x+15$, subtraindo $26$ se $f(x)$ for maior que $26$.

\begin{tikzpicture}[every node/.style={black}, every path/.style={black}]
\draw[scale=.5](0,0)grid(26,26);
\foreach \i [count=\x from 0] in{{A}, {B},{C}, {D}, {E}, {F}, {G}, {H},{I},{J},{K},{L},{M},{N},{O},{P},{Q},{R},{S},{T},{U},{V},{W},{X},{Y},{Z}}
\draw (.2+.5*\x,-.4) node {\i};
\foreach \i [count=\x from 0] in{{A}, {B},{C}, {D}, {E}, {F}, {G}, {H},{I},{J},{K},{L},{M},{N},{O},{P},{Q},{R},{S},{T},{U},{V},{W},{X},{Y},{Z}}
\draw (-.4,.2+.5*\x) node {\i};
\fill[color=primario!80] ( 0.0 , 0.0 )--( 0.5 , 0.0 )--( 0.5 , 0.5 )--( 0.0 , 0.5 );
\fill[color=primario!80] ( 0.5 , 1.5 )--( 1.0 , 1.5 )--( 1.0 , 2.0 )--( 0.5 , 2.0 );
\fill[color=primario!80] ( 1.0 , 4.0 )--( 1.5 , 4.0 )--( 1.5 , 4.5 )--( 1.0 , 4.5 );
\fill[color=primario!80] ( 1.5 , 7.5 )--( 2.0 , 7.5 )--( 2.0 , 8.0 )--( 1.5 , 8.0 );
\fill[color=primario!80] ( 2.0 , 12.0 )--( 2.5 , 12.0 )--( 2.5 , 12.5 )--( 2.0 , 12.5 );
\fill[color=primario!80] ( 2.5 , 4.5 )--( 3.0 , 4.5 )--( 3.0 , 5.0 )--( 2.5 , 5.0 );
\fill[color=primario!80] ( 3.0 , 11.0 )--( 3.5 , 11.0 )--( 3.5 , 11.5 )--( 3.0 , 11.5 );
\fill[color=primario!80] ( 3.5 , 5.5 )--( 4.0 , 5.5 )--( 4.0 , 6.0 )--( 3.5 , 6.0 );
\fill[color=primario!80] ( 4.0 , 1.0 )--( 4.5 , 1.0 )--( 4.5 , 1.5 )--( 4.0 , 1.5 );
\fill[color=primario!80] ( 4.5 , 10.5 ) rectangle ( 5.0 , 11.0 );
\fill[color=primario!80] ( 5.0 , 8.0 ) rectangle ( 5.5 , 8.5 );
\fill[color=primario!80] ( 5.5 , 6.5 ) rectangle ( 6.0 , 7.0 );
\fill[color=primario!80] ( 6.0 , 6.0 ) rectangle ( 6.5 , 6.5 );
\fill[color=primario!80] ( 6.5 , 6.5 ) rectangle ( 7.0 , 7.0 );
\fill[color=primario!80] ( 7.0 , 8.0 ) rectangle ( 7.5 , 8.5 );
\fill[color=primario!80] ( 7.5 , 10.5 ) rectangle ( 8.0 , 11.0 );
\fill[color=primario!80] ( 8.0 , 1.0 ) rectangle ( 8.5 , 1.5 );
\fill[color=primario!80] ( 8.5 , 5.5 ) rectangle ( 9.0 , 6.0 );
\fill[color=primario!80] ( 9.0 , 11.0 ) rectangle ( 9.5 , 11.5 );
\fill[color=primario!80] ( 9.5 , 4.5 ) rectangle ( 10.0 , 5.0 );
\fill[color=primario!80] ( 10.0 , 12.0 ) rectangle ( 10.5 , 12.5 );
\fill[color=primario!80] ( 10.5 , 7.5 ) rectangle ( 11.0 , 8.0 );
\fill[color=primario!80] ( 11.0 , 4.0 ) rectangle ( 11.5 , 4.5 );
\fill[color=primario!80] ( 11.5 , 1.5 ) rectangle ( 12.0 , 2.0 );
\fill[color=primario!80] ( 12.0 , 0.0 ) rectangle ( 12.5 , 0.5 );
\fill[color=primario!80] ( 12.5 , 12.5 ) rectangle ( 13.0 , 13.0 );
\draw(12.3,-1)node{alfabeto};
\draw(-1,12.3) node[rotate=90.]{Código};   
\end{tikzpicture}

Para valores maiores que $26$ devemos subtrair $26$ sucessivamente até encontrar um valor positivo menor que ou igual a $26$ e então encontrar a letra correspondente. Isso equivale a tomar o resto da divisão por $26$.

\item Impossível decodificar, pois os códigos P, Q e X não têm correspondente no alfabeto e os códigos G e J têm mais de uma opção de escolha.

\item Códigos de D a K têm duas letras do alfabeto associadas a cada um. Códigos de P a T e de X a Z não têm correspondente no alfabeto.

\item Todo código deve possuir um único correspondente no alfabeto. Ou seja, a relação (código, alfabeto) deve ser uma função.

\end{enumerate}
\end{solucao}
\fi

\end{document}