\documentclass[10 pt,usenames,dvipsnames, oneside]{article}
\usepackage{../../../modelo-ensino-medio}


\begin{document}

\begin{center}
  \begin{minipage}[l]{3cm}
\includegraphics[width=2cm]{logo}    
\end{minipage}\hfill
\begin{minipage}[r]{.8\textwidth}
 {\Large \scshape Atividade: Planos telefônicos}  
\end{minipage}
\end{center}
\vspace{.2cm}

\ifdefined\prof
\begin{objetivos}
\item \textbf{EM13MAT404} Analisar funções definidas por uma ou mais sentenças (tabela do Imposto de Renda, contas de luz, água, gás etc.), em suas representações algébrica e gráfica, identificando domínios de validade, imagem, crescimento e decrescimento, e convertendo essas representações de uma para outra, com ou sem apoio de tecnologias digitais
\end{objetivos}

\begin{goals}
\begin{enumerate}

\item[OE1] Interpretar informações registradas textual e graficamente.

\item[OE2] Visualizar o gráfico de uma função definida por mais de uma sentença.

\item[OE3] Utilizar informações de um gráfico para tomada de decisão.

\end{enumerate}

\tcblower

\begin{itemize}
\item Não é esperado nesse momento que os estudantes apresentem uma expressão para a função cujo gráfico é apresentado.

\item Estimule uma discussão sobre os diferentes planos oferecidos pelas operadoras e as experiências de cada um.
\end{itemize}

\end{goals}

\bigskip
\begin{center}
{\large \scshape Atividade}
\end{center}
\fi

Você deseja trocar o plano do seu telefone e ao consultar a sua operadora tem a opção de escolher entre dois planos: plano Prata e plano Ouro. No seu plano atual, você paga R\$$70{,}00$ por 500MB de internet e os dados além disso custam R\$$0{,}20$ por MB. 

O plano Ouro cobra R\$$140{,}00$ por dados ilimitados e o plano Prata tem a mesma estrutura do seu plano atual. Os valores cobrados pelo plano Prata estão representados no gráfico a seguir.

\begin{figure}[H]
\centering


\begin{tikzpicture}[yscale=.5,scale=.75, every node/.style={scale=.75}]

\draw [->] (0,0) -- (14,0) node [below left, yshift=-.5cm] {Dados (MB)};
\draw [->] (0,0) -- (0,20) node [above left, rotate=90, yshift=.5cm] {Valor (R\$)};
\draw [help lines] (0,0) grid (14,20);

\foreach \x in {2,4,...,20} \node [left] at (0,\x) {\x0};
\foreach \x in {1,2,...,14} \node [below] at (\x,0) {\x00};

\node [below left] at (0,0) {0};

\draw [thick, session1] (0,8) -- (6,8) -- (14,20);

\end{tikzpicture}
\end{figure}

\begin{enumerate}
\item Qual o valor fixo cobrado no plano Prata e que quantidade de dados ele cobre?
\item Qual o valor por MB excedente do valor estipulado?
\item A partir de que quantidade de dados consumidos o plano Ouro passa a ser mais vantajoso?
\item Represente no sistema de coordenadas acima o gráfico do preço a pagar pelo plano Ouro.
\end{enumerate}


\ifdefined\prof
\begin{solucao}
\begin{enumerate}
\item R\$$80{,}00$
\item R\$$0{,}15$
\item 1000MB
\item\adjustbox{valign=t}{
	
\begin{tikzpicture}[yscale=.5,scale=.75, every node/.style={scale=.75}]

\draw [->] (0,0) -- (14,0) node [below left, yshift=-.5cm] {Dados (MB)};
\draw [->] (0,0) -- (0,20) node [above left, rotate=90, yshift=.75cm] {Valor (R\$)};
\draw [help lines] (0,0) grid (14,20);

\foreach \x in {2,4,...,20} \node [left] at (0,\x) {\x0};
\foreach \x in {1,2,...,14} \node [below] at (\x,0) {\x00};

\node [below left] at (0,0) {0};

\draw [thick, session1] (0,8) -- (6,8) -- (14,20);

\draw [thick, session3] (0,14) -- (14,14);

\end{tikzpicture}

}
\end{enumerate}
\end{solucao}
\fi

\end{document}