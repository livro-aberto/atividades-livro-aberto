\documentclass[10 pt,usenames,dvipsnames, oneside]{article}
\usepackage{../../../modelo-ensino-medio}



\begin{document}

\begin{center}
  \begin{minipage}[l]{3cm}
\includegraphics[width=2cm]{logo}    
\end{minipage}\hfill
\begin{minipage}[r]{.8\textwidth}
 {\Large \scshape Atividade: Enchendo o cone}  
\end{minipage}
\end{center}
\vspace{.2cm}

\ifdefined\prof
\begin{objetivos}
\item \textbf{LAF1} Compreender função como uma relação de dependência entre duas variáveis, as ideias de domínio, contradomínio e imagem, e suas representações algébricas e gráficas e utilizá-las para analisar, interpretar e resolver problemas em contextos diversos, inclusive fenômenos naturais, sociais e de outras áreas.
\end{objetivos}

\begin{goals}
\begin{enumerate}

\item[OE1] Determinar valores da imagem (respectivamente, do domínio) de uma função a partir da sua expressão analítica e de ponto do domínio (respectivamente, da imagem).

\item[OE2] Interpretar valores do domínio e da imagem de uma função dada que modela uma situação real específica.

\end{enumerate}

\tcblower

\begin{itemize}
\item Nível de abstração \textbf{Ação}.

\item É importante que o estudante identifique a relação existente entre a altura do nível da água no reservatório e o volume do mesmo.

\item Essa pode também ser uma oportunidade para explorar conversão de unidades. Sabemos que a expressão $V=\frac{1}{3}(\pi r^2)h$ fornece o volume do cone em função do raio $r$ e da altura $h$ do nível de água, desde que raio e altura estejam expressos na mesma unidade. A partir das dimensões dadas no enunciado, tem-se $r=\frac{h}{2}$ e, portanto, $V(h)=\frac{1}{3}\pi\frac{h^3}{4}$ é o volume de água no reservatório, em metros cúbicos, correspondente a uma altura de $h$ em metros. Considerando $3$ como aproximação de $\pi$ obtem-se que o volume, em metros cúbicos, é dado, aproximadamente, por $V(h)=\frac{h^3}{4}$, o que equivale em litros a $V(h)=250h^3$.

\item Destaque a “não proporcionalidade” da situação, observando por exemplo, que $2$ é a metade de $4$, mas $2000$ não é a metade de $16000$.
\end{itemize}

\end{goals}

\bigskip
\begin{center}
{\large \scshape Atividade}
\end{center}
\fi

O reservatório representado a seguir tem a forma de um cone cuja altura é \(6 m\) e a base é um círculo de raio \(3 m\). O volume \(V\) em litros de água no reservatório pode ser estimado a partir altura do nível da água \(h\) (em metros) de acordo com a seguinte expressão:
\begin{equation*}
\begin{split}V(h)=250h^3\end{split}
\end{equation*}\begin{center}\begin{tikzpicture}
\fill[thick,color=session1!80,fill=session1!80,fill opacity=0.10000000149011612, left color =white, right color =session1!80] (1.,0.) -- (-0.5,3.) -- (2.5,3.) -- cycle;
\draw [rotate around={-180:(1.0047836744699097,3.1435102340973167)},thick,left color=session1!80, right color=session1!80, middle color=white] (1.0047836744699097,3.1435102340973167) ellipse (1.5611029721362464cm and 0.5184113668542463cm);
\draw [rotate around={-180:(1.0071755117048646,4.715265351145975)},thick, left color=gray!80, right color=gray!60, middle color=white] (1.0071755117048646,4.715265351145975) ellipse (2.341654458204363cm and 0.7776170502813675cm);
\draw [thick] (1.,0.)-- (-1.3303743315507686,4.660748663101537);
\draw [thick] (1.,0.)-- (3.347109515260305,4.69421903052061);
\draw[dashed](1,0) -- (3.4,0);
\draw[|-|, dashed](2.6,0)--(2.6,3);
\draw (2.7,1.6) node[right] {$h$};
\draw[|-|, dashed](3.4,0)--(3.4,4.6);
\draw (3.5,2) node[right] {6 m};
\end{tikzpicture}\end{center}\begin{enumerate}
\item {} 
Determine \(V(2), V(3)\) e \(V(4)\) e explique os seus significados no contexto.

\item {} 
Quais os volumes de água, mínimo e máximo, que o reservatório comporta?

\item {} 
A que altura do nível da água corresponde o volume igual a \(3 456\) litros?

\end{enumerate}

\ifdefined\prof
\begin{solucao}
\begin{enumerate}

\item $V(2)$, $V(3)$ e $V(4)$ são, respectivamente iguais a $2000, 6750$ e $16000$ litros e correspondem aos volumes quando a altura da água no reservatório é igual a $2,3$ e $4$ metros, respectivamente.

\item O menor volume observado é $V=0$ litros, que corresponde a $h=0$m, e o maior volume é $V(6)=54000$ litros.

\item Corresponde a uma altura de $2,4$ metros.

\end{enumerate}
\end{solucao}
\fi

\end{document}