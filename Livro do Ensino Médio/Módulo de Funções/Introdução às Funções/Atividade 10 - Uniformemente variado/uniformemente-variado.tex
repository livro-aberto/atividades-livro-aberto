\documentclass[10 pt,usenames,dvipsnames, oneside]{article}
\usepackage{../../../modelo-ensino-medio}



\begin{document}

\begin{center}
  \begin{minipage}[l]{3cm}
\includegraphics[width=2cm]{logo}    
\end{minipage}\hfill
\begin{minipage}[r]{.8\textwidth}
 {\Large \scshape Atividade: Uniformemente Variado}  
\end{minipage}
\end{center}
\vspace{.2cm}

\ifdefined\prof
\begin{objetivos}
\item \textbf{LAF1} Compreender função como uma relação de dependência entre duas variáveis, as ideias de domínio, contradomínio e imagem, e suas representações algébricas e gráficas e utilizá-las para analisar, interpretar e resolver problemas em contextos diversos, inclusive fenômenos naturais, sociais e de outras áreas.
\end{objetivos}

\begin{goals}
\begin{enumerate}

\item[OE1] Compreender funções a partir de sua representação analítica, relacionando-a ao contexto descrito pelo problema.

\end{enumerate}

\tcblower

\begin{itemize}
\item Nível de abstração: \textbf{Ação}.

\item Chamar atenção do estudante para o importante papel que as funções desempenham na Física, em especial na Mecânica Clássica, relacionando grandezas como tempo, deslocamento, velocidade e aceleração.
\end{itemize}

\end{goals}

\bigskip
\begin{center}
{\large \scshape Atividade}
\end{center}
\fi

A posição \(S\) (em quilômetros), medida a partir de um referencial, de um veículo que se desloca segundo um movimento retilíneo uniformemente variado (MRUV) é dada em função do tempo \(t\) (medido em horas) pela seguinte expressão:
\begin{equation*}
\begin{split}S(t)=2t^2-4t+2\end{split}
\end{equation*}\begin{enumerate}
\item {} 
Determine a posição inicial do veículo. Explique o significado desse resultado a partir do contexto.

\item {} 
Após quanto tempo o veículo estará a 18km da origem?

\end{enumerate}


\ifdefined\prof
\begin{solucao}
\begin{enumerate}
\item Inicialmente o veículo está posicionado a $S(0)=2$ quilômetros da origem $O$.

\item Após $4$ horas.

\end{enumerate}
\end{solucao}
\fi

\end{document}