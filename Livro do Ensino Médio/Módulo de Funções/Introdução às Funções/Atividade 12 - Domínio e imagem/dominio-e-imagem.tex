\documentclass[10 pt,usenames,dvipsnames, oneside]{article}
\usepackage{../../../modelo-ensino-medio}


\begin{document}

\begin{center}
  \begin{minipage}[l]{3cm}
\includegraphics[width=2cm]{logo}    
\end{minipage}\hfill
\begin{minipage}[r]{.8\textwidth}
 {\Large \scshape Atividade: Domínio e Imagem}  
\end{minipage}
\end{center}
\vspace{.2cm}

\ifdefined\prof
\begin{objetivos}
\item \textbf{LAF1} Compreender função como uma relação de dependência entre duas variáveis, as ideias de domínio, contradomínio e imagem, e suas representações algébricas e gráficas e utilizá-las para analisar, interpretar e resolver problemas em contextos diversos, inclusive fenômenos naturais, sociais e de outras áreas.
\end{objetivos}

\begin{goals}
\begin{enumerate}

\item[OE1] Determinar a partir da expressão algébrica os conjuntos domínio e imagem.

\end{enumerate}

\tcblower
\begin{itemize}
\item Nível de abstração: \textbf{Ação}.

\item É importante que o estudante perceba as restrições para a escolha de $x$ impostas por algumas das expressões dadas.
\end{itemize}

\end{goals}

\bigskip
\begin{center}
{\large \scshape Atividade}
\end{center}
\fi

Considere a seguinte lista de expressões algébricas.
\begin{multicols}{3}
\begin{enumerate}
\item {} 
\(f(x)=\sqrt{x}\)

\item {} 
\(G(z)=\sqrt{z-5}\)

\item {} 
\(h(s)=\frac{1}{3-s}\)

\item {} 
\(J(t)=\frac{1}{t+8}\)

\item {} 
\(T(x)=\frac{1}{\sqrt{x}}\)

\item {} 
\(R(x)=(x-2)^2+7\)

\item {} 
\(g(u)=5u^2+8\)

\item {} 
\(F(x)=(x+1)^2-3\)

\end{enumerate}
\end{multicols}

Veja que, em algumas das expressões, a variável independente não pode assumir alguns valores, por exemplo, na letra (a) \(x\) não pode assumir valores negativos. Complete a tabela abaixo com o maior conjunto domínio possível que cada uma das funções pode ter e o correspondente conjunto imagem.


\begin{table}[H]
\centering
\begin{tabu} to \textwidth{|c|c|c|}
\hline
\thead
Expressão & Domínio $A$ & Imagem \\
\hline
\((a)\) & \(\mathbb{R}^+\) & \\
\hline
\((b)\) & & \\
\hline
\((c)\) & & \(\mathbb{R}\setminus \{0\}\) \\
\hline
\((d)\) & \(\mathbb{R}\setminus \{-8\}\) & \\
\hline
\((e)\) & & \\ 
\hline
\((f)\) & & \([7,+\infty[\) \\
\hline
\((g)\) & & \\
\hline
\((h)\) & & \\
\hline
\end{tabu}
\end{table}




\ifdefined\prof
\begin{solucao}

Ajude o estudante a completar a tabela.

\begin{table}[H]
\centering
\begin{tabu} to \textwidth{|c|c|c|}
\hline
\thead
Expressão & Domínio $A$ & Imagem \\
\hline
\((a)\) & \(\mathbb{R}^+\) & $\mathbb{R}^+$\\
\hline
\((b)\) & $[5,+\infty[$ & $\mathbb{R}^+$\\
\hline
\((c)\) & $\mathbb{R}\setminus\{3\}$ & \(\mathbb{R}\setminus \{0\}\) \\
\hline
\((d)\) & \(\mathbb{R}\setminus \{-8\}\) & $\mathbb{R}\setminus\{0\}$ \\
\hline
\((e)\) & $]0,+\infty[$ & $]0,+\infty[$ \\ 
\hline
\((f)\) & $\mathbb{R}$ & \([7,+\infty[\) \\
\hline
\((g)\) & $\mathbb{R}$  & $[8,+\infty[$ \\
\hline
\((h)\) & $\mathbb{R}$  & $[-3,+\infty[$ \\
\hline
\end{tabu}
\end{table}

\end{solucao}
\fi

\end{document}