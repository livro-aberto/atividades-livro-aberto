\documentclass[10 pt,usenames,dvipsnames, oneside]{article}
\usepackage{../../../modelo-ensino-medio}


\begin{document}

\begin{center}
  \begin{minipage}[l]{3cm}
\includegraphics[width=2cm]{logo}    
\end{minipage}\hfill
\begin{minipage}[r]{.8\textwidth}
 {\Large \scshape Atividade: Arranha-céu}  
\end{minipage}
\end{center}
\vspace{.2cm}

\ifdefined\prof
\begin{objetivos}
\item \textbf{LAF1} Compreender função como uma relação de dependência entre duas variáveis, as ideias de domínio, contradomínio e imagem, e suas representações algébricas e gráficas e utilizá-las para analisar, interpretar e resolver problemas em contextos diversos, inclusive fenômenos naturais, sociais e de outras áreas.
\end{objetivos}

\begin{goals}
\begin{enumerate}

\item[OE1] econhecer uma relação de dependência entre variáveis apresentada em forma de tabela.

\item[OE2] Interpretar tabela que representa relação de dependência entre variáveis.

\end{enumerate}

\tcblower

\begin{itemize}
\item Nível de abstração: \textbf{Processo}.

\item A escolha dessa atividade se apoia no fato de que os estudantes têm familiaridade com a noção de proporcionalidade, que é explorada em álgebra e em geometria desde os anos iniciais do Ensino Fundamental.

\item Deseja-se, entretanto, que os estudantes levem em conta o contexto do problema.
\end{itemize}

\end{goals}

\bigskip
\begin{center}
{\large \scshape Atividade}
\end{center}
\fi

Imagine um arranha-céu de \(40\) andares cujas diferentes alturas que correspondem a alguns andares estão representadas na tabela abaixo.

\begin{table}[H]
\centering

\begin{tabu} to \textwidth{|c|c|@{}|*{9}{p{.5cm}@{}|}}
\hline
\cellcolor{primario}{\textcolor{white}{\textbf{Número do Andar}}} & Garagem (0) & 1 & 2 & 3 & 4 & … & 10 & … & & \\
\hline
\cellcolor{primario}{\textcolor{white}{\textbf{Altura (metros)}}} & -1 & 3 & 7 & 11 & 15 & … & & … & & 91 \\
\hline
\end{tabu}
\end{table}

\begin{quote}

Considere que a altura de um andar é medida a partir do nível da rua até o piso desse andar e que a altura entre os andares seja sempre a mesma, conforme o esquema abaixo.
\end{quote}

\begin{figure}[H]
\centering

\noindent\includegraphics[width=200bp]{{Arranha-ceu_1}.png}
\end{figure}
\begin{enumerate}
\item {} 
Qual a altura entre os andares?

\item {} 
Qual a altura  do 10º andar?

\item {} 
O que significa o sinal negativo do andar da garagem?

\item {} 
A que andar corresponde a altura de 91 m?

\item {} 
Qual é a altura total desse prédio?

\item {} 
Realize uma pesquisa na internet e descubra o maior arranha-céu brasileiro atualmente. Dividindo a altura total desse arranha-céu pela quantidade de andares, determine a altura média de um andar.

\end{enumerate}



\ifdefined\prof
\begin{solucao}
\begin{enumerate}
\item {} 
$4$ metros
\item {} 
$39$ metros

\item {} 
Significa que a garagem está abaixo do nível da rua

\item {} 
$23^{\circ}$ andar.

\item {} 
O $40^{\circ}$ andar está localizado a $159$ metros do solo, e como cada andar possiu altura $4$ metros,a altura total do prédio é $163$ metros.

\item A resposta depende do período em que a pesquisa for realizada. Em setembro de $2017$ o maior arranha-céu brasileiro é o Millennium Palace, localizado em Balneário Camboriú, Santa Catarina, com $177$ metros de altura e $46$ andares.

\end{enumerate}
\end{solucao}
\fi

\end{document}