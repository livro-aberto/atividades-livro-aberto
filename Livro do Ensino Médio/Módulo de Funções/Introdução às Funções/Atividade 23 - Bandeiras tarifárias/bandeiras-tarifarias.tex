\documentclass[10 pt,usenames,dvipsnames, oneside]{article}
\usepackage{../../../modelo-ensino-medio}


\begin{document}

\begin{center}
  \begin{minipage}[l]{3cm}
\includegraphics[width=2cm]{logo}    
\end{minipage}\hfill
\begin{minipage}[r]{.8\textwidth}
 {\Large \scshape Atividade: Bandeiras tarifárias}  
\end{minipage}
\end{center}
\vspace{.2cm}

\ifdefined\prof
\begin{objetivos}
\item \textbf{EM13MT404} Analisar funções definidas por uma ou mais sentenças (tabela do Imposto de Renda, contas de luz, água, gás etc.), em suas representações algébrica e gráfica, identificando domínios de validade, imagem, crescimento e decrescimento, e convertendo essas representações de uma para outra, com ou sem apoio de tecnologias digitais
\end{objetivos}

\begin{goals}

\begin{enumerate}

\item[OE1] Interpretar informações registradas textual e graficamente.

\end{enumerate}

\tcblower

\begin{itemize}
\item Este é um tema que tem potencial interdisciplinar. Considere conversar com os professores de Geografia, Física, Química, Biologia, Sociologia para tratar de temas como: energia, formas alternativas de geração, custo de produção e distribuição, políticas públicas de fornecimento, roubos de carga, etc.

\item Caso haja a possibilidade assista com seus alunos o vídeo oficial da ANEEL sobre as bandeiras tarifárias \url{https://youtu.be/w1rS7_tGSvM}.
\end{itemize}

\end{goals}

\bigskip
\begin{center}
{\large \scshape Atividade}
\end{center}
\fi

Desde o ano de 2015, as contas de energia passaram a trazer uma novidade: o Sistema de Bandeiras Tarifárias, que apresenta as seguintes modalidades: verde, amarela e vermelha - as mesmas cores dos semáforos - e indicam se haverá ou não acréscimo no valor de energia a ser repassada ao consumidor final, em função das condições de geração de eletricidade. Cada modalidade apresenta as seguintes características:

%adicionar bandeiras
\begin{itemize}
\item \textbf{Bandeira verde}: condições favoráveis de geração de energia. A tarifa não sofre nenhum acréscimo;

\item \textbf{Bandeira amarela}: condições de geração menos favoráveis. A tarifa sobre acréscimo de R\$$0{,}01343$ para cada quilowatt-hora (kWh) consumidos;

\item \textbf{Bandeira vermelha - patamar 1}: condições mais custosas de geração. A tarifa sofre acréscimo de R\$$0{,}04169$ para cada quilowatt-hora (kWh) consumido.

\item \textbf{Bandeira vermelha - patamar 2}: condições mais custosas de geração. A tarifa sofre acréscimo de R\$$0{,}06243$ para cada quilowatt-hora (kWh) consumido.
\end{itemize}

\flushright{\small

Texto extraído da página da ANEEL em 28/03/2020 \\ \url{https://www.aneel.gov.br/bandeiras-tarifárias}}

\justify
O sistema de coordenadas abaixo contém os gráficos para as funções que relacionam o preço a pagar pela energia em relação ao consumo em quilowatt-hora (kWh) para cada uma das bandeiras tarifárias, em uma cidade vizinha. Com base nas informações do gráfico a seguir, responda:

\begin{figure}[H]
\centering

\begin{tikzpicture}[yscale=2.5,scale=.75, every node/.style={scale=.75}]

\draw [->] (0,5) -- (11,5) node [below left, yshift=-.5cm] {consumo (kWh)};
\draw [->] (0,5) -- (0,9) node [above left, , rotate=90, yshift=1.2cm] {Preço a pagar (R\$)};

\foreach \x in {2,4,...,10} {\node [below] at (\x,5) {\x00};
\draw [help lines] (\x,5) -- (\x,9);
};
\foreach \x/\y in {8/{800,00},8.1343/{813,43},8.4169/{841,89},8.6243/{862,43}} {

\node [left] at (0,\x) {\y};

\draw [thick, session3] (0,5) -- (10,\x);
\draw [help lines] (0,\x) -- (10,\x);
};

\end{tikzpicture}
\end{figure}
\ifdefined\prof
\begin{solucao}

\begin{enumerate}
\item R\$$0{,}80$ por kWh
\item R\$$284{,}70$
\item $356$ kWh
\end{enumerate}

\end{solucao}
\fi

\end{document}