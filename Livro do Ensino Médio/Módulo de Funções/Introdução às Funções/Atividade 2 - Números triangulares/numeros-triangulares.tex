\documentclass[10 pt,usenames,dvipsnames, oneside]{article}
\usepackage{../../../modelo-ensino-medio}




\begin{document}

\begin{center}
  \begin{minipage}[l]{3cm}
\includegraphics[width=2cm]{logo}    
\end{minipage}\hfill
\begin{minipage}[r]{.8\textwidth}
 {\Large \scshape Atividade: Números triangulares}  
\end{minipage}
\end{center}
\vspace{.2cm}

\ifdefined\prof
\begin{objetivos}
\item \textbf{LAF1} Compreender função como uma relação de dependência entre duas variáveis, as ideias de domínio, contradomínio e imagem, e suas representações algébricas e gráficas e utilizá-las para analisar, interpretar e resolver problemas em contextos diversos, inclusive fenômenos naturais, sociais e de outras áreas.
\end{objetivos}

\begin{goals}
\begin{enumerate}

\item[OE1] Reconhecer a relação de dependência entre a ordem e os termos de uma sequência.

\item[OE2] Reconhecer, a partir de um padrão geométrico, os primeiros termos de uma sequência e ser capaz de, a partir do padrão identificado, inferir os próximos termos da sequência.

\item[OE3] Generalizar, ainda que em palavras, a determinação de um termo qualquer da sequência a partir da sua ordem segundo um padrão identificado.

\end{enumerate}

\tcblower

\begin{itemize}
\item Nível de abstração: \textbf{Ação}.


\item Muito provavelmente os estudantes descreverão a sequência de formas diferentes, mas obtendo o mesmo resultado para o sexto, o sétimo e o oitavo números triangulares. Por exemplo, um estudante poderá dizer que, para identificar os números triangulares solicitados, “constrói”{} os triângulos “de cima para baixo”{}. Já ouro pode argumentar que o faz “de baixo para cima”. Outro ainda pode agumentar a partir da observação do padrão recursivo: “basta acrescentar uma linha ao último triângulo construído”. Assim, como a resposta ao ítem (b) não é única, procure aproveitar e explorar as diferentes respostas na discussão com a turma: os resultados são os mesmos para essas diferentes formas de descrever a sequência? Por que? Por exemplo, “somar de cima para baixo” produz o mesmo resultado que “somar de baixo para cima”, pois a adição é comutativa.


\item Pela mesma razão apontada no ítem (b), a resposta do item (d) não é única.

\item Não é objetivo, neste momento, que o estudante expresse a relação por meio da linguagem simbólica matemática, escrevendo, por exemplo, $T_n=T_n−1+n$, mas que seja matematicamente preciso em suas palavras, dizendo, por exemplo, que “o $n$-ésimo termo da sequência é obtido a partir do termo anterior acrescido de mais uma fileira com $n$”{} ou que “o $n$-ésimo triângulo da sequência é obtido a partir do triângulo anterior acrescido de mais uma fileira com $n$ círculos", portanto, “o $n$-ésimo número triangular é obtido a partir do termo anterior acrescido de $n$“.


\item É possível que algum estudante descreva o $n$-ésimo número triangular como a soma dos primeiros $n$ números naturais. Nesse caso, você pode mostrar que essa maneira de descrever o procedimento é equivalente à recursiva. Não apenas testando exemplos, mas sim fazendo uso da propriedade associativa da adição: seja qual for o $n$ tem-se

\begin{align*}
T_n &= 1+2+...+(n-1)+n \\
&=[1+2+...+(n-1)]+n \\
&=T_{n-1}+n
\end{align*}

\end{itemize}
\end{goals}

\bigskip
\begin{center}
{\large \scshape Atividade}
\end{center}
\fi

\begin{figure}[H]
\centering

\begin{tikzpicture}
\tikzstyle{circ}=[circle,draw,minimum size=1cm, fill=session1];
\begin{scope}
\node (A) [circ] {};
\node [below of=A] {$T_1=1$};
\end{scope}

\begin{scope}[xshift=1.3cm,node distance=1cm]

\node (A) [circ] {};
\node (B) [circ, right of=A] {};
\node (C) at ($(A)!.5!(B)$) [circ, 
yshift=.86602cm] {};

\node [below of=A, xshift=.5cm] {$T_2=3$};
\end{scope}

\begin{scope}[xshift=3.4cm,node distance=1cm]

\node (A) [circ] {};
\node (B) [circ, right of=A] {};
\node (C) at ($(A)!.5!(B)$) [circ, 
yshift=.86602cm] {};
\node (D) [circ, right of=B] {};
\node (E) [circ, right of=C] {};
\node (F) at ($(C)!.5!(E)$) [circ, yshift=.86602cm] {};

\node [below of=B] {$T_3=6$};
\end{scope}

\begin{scope}[xshift=6.6cm,node distance=1cm]

\node (A) [circ] {};
\node (B) [circ, right of=A] {};
\node (C) at ($(A)!.5!(B)$) [circ, 
yshift=.86602cm] {};
\node (D) [circ, right of=B] {};
\node (E) [circ, right of=C] {};
\node (F) at ($(C)!.5!(E)$) [circ, yshift=.86602cm] {};
\node (G) [circ, right of=D] {};
\node (H) [circ, right of=E] {};
\node (I) [circ, right of=F] {};
\node (J) at ($(F)!.5!(I)$) [circ, yshift=.86602cm] {};

\node [below of=B, xshift=.5cm] {$T_4=10$};
\end{scope}

\begin{scope}[xshift=10.8cm,node distance=1cm]

\node (A) [circ] {};
\node (B) [circ, right of=A] {};
\node (C) at ($(A)!.5!(B)$) [circ, 
yshift=.86602cm] {};
\node (D) [circ, right of=B] {};
\node (E) [circ, right of=C] {};
\node (F) at ($(C)!.5!(E)$) [circ, yshift=.86602cm] {};
\node (G) [circ, right of=D] {};
\node (H) [circ, right of=E] {};
\node (I) [circ, right of=F] {};
\node (J) at ($(F)!.5!(I)$) [circ, yshift=.86602cm] {};
\node (K) [circ, right of=G] {};
\node (L) [circ, right of=H] {};
\node (M) [circ, right of=I] {};
\node (N) [circ, right of=J] {};
\node (O) at ($(N)!.5!(J)$) [circ, yshift=.86602cm] {};

\node [below of=D] {$T_5=15$};
\end{scope}

\end{tikzpicture}
\end{figure}


Considere a sequência de números ilustrada acima. Ela é conhecida como a sequência dos \emph{números triangulares}. O \(n\)-ésimo número triangular, \(T_n\), é igual a quantidade total de círculos congruentes necessários para formar um triângulo equilátero cujo lado tem \(n\) círculos. Por exemplo, o quarto número triangular é \(T_4=10\), porque são necessários \(10\) círculos congruentes para formar um triângulo cujo lado tem \(4\) desses círculos.
\begin{enumerate}
\item {} 
Determine o 6º, o 7º e o 8º números triangulares.

\item {} 
Descreva o procedimento que você usou para determinar \(T_6\), \(T_7\) e \(T_8\) no item anterior.

\item {} 
Determine o milésimo número triangular, \(T_{1000}\).

\item {} 
Descreva um procedimento que permita determinar qualquer número triangular a partir da sua ordem na sequência? Explique.

\item {} 
Quais são as variáveis relacionadas?

\end{enumerate}



\ifdefined\prof
\begin{solucao}
\begin{enumerate}
\item {} 
$21$, $28$ e $36$

\item {} 
Uma resposta possível seria a partir de um raciocínio aditivo baseado em contagem: $T_6$ é obtido adicionando $6$ círculos a um dos lados do triângulo equilátero que corresponde a $T_5$ e efetuando a soma dos círculos presentes nesse novo triângulo equilátero: $T_6=1+2+3+4+5+6=21$. Outra maneira é a partir do raciocínio recursivo. Assim $T_6$ é obrido adicionando $6$ círculos ao total de círculos do triângulo equilátero que corresponde a $T_5$: $T_6=T_5+6=15+6=21$. Os números triangulares $T_7$ e $T_8$ podem ser obtidos de formas análogas.

\item {} 
$T_1000=1+2+3+4+5+6+...+1000=500500$.

\item {} 
Uma resposta possível é: o número triangular $T_n$ é obtido somando $n$ ao número triangular anterior

\item {} 
$n$ e $T_n$.

\end{enumerate}
\end{solucao}
\fi

\end{document}