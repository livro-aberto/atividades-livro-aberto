\documentclass[10 pt,usenames,dvipsnames, oneside]{article}
\usepackage{../../../modelo-ensino-medio}


\begin{document}

\begin{center}
  \begin{minipage}[l]{3cm}
\includegraphics[width=2cm]{logo}    
\end{minipage}\hfill
\begin{minipage}[r]{.8\textwidth}
 {\Large \scshape Atividade: Leia no gráfico}  
\end{minipage}
\end{center}
\vspace{.2cm}

\ifdefined\prof
\begin{objetivos}
\item \textbf{LAF1} Compreender função como uma relação de dependência entre duas variáveis, as ideias de domínio, contradomínio e imagem, e suas representações algébricas e gráficas e utilizá-las para analisar, interpretar e resolver problemas em contextos diversos, inclusive fenômenos naturais, sociais e de outras áreas.
\end{objetivos}

\begin{goals}
\begin{enumerate}

\item[OE1] Calcular, a partir da representação gráfica de uma função real de variável real, os valores de $f(x)$ e $x$ solicitados.

\end{enumerate}

\tcblower

\begin{itemize}
\item Todos os valores solicitados são exatos, esta opção foi feita com o intuito de facilitar a feitura da atividade. Caso julgue adequado você poderá explorar a determinação de valores aproximados, como por exemplo: $f(0,5)$ ou os valores aproximados de $x$ tais que $f(x)=0$.
\end{itemize}

\end{goals}

\bigskip
\begin{center}
{\large \scshape Atividade}
\end{center}
\fi

Seja \(f\) a função real cuja representação gráfica é apresentada a seguir.

\begin{figure}[H]
\centering

\begin{tikzpicture}
\tikzstyle{ponto}=[circle, minimum size=3pt, inner sep=0, draw=black, fill=black, shift only]
\draw[gray!40](0,-1.5)grid[xstep=.25,ystep=.25,line width =1pt](5.5,3);
\draw(0,-1.5)grid[xstep=1,ystep=1,line width =1pt, help lines](5.5,3);
\draw[->,thick](-.5,0)--(5.5,0) node[right]{$x$};
\draw[->,thick](0,-1.5)--(0,3)node[above]{$y$};
\draw[very thick, session1!80](0,-.5)--(.5,1)--(1,1.5)--(1.5,1.5)--(2,.5)--(2.5,0)--(3,-1)--(4,2)--(4.5,2.5)--(5,2.75);
\foreach \x in{2, 4, 6, ..., 10}
\draw(.5*\x,0)[below]node{\x};
\foreach \y in{-2, 2,4,6}
\draw(0,.5*\y)[left]node{\y};
\node[below left]at (0,0){0};
\end{tikzpicture}
\end{figure}

A partir da representação gráfica, calcule os seguintes valores:

\begin{table}[H]
\centering
\begin{tabu} to \textwidth{|l|c|}
\hline
\thead
Notação & Valor \\
\hline
\(f(1)-f(0)\) & \\
\hline
\(4\cdot f(3)\) & \\
\hline
\(f(4)/f(2)\) & \\
\hline
\(f(6)\cdot f(2)\) & \\
\hline
\(x\) quando \(f(x)=-2\) & \\
\hline
\(x\) quando \(f(x)=0\) & \\
\hline
\(f(3\cdot 2)-4\cdot f(\sqrt{81})+1\) & \\
\hline
\end{tabu}
\end{table}

\ifdefined\prof
\begin{solucao}
\begin{table}[H]
\centering
\begin{tabu} to \textwidth{|l|>{$}c<{$}|}
\hline
\cellcolor{primario}\textcolor{white}{\textbf{Notação}} & $\cellcolor{primario}\textcolor{white}{\textbf{Valor}}$ \\
\hline
\(f(1)-f(0)\) & 3 \\
\hline
\(4\cdot f(3)\) & 12 \\
\hline
\(f(4)/f(2)\) & 1/3 \\
\hline
\(f(6)\cdot f(2)\) & -6 \\
\hline
\(x\) quando \(f(x)=-2\) & x=6 \\
\hline
\(x\) quando \(f(x)=0\) & x=8 \\
\hline
\(f(3\cdot 2)-4\cdot f(\sqrt{81})+1\) & -21 \\
\hline
\end{tabu}
\end{table}
\end{solucao}
\fi

\end{document}