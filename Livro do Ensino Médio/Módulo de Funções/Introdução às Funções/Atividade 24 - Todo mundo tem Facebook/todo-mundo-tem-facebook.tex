\documentclass[10 pt,usenames,dvipsnames, oneside]{article}
\usepackage{../../../modelo-ensino-medio}


\begin{document}

\begin{center}
  \begin{minipage}[l]{3cm}
\includegraphics[width=2cm]{logo}    
\end{minipage}\hfill
\begin{minipage}[r]{.8\textwidth}
 {\Large \scshape Atividade: Todo mundo tem Facebook}  
\end{minipage}
\end{center}
\vspace{.2cm}

\ifdefined\prof
\begin{objetivos}
\item \textbf{LAF1} Compreender função como uma relação de dependência entre duas variáveis, as ideias de domínio, contradomínio e imagem, e suas representações algébricas e gráficas e utilizá-las para analisar, interpretar e resolver problemas em contextos diversos, inclusive fenômenos naturais, sociais e de outras áreas.
\end{objetivos}

\begin{goals}
\begin{enumerate}

\item[OE1] Utilizar os conhecimentos adquiridos ao longo do Capítulo para investigar o crescimento do número de usuários ativos na rede social Facebook.

\item[OE2] Fazer inferência baseado em um modelo matemático.

\end{enumerate}
\tcblower

\begin{itemize}
\item 
\end{itemize}
No item (e) os dados indicam que o número de usuários não irá ultrapassar $1.500.000.000$, mas isso pode não ser facilmente percebido. Espera-se, caso o estudante acredite que o número de usuários atinja os $2$ bilhões, que isso ocorra depois de um grande intervalo de tempo.
\end{goals}

\bigskip
\begin{center}
{\large \scshape Atividade}
\end{center}
\fi

A rede social virtual \emph{Facebook} é um grande sucesso. O Facebook criado por Mark Zuckerberg em outubro de 2003, com o nome de \emph{Facemash}, quando ele era  um estudante do segundo ano em Harvard. Inicialmente \(450\) visitantes geraram \(22.000\) visualizações de fotos em suas primeiras \(4\) horas online. Em fevereiro de \(2004\), agora com o nome de \emph{Thefacebook}, ele já contava com a participação de mais da metade dos alunos de Harvard, e um mês depois, estudantes das Universidades de Stanford, Columbia, Yale, Boston, Nova Iorque e MIT tiveram acesso à rede social criada por Mark Zuckerberg. A partir de setembro de \(2005\), funcionários de várias empresas, dentre elas \emph{Apple} e \emph{Microsoft}, puderam ter acesso ao \emph{Facebook} e no final de \(2006\) o serviço ficou disponível para qualquer pessoa maior de \(13\) anos e com um endereço válido de \emph{e-mail}.

A tabela a seguir mostra o número de usuários ativos do \emph{Facebook} em janeiro dos anos de \(2004\) a \(2015\).

\begin{table}[H]
\centering
\begin{tabu} to \textwidth{|c|l|c|}
\hline
\thead
Ano & Número de usuários & Crescimento percentual \\
\hline
2004 & 5 & \textendash{} \\
\hline
2005 & 1.000.000 & \\
\hline
2006 & 5.500.000 & 450\% \\
\hline
2007 & 12.000.000 & \\
\hline
2008 & 70.000.000 & \\
\hline
2009 & 150.000.000 & \\
\hline
2010 & 370.000.000 & \\
\hline
2011 & 600.000.000 & \\
\hline
2012 & 800.000.000 & \\
\hline
2013 & 1.056.000.000 & \\
\hline
2014 & 1.228.000.000 & \\
\hline
2015 & 1.317.000.000 & \\
\hline
\end{tabu}
\end{table}


Imagine que queremos investigar o crescimento anual do número de usuários. E, a partir da investigação formular um modelo que nos permita fazer previsões sobre a base de usuários para os próximos anos.
\begin{enumerate}
\item {} 
Vamos começar investigando o crescimento percentual, preenchendo as lacunas da terceira coluna da tabela acima.

\item {} 
Marque no plano cartesiano os pontos correspondentes aos dados fornecidos pelas duas primeiras colunas da tabela, usando a seguinte escala: no eixo das abscissas \(1\) cm corresponde a \(1\) ano e no eixo das ordenadas \(1\) cm corresponde a \(200\) milhões de usuários ativos.

\item {} 
Como você descreveria o crescimento do número de usuários ativos do \emph{Facebook}? Você acha que o crescimento está com tendência a diminuir, a aumentar ou a permanecer estável?

\item {} 
Baseado no item (c), faça uma previsão para o número de usuários para os anos de 2016 e 2017.

\item {} 
Usando os dados da tabela e a representação gráfica feita no item (b), faça uma previsão para o futuro do \emph{Facebook}. Você acha que os números continuarão a aumentar? Se sim, quando ele atingirá a marca de \(2\) bilhões de usuários? Explique seu raciocínio.

\item {} 
Um modelo matemático que fornece uma aproximação para a relação entre os dados das duas primeiras colunas da tabela é dado por uma função \(f\) que tem a seguinte expressão
\begin{equation*}
\begin{split}f(x)=\dfrac{980}{0,7+670 \cdot 0,45^{(x+1)}}\end{split}
\end{equation*}
em que \(x\) representa o tempo decorrido desde \(2004\), isto é, para \(2010\) tem-se \(x=6\), e \(f(6)\) é o valor em milhões de usuários ativos no \emph{Facebook} naquele ano. Com a ajuda de uma calculadora científica, use a expressão acima para calcular a estimativa do número de usuários nos anos de \(2013\) e de \(2014\), e em seguida compare com a tabela.

\item {} 
Use a expressão anterior e calcule a estimativa para os anos de \(2016\) e \(2017\) e compare com as suas previsões do item (d).

\end{enumerate}

Os dados reais para os meses de janeiro de \(2016\) e \(2017\) são \(1.654.000.000\) e \(1.936.000.000\), respectivamente. Isso significa que apesar do modelo descrever de forma satisfatória o comportamento do crescimento do número de usuários até o ano de \(2015\), para os anos seguintes ele não se mostra adequado. Existia de fato uma tendência para diminuição do crescimento, no entanto essa trajetória foi possivelmente modificada por ações que foram tomadas pela empresa ao perceber tal comportamento.

Situações como essa são bastante comuns em Modelagem Matemática. O modelo se mostra adequado sob certas condições, mas quando outras variáveis são consideradas (investimento em propaganda, alteração no algoritmo que escolhe as atualizações que serão exibidas para cada usuário, etc) ele pode perder sua acurácia, momento em que se fazem necessárias revisões.


\ifdefined\prof
\begin{solucao}
\begin{enumerate}
\item $19999900\% , 450\%, 118\%, 483\%, 114\%, 147\%, 62\%, 33\%, 32\%, 16\%, 7\%$.

\item\adjustbox{valign=t}{
\begin{tikzpicture}[scale=1.5, every node/.style={black}, every path/.style={black}]
\draw [help lines, xstep=.5cm,ystep=.25cm] (-.1,-.1) grid (7.5,4.1);
\foreach \x in {0,1, 2,3, 4, 5, 6,7,8,9,10,11,12,13,14, 15}
\draw[shift={(.5*\x,0)},color=black] (0pt,-2pt) -- (0pt,-2pt) node[below] { $\x$};
\foreach \y in {100,200,300,400,500,600,700,800,900,1000,1100,1200,1300,1400,1500,1600}
\draw[shift={(-.3,.18 +.0025*\y)},color=black] (0pt,-2pt) -- (0pt,-2pt) node[below] { $\y$};
\draw[thick, ->](-.1,0)--(7.6,0);
\draw[thick, ->](0,-.1)--(0,4.1);
\draw[fill =session1](.5,0) circle(1pt);
\draw[fill =session1](1,.06) circle(1pt);
\draw[fill =session1](1.5,.08) circle(1pt);
\draw[fill =session1](2,.09) circle(1pt);
\draw[fill =session1](2.5,.2) circle(1pt);
\draw[fill =session1](3,.37) circle(1pt);
\draw[fill =session1](3.5,.92) circle(1pt);   
\draw[fill =session1](4,1.5) circle(1pt);
\draw[fill =session1](4.5,2) circle(1pt);
\draw[fill =session1](5,2.62) circle(1pt);
\draw[fill =session1](5.5,3.08) circle(1pt);
\draw[fill =session1](6,3.35) circle(1pt);
\end{tikzpicture}}

\item No primeiro ano, observa-se um grande crescimento no número de usuários ativos, entre os anos de $2006$ e $2010$, o crescimento percentual oscila e, a partir de $2011$, é cada vez menor, indicando que o crescimento do número de usuários está com tendência a diminuir.

\item Espera-se para $2016$ um valor acima de $1.317.000.000$ e abaixo de $1.400.000.000$. Para $2017$ um valor maior que o anterior e que não ultrapasse $1.500.000.000$.

\item É razoável imaginar que o número de usuários continuará a aumentar. Com um crescimento percentual cada vez menor a tendência observada é que a marca de $2$ bilhões de usuários não será atingida.

\item Para o ano de $2013$ tem-se $f(9)=1.055.876.085$ e para o ano de $2014$ tem-se $f(10)=1.220.936.348$.

\item Para o ano de $2016$ o modelo prevê um número de usuários de $f(12)=1.359.620.842$ e para $2017$, $f(13)=1.381.536.488$.

\end{enumerate}
\end{solucao}
\fi

\end{document}