\documentclass[10 pt,usenames,dvipsnames, oneside]{article}
\usepackage{../../../modelo-ensino-medio}



\begin{document}

\begin{center}
  \begin{minipage}[l]{3cm}
\includegraphics[width=2cm]{logo}    
\end{minipage}\hfill
\begin{minipage}[r]{.8\textwidth}
 {\Large \scshape Atividade: A família}  
\end{minipage}
\end{center}
\vspace{.2cm}

\ifdefined\prof
\begin{objetivos}
\item \textbf{LAF1} Compreender função como uma relação de dependência entre duas variáveis, as ideias de domínio, contradomínio e imagem, e suas representações algébricas e gráficas e utilizá-las para analisar, interpretar e resolver problemas em contextos diversos, inclusive fenômenos naturais, sociais e de outras áreas.
\end{objetivos}

\begin{goals}
\begin{enumerate}

\item[OE1] Identificar uma relação a partir de sua representação no plano cartesiano.

\item[OE2] Identificar a univocidade (ou não) de uma relação a partir de sua representação no plano cartesiano.

\end{enumerate}

\tcblower

\begin{itemize}
\item Nível de abstração \textbf{Processo}.

\item No item (b) o objetivo é que os estudantes percebam que, como as pessoas representadas pelos pontos $C$ (Márcia) e $D$ (Júlio) têm a mesma idade mas alturas diferentes, a relação apresentada no gráfico, que associa a idade com a altura nessa ordem, não é função.
\end{itemize}

\end{goals}

\bigskip
\begin{center}
{\large \scshape Atividade}
\end{center}
\fi

Cada ponto do gráfico a seguir representa uma das seguintes pessoas.

\begin{enumerate}
\item {} 
Associe cada ponto do gráfico à pessoa correspondente.

\item {} 
A relação expressa pelos pares ordenados (idade, altura) apresentados no gráfico é função? Por que?
\end{enumerate}


\begin{center}\begin{tikzpicture}[scale=1.4]
\draw[->](-0,0)--(6,0);
\draw(5.7,0) node [below]{idade};
\draw[->](0,-0)--(0,4);
\draw(0,3.95) node [above left, rotate =90]{altura};
\draw[fill](5.5,1.5) circle(1pt) node[right]{$F$};
\draw[dotted] (5.5,0)--(5.5,1.5)--(0,1.5);
\draw[fill](4.5,3.5) circle(1pt) node[above]{$E$};
\draw[dotted](4.5,0)--(4.5,3.5)--(0,3.5);
\draw[fill](3,3.5) circle(1pt) node[above]{$D$};
\draw[dotted](3,0)--(3,3.5);
\draw[fill](3,2.5) circle(1pt) node[right]{$C$};
\draw[dotted](3,2.5)--(0,2.5);
\draw[fill](2,1) circle(1pt) node[right]{$B$};
\draw[dotted](2,0) -- (2,1) --(0,1);
\draw[fill](.5,.3) circle(1pt) node[right]{$A$};
\draw[dotted](.5,0)--(.5,.3)--(0,.3);
\end{tikzpicture}\end{center}

\begin{figure}[H]
\centering



\noindent\includegraphics[width=300bp]{{familia}.png}
\label{\detokenize{AF106-2:fig-altura-idade}}\end{figure}


{\color{red}\bfseries{}*}Adaptado de The Language of Functions and Graphs, Shell Centre for Mathematical Education Publications Ltd., 1985.


\ifdefined\prof
\begin{solucao}

\begin{enumerate}
\item O ponto $A$ representa o bebê Miguel, ponto $B$ Sofia, ponto $C$ Márcia, $D$ Júlio, $E$ Antônio e o ponto $F$ Laura.

\item Não é função, pois Márcia e Júlio têm a mesma idade mas alturas diferentes; no plano, os pontos $C$ e $D$ têm a mesma abscissa e ordenadas diferentes.

\end{enumerate}
\end{solucao}
\fi

\end{document}