\documentclass[10 pt,usenames,dvipsnames, oneside]{article}
\usepackage{../../../modelo-ensino-medio}



\begin{document}

\begin{center}
  \begin{minipage}[l]{3cm}
\includegraphics[width=2cm]{logo}    
\end{minipage}\hfill
\begin{minipage}[r]{.8\textwidth}
 {\Large \scshape Atividade: Não é função!}  
\end{minipage}
\end{center}
\vspace{.2cm}

\ifdefined\prof
\begin{objetivos}
\item \textbf{LAF1} Compreender função como uma relação de dependência entre duas variáveis, as ideias de domínio, contradomínio e imagem, e suas representações algébricas e gráficas e utilizá-las para analisar, interpretar e resolver problemas em contextos diversos, inclusive fenômenos naturais, sociais e de outras áreas.
\end{objetivos}

\begin{goals}
\begin{enumerate}

\item[OE1] Identificar a univocidade (ou não) em uma relação

\end{enumerate}

\tcblower

\begin{itemize}
\item Nível de abstração: \textbf{Processo}.

\item Esta é a oportunidade para reforçar as condições que garantem que uma relação é função, em particular, a univocidade.
\end{itemize}

\end{goals}

\bigskip
\begin{center}
{\large \scshape Atividade}
\end{center}
\fi

Considere a relação formada por todos \((a,b)\) de números naturais tais que \(b\) é múltiplo de \(a\). Assim, \((2,4)\), \((2,6)\), \((3,6)\) e \((9, 9)\) são pares ordenado dessa relação, pois \(4\) é múltiplo de \(2\), \(6\) é múltiplo de \(2\) e de \(3\) e \(9\) é múltiplo de \(9\) . No entanto, \((4,2)\) e \((7,17)\) são pares ordenados de números naturais, mas não são pares dessa relação.
\begin{enumerate}
\item {} 
Exiba outros quatro pares ordenados dessa relação.

\item {} 
Explique por que essa relação não é uma função.

\item {} 
\((5, 405)\) é um par ordenado dessa relação. Quantos outros pares ordenados dessa relação têm 5 como primeiro elemento?

\item {} 
Dê exemplo de uma ou mais relações que não sejam funções. Não precisam ser exemplos numéricos.

\end{enumerate}

\ifdefined\prof
\begin{solucao}
\begin{enumerate}
\item {} 
$(2,8),(3,9), (1,1)$ e $(5,10)$ pertencem à relação.

\item {} 
Por exemplo, os pares $(3,12)$ e $(3,15)$ pertencem à relação e isso nos mostra que o número natural $3$ está associado a $12$ e a $15$. Portanto, a relação não pode ser função.

\item {} 
Infinitos.

\item {} 
Um exemplo não númerico: a relação associa cada livro ao seu autor

\end{enumerate}
\end{solucao}
\fi

\end{document}