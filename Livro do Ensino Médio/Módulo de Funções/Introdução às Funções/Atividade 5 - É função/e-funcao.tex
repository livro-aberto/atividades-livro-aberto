\documentclass[10 pt,usenames,dvipsnames, oneside]{article}
\usepackage{../../../modelo-ensino-medio}


\begin{document}

\begin{center}
  \begin{minipage}[l]{3cm}
\includegraphics[width=2cm]{logo}    
\end{minipage}\hfill
\begin{minipage}[r]{.8\textwidth}
 {\Large \scshape Atividade: É função?}  
\end{minipage}
\end{center}
\vspace{.2cm}

\ifdefined\prof
\begin{goals}
\begin{enumerate}

\item[OE1] Identificar, em um contexto, diferentes relações de dependência entre conjuntos de dados, reconhecendo quais são funções.

\item[OE2] Identificar a univocidade (ou não) de uma relação.

\end{enumerate}

\tcblower

\begin{itemize}
\item Nível de abstração: \textbf{Processo}.

\item Esta é a oportunidade para reforçar as condições que garantem que uma relação é função, em particular, a univocidade.
\end{itemize}

\end{goals}

\bigskip
\begin{center}
{\large \scshape Atividade}
\end{center}
\fi

No contexto da atividade anterior são observados diferentes conjuntos de dados: O conjunto dos tempos de travessia da ponte, \(A=\{13, 14, 15, 16, 18, 23\}\); O conjunto das cores que compoõem a escala, \(B=\{\) Verde, Laranja, Vermelho, Vinho \(\}\); e o conjunto de velocidades obtidas,{}`C{}`. Considere as diferentes relações de dependências estabelecidas entre esses conjuntos. Quais são funções?

\begin{table}[H]
\centering
\begin{tabu} to \textwidth{|c|c|>{\centering}m{6cm}|}
\hline
\thead
Relação & É função? &  Se não, por que? \\
\hline
De A em B
&&\\
\hline
De B em A
&&\\
\hline
De A em C
&&\\
\hline
De C em A
&&\\
\hline
De B em C
&&\\
\hline
De C em B
&&\\
\hline
\end{tabu}
\end{table}



\ifdefined\prof
\begin{solucao}

Apenas as relações de $B$ em $A$ e de $B$ em $C$ não são funções. A primeira porque a uma mesma cor estão associados diferentes tempos de travessia, e a segunda porque a uma mesma cor estão assiciadas velocidades médias diferentes.


\end{solucao}
\fi

\end{document}