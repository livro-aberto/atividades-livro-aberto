\documentclass[10 pt,usenames,dvipsnames, oneside]{article}
\usepackage{../../../modelo-ensino-medio}



\begin{document}

\begin{center}
  \begin{minipage}[l]{3cm}
\includegraphics[width=2cm]{logo}    
\end{minipage}\hfill
\begin{minipage}[r]{.8\textwidth}
 {\Large \scshape Atividade: Praticando a notação}  
\end{minipage}
\end{center}
\vspace{.2cm}

\ifdefined\prof
\begin{objetivos}
\item \textbf{LAF1} Compreender função como uma relação de dependência entre duas variáveis, as ideias de domínio, contradomínio e imagem, e suas representações algébricas e gráficas e utilizá-las para analisar, interpretar e resolver problemas em contextos diversos, inclusive fenômenos naturais, sociais e de outras áreas.
\end{objetivos}

\begin{goals}
\begin{enumerate}

\item[OE1] Compreender funções a partir de sua representação analítica.


\end{enumerate}

\tcblower

\begin{itemize}
\item Nível de abstração: \textbf{Ação}.

\item Muitos estudantes cometem erros relacionados ao uso da expressão analítica que representa a função. É comum, por exemplo, que o cálculo de $f(-2)$ para $f(x)=x^2$ seja feito da seguinte forma: $f(-2)=-2^2=-4$. O que claramente está errado. Muito frequentemente, esse tipo de erro está relacionado à falta de compreensão do papel de uma varíavel em uma expressão algébrica. Aproveite a atividade para fazer uma revisão.
\end{itemize}

\end{goals}

\bigskip
\begin{center}
{\large \scshape Atividade}
\end{center}
\fi

Considere as funções \(f\), \(g\), \(k\) e \(h\), todas de domínio \(\mathbb{R}\), tais que:
\begin{equation*}
\begin{split}f(x)=3x^2+5x\quad ; \quad g(x)=\frac{x-1}{x^3+3}\quad ; \quad k(x)=(x-2)^2+6\quad ; \quad h(x)=2x-7\end{split}
\end{equation*}
Determine o valor de:


\begin{table}[H]
\centering
\begin{tabu} to \textwidth{|l|c|}
\hline
\thead
Função & Valor \\
\hline
\(f(3)\) & \\ 
\hline
\(g(-1)\) & \\
\hline
\(k(2)\) & \\
\hline
\(f(1)+g(1)\) & \\
\hline
\(g(2)-k(-1)\) & \\
\hline
\(k(0).f(-2)\) & \\
\hline
\(f(0)+h(0)-1\) & \\
\hline
\(f(-2).g(-2)+k(2)\) & \\
\hline
\(\dfrac{f(-3)}{k(0)}\) & \\
\hline
\(x\) quando \(h(x)=0\) & \\
\hline
\(x\) quando \(h(x)=3\) & \\
\hline
\end{tabu}
\end{table}



\ifdefined\prof
\begin{solucao}

\begin{table}[H]
\centering
\begin{tabu} to \textwidth{|l|>{$}c<{$}|}
\hline
\cellcolor{primario}\textcolor{white}{\textbf{Função}} & $\cellcolor{primario}\textcolor{white}{\textbf{Valor}}$ \\
\hline
\(f(3)\) & 42\\ 
\hline
\(g(-1)\) & -1\\
\hline
\(k(2)\) & 6\\
\hline
\(f(1)+g(1)\) & 8\\
\hline
\(g(2)-k(-1)\) & -\frac{46}{3}\\
\hline
\(k(0).f(-2)\) & 20\\
\hline
\(f(0)+h(0)-1\) & -8\\
\hline
\(f(-2).g(-2)+k(2)\) & \frac{36}{5}\\
\hline
\(\dfrac{f(-3)}{k(0)}\) & \frac{6}{5}\\
\hline
\(x\) quando \(h(x)=0\) & \frac{7}{2}\\
\hline
\(x\) quando \(h(x)=3\) & 5\\
\hline
\end{tabu}
\end{table}

\end{solucao}
\fi

\end{document}