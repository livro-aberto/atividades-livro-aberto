\documentclass[10 pt,usenames,dvipsnames, oneside]{article}
\usepackage{../../../modelo-ensino-medio}



\begin{document}

\begin{center}
  \begin{minipage}[l]{3cm}
\includegraphics[width=2cm]{logo}    
\end{minipage}\hfill
\begin{minipage}[r]{.8\textwidth}
 {\Large \scshape Atividade: Abastecendo a caixa}  
\end{minipage}
\end{center}
\vspace{.2cm}

\ifdefined\prof
\begin{objetivos}
\item \textbf{EM13MAT302} Construir modelos empregando as funções polinomiais de 1º ou 2º graus, para resolver problemas em contextos diversos, com ou sem apoio de tecnologias digitais.
\end{objetivos}

\begin{goals}
\begin{enumerate}
\item Identificar a taxa de variação gerada por duas razões distintas.
\item Reconhecer que a taxa de variação é negativa na situação descrita.
\item Relacionar o preenchimento do quadro com a expressão algébrica que modela a situação, sem a necessidade da representação gráfica.
\end{enumerate}

\tcblower
\begin{itemize}
\item Discuta com seus alunos a importância da utilização dos conceitos trabalhados no processo de controle do desperdício de água potável. E como ações simples, pautadas em dados quantitativos podem influenciar na economia de água.
\item Durante a aplicação da atividade, conduza as discussões para que seus alunos argumentem à respeito do sinal da taxa de variação.
\item Possibilite à seus alunos a oportunidade de apresentar soluções diferentes das usuais, seja utilizando conceitos de Progressões aritméticas ou até mesmo de proporcionalidade (fazendo os ajustes necessários, já que $V(0)$ não é zero).
\end{itemize}

\end{goals}

\bigskip
\begin{center}
{\large \scshape Atividade}
\end{center}
\fi

Uma caixa de água é abastecida por uma torneira cujo fluxo de água é constante e igual a \(10\) litros por minuto e, simultaneamente, seu conteúdo escoa, por um ralo, cujo fluxo de água é controlado à razão constante de \(15\) litros por minuto. Em certo instante, o volume de água dentro da caixa é de \(100\) litros, estando a torneira e o ralo ambos abertos.
\begin{enumerate}
\item {} 
Sendo V(t) o volume de água na caixa após t minutos do instante citado. Exiba uma sentença matemática para V(t).

\item {} 
Complete a tabela abaixo com os valores correspondentes ao volume de água na caixa.

\begin{table}[H]
\centering
\begin{tabu} to \textwidth{|l|c|c|c|c|c|c|c|c|}
\hline
\thead
Tempo (minutos) & 0 & 1 & 2 & 3 & 4 & 5 & 10 & 20 \\
\hline
Volume (litros) & & 4 & & & & & & \\
\hline
\end{tabu}
\end{table}

\item {} 
À medida que os valores do tempo aumentam, o que ocorre com os valores correspondentes ao volume de água da caixa?

\item {} 
Quando os valores do tempo aumentam de \(t=1\) a \(t=2\), o quanto variam os valores correspondentes ao volume de água da caixa? E quando estes valores aumentam de \(t=12\) a \(t=13\)?

\item {} 
Quando os valores do tempo aumentam em uma unidade, a partir de um instante qualquer, o quanto variam os valores correspondentes ao volume de água da caixa?

\end{enumerate}


\ifdefined\prof
\begin{solucao}
\begin{enumerate}
\item $V(t)=−5t+100$.
\item Diminuem.
\item Reduzem $5$ litros em ambos os casos.
\item Em $5$ litros.

\end{enumerate}
\end{solucao}
\fi

\end{document}