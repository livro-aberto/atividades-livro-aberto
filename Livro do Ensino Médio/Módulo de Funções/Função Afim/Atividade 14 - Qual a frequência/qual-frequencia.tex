\documentclass[10 pt,usenames,dvipsnames, oneside]{article}
\usepackage{../../../modelo-ensino-medio}



\begin{document}

\begin{center}
  \begin{minipage}[l]{3cm}
\includegraphics[width=2cm]{logo}    
\end{minipage}\hfill
\begin{minipage}[r]{.8\textwidth}
 {\Large \scshape Atividade: Qual a frequência?}  
\end{minipage}
\end{center}
\vspace{.2cm}

\ifdefined\prof
\begin{objetivos}
\item \phantom{a}
\end{objetivos}

\begin{goals}
\begin{enumerate}
\item Utilizar a ideia de progressão aritmética para resolver um problema.
\end{enumerate}

\tcblower
\begin{itemize}
\item Evite usar a fórmula $a_n=a_1+(n−1)r$ como único recurso para resolver problemas de P.A. Ela, apesar de parecer prática, pode esconder as ideias simples dos problemas de progressão aritmética. Por exemplo, da 1ª para 70ª frequencia, somamos $69$ vezes a razão. E depois da 70ª para a 86ª, somamos 16 vezes a razão. Isso é mais simples do que escrever $a_70=a_1+(70−1)r$ e $a_86=a_1+(86−1)r$.
\end{itemize}

\end{goals}

\bigskip
\begin{center}
{\large \scshape Atividade}
\end{center}
\fi

A Agência Nacional de Telecomunicações (ANATEL) determina que as emissoras de rádio FM utilizem as frequências de 87,9 a 107,9 MHz, e que haja uma diferença de 0,2 MHz entre emissoras com frequências vizinhas.
\begin{enumerate}
\item Em uma determinada região as frequências entre a 70ª e 86ª são reservadas rádios comunitárias. Determine a frequência mínima e máxima para uma rádio comunitária.
\item Determine quantas emissoras FM podem funcionar em uma mesma região.
\item Lembre-se da frequência da rádio que você costuma ouvir e determine a posição dela na sequência das frequências do problema.
\end{enumerate}
\ifdefined\prof
\begin{solucao}
\begin{enumerate}

\item 101,7 a 104,9.
\item 101 emissoras FM.
\item Resposta individual.

\end{enumerate}
\end{solucao}
\fi

\end{document}