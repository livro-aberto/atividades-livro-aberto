\documentclass[10 pt,usenames,dvipsnames, oneside]{article}
\usepackage{../../../modelo-ensino-medio}



\begin{document}

\begin{center}
  \begin{minipage}[l]{3cm}
\includegraphics[width=2cm]{logo}    
\end{minipage}\hfill
\begin{minipage}[r]{.8\textwidth}
 {\Large \scshape Atividade: }  
\end{minipage}
\end{center}
\vspace{.2cm}

\ifdefined\prof
\begin{objetivos}
\item \textbf{EM13MAT302} Construir modelos empregando as funções polinomiais de 1º ou 2º graus, para resolver problemas em contextos diversos, com ou sem apoio de tecnologias digitais.
\item \textbf{LAF2} Compreender a taxa de variação como uma medida de covariação entre grandezas e utilizá-la para interpretar situações reais.
\end{objetivos}

\begin{goals}
\begin{enumerate}
\item Perceber com o auxílio da representação gráfica a relação entre taxa de variação média negativa e função linear decrescente.
\end{enumerate}

\tcblower

É possível que os estudantes utilizem regra de três para responder as questões propostas no item a). A seguir iremos construir a representação gráfica da função linear, por isso é importante fazer a conexão da regra de três com sua interpretação geométrica, destacando o uso da semelhança de triângulos.

\begin{figure}[H]
\centering

\begin{tikzpicture}[yscale=.5]
\tikzstyle{ponto}=[circle, minimum size=3pt, inner sep=0,    draw=black, fill=black, shift only]
\draw[->, thick](-1,0)--(5,0);
\draw[->, thick](0,-13)--(0,2);
\draw[primario, thick](0,0)--(4,-12);
\draw[dashed](0,-12)--(4,-12)--(4,0);
\node[ponto]at (0,0){};
\node[ponto]at (4,-12){};
\node[above,rotate=90] at(-1,-10){Temperatura $^\circ$C};
\node[above] at(5,0){Tempo (h)};
\node[above] at(4,0){8};
\node[left] at(0,-12){$-24$};
\node[above] at(2,0){$t$};
\node[left] at(0,-6){$f(t)$};
\draw[dashed](2,0)--(2,-6)--(0,-6);
\node[ponto]at (2,-6){};
\end{tikzpicture}
\end{figure}
\end{goals}

\bigskip
\begin{center}
{\large \scshape Atividade}
\end{center}
\fi

Uma câmara frigorífica está programada para diminuir sua temperatura segundo uma taxa constante em \(^\circ C\) por hora. Na primeira observação constata-se que ela está a \(0^\circ C\). Após \(8\) horas, realiza-se uma nova observação e seu visor mostra a temperatura de \(-24^\circ C\) e também o seguinte gráfico para a evolução da temperatura em função do tempo.

\begin{figure}[H]
\centering

\begin{tikzpicture}[yscale=.75, xscale=1.25]
\tikzstyle{ponto}=[circle, minimum size=3pt, inner sep=0, draw=black, fill=black, shift only]
\draw(-3,-.05) grid(5,.05);
\draw(-.05,-12) grid(.05,2);
\draw[->, thick](-3,0)--(5,0);
\draw[->, thick](0,-12)--(0,2);
\draw[\currentcolor!80, thick](0,0)--(4,-12);
\draw[dashed](0,-12)--(4,-12)--(4,0);
\node[ponto]at (0,0){};
\node[ponto]at (4,-12){};
\node[above,rotate=90] at(-1,-10){Temperatura $^\circ$C};
\node[above] at(4,0){Tempo (h)};
\foreach\x in{-4, -2, 0, 2, 4, 6, 8}
\node[below left] at (.5*\x, 0){\x};
\foreach \y in{-24, -22, -20, ..., -2}
\node[left]at(0,.5*\y){\y};
\node[left]at(0,1){2};
\node[left]at(0,2){4};
\end{tikzpicture}
\end{figure}
\begin{enumerate}
\item {} 
Qual a temperatura da câmara \(1\) hora após a primeira observação? E \(5\) horas após a primeira observação? E \(t\) horas após a primeira observação?

\item {} 
Qual o valor da taxa (de variação média) constante segundo a qual a temperatura diminui?

\item {} 
Determine a função que relaciona temperatura e tempo nesse contexto, considerando para seu domínio o intervalo de números reais \([0,8]\). Ela é uma função crescente ou decrescente? Por que?

\item {} 
Como seria o gráfico se a temperatura, no mesmo intervalo de tempo, ao invés de diminuir, estivesse aumentando \(1,5^\circ C/h\)? Qual seria a expressão da função, nesse caso? Teríamos uma função crescente ou decrescente? Por que?

\end{enumerate}

\ifdefined\prof
\begin{solucao}
\begin{enumerate}
\item Após 1 hora desde a primeira observaço, a temperatura será de $-3^{\circ}$ C. Após 5 horas, a temperatura será de $-15^{\circ}$, e $t$ horas após a primeira observação, a temperatura 'será de $-3t^{\circ}$ C

\item $-3^{\circ}$ C/h

\item $f:[0,8]\rightarrow\R, f(t)=-3t$ é uma função decrescente, pois à medida que o tempo aumenta, a temperatura correspendente diminui. Ou ainda, para quaisquer tempos $t_1$ e $t_2$ tais que $t_1<t_2$ tem-se que $-3t_1>-3t_2$, isto é $f(t_1)>f(t_2)$.

\item A expressão da função é $f(t)=1{,}5\cdot t$. É uma função crescente, pois à medida que o tempo aumenta, a temperatura correspendente também aumenta.

\begin{tikzpicture}
\tikzstyle{ponto}=[circle, minimum size=2pt, inner sep=0, draw=black, fill=black, shift only]
\draw[->, thick](-1,0)--(6,0)node[above, xshift=-.5cm]{Tempo (h)};
\draw[->, thick](0,-1)--(0,7)node[left,xshift=-.7cm, rotate=90]{Temperatura ($^\circ$C)};
\draw[primario, domain=0:4.4]plot(\x, 1.5*\x);
\node[below]at (1,0){1};
\node[below]at (4,0){8};
\node[left]at (0,1.5){1.5};
\node[left]at (0,6){12};
\draw[dashed](0,1.5)--(1,1.5)--(1,0);
\draw[dashed](0,6)--(4,6)--(4,0);
\end{tikzpicture}

\item Quando a taxa de variação média de uma função linear é um número real \textit{positivo}, a função é \textit{crescente} e quando a taxa é um número real \textit{negativo}, a função é \textit{descrescente}.
\end{enumerate}
\end{solucao}
\fi

\end{document}