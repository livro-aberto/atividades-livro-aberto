\documentclass[10 pt,usenames,dvipsnames, oneside]{article}
\usepackage{../../../modelo-ensino-medio}



\begin{document}

\begin{center}
  \begin{minipage}[l]{3cm}
\includegraphics[width=2cm]{logo}    
\end{minipage}\hfill
\begin{minipage}[r]{.8\textwidth}
 {\Large \scshape Atividade: Na Piscina}  
\end{minipage}
\end{center}
\vspace{.2cm}

\ifdefined\prof
\begin{objetivos}
\item \textbf{EM13MAT501} Investigar relações entre números expressos em tabelas para representá-los no plano cartesiano, identificando padrões e criando conjecturas para generalizar e expressar algebricamente essa generalização, reconhecendo quando essa representação é de função polinomial de 1º grau.
\end{objetivos}

\begin{goals}
\begin{enumerate}
\item Reconhecer uma relação de proporcionalidade entre grandezas a partir da análise gráfica e da construção e análise dos dados em uma tabela;

\item Conjecturar sobre a representação gráfica de grandezas diretamente proporcionais, associando-a a um conjunto de pontos colineares
\end{enumerate}

\tcblower

\begin{itemize}
\item Como primeira atividade do capítulo, priorize as ideias em detrimento do rigor matemático. Ajude seus estudantes a transcreverem suas ideias de maneira precisa, ainda que informais.
No item (b), podem surgir respostas como: a primeira coluna aumenta de “uma em uma hora” enquanto a segunda aumenta de “200 em 200 litros”; a segunda coluna é obtida multiplicando a primeira por 200; tabela gerada pela função f(n)=200⋅n com domínio {0,1,2,3,4,5}. Apesar de não serem consequências diretas da definição, estão corretas e serão tratadas ao longo do capítulo.
\end{itemize}
\end{goals}

\bigskip
\begin{center}
{\large \scshape Atividade}
\end{center}
\fi

Duas piscinas de 1000 litros cada estão sendo enchidas simultaneamente. A piscina 1 leva 5 horas para ficar completamente cheia e a piscina 2, 8 horas. A cada hora, o volume total de água em cada piscina foi sendo registrado em dois gráficos


\begin{figure}[H]
\centering

\begin{tikzpicture}[scale=1.5, every node/.style={scale=1}]
\tikzstyle{ponto}=[circle, minimum size=3pt, inner sep=0, draw=\currentcolor!80, fill=\currentcolor!80, shift only]
\begin{scope}[yscale=.5]
\draw[lightgray](0,0)grid[xstep=.25,ystep=.25](6,10);
\draw[gray](0,0)grid(6,10);
\draw[thick, ->](0,0)--(6,0)node [below, shift={(-0.55,-0.2)}]{tempo(horas)};
\draw[thick, ->](-0,0)--(0,10);
\node[right, rotate=90]at (-.9,6){capacidade(litros)};
\foreach \x in{0,1, 2, 3, 4, 5}
\node[ponto]at(\x,2*\x){};
\foreach\x in{0,1, 2, 3, 4, 5, 6}
\node[below] at(\x, 0){ \x};
\foreach\y in{100, 200, 300, ..., 1000}
\node[left]at(0,.01*\y){ \y};
\end{scope}
\end{tikzpicture}
\caption{Piscina 1}
\label{piscina1}
\end{figure}
\begin{figure}[H]
\centering

\begin{tikzpicture}[scale=1.5, every node/.style={scale=1}]
\tikzstyle{ponto}=[circle, minimum size=3pt, inner sep=0, draw=\currentcolor!80, fill=\currentcolor!80, shift only]
\begin{scope}[yscale=.5]
\draw[lightgray](0,0)grid[xstep=.25,ystep=.25](8,10);
\draw[gray](0,0)grid(8,10);
\draw[thick, ->](0,0)--(8,0)node [below, shift={(-0.55,-0.3)}]{tempo(horas)};
\draw[thick, ->](-0,0)--(0,10);
\node[right, rotate=90]at (-.9,6){ capacidade(litros)};
\node[ponto]at(0,0){};
\node[ponto]at(1,1.5){};
\node[ponto]at(2,2){};
\node[ponto]at(3,3){};
\node[ponto]at(4,5){};
\node[ponto]at(5,8){};
\node[ponto]at(6,9){};
\node[ponto]at(7,9.5){};
\node[ponto]at(8,10){};
\foreach\x in{0,1, 2, 3, 4, 5, 6, 7, 8}
\node[below] at(\x, 0){\x};
\foreach\y in{100, 200, 300, ..., 1000}
\node[left]at(0,.01*\y){\y};
\end{scope}
\end{tikzpicture}
\caption{Piscina 2}
\label{piscina2}

\end{figure}

\begin{enumerate}
\item {} 
Construa uma tabela com os dados de cada gráfico.

\item {} 
As grandezas volume total de água e tempo de enchimento da piscina 1 são diretamente proporcionais? Explique.

\item {} 
As grandezas volume total de água e tempo de enchimento da piscina 2 são diretamente proporcionais? Explique.

\end{enumerate}


\ifdefined\prof
\begin{solucao}
\begin{enumerate}
\item Piscina 1

\adjustbox{valign=t}
{
\begin{tabu} to \textwidth{|c|c|}
\hline
\thead
(Tempo ($\rm{h}$) & volume (litros) \\
\hline
0 & 0 \\
\hline
1 & 200 \\
\hline
2 & 400 \\
\hline
3 & 600 \\
\hline
4 & 800 \\
\hline
5 & 1000 \\
\hline
\end{tabu}
}

\vspace{2em}

Piscina 2

\adjustbox{valign=t}
{
\begin{tabu} to \textwidth{|c|c|}
\hline
\thead
(Tempo ($\rm{h}$) & volume (litros) \\
\hline
0 & 0 \\
\hline
1 & 150 \\
\hline
2 & 200 \\
\hline
3 & 300 \\
\hline
4 & 500 \\
\hline
5 & 800 \\
\hline
6 & 900 \\
\hline 
7 & 950 \\
\hline
8 & 1000 \\
\hline
\end{tabu}
}

\vspace{2em}


\item Sim, pois para $k\in\{0,2,3,4,5\}$ tempo
\[\begin{array}{ccc}
x\quad &\overline{\quad \quad \quad}& \quad y \\
k\cdot 1 \quad &\overline{\quad \quad \quad}& \quad k\cdot 200
\end{array}\]

\item Não, pois ao final da primeira hora o volume total de água aumentou 150 litros e na hora seguinte aumentou apenas 50 litros. Para haver proporcionalidade direta, deveria ter aumentado também 150 litros na segunda hora, totalizando 300 litros.
\end{enumerate}
\end{solucao}
\fi

\end{document}