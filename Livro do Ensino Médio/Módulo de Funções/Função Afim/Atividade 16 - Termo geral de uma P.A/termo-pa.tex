\documentclass[10 pt,usenames,dvipsnames, oneside]{article}
\usepackage{../../../modelo-ensino-medio}



\begin{document}

\begin{center}
  \begin{minipage}[l]{3cm}
\includegraphics[width=2cm]{logo}    
\end{minipage}\hfill
\begin{minipage}[r]{.8\textwidth}
 {\Large \scshape Atividade: Termo geral de uma P.A.}  
\end{minipage}
\end{center}
\vspace{.2cm}

\ifdefined\prof
\begin{objetivos}
\item \phantom{a}
\end{objetivos}

\begin{goals}
\begin{enumerate}
\item Objetivos específicos
\item Introduzir a nomenclatura “termo geral” para uma P.A.
\item Identificar os elementos principais dessa fórmula: $a_n, a_1,r$
\end{enumerate}

\tcblower

\begin{itemize}
\item Evite estimular a memorização dessa fórmula. Procure explorar o significado da relação entre $a_n$ e $a_1$.
\item As expressões do termo geral podem ficar em função de $n$ ou de $n-1$.
\item Considere extrapolar a fórmula para a relação com outros termos diferentes do primeiro:
\begin{equation*}
a_n=a_k+(n-k)r.
\end{equation*}

\end{itemize}

\end{goals}

\bigskip
\begin{center}
{\large \scshape Atividade}
\end{center}
\fi

A função afim que relaciona um termo genérico de uma progressão aritmética com o primeiro termo e a razão é comumente chamada de \textbf{fórmula do termo geral} da progressão. Ou seja, para a P.A. $(a_1,a_2,a_3,...)$ de razão $r$ a fórmula do termo geral é
\begin{equation*}
a_n=a_1+(n-1)r
\end{equation*}

Complete a tabela abaixo com as progressões ou as fórmulas dos termos gerais.

\begin{table}[H]
\setlength\tabulinesep{5pt}
\centering
\begin{tabu} to \textwidth{|>{$\displaystyle}c<{$}|>{\centering $}m{2cm}<{$}|>{$}c<{$}|>{$\displaystyle}l<{$}|}
\hline
$\tcolor{P.A.}$ & 
$\tcolor{\makecell{Primeiro\\termo}}$ & 
$\tcolor{Razão}$ & 
$\centering\tcolor{ Termo Geral}$ \tabularnewline
\hline
(a_1,a_2,a_3) & a_1 & r & a_n=1+2(n-1)=2n-1 \\
\hline
(1,3,5,7,9,...) & 1 & 2 & a_n=1+2(n-1)=2n-1 \\
\hline
(2,4,6,8,10,...) & & & \\
\hline
& 3 & -1 & \\
\hline
& & & a_n=10-\frac{n}{5} \\
\hline
\Big(\pi,\frac{5\pi}{4},\frac{9\pi}{4},...\Big) & & & \\
\hline
& 4 & & a_n=2+2n \\
\hline
\end{tabu}
\end{table}

\ifdefined\prof
\begin{solucao}

\begin{table}[H]
\setlength\tabulinesep{5pt}
\centering
\begin{tabu} to \textwidth{|>{$\displaystyle}c<{$}|>{\centering $}m{2cm}<{$}|>{$}c<{$}|>{$\displaystyle}l<{$}|}
\hline
$\tcolor{P.A.}$ & 
$\tcolor{\makecell{Primeiro\\termo}}$ & 
$\tcolor{Razão}$ & 
$\tcolor{\centering Termo Geral}$ \tabularnewline
\hline
(a_1,a_2,a_3) & a_1 & r & a_n=1+2(n-1)=2n-1 \\
\hline
(1,3,5,7,9,...) & 1 & 2 & a_n=1+2(n-1)=2n-1 \\
\hline
(2,4,6,8,10,...) & 2 & 2 & a_n=2+2(n-1)=2n \\
\hline
(3,2,1,0,-1,...)& 3 & -1 & a_n=3-(n-1)=4-n\\
\hline
\Big(\frac{49}{5},\frac{48}{5},\frac{47}{5},\frac{46}{5},\frac{45}{5},...\Big) & \frac{49}{5} & \frac{-1}{5} & a_n=10-\frac{n}{5} \\
\hline
\Big(\pi,\frac{5\pi}{4},\frac{9\pi}{4},...\Big) & \pi & \frac{\pi}{4} &a_n=\pi+\frac{\pi}{4}(n-1)=\frac{3\pi}{4}+\frac{\pi}{r}n \\
\hline
(4,6,8,10,12,...)& 4 & 2 & a_n=2+2n \\
\hline
\end{tabu}
\end{table}

\end{solucao}
\fi

\end{document}