\documentclass[10 pt,usenames,dvipsnames, oneside]{article}
\usepackage{../../../modelo-ensino-medio}



\begin{document}

\begin{center}
  \begin{minipage}[l]{3cm}
\includegraphics[width=2cm]{logo}    
\end{minipage}\hfill
\begin{minipage}[r]{.8\textwidth}
 {\Large \scshape Atividade: Antecipando o pagamento de uma dívida}  
\end{minipage}
\end{center}
\vspace{.2cm}

\ifdefined\prof
\begin{objetivos}
\item \textbf{EM13MAT501} Investigar relações entre números expressos em tabelas para representá-los no plano cartesiano, identificando padrões e criando conjecturas para generalizar e expressar algebricamente essa generalização, reconhecendo quando essa representação é de função polinomial de 1º grau.
\item \textbf{EM13MAT507} Identificar e associar progressões aritméticas (PA) a funções afins de domínios discretos, para análise de propriedades, dedução de algumas fórmulas e resolução de problemas.
\end{objetivos}

\begin{goals}
\begin{enumerate}
\item Relacionar função afim de domínio discreto com progressões aritméticas.
\item Perceber nas sequências apresentadas que a diferença entre termos consecutivos é constante, e portanto elas formam uma P.A.
\end{enumerate}

\tcblower

\begin{itemize}
\item Questione seus estudantes sobre o fato de, apesar da variação ser sempre de \textit{250} reais, em um caso a primeira função (valor a pagar) é decrescente e a segunda função (desconto) é crescente, relacionando com as expressões obtidas no item f).
\item No item d) espera-se que a resposta seja em forma de porcentagem. Caso julgue necessário, aproveite o momento para fazer uma rápida revisão sobre o assunto.
\item Possivelmente os estudantes não estão familizarizados com alguns dos termos apresentados nessa atividade: \textit{\textbf{antecipação de uma dívida, desconto comercial, tempo de antecipação}}. Antes de iniciar a atividade esclareça para eles o significado de cada um desses termos.
\end{itemize}

\end{goals}

\bigskip
\begin{center}
{\large \scshape Atividade}
\end{center}
\fi

O cliente de um banco foi à sua agência de relacionamento para negociar a antecipação de uma dívida no valor de \(R\$ 5.000,00\), que venceria no prazo de 8 meses. Lá, foi informado pelo gerente que a modalidade praticada era a de desconto comercial, no qual o valor a ser descontado é calculado a partir do valor da dívida e é proporcional ao tempo de antecipação. Para o caso em questão, o gerente gerou a seguinte quadro:

\begin{table}[H]
\centering
\begin{tabu} to \textwidth{|l|c|c|c|c|c|}
\hline
\thead
Antecipação (meses) & 0 & 1 & 2 & 3 & 4 \\
\hline
Valor a pagar (R\$) & 5.000 & 4.750 & 4.500 & 4.250 & 4.000 \\
\hline
\end{tabu}
\end{table}

\begin{enumerate}
\item {} 
Observando o padrão de desconto, complete a tabela até o 8º mês de antecipação.

\item {} 
Descreva a variação sofrida pelo valor a pagar à medida que aumenta o tempo de antecipação. O que se pode afirmar sobre a variação observada no valor a pagar entre dois meses consecutivos?

\item {} 
Descreva a variação sofrida pelo valor do desconto à medida que aumenta o tempo de antecipação. O que se pode afirmar sobre a variação observada no valor de desconto entre dois meses consecutivos?

\item {} 
Qual é, então, a taxa mensal de desconto comercial praticada pelo banco?

\item {} 
No plano cartesiano a seguir estão representados os pares ordenados \((n,D(n))\) em que \(n\) é o tempo de antecipação e \(D(n)\) o valor do desconto correspondente. Represente nele os pontos que correspondem aos pares ordenados \((n,V(n))\) em que \(V(n)\) é o valor a pagar no tempo \(n\).

\begin{figure}[H]
\centering

\begin{tikzpicture}[xscale=2, scale=.5]
\draw [help lines, ystep=.5] (0,0) grid (9,11);
\draw [-{>[length=2mm,width=3mm]},thick] (0,0) -- (9,0);
\draw [-{>[length=2mm,width=3mm]},thick] (0,0) -- (0,11);
\foreach \x in {1,...,8}
{
\node [below,scale=.75] at (\x,0) {\x};
};
\node [below,left,scale=.75] at (0,0) {0};

\foreach \x/\y in {1/500,2/1000,3/1500,4/2000,5/2500,6/3000,7/3500,8/4000,9/4500,10/5000}
{
\node [left,scale=.75] at (0,\x) {\y};
};

\foreach \x in {0,1,2,3,4,5,6,7,8}{
	\node [fill, circle, inner sep=1pt] at (\x,\x*.5) {};
	}

\end{tikzpicture}
\end{figure}

\end{enumerate}

\ifdefined\prof
\begin{solucao}
\begin{enumerate}
\item \adjustbox{valign=t}
{
\begin{tabu} to \textwidth{|l|c|c|c|c|}
\hline
\thead
Antecipação (meses) & 5 & 6 & 7 & 8 \\
\hline
Valor a pagar (R\$) & 3.750 & 3.500 & 3.250 & 3.000 \\
\hline
\end{tabu}
}

\item A medida que o tempo aumenta, o valor a pagar diminui. Entre dois meses consecutivos a variação do valor a pagar é de R\$ $250,00$.

\item A medida que o tempo aumenta, o valor do desconto aumenta. Entre dois meses consecutivos a variação do desconto é de R\$ $250,00$.

\item $5\%$ do valor da dívida.

\item \adjustbox{valign=t}
{
\begin{tikzpicture}[xscale=2, scale=.5]
\draw [help lines, ystep=.5] (0,0) grid (9,11);
\draw [-{>[length=2mm,width=3mm]},thick] (0,0) -- (9,0);
\draw [-{>[length=2mm,width=3mm]},thick] (0,0) -- (0,11);
\foreach \x in {1,...,8}
{
\node [below,scale=.75] at (\x,0) {\x};
};
\node [below,left,scale=.75] at (0,0) {0};

\foreach \x/\y in {1/500,2/1000,3/1500,4/2000,5/2500,6/3000,7/3500,8/4000,9/4500,10/5000}
{
\node [left,scale=.75] at (0,\x) {\y};
};

\foreach \x in {0,1,2,3,4,5,6,7,8}{
	\node [fill, circle, inner sep=1pt] at (\x,\x*.5) {};
	\node [fill, circle, inner sep=1pt] at (\x,10-\x*.5) {};
};

\end{tikzpicture}
}

\end{enumerate}
\end{solucao}
\fi

\end{document}