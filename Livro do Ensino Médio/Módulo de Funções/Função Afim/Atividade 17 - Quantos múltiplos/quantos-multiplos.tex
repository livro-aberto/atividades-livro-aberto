\documentclass[10 pt,usenames,dvipsnames, oneside]{article}
\usepackage{../../../modelo-ensino-medio}



\begin{document}

\begin{center}
  \begin{minipage}[l]{3cm}
\includegraphics[width=2cm]{logo}    
\end{minipage}\hfill
\begin{minipage}[r]{.8\textwidth}
 {\Large \scshape Atividade: Quantos múltiplos?}  
\end{minipage}
\end{center}
\vspace{.2cm}

\ifdefined\prof
\begin{objetivos}
\item \textbf{EM13MAT507} Identificar e associar progressões aritméticas (PA) a funções afins de domínios discretos, para análise de propriedades, dedução de algumas fórmulas e resolução de problemas.
\end{objetivos}

\begin{goals}
\begin{enumerate}
\item Utilizar a mesma estratégia da atividade “Quadros na parede” em um contexto matemático abstrato.
\end{enumerate}

\tcblower
\begin{itemize}
\item Estimule que os estudantes descrevam suas estratégias e critiquem o raciocínio uns dos outros.
\item Caso alguém comece enumerando os múltiplos, peça que tente desenvolver uma estratégia diferente.

\end{itemize}

\end{goals}

\bigskip
\begin{center}
{\large \scshape Atividade}
\end{center}
\fi

Os múltiplos de um número inteiro positivo $m$, quando representados na reta numérica ficam igualmente espaçados entre si.
\begin{enumerate}
\item Quantos múltiplos de 13 há entre 100 e 200? Explique sua estratégia.
\item Quantos múltiplos de 7 há entre 1000 e 2000?
\end{enumerate}


\ifdefined\prof
\begin{solucao}
\begin{enumerate}
\item O primeiro e o último múltiplos de $13$ dentro do intervalo dado são $104$ e $195$, portanto a relação$ 195=104+13(n-1)$, fornece $n=8$.
\item O primeiro e o último múltiplos de $7$ dentro do intervalo dado são $1001$ e $1995$, portanto a relação $1995=1001+7(n-1)$, fornece $n=143$.


\end{enumerate}
\end{solucao}
\fi

\end{document}