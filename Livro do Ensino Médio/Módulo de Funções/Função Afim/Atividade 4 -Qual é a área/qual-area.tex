\documentclass[10 pt,usenames,dvipsnames, oneside]{article}
\usepackage{../../../modelo-ensino-medio}



\begin{document}

\begin{center}
  \begin{minipage}[l]{3cm}
\includegraphics[width=2cm]{logo}    
\end{minipage}\hfill
\begin{minipage}[r]{.8\textwidth}
 {\Large \scshape Atividade: Qual é a área?}  
\end{minipage}
\end{center}
\vspace{.2cm}

\ifdefined\prof
\begin{objetivos}
\item \textbf{EM13MAT401} Converter representações algébricas de funções polinomiais de 1º grau em representações geométricas no plano cartesiano, distinguindo os casos nos quais o comportamento é proporcional, recorrendo ou não a softwares ou aplicativos de álgebra e geometria dinâmica.
\end{objetivos}

\begin{goals}
\begin{enumerate}
\item Em um círculo dado, reconhecer a relação de dependência entre a medida do ângulo central e a medida da área do setor circular.

\item Inferir que a medida da área do setor é diretamente proporcional a medida do ângulo central.

\item Determinar a medida da área do setor circular dada a medida do ângulo central e vice-versa.

\end{enumerate}

\tcblower

\begin{itemize}
\item Nos dois primeiros itens procure incentivar os alunos a resolver o problema utilizando apenas processos mentais, ou ao menos na hora de discutir a solução, utilize argumentações que valorizem a estimativa, tais como:

\begin{itemize}
\item Como $\frac{1}{4}$ de $20$ é $5$, e $14$ é um valor um pouco menor que $34$ de $20$ então o setor circular de área $14$ tem que ser menor do que $34$ do círculo.

\item Ao analisar as opções descartamos a opção “2” por ser uma região menor que $34$ da área do círculo, descartamos também a opção “3” por se tratar de um valor entre $15$ e $20$ mais próximo de $15$, logo a resposta correta está representada pela opção “1”.
\end{itemize}

\item Nos itens c) e d), discuta com a turma a importância de ter sido apresentado a medida do ângulo.
\end{itemize}
\end{goals}

\bigskip
\begin{center}
{\large \scshape Atividade}
\end{center}
\fi

Caso tenha disponibilidade, sugerimos o uso da construção GeoGebra disponível \href{https://www.geogebra.org/m/Xjjym4e7}{neste link}, que é a versão eletrônica dessa atividade.

\begin{figure}[H]
\centering

\noindent\includegraphics[width=100bp]{{codigo-qual-area}.png}
\end{figure}



\begin{enumerate}
\item {}
Cada círculo representado a seguir tem área total \(20\). Um   dos setores circulares destacados em verde nesses círculos tem área \(14\). Qual é esse setor?

\begin{figure}[H]
\centering

  \begin{tikzpicture}
    \draw(0,0)circle(1);
    \draw[fill=\currentcolor!80] (1,0)--(0,0) --(210:1) arc (210:0:1);
    \node at(-1,1){1)};
\end{tikzpicture}\quad\quad\quad
 \begin{tikzpicture}
    \draw(0,0)circle(1);
    \draw[fill=\currentcolor!80] (1,0)--(0,0) --(250:1) arc (250:0:1);
    \node at(-1,1){2)};
 \end{tikzpicture}\quad\quad\quad
      \begin{tikzpicture}
    \draw(0,0)circle(1);
    \draw[fill=\currentcolor!80] (1,0)--(0,0) --(270:1) arc (270:0:1);
    \node at(-1,1){3)};
 \end{tikzpicture}
\end{figure}

\item{}
Agora, um dos setores circulares em verde tem área \(18\). Qual é esse setor?

\begin{figure}[H]
\centering


\begin{tikzpicture}
  \draw(0,0)circle(1);
  \draw[fill=\currentcolor!80] (1,0)--(0,0) --(330:1) arc (330:0:1);
  \node at(-1,1){1)};
 \end{tikzpicture}\quad\quad\quad
      \begin{tikzpicture}
  \draw(0,0)circle(1);
  \draw[fill=\currentcolor!80] (1,0)--(0,0) --(250:1) arc (250:0:1);
  \node at(-1,1){2)};
 \end{tikzpicture}\quad\quad\quad
      \begin{tikzpicture}
  \draw(0,0)circle(1);
  \draw[fill=\currentcolor!80] (1,0)--(0,0) --(300:1) arc (300:0:1);
  \node at(-1,1){3)};
\end{tikzpicture}

\end{figure}

 \item{}
Explique a estratégia matemática que você utilizou para resolver os itens anteriores? Dentre os setores circulares apresentados a seguir, um deles tem área \(7\). Aplique sua estratégia para determinar qual é esse setor.

\begin{figure}[H]
\centering


 \begin{tikzpicture}

   \draw(0,0)circle(1);
   \draw[fill=\currentcolor!80] (1,0)--(0,0) --(110:1) arc (110:0:1);
   \node at(-1,1){1)};
 \end{tikzpicture}\quad\quad\quad
      \begin{tikzpicture}
        \draw(0,0)circle(1);\draw[fill=\currentcolor!80] (1,0)--(0,0) --(126:1) arc (126:0:1);\node at((-1,1){2)};
         \end{tikzpicture}\quad\quad\quad
      \begin{tikzpicture}
        \draw(0,0)circle(1);
        \draw[fill=\currentcolor!80] (1,0)--(0,0) --(142:1) arc (142:0:1);
        \node at(-1,1){2)};
      \end{tikzpicture}
 \end{figure}
      
\item{}
Possivelmente você encontrou alguma dificuldade para determinar a resposta correta no item anterior. Que tal acrescentarmos uma informação a mais para ajudar na decisão?

\begin{figure}[H]
\centering


\begin{tikzpicture}
  \draw(0,0)circle(1);
  \draw[fill=\currentcolor!80] (1,0)--(0,0) --(110:1) arc (110:0:1);
  \node at(-1,1){1};
  \draw[atento] (.2,0) arc (0:110:.2);
  \node at(.5,.35){\small $ 110^\circ$};
   \end{tikzpicture}\quad\quad\quad
      \begin{tikzpicture}
        \draw(0,0)circle(1);
        \draw[fill=\currentcolor!80] (1,0)--(0,0) --(126:1) arc (126:0:1);
        \node at(-1,1){2)};\draw[atento] (.2,0) arc (0:126:.2);
        \node at(.5,.35){\small $126^\circ$};
 \end{tikzpicture}\quad\quad\quad
      \begin{tikzpicture}
        \draw(0,0)circle(1);
        \draw[fill=\currentcolor!80] (1,0)--(0,0) --(142:1) arc (142:0:1);
        \node at(-1,1){23)};
        \draw[atento] (.2,0) arc (0:142:.2);
        \node at(.5,.35){\small $ 142^\circ$};
 \end{tikzpicture}
 \end{figure}

\item{}
E agora? Como você usou a medida do ângulo que determina o setor circular para ajudar no cálculo da área? Vamos fazer mais uma vez! Um dos setores apresentados a seguir tem área \(4\). Determine esse setor.

\begin{figure}[H]
\centering


 \begin{tikzpicture}
\draw(0,0)circle(1);\draw[fill=\currentcolor!80] (1,0)--(0,0) --(72:1) arc (72:0:1);\node at(-1,1){1}; \draw[atento] (.2,0) arc (0:72:.2);\node at(.55,.22){\small $ 72^\circ$}; \end{tikzpicture}\quad\quad\quad
      \begin{tikzpicture}
\draw(0,0)circle(1);\draw[fill=\currentcolor!80] (1,0)--(0,0) --(60:1) arc (60:0:1);\node at(-1,1){2)};\draw[atento] (.2,0) arc (0:60:.2);\node at(.55,.25){\small $ 60^\circ$}; \end{tikzpicture}\quad\quad\quad
      \begin{tikzpicture}
\draw(0,0)circle(1);\draw[fill=\currentcolor!80] (1,0)--(0,0) --(45:1) arc (45:0:1);\node at(-1,1){3)};\draw[atento] (.2,0) arc (0:45:.2);\node at(.55,.25){\small $ 45^\circ$};
 \end{tikzpicture}%\label{fig-setor5}
\end{figure}

\item{}
Determine a função que relaciona a área do setor circular com o seu ângulo central, especificando seu domínio.

\end{enumerate}

\ifdefined\prof
\clearpage
\begin{solucao}
\begin{enumerate}
\item 2)
\item 1)
\item Uma possível resposta seria: sendo a área total do círculo igual a $20$, então $14$ do círculo equivale a uma área 5. No entanto, como as áreas destacadas nos itens apresentados estão muito próximas esse critério não nos permite concluir com exatidão qual seria a resposta correta, que no caso é o item 2).
\item 2)
\item Fazendo uma regra de três, item 1).
\item $S:[0,360]\rightarrow\R$ em que $S(x)=\frac{x}{18}$
\end{enumerate}
\end{solucao}
\fi

\end{document}