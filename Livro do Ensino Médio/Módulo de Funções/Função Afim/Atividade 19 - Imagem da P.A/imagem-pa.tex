\documentclass[10 pt,usenames,dvipsnames, oneside]{article}
\usepackage{../../../modelo-ensino-medio}



\begin{document}

\begin{center}
  \begin{minipage}[l]{3cm}
\includegraphics[width=2cm]{logo}    
\end{minipage}\hfill
\begin{minipage}[r]{.8\textwidth}
 {\Large \scshape Atividade: Imagem da P.A.}  
\end{minipage}
\end{center}
\vspace{.2cm}

\ifdefined\prof
\begin{objetivos}
\item \textbf{EM13MAT507} Identificar e associar progressões aritméticas (PA) a funções afins de domínios discretos, para análise de propriedades, dedução de algumas fórmulas e resolução de problemas.
\end{objetivos}

\begin{goals}
\begin{enumerate}
\item Deduzir que as funções afins preservam progressões aritméticas.
\item Reconhecer a relação entre as razões das P.A. e a taxa de variação das funções afins.
\end{enumerate}

\tcblower

\begin{itemize}
\item Verifique se as conjecturas do item (e) estão formuladas de maneira precisa, mesmo que sejam textuais.
\item Como a demonstração desse fato é simples, conduza os estudantes para a justificativa/demonstração.
\end{itemize}

\end{goals}

\bigskip
\begin{center}
{\large \scshape Atividade}
\end{center}
\fi

Considere as funções afins $f,g$ definidas por $f(x)=3x+1$ e $g(x)=-5x+2$.
\begin{enumerate}
\item Complete a tabela abaixo com as imagens pedidas

\begin{table}[H]
\centering
\begin{tabu} to \textwidth{|>{$}c<{$}|*{9}{c|}}
\hline
\cellcolor{\currentcolor!80}\textcolor{white}{\bm{x}} & & & & & & & & & \\
\hline
\cellcolor{\currentcolor!80}\textcolor{white}{\bm{f(x)}} & & & & & & & & & \\
\hline
\cellcolor{\currentcolor!80}\textcolor{white}{\bm{g(x)}} & & & & & & & & & \\
\hline
\end{tabu}
\end{table}
\item Qual a razão da P.A. da primeira linha da tabela?
\item As imagens pela função $f$ formam tabém uma P.A.? Caso positivo, qual a razão?
\item E as imagens pela função $g$? Caso positivo, qual a razão?
\item Faça uma conjectura sobre o que acontece com as imagens de uma P.A. por uma função afim.
\end{enumerate}

\ifdefined\prof
\begin{solucao}
\begin{enumerate}
\item \adjustbox{valign=t}
{
\begin{tabu} to \textwidth{|>{$}c<{$}|*{9}{c|}}
\hline
\cellcolor{\currentcolor!80}\textcolor{white}{\bm{x}} & 2 & 5 & 8 & 11 & 14 & 17 & 20 & 23 & 26 \\
\hline
\cellcolor{\currentcolor!80}\textcolor{white}{\bm{f(x)}} & 7 & 16 & 25 & 34 & 43 & 52 & 61 & 70 & 79 \\
\hline
\cellcolor{\currentcolor!80}\textcolor{white}{\bm{g(x)}} & -8 & -23 & -38 & -53 & -68 & -83 & -98 & -113 & -128 \\
\hline
\end{tabu}
}

\item Razão $3$.
\item Sim, uma P.A. de razão $9$.
\item Sim, uma P.A. de razão $-15$.
\item As imagens de uma P.A. por uma função afim também formam uma P.A..
\end{enumerate}
\end{solucao}
\fi

\end{document}