\documentclass[10 pt,usenames,dvipsnames, oneside]{article}
\usepackage{../../../modelo-ensino-medio}



\begin{document}

\begin{center}
  \begin{minipage}[l]{3cm}
\includegraphics[width=2cm]{logo}    
\end{minipage}\hfill
\begin{minipage}[r]{.8\textwidth}
 {\Large \scshape Atividade: Taxa de câmbio}  
\end{minipage}
\end{center}
\vspace{.2cm}

\ifdefined\prof
\begin{objetivos}
\item \phantom{a}
\end{objetivos}

\begin{goals}
\begin{enumerate}
\item Utilizar a taxa de câmbio fornecida para realizar a conversão do valor dado em moeda estrangeira para o valor correspondente em reais.

\item Obter a partir das informações fornecidas a função linear que converte dólar americano em reais.

\end{enumerate}

\tcblower

\begin{enumerate}
\item É bastante provável que no item c) os estudantes apresentem o seguinte raciocínio: se 1 dólar americano equivale a R\$ $3{,}20$ reais então $x$ dólares americanos irão corresponder a $y$ reais, isto é, $y=3,20\cdot x$. Em analogia ao que foi feito anteriormente, é importante chamar atenção de que se $y=f(x)$ é a função que fornece a quantia equivalente em reais a x dólares americanos, como as grandezas envolvidas são diretamente proporcionais e $f(1)=3{,}20$, então $f(x)=x\cdot f(1)$ e portanto $f(x)=3{,}20{,}x$.

\item Ainda no item c) o questionamente apresentado sobre o domínio da função tem como objetivo levar a uma reflexão de que na prática não faz sentido, por exemplo, converter $\sqrt{5}$ dólares americanaos para reais.
\end{enumerate}

\end{goals}

\bigskip
\begin{center}
{\large \scshape Atividade}
\end{center}
\fi

Segundo o site do Banco Central do Brasil (\url{http://www.bcb.gov.br/pre/bc\_atende/port/taxCam.asp}), a \emph{taxa de câmbio} é o preço de uma moeda estrangeira medido em unidades ou frações (centavos) da moeda nacional. Em um determinado dia as taxas de câmbio do dólar americano e do euro eram respectivamente R\$ $3{,}20$ e R\$ $4{,}00$.
\begin{enumerate}
\item {} 
Nesse mesmo dia você deseja comprar \(100\) dólares. Qual seria o valor em reais necessário para realizar essa compra?

\item {} 
Para adquirir nesse mesmo dia \(200\) euros, qual o valor em reais deverá ser desembolsado?

\item {} 
A partir da taxa praticada nesse dia, apresente uma função que converta dólar americano para reais. Qual o conjunto domínio mais adequado a ser considerado para essa função? Justifique.

\item {} 
Com a taxa de câmbio que está sendo praticada nesse dia, quantos dólares americanos podem ser comprados com R\$ $2000{,}00$. Com os mesmos R\$ $2000{,}00$, quantos euros podem ser adquiridos?

\end{enumerate}

\ifdefined\prof
\begin{solucao}
\begin{enumerate}

\item A partir da taxa de câmbio fornecida sabemos que 1 dólar americano é equivalente a R\$ $320{,}00$, e portanto, para comprar 100 dólares americanos serão necessários R\$ $320{,}00$. \si{kg}

\item Como nesse dia 1 euro é equivalente a R\$ $4{,}00$, então será necessário desembolsar R\$ $800{,}00$ para a compra de $200$ euros.

\item Vamos chamar de $y=f(x)$ a função que fornece a quantia equivalente em reais a x dólares americanos. Como as grandezas envolvidas são diretamente proporcionais e $f(1)=3{,}20$ (veja que isso é a tradução, usando a linguagem de função, de que 1 dólar americano equivale a R\$ $3{,}20$), então $f(x)=x\cdot f(1)$ e portanto $f(x)=3{,}20\cdot x$. Como na prática não existem quantias irracionais de dólares americanos e de reais, devemos considerar $f:\Q\rightarrow\Q$.

\item Utilizando a função obtida no item anterior vemos que R\$ $2000,00$ equivalem a $x\displaystyle=\frac{2000}{3}{,}20=625$ reais. Raciocinando de forma análoga obtemos que com R\$ $2000{,}00$ poderão ser adquiridos $\displaystyle\frac{2000}{4}=500$ euros.
\end{enumerate}
\end{solucao}
\fi

\end{document}