\documentclass[10 pt,usenames,dvipsnames, oneside]{article}
\usepackage{../../../modelo-ensino-medio}



\begin{document}

\begin{center}
  \begin{minipage}[l]{3cm}
\includegraphics[width=2cm]{logo}    
\end{minipage}\hfill
\begin{minipage}[r]{.8\textwidth}
 {\Large \scshape Atividade: Temperatura controlada}  
\end{minipage}
\end{center}
\vspace{.2cm}

\ifdefined\prof
\begin{objetivos}
\item \textbf{EM13MAT401} Converter representações algébricas de funções polinomiais de 1º grau em representações geométricas no plano cartesiano, distinguindo os casos nos quais o comportamento é proporcional, recorrendo ou não a softwares ou aplicativos de álgebra e geometria dinâmica.
\end{objetivos}

\begin{goals}
\begin{enumerate}
\item Obter a expressão algébrica de uma função afim a partir de dois pontos dados no plano cartesiano.
\item Interpretar o ponto de interseção entre as funções que modelam a situação apresentada
\end{enumerate}

\end{goals}


\bigskip
\begin{center}
{\large \scshape Atividade}
\end{center}
\fi

Num laboratório, um químico conseguiu controlar a variação de temperatura de dois compostos. A variação de ambos está associada às funções afins \(f\) e \(g\), de maneira que a taxa de variação das temperaturas de cada um dos compostos seja constante. Observe o gráfico, onde o eixo das ordenadas indica a temperatura (em graus Celsius) de cada composto em função do tempo \(t\), em minutos. O gráfico da figura a seguir modela a situação:

\begin{figure}[H]
\centering

\begin{tikzpicture}[yscale=.75]
\tikzstyle{ponto}=[fill,circle,inner sep=1.25pt]

\node [ponto] at (0,-4) {};
\node [ponto] at (4,0) {};
\node [ponto] at (0,2) {};
\node [ponto] at (2,0) {};

\draw [->] (-.5,0) -- (7,0) node [below] {$t$ (minutos)};
\draw [->] (0,-5.5) -- (0,4) node [right] {T ($^{\circ}$C)};

\draw [thick] (0,-4) -- (7,3) node [above left] {$f$};

\draw [thick]  (0,2) -- (7,-5) node [above] {$g$};
\end{tikzpicture}
\end{figure}


O gráfico da função \(f\) passa pelos pontos \(A=(0,-4)\) e \(B=(4,0)\), indicando que o composto associado à \(f\) está com uma temperatura de \(-4\,^{\circ}\mathrm{C}\) no início da medição e após \(4\) minutos a temperatura atinge \(0\,^{\circ}\mathrm{C}\).

O gráfico da função \(g\) passa pelos pontos \(C=(0,2)\) e \(D=(2,0)\), indicando que o composto associado à \(g\) está com uma temperatura de \(2\,^{\circ}\mathrm{C}\) no início da medição e após \(2\) minutos a temperatura atinge \(0\,^{\circ}\mathrm{C}\).

Com base nas informações do texto responda as perguntas a seguir:
\begin{enumerate}
\item {} 
Determine as expressões das funções afins \(f\) e \(g\).

\item {} 
A temperatura do composto associado à função \(f\) estão aumentando ou diminuindo? E do composto associado à função \(g\)?

\item {} 
Em quanto tempo cada composto atinge a temperatura de

\begin{enumerate}
\item \(1\,^{\circ}\mathrm{C}\)?

\item \(-3\,^{\circ}\mathrm{C}\)?

\item \(-8\,^{\circ}\mathrm{C}\)?

\item \(10\,^{\circ}\mathrm{C}\)?
\end{enumerate}
\item {} 
Após quantos minutos os dois compostos terão a mesma temperatura? E que temperatura é essa?

\end{enumerate}

\ifdefined\prof
\begin{solucao}
\begin{enumerate}

\item Como $f$ intersecta o eixo das ordenadas no ponto $A=(0,−4)$ temos que $b=−4$; substituindo o ponto $B=(4,0)$, ou seja, fazendo $f(4)=0$ encontramos $a=1$. Do mesmo modo, $g$ intersecta o eixo das ordenadas no ponto $C=(0,2)$, logo temos que $n=2$; substituindo o ponto $D=(2,0)$, ou seja, fazendo $f(2)=0$ encontramos $m=−1$.

\item Observando os gráficos temos que: a temperatura do composto associado à função $f$ está aumentando, e a temperatura do composto associado à função g está diminuindo.
\begin{enumerate}
\item Fazendo $f(t)=1$ encontramos $t=5$, ou seja o composto associado à função afim $f$, atinge $1$ $^{\circ}$C após $5$ minutos. E fazendo $g(t)=1$ encontramos $t=1$, ou seja o composto associado à função afim $g$, atinge $1$ $^{\circ}$C após $1$ minuto.

\item Fazendo $f(t)=−3$ encontramos $t=1$, ou seja o composto associado à função afim $f$, atinge $−3$ $^{\circ}$C após $1$ minuto. E fazendo $g(t)=−3$ encontramos $t=5$, ou seja o composto associado à função afim $g$, atinge $−3$ $^{\circ}$C após $5$ minutos.

\item Fazendo $f(t)=−8$ encontramos $t=−4$, ou seja o composto associado à função afim $f$, nunca atingirá essa temperatura, já que $f$ é sempre maior ou igual a $−4$ $^{\circ}$C. E fazendo $g(t)=−8$ encontramos $t=10$, ou seja o composto associado à função afim $g$, atinge $−8$ $^{\circ}$C após $10$ minutos.

\item Fazendo $f(t)=10$ encontramos $t=14$, ou seja o composto associado à função afim $f$, atinge $10$ $^{\circ}$C após $14$ minutos. E fazendo $g(t)=10$ encontramos $t=−8$, ou seja o composto associado à função afim $g$, nunca atingirá ess a temperatura, já que $g$ é sempre menor ou igual a $2$ $^{\circ}$C.

\end{enumerate}
\item Basta fazermos $f(t)=g(t)$, ou seja $t−4=−t+2$, resolvendo encontramos $t=3$ minutos, e a temperatura é igual a $f(3)=g(3)=−1$ $^{\circ}$C. Portanto, os dois compostos atigem $−1$ $^{\circ}$C após $3$ minutos de observação.


\end{enumerate}
\end{solucao}
\fi

\end{document}