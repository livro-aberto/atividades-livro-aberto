\documentclass[10 pt,usenames,dvipsnames, oneside]{article}
\usepackage{../../../modelo-ensino-medio}



\begin{document}

\begin{center}
  \begin{minipage}[l]{3cm}
\includegraphics[width=2cm]{logo}    
\end{minipage}\hfill
\begin{minipage}[r]{.8\textwidth}
 {\Large \scshape Atividade: Comprando vinho}  
\end{minipage}
\end{center}
\vspace{.2cm}

\ifdefined\prof
\begin{objetivos}
\item \textbf{EM13MAT302} Construir modelos empregando as funções polinomiais de 1º ou 2º graus, para resolver problemas em contextos diversos, com ou sem apoio de tecnologias digitais.
\end{objetivos}

\begin{goals}
\begin{enumerate}
\item Compreender função afim por partes.
\end{enumerate}

\tcblower
\begin{itemize}
\item Chame atenção para o significado das bolas “abertas”{}e ”fechadas”, representadas na figura da solução, respectivamente pelas bolas brancas e pretas.
\item A experiência tem mostrado que os estudantes apresentam dificuldade em perceber que a representação gráfica que ilustra a situação da atividade inicia a partir do ponto $(0,5)$.
\end{itemize}
\end{goals}

\bigskip
\begin{center}
{\large \scshape Atividade}
\end{center}
\fi

Em uma vinícola podemos comprar vinho por litro. Neste caso, o vinho é colocado em garrafões com capacidade de \(5\) litros. O vinho é vendido a R\$ \(10,00\) por litro e cada garrafão é vendido a R\$ \(5,00\).
\begin{enumerate}
\item {} 
Calcule o preço que um cliente deverá pagar por \(2\) litros, por \(5\) litros e por \(7\) litros. Explique seus cálculos.

\item {} 
Determine uma expressão para o preço \(p\) (em reais) em função do volume \(x\) (expresso em litros) de vinho adquirido. Considere \(x\) compreendido entre \(0\) e \(15\).

\item {} 
Trace a curva que representa a função \(p\) no plano cartesiano. Utilize a escala de \(1\) cm para \(1\) litro no eixo das abscissas e \(1\) cm para \(10\) reais nas ordenadas.

\end{enumerate}

\ifdefined\prof
\begin{solucao}
\begin{enumerate}
\item R\$ $25{,}00$ por $2$ litros, R\$ $55{,}00$ por $5$ litros e R\$ $80{,}00$ pelos $7$ litros

\item 
\begin{align*}
p(x)&=10x+5,\text{ se }x\in(0,5]\\
p(x)&=10x+10,\text{ se }x\in (5,10]\\
p(x)&=10x+15, \text{ se }x\in(10,15]
\end{align*}
\clearpage
\item \adjustbox{valign=t}
{
\begin{tikzpicture}[scale=.35]
\tikzstyle{ponto}=[circle, minimum size=7pt, inner sep=0, draw=black, fill=black, shift only, label={}]
\draw [help lines, secundario!20] (0,0) grid (38,34);
\draw [thick, <->] (38,0) -- (0,0) -- (0, 34);
\node [below] at (34.5,-0.5) {quantidade (litros)};
\node [above, rotate=90] at (-1, 28) { pre\c{c}o (reais)};
\node [below] at (0,0) {0};
\node [below] at (10,0) {5};
\node [below] at (20,0) {10};
\node [below] at (30,0) {15};
\node [left] at (0,1) {5};
\node [left] at (0,11) {55};
\node [left] at (0,12) {60};
\node [left] at (0,22) {110};
\node [left] at (0,23) {115};
\node [left] at (0,33) {165};	
\draw [thick, dashed] (10,0) -- (10,11) -- (0,11);
\draw [thick, dashed] (0,12) -- (10,12) ;
\draw [thick, dashed] (20,0) -- (20,22) -- (0,22);
\draw [thick, dashed] (0,23) -- (20,23) ;
\draw [thick, dashed] (30,0) -- (30,33) -- (0,33);
\node [ponto, color=primario] at (10,11) {};
\node [ponto, color=primario] at (20,22) {};
\node [ponto, color=primario] at (30,33) {};
\draw [very thick, color=primario] (0,1) -- (10,11);
\node [ponto, draw=primario, fill=white] at (0,1) {};
\draw [very thick, color=primario] (10,12) -- (20,22);
\node [ponto, draw=primario, fill=white] at (10,12) {};
\draw [very thick, color=primario] (20,23) -- (30,33);
\node [ponto, draw=primario, fill=white] at (20,23) {};
\end{tikzpicture}
}
\end{enumerate}
\end{solucao}
\fi

\end{document}