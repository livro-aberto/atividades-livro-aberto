\documentclass[10 pt,usenames,dvipsnames, oneside]{article}
\usepackage{../../../modelo-ensino-medio}



\begin{document}

\begin{center}
  \begin{minipage}[l]{3cm}
\includegraphics[width=2cm]{logo}    
\end{minipage}\hfill
\begin{minipage}[r]{.8\textwidth}
 {\Large \scshape Atividade: Teor de álcool sanguíneo}  
\end{minipage}
\end{center}
\vspace{.2cm}

\ifdefined\prof
\begin{objetivos}
\item \phantom{a}
\end{objetivos}

\begin{goals}
\begin{enumerate}
\item Conjecturar que taxa de variação média de uma função linear qualquer é a mesma para qualquer intervalo.
\end{enumerate}

\tcblower
\begin{itemize}
\item A atividade aborda assuntos relacionados a temas transversais, como saúde e consumo de álcool. Sugerimos que procure fazer um trabalho colaborativo com os professores de Biologia, Química e de Geografia para ampliar a discussão com os alunos em questões como os processos bioquímicos do metabolismo do álcool, ou mesmo em questões sobre o a relação entre álcool e direção. No site referenciado há informações adicionais que podem enriquecer a discussão.

\item Caso necessário, faça uma revisão sobre taxa de variação média, vista no capítulo de funções.
\end{itemize}
\end{goals}

\bigskip
\begin{center}
{\large \scshape Atividade}
\end{center}
\fi

De acordo com o site \href{https://pt.wikihow.com/Calcular-o-N\%C3\%ADvel-de-\%C3\%81lcool-no-Sangue}{wikiHow} o Teor Alcoólico Sanguíneo, ou TAS, é a medida da proporção de álcool no sangue de uma pessoa. Um TAS de \(0{,}08\) indica que há \(80\) mg de álcool por \(100\) ml de sangue. O álcool é absorvido de forma diferente pelos homens e pelas mulheres. O corpo masculino geralmente tem mais água (\(61\%\) \emph{versus} \(52\%\)) e, portanto, dilui melhor o álcool, gerando TAS mais baixos.

O TAS é proporcional ao número de doses de bebida consumidas, de maneira que para um homem de \(75\) kg, a função linear \(h(x)\) que relaciona o TAS com o número de doses \(x\) de bebida é dada pela expressão
\begin{equation*}
\begin{split}h(x)=0{,}0205 \cdot x.\end{split}
\end{equation*}
Para uma mulher que pesa \(60\) kg, a mesma relação é dada pela função linear
\begin{equation*}
\begin{split}m(x)=0{,}0307 \cdot x.\end{split}
\end{equation*}\begin{enumerate}
\item {} 
Complete a tabela a seguir que relaciona os valores de \(h(x)\) e de \(m(x)\) correspondentes a valores inteiros de \(x\), de \(0\) a \(5\).

\begin{table}[H]
\centering
\begin{tabu} to \textwidth{|l|c|c|}
\hline
\thead
\(\bm{x}\) & \(\bm{h(x)}\) & \(\bm{m(x)}\) \\
\hline
0 & & \\
\hline
1 & & \\
\hline
2 & & \\
\hline
3 & & \\
\hline
4 & & \\
\hline
5 & & \\
\hline
\end{tabu}
\end{table}

\item {} 
Calcule, para a função \(h(x)\), as taxas de variação médias nos seguintes intervalos de valores de \(x\):

\begin{enumerate}
\item entre \(x=0\) e \(x=1\);

\item entre \(x=1\) e \(x=3\);

\item entre \(x=2\) e \(x=5\);
\end{enumerate}

\item {} 
Repita o item anterior para a função \(m(x)\) nos intervalos:

\begin{enumerate}
\item entre \(x=2\) e \(x=3\);

\item entre \(x=1\) e \(x=4\);

\item entre \(x=0\) e \(x=5\);
\end{enumerate}

\item {} 
A partir dos itens anteriores, faça uma conjectura sobre as taxas de variação médias de uma função linear qualquer.

\end{enumerate}

\ifdefined\prof
\begin{solucao}
\begin{enumerate}
\item \adjustbox{valign=t}
{
\begin{tabu} to \textwidth{|>{$}c<{$}|>{$}c<{$}|>{$}c<{$}|}
\hline
\rowfont{\color{white}}\rowcolor{\currentcolor!80}
\bm{x} & \bm{h(x)} & \bm{m(x)}\tabularnewline
\hline
0 & 0 & 0 \tabularnewline
\hline
1 & 0{,}0205 & 0{,}3075 \tabularnewline
\hline
2 & 0{,}041 & 0{,}615 \tabularnewline
\hline
3 & 0{,}615 & 0{,}9225 \tabularnewline
\hline
4 & 0{,}082 & 1{,}23 \tabularnewline
\hline
5 & 0{,}1025 &1{,}5375 \tabularnewline
\hline
\end{tabu}

}

\item 
\begin{enumerate}
\item 0{,}0205
\item 0{,}0205
\item 0{,}0205
\end{enumerate}
\item
\begin{enumerate}
\item 0{,}3075
\item 0{,}3075
\item 0{,}3075
\end{enumerate}

\item A conjectura é que a taxa de variação média de uma função linear qualquer deve ser constante.
\end{enumerate}
\end{solucao}
\fi

\end{document}