
\documentclass[10 pt,usenames,dvipsnames, oneside]{article}
\usepackage{../../../modelo-ensino-medio}



\begin{document}

\begin{center}
  \begin{minipage}[l]{3cm}
\includegraphics[width=2cm]{logo}    
\end{minipage}\hfill
\begin{minipage}[r]{.8\textwidth}
 {\Large \scshape Atividade: }  
\end{minipage}
\end{center}
\vspace{.2cm}

\ifdefined\prof
\begin{objetivos}
\item \phantom{a}
\end{objetivos}

\begin{goals}
\begin{enumerate}
\item Perceber, a partir da taxa de variação média constante, que o gráfico de uma função linear está contido em uma reta.
\end{enumerate}

\tcblower

\begin{itemize}
\item No item d) é possível que os estudantes façam direto a “regra de três”; o que está correto. Contudo, peça para que justifiquem o procedimento usando alguma justificativa geométrica envolvendo os pontos do gráfico. A ideia é que, nesse item eles percebam os triângulos semelhantes que podem ser considerados para a solução.
\end{itemize}
\end{goals}

\bigskip
\begin{center}
{\large \scshape Atividade}
\end{center}
\fi

\begin{figure}[H]
\centering

\noindent\includegraphics[width=100bp]{{celular}.jpg}
\end{figure}

O tempo total de recarga da bateria (de \(0\%\) a \(100\%\)) de um determinado modelo de telefone celular é  de \(2\) horas e \(5\) minutos. Supondo que o carregamento ocorre segundo uma taxa constante:

\begin{enumerate}
\item {} 
Faça uma tabela que forneça o percentual de carga na bateria a cada \(25\) minutos, a partir de zero.

\item {} 
Represente em um plano cartesiano os pontos da tabela do item anterior.

\item {} 
Descreva uma estratégia que permita, a partir da representação gráfica obtida no item anterior, determinar o percentual de carga na bateria após \(40\) minutos de carregamento.

\item {} 
Determine a função que modela o carregamento desse modelo de telefone, especificando seus domínio e conjunto imagem.

\item {} 
Qual é a taxa de carregamento desse modelo de telefone celular.
\end{enumerate}


\ifdefined\prof
\begin{solucao}
\begin{enumerate}

\item \adjustbox{valign=t}
{
\begin{tabu} to \textwidth{|c|c|}
\hline
\thead
$t$(min) & Porcentagem de recarga \\
\hline
0 & 0 \\
25 & 20 \\
\hline
50 & 40 \\
\hline 
75 & 60 \\
\hline 
100 & 80 \\
\hline
125 & 100 \\
\hline
\end{tabu}
}


\item\adjustbox{valign=t}
{
\begin{tikzpicture}
\tikzstyle{ponto}=[circle, minimum size=2pt, inner sep=0, draw=black, fill=black, shift only]         
\draw[->, thick](-1,0)--(6,0) node[above]{tempo(min)};
\draw[->, thick](0,-1)--(0,6);
\foreach \x/\y in{25/20, 50/40, 75/60, 100/80, 125/100}
\node[ponto]at(.04*\x, .05*\y){};
\foreach \x/\y in{25/20, 50/40, 75/60, 100/80, 125/100}
\draw[dashed](.04*\x,0)--(.04*\x,.05*\y)--(0,.05*\y) node[left]{\y} node[below] at(.04*\x,0){\x};
\node[ponto]at(0,0){};
\node[below left]at(0,0){0};
\end{tikzpicture}
}

\item A partir da representação dos pontos no plano cartesiano pode-se concluir, usando semelhança de triângulos, que se em 25 minutos a carga na bateria é de $20\%$ então em 40 minutos a carga será de $32\%$.

\item $f(t)=\displaystyle\frac{45}{t}=0{,}8t$, com domínio sendo o conjunto ${0,1,2,...,125}$ e a imagem o conjunto ${0,1,2,...,100}$.

\item A bateria carrega a uma taxa de 0{,}8\% a cada minuto, isto é, $0{,}8$/min.
\end{enumerate}
\end{solucao}
\fi

\end{document}