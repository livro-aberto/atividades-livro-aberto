\documentclass[10 pt,usenames,dvipsnames, oneside]{article}
\usepackage{../../../modelo-ensino-medio}



\begin{document}

\begin{center}
  \begin{minipage}[l]{3cm}
\includegraphics[width=2cm]{logo}    
\end{minipage}\hfill
\begin{minipage}[r]{.8\textwidth}
 {\Large \scshape Atividade: Proporcionalidade na construção de retângulos}  
\end{minipage}
\end{center}
\vspace{.2cm}

\ifdefined\prof
\begin{objetivos}
\item \textbf{EM13MAT401} Converter representações algébricas de funções polinomiais de 1º grau em representações geométricas no plano cartesiano, distinguindo os casos nos quais o comportamento é proporcional, recorrendo ou não a softwares ou aplicativos de álgebra e geometria dinâmica.
\end{objetivos}

\begin{goals}
\begin{enumerate}
\item Levar o estudante a relacionar os conceitos de proporcionalidade e semelhança de figuras e função linear.

\item Construir retângulos que sejam semelhantes a um retângulo dado.

\end{enumerate}

\end{goals}

\bigskip
\begin{center}
{\large \scshape Atividade}
\end{center}
\fi

Considere o retângulo \(R\) abaixo, de lados \(3\) e \(1,5\), e responda as questões propostas.

\begin{figure}[H]
\centering

\begin{tikzpicture}
\tikzstyle{ponto}=[circle, minimum size=2pt, inner sep=0, draw=black, fill=black, shift only]
\draw[thick,black,fill=\currentcolor!80] (0.,0.) -- (3.,0.) -- (3.,1.5) -- (0.,1.5) -- cycle;
\draw (0.2,0.) -- (0.2,0.2) -- (0.,0.2) -- (0.,0.);
\draw (0.,1.3) -- (0.2,1.3) -- (0.2,1.5) -- (0.,1.5);
\draw (2.8,1.5) -- (2.8,1.3) -- (3.,1.3) -- (3.,1.5);
\draw(3.,0.2) -- (2.8,0.2) -- (2.8,0.) -- (3.,0.);
\node[ponto]at(0,0){};
\node[ponto]at(3,0){};
\node[ponto]at(3,1.5){};
\node[ponto]at(0,1.5){};
\node[below ]at(0,0){$$};
\node[below ]at(3,0){$$};
\node[above ]at(3,1.5){$$};
\node[above ]at(0,1.5){$$};
\node[above]at(1.7,-.7){$3$};
\node[right]at(3,.75){$1.5$};
\end{tikzpicture}

\end{figure}

\begin{enumerate}
\item {} 
Observe o retângulo da figura a seguir e determine se ele é semelhante ou não ao retângulo \(R\).

\begin{figure}[H]
\centering

\begin{tikzpicture}
\tikzstyle{ponto}=[circle, minimum size=2pt, inner sep=0, draw=black, fill=black, shift only]
\draw[thick,black,fill=\currentcolor!80] (0.,0.) -- (6.,0.) -- (6.,1) -- (0.,1.)-- cycle;
\draw (0.2,0.) -- (0.2,0.2) -- (0.,0.2) -- (0.,0.);
\draw (0.,.8) -- (0.2,.8) -- (0.2,1) -- (0.,1);
\draw (5.8,1) -- (5.8,.8) -- (6.,.8) -- (6.,1);
\draw(6.,0.2) -- (5.8,0.2) -- (5.8,0.) -- (6.,0.);
\node[ponto]at(0,0){};
\node[ponto]at(6,0){};
\node[ponto]at(6,1){};
\node[ponto]at(0,1){};
\node[below ]at(0,0){$$};
\node[below ]at(6,0){$$};
\node[above ]at(6,1){$$};
\node[above ]at(0,1){$$};
\node[above]at(3,-.7){$6$};
\node[right]at(6,.5){$1.5$};
\end{tikzpicture}

\end{figure}

\item {} 
Na figura a seguir temos a medida base de um retângulo em destaque, qual deve ser a medida de sua altura para que o retângulo gerado seja semelhante a \(R\)? Qual a função linear que relaciona esses dois retângulos?

\begin{figure}[H]
\centering


\begin{tikzpicture}
\tikzstyle{ponto}=[circle, minimum size=2pt, inner sep=0, draw=black, fill=black, shift only]
\fill[bottom color=\currentcolor!80,top color =white] (0.,0.) -- (6.,0.) -- (6.,.5) -- (0.,.5) -- cycle;
\draw (0.2,0.) -- (0.2,0.2) -- (0.,0.2) -- (0.,0.);
\draw(6.,0.2) -- (5.8,0.2) -- (5.8,0.) -- (6.,0.);
\draw(0.,.5)--(0.,0.) -- (6.,0.) -- (6.,.5);
\node[ponto]at(0,0){};
\draw[fill](0,.6)circle(.5pt);
\draw[fill](0,.7)circle(.5pt);
\draw[fill](0,.8)circle(.5pt);
\node[ponto]at(6,0){};
\draw[fill](6,.6)circle(.5pt);
\draw[fill](6,.7)circle(.5pt);
\draw[fill](6,.8)circle(.5pt);
\node[below ]at(0,0){$$};
\node[below ]at(6,0){$$};
\node[above ]at(6,1.5){$$};
\node[above ]at(0,1.5){$$};
\node[above]at(3,-.7){$6$};
\end{tikzpicture}

\end{figure}
\item {} 
Seguindo a mesma ideia do item anterior, qual deve ser a medida da altura desse novo retângulo de base \(5\), para que ele seja semelhante a \(R\)? E neste caso, qual a função linear entre os retângulos?

\begin{figure}[H]
\centering


\begin{tikzpicture}
\tikzstyle{ponto}=[circle, minimum size=2pt, inner sep=0,   draw=black, fill=black, shift only]
\fill[bottom color=\currentcolor!80,top color =white] (0.,0.) -- (5.,0.) -- (5.,.5) -- (0.,.5) -- cycle;
\draw (0.2,0.) -- (0.2,0.2) -- (0.,0.2) -- (0.,0.);
\draw(5.,0.2) -- (4.8,0.2) -- (4.8,0.) -- (5.,0.);
\draw(0.,.5)--(0.,0.) -- (5.,0.) -- (5.,.5);
\node[ponto]at(0,0){};
\draw[fill](0,.6)circle(.5pt);
\draw[fill](0,.7)circle(.5pt);
\draw[fill](0,.8)circle(.5pt);
\node[ponto]at(5,0){};
\draw[fill](5,.6)circle(.5pt);
\draw[fill](5,.7)circle(.5pt);
\draw[fill](5,.8)circle(.5pt);
\node[below ]at(0,0){$$};
\node[below ]at(5,0){$$};
\node[above ]at(5,1.5){$$};
\node[above ]at(0,1.5){$$};
\node[above]at(2.5,-.7){$5$};
\end{tikzpicture}

\end{figure}
\item {} 
Já na figura a seguir, apresentamos um retângulo de altura \(4\), qual deve ser a medida da base desse novo retângulo, para que ele seja semelhante a \(R\)?

\begin{figure}[H]
\centering

\begin{tikzpicture}
\tikzstyle{ponto}=[circle, minimum size=2pt, inner sep=0, draw=black, fill=black, shift only]
\fill[left color = white, right color =\currentcolor!80,] (2.,0.) -- (3.,0.) -- (3.,2.5) -- (2.,2.5) -- cycle;
\draw[thick] (2.,0.) -- (3.,0.) -- (3.,2.5) -- (2.,2.5) ;
\draw (2.8,2.5) -- (2.8,2.3) -- (3.,2.3) -- (3.,2.5);
\draw(3.,0.2) -- (2.8,0.2) -- (2.8,0.) -- (3.,0.);
\node[ponto]at(0,0){};
\node[ponto]at(3,0){};
\node[ponto]at(3,2.5){};
\node[ponto]at(0,2.5){};
\node[below ]at(0,0){$$};
\node[below ]at(3,0){$$};
\node[above ]at(3,2.5){$$};
\node[above ]at(0,2.5){$$};
\node[right]at(3,1.25){$4$};
\draw[fill](1.7,0)circle(.5pt);
\draw[fill](1.8,0)circle(.5pt);
\draw[fill](1.9,0)circle(.5pt);
\draw[fill](1.7,2.5)circle(.5pt);
\draw[fill](1.8,2.5)circle(.5pt);
\draw[fill](1.9,2.5)circle(.5pt);
\end{tikzpicture}

\end{figure}
\item {} 
Na figura a seguir, apresentamos um retângulo cuja base tem a mesma medida da base de \(R\) (igual a \(3\)), e cuja altura coincide com a de um triângulo equilátero de lado medindo \(3\). Esse retângulo é semelhante a \(R\)?

\begin{figure}[H]
\centering


\begin{tikzpicture}
\tikzstyle{ponto}=[circle, minimum size=2pt, inner sep=0, draw=black, fill=black, shift only]
\draw[fill=\currentcolor!80,very thick](0,0)--(4,0)--(4,3.46)--(0,3.46)--cycle;
\draw[fill=terciario,very thick](0,0)--(4,0)--(2,3.46)--cycle;
\node[ponto]at(0,0){};
\node[ponto]at(4,0){};
\node[ponto]at(4,3.46){};
\node[ponto]at(0,3.46){};
\node[ponto]at(2,3.46){};
\node[below]at(0,0){$$};
\node[below]at(4,0){$$};
\node[above]at(4,3.46){$$};
\node[above]at(0,3.46){$$};
\node[above]at(2,3.46){$$};
\node[above]at(2,-.8){$3$};
\end{tikzpicture}

\end{figure}
\item {} 
Se utlizarmos a altura do retângulo da figura anterior na construção de um novo retângulo, qual deve ser a medida de sua base para que seja semelhante a \(R\)?

\end{enumerate}


\ifdefined\prof
\begin{solucao}
\begin{enumerate}
\item Não, pois a medida da base dobrou e a altura se manteve.

\item $3$, pois se a medida da base dobrou a altura deve dobrar $1{,}5\cdot2=3$. Os retângulos se relacionam por meio da função linear $f(x)=2\cdot x$.

\item $2{,}5$, pois em todos os retângulos a razão de semelhança, entre a base e a altura é de $12$, portando a altura deve ser a metade da base. Neste caso os retângulos se relacionam por meio da função linear $f(x)=53\cdot x$.

\item $8$, pelo mesmo motivo citado anteriormente, a base deve ser o dobro a altura.

\item Não, pois a razão entre base e altura não é de $12$.

\item $\sqrt{33}$, pois a altura de um triângulo equilátero de lado $3$ é $\frac{3\sqrt{33}}{2}$, ao assumir essa medida como altura do retângulo, sua base deve ser o dobro dessa medida.

\end{enumerate}
\end{solucao}
\fi

\end{document}