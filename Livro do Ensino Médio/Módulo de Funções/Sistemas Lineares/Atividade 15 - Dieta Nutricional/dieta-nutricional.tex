\documentclass[10 pt,usenames,dvipsnames, oneside]{article}
\usepackage{../../../modelo-ensino-medio}



\begin{document}

\begin{center}
  \begin{minipage}[l]{3cm}
\includegraphics[width=2cm]{logo}    
\end{minipage}\hfill
\begin{minipage}[r]{.8\textwidth}
 {\Large \scshape Atividade: Dieta Nutricional}  
\end{minipage}
\end{center}
\vspace{.2cm}

\ifdefined\prof
%Habilidades da BNCC
% \begin{objetivos}
% \item 
% \end{objetivos}

%Caixa do Para o Professor
\begin{goals}
%Objetivos específicos
\begin{enumerate}
\item Aplicar sistemas lineares em contexto nutricional.
 \item Resolver um sistema linear por escalonamento. 
\end{enumerate}

\tcblower

%Orientações e sugestões
Neste exercício primeiro o aluno precisa escrever as equações que modelam o fenômeno (isso o aluno já está acostumado, pois já fez algumas modelagens nas atividades \textit{\hyperref[teatro]{Faturamento de um teatro}, \hyperref[clube]{Clube de Esportes}, \hyperref[mais_bolos]{Fazendo mais bolos}} e \textit{\hyperref[trafego]{Rede de Tráfego}} ) e, em seguida, deve escalonar o sistema obtido, seguindo os passos apresentados anteriormente.
\end{goals}

\bigskip
\begin{center}
{\large \scshape Atividade}
\end{center}
\fi

Um nutricionista está elaborando uma dieta para um paciente que está com deficiência de vitamina C, cálcio e magnésio. Uma das refeições da dieta será composta por três alimentos com quantidades medidas em porções de $50$ g, de forma que os totais em miligramas (mg) dos nutrientes necessários sejam atingidos. A tabela abaixo contém as informações nutricionais dos três alimentos desta refeição:

\begin{table}[H]
\centering
\begin{tabu} to \textwidth{|c|c|c|c|e{3,5cm}|}
\hline
\tmcol{5}{|c|}{Miligramas de nutrientes por porção de 50g de alimento} \\
\hline
\tcolor{Nutriente} & \tcolor{Alimento 1 (mg)} & \tcolor{Alimento 2 (mg)} & \tcolor{Alimento 3 (mg)} & \tcolor{Total de  nutrientes  necessários (mg)} \\
\hline
Vitamina C & 10 & 10 & 20 & 100  \\
\hline
Cálcio & 50 & 40 & 10 & 300 \\
\hline
Magnésio & 30 & 10 & 40 & 200 \\
\hline
\end{tabu}
\end{table}

\begin{enumerate}
\item{}
Denote por $x_1$, $x_2$ e $x_3$ a quantidade de porções dos alimentos 1, 2 e 3 respectivamente que compõem uma refeição contendo exatamente a quantidade necessária de Vitamina C, Cálcio e Magnésio a serem ingeridas pelo paciente. Utilizando a tabela acima, escreva um sistema linear nas incógnitas $x_1$, $x_2$ e $x_3$ que represente essa situação;

\item{}

Escalone e resolva o sistema obtido no item \titem{a)}, classificando-o.

\item{}
Qual será o peso total de comida ingerida nesta refeição, considerando que ela contém apenas os alimentos 1, 2 e 3? 

\end{enumerate}

\ifdefined\prof
\clearpage
\begin{solucao}

\begin{enumerate}
\item 
$
\left \{
\begin{aligned}
10x_1+10x_2+20x_3&=100\\
50_x1+40_x2+10_x3&=300\\
30x_1+10x_2+40x_3&=200
\end{aligned}
\right.  
$
\item $\displaystyle(x_1,x_2,x_3)=(\frac{25}{8},\frac{25}{8},\frac{15}{8})$.
\item $\displaystyle\frac{1625}{4}$ gramas.
\end{enumerate}

\end{solucao}
\fi

\end{document}