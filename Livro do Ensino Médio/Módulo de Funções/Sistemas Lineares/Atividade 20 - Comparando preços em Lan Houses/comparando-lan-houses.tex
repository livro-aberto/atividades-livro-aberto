\documentclass[10 pt,usenames,dvipsnames, oneside]{article}
\usepackage{../../../modelo-ensino-medio}



\begin{document}

\begin{center}
  \begin{minipage}[l]{3cm}
\includegraphics[width=2cm]{logo}    
\end{minipage}\hfill
\begin{minipage}[r]{.8\textwidth}
 {\Large \scshape Atividade: Comparando preços em \textit{Lan Houses}}  
\end{minipage}
\end{center}
\vspace{.2cm}

\ifdefined\prof
%Habilidades da BNCC
% \begin{objetivos}
% \item 
% \end{objetivos}

%Caixa do Para o Professor
\begin{goals}
%Objetivos específicos
\begin{enumerate}
\item Modelar matematicamente fenômenos usando funções afins. 
\item Resolver uma inequação do primeiro grau algebricamente.
\item Resolver uma inequação do primeiro grau geometricamente.
\end{enumerate}

\tcblower

%Orientações e sugestões
\begin{itemize}
\item Comente com os alunos sobre a importância de considerar o tempo em horas para estudar essa situação, por exemplo, $3$h$15$min são $3,25$ horas, e não $3,15$.
\item No itens \titem{a)} e \titem{b)} faça os alunos calcularem começarem a pensar em como poderiam generalizar esse cálculo.
\item Para responder o item \titem{e)} incentive seus aluno a construir os gráfico das funções encontradas nos itens \titem{c)} e \titem{d)} e lembre-os de que as \textit{Lan Houses} só ficam abertas por $12$ hrs. 
\item Faça com que os alunos entendam também que, responder quando a Mega-Conexão é mais barata que a Net-Ágil é equivalente a resolver a inequação $2x<5-x$ e que responder quando a Net-Ágil é mais barata que a Mega-Conexão é equivalente a resolver a inequação $2x>5-x$
\end{itemize}
\end{goals}

\bigskip
\begin{center}
{\large \scshape Atividade}
\end{center}
\fi

Em alguns lugares onde é há baixo acesso à internet, ainda existem as \emph{Lan Houses}, que são estabelecimentos comerciais que ofertam serviço de internet para seus clientes usarem durante um certo período de tempo. A \emph{Lan house} Net-Ágil cobra 5 reais por acesso e mais 1 real por hora utilizada. Já na \emph{Lan House} Mega-Conexão, a cobrança é de 2 reais por hora. Ambas as Lan Houses cobram por frações de hora.

\begin{enumerate}
\item{}	
Se um cliente permanecer na \emph{Lan House} NetÁgil por 3h15min, qual será o valor a ser pago? E se escolher a \emph{Lan House} Mega-Conexão, quanto deverá pagar ao final desse período?

\item{}
Para usar os computadores e internet por 6h10min, qual das duas \emph{Lan House} seria a melhor escolha?

\item{}
Escreva a expressão algébrica que define a função $f$ e que retorna a cada tempo $x$ em horas o valor a ser pago em reais na \emph{Lan House} Net-Ágil.

\item{}	
Escreva a expressão algébrica que define a função $g$ e que retorna a cada tempo $x$ em horas o valor a ser pago em reais na \emph{Lan House} Mega-Conexão.

\item{}	
Suponha que as duas \emph{Lan Houses} funcionam das 8h às 20h. Para que valores de x a \emph{Lan House} Mega-Conexão é mais vantajosa?
\end{enumerate}

\ifdefined\prof
\begin{solucao}

\begin{enumerate}
\item Net-Ágil: R\$ $8{,}25$, Mega-Conexão:  R\$ $6{,}25$.
\item  O custo na Net-Ágil é de  R\$ $11{,}17$ enquanto na Mega-Conexão é de R\$ $12{,}33$, portanto a Net-Ágil é a escolha mais econômica.
\item $f(x)=5+x$.
\item $g(x)=2x$.
\item De $1$ min a $5$ h, a melhor escolha é a Mega-Conexão. Entre $5$ e $12$ h, Net-Ágil.
\end{enumerate}

\end{solucao}
\fi

\end{document}