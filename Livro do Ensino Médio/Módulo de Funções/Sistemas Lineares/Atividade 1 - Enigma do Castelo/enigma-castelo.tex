\documentclass[10 pt,usenames,dvipsnames, oneside]{article}
\usepackage{../../../modelo-ensino-medio}



\begin{document}

\begin{center}
  \begin{minipage}[l]{3cm}
\includegraphics[width=2cm]{logo}    
\end{minipage}\hfill
\begin{minipage}[r]{.8\textwidth}
 {\Large \scshape Atividade: Enigma do Castelo}  
\end{minipage}
\end{center}
\vspace{.2cm}

\ifdefined\prof
%Habilidades da BNCC
% \begin{objetivos}
% \item 
% \end{objetivos}

%Caixa do Para o Professor
\begin{goals}
%Objetivos específicos
\begin{enumerate}
\item Trabalhar o significado de um elemento ser, ou não, solução de uma equação.
\item Introduzir o aluno à representação geométrica do conjunto solução de uma equação.
\end{enumerate}

\tcblower

%Orientações e sugestões
\begin{itemize}
\item Deixe os alunos testarem valores livremente. A ideia é que a sequência dos itens faça o aluno chegar às conclusões esperadas.
\item No item \titem{g)}, espera-se que os alunos consigam perceber, do ponto de vista intuitivo, que as soluções da equação $x^2+y^2=5$ formem uma circunferência de centro na origem e raio 5. Não se espera e nem se deseja que uma justificativa formal para esse fato seja realizada nesse momento. Caso os alunos disponham de celulares com o GeoGebra instalado, é interessante plotar a curva associada à referida equação no aplicativo, para contribuir com o convencimento dos alunos sobre o resultado conjecturado por eles.
\end{itemize}
\end{goals}

\bigskip
\begin{center}
{\large \scshape Atividade}
\end{center}
\fi

Numa antiga série de TV chamada "Castelo Rá-tim-bum", para os personagens entrarem no castelo, eles precisavam responder a um enigma proposto pelo porteiro. O enigma proposto era "Diga $2$ números reais cuja soma dos quadrados seja igual a $25$".

\begin{figure}[H]
\centering

\noindent\includegraphics[width=150bp]{ratimbum.png}
\end{figure}



\begin{enumerate}
\item {} 
Os números $-3$ e $4$ são uma possível resposta para o enigma? E os números 5 e 3?

\item {} 
Você consegue outras respostas para o enigma formadas apenas por pares de números inteiros?

\item {} 
Ao responder ao enigma, se qualquer um dos dois números escolhidos for maior que $5$, esta resposta estará correta? E se um deles for menor que $-5$? Justifique.

\item {}
Escolhendo qualquer número entre $-5$ e $5$, é possível encontrar um segundo número de forma que, junto com o primeiro, eles formem uma solução para o enigma? Se sim, determine algumas dessas soluções.

\item {}
Estabelecendo uma ordem para os dois números escolhidos, chamemos o primeiro número de $x$ e o segundo de $y$. Qual a equação nas incógnitas $x$ e $y$ descreve o acerto do enigma?

\item {}
Marque suas respostas aos nos itens \titem{a}, \titem{b} e \titem{d} no plano cartesiano, associando ao primeiro número à coordenada $x$ e ao segundo número, a coordenada $y$. Se você preferir, pode usar o GeoGebra para isso!

\item {}
Discuta com os seus colegas se há algum padrão na disposição dos pontos marcados pelas possíveis respostas do enigma.

\end{enumerate}

\ifdefined\prof
\begin{solucao}

\begin{enumerate}
\item Sim. Não.
\item $\{-5,0\},\{-5,0\},\{-4, -3\}, \{-4,3\}, \{4, -3\}, \{4, 3\}$. Note que ainda não estamos falando de pares ordenados, portanto, $\{0,5\}$ e $\{5,0\}$ são iguais.
\item Não, pois se um dos números é maior que $5$, seu quadrado já é maior que $25$.  Não, raciocínio análogo.
\item Sim, esse número pode não ser, necessariamente, inteiro.
\item $x^2+y^2=25$.
\item Ao dispor as soluções no plano cartesiano, se tem um círculo de raio $5$.
\end{enumerate}

\end{solucao}
\fi

\end{document}