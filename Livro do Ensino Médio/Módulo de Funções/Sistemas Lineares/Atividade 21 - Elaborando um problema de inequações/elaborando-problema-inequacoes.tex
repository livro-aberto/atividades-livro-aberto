\documentclass[10 pt,usenames,dvipsnames, oneside]{article}
\usepackage{../../../modelo-ensino-medio}



\begin{document}

\begin{center}
  \begin{minipage}[l]{3cm}
\includegraphics[width=2cm]{logo}    
\end{minipage}\hfill
\begin{minipage}[r]{.8\textwidth}
 {\Large \scshape Atividade: }  
\end{minipage}
\end{center}
\vspace{.2cm}

\ifdefined\prof
%Habilidades da BNCC
% \begin{objetivos}
% \item 
% \end{objetivos}

%Caixa do Para o Professor
\begin{goals}
%Objetivos específicos
\begin{enumerate}
\item Elaborar problemas envolvendo inequações de primeiro grau
\end{enumerate}

\tcblower

%Orientações e sugestões
Professor, nesse exercício, sugerimos que os alunos proponham os problemas em grupos e que troquem entre si. Cabe lembrar que a elaboração de problemas é uma habilidade enfatizada pela base nacional comum curricular e que estimula a criatividade e o vínculo entre objetos matemáticos e situações cotidianas para o aluno.
\end{goals}

\bigskip
\begin{center}
{\large \scshape Atividade}
\end{center}
\fi
Elabore problemas do seu cotidiano que sejam representados pelas inequações abaixo. Em seguida, resolva as inequações.
\begin{enumerate}
\item{}
$2x - 3 > 0$

\item{}
$4  - x < 1 + 8x$.
\end{enumerate}


% \ifdefined\prof
% \begin{solucao}

% \begin{enumerate}
% \item
% \end{enumerate}

% \end{solucao}
% \fi

\end{document}