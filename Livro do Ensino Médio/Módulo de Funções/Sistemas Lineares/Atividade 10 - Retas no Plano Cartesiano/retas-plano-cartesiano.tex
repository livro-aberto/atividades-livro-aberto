\documentclass[10 pt,usenames,dvipsnames, oneside]{article}
\usepackage{../../../modelo-ensino-medio}



\begin{document}

\begin{center}
  \begin{minipage}[l]{3cm}
\includegraphics[width=2cm]{logo}    
\end{minipage}\hfill
\begin{minipage}[r]{.8\textwidth}
 {\Large \scshape Atividade: Retas no Plano Cartesiano}  
\end{minipage}
\end{center}
\vspace{.2cm}

\ifdefined\prof
%Habilidades da BNCC
% \begin{objetivos}
% \item 
% \end{objetivos}

%Caixa do Para o Professor
\begin{goals}
%Objetivos específicos
\begin{enumerate}
\item Identificar equações de retas paralelas.
\end{enumerate}

\tcblower

%Orientações e sugestões
\begin{itemize}
\item Professor, vale a pena estimular os alunos a construir as representações gráficas das três retas em um mesmo sistema de eixos, seja no papel, seja usando o GeoGebra.
\item Essa atividade pode ser feita tanto no Geogebra quanto no papel. Se fizer no Geogebra, basta entrar com as equações das retas para que se faça a representação gráfica dela. Se fizer no papel, lembre que da Geometria Euclidiana Plana,  para determinar uma reta no plano, basta conhecer dois de seus pontos.  
\item Na resolução do item \titem{b)} ajude seus alunos a encontrar soluções de uma das equações, levando-os a verificar que as mesmas também são solução para a outra.
\item Na resolução do item \titem{c}) sugira que os alunos manipulem algebricamente a equação $2$ de forma que se chegue na equação $1$.
\end{itemize}
\end{goals}

\bigskip
\begin{center}
{\large \scshape Atividade}
\end{center}
\fi

Na última seção, vimos que o conjunto solução de uma equação linear em duas incógnitas é representado por uma reta no plano. Considere as equações lineares a seguir:
\begin{align*}
\mbox{Equação} \ 1: 2x+3y=1;   &   \ \ \ \  \mbox{Equação} \ 2: -6x - 9y = -3;\\
\mbox{Equação}\ 3: 2x+3y=4;    & \ \ \ \ \mbox{Equação} \ 4: 3x + y = 5. \\
\end{align*}
\begin{enumerate}
\item{} 
Esboce a reta associada a Equação 1, em seguida, destaque 2 pontos dessa reta. Esses dois pontos são soluções da Equação? Justifique.

\item{}
Esboce a reta da Equação 2. Qual é a relação entre essa reta e a da Equação 1? Discuta com seus colegas o que isso significa com relação às soluções das Equações 1 e 2.

\item{}
Qual é a relação algébrica entre as Equações 1 e 2? Elabore exemplos de outras equações lineares que tenham a mesma relação.

\item{}
Esboce a reta da Equação 3. Qual é a relação entre as retas das Equações 1 e 3? Baseado nisso, diga se um ponto $(x_0,y_0)$ pode ser solução das Equações 1 e 3 simultaneamente? Justifique a sua resposta.

\item{}
Qual é a relação algébrica entre as Equações 1 e 3? Dê exemplos de outras equações que tenham a mesma relação.

\item{}
Trace a reta associada a Equação 4. Qual a relação dessa reta com as traçadas nos itens anteriores?

\item{}
Resolva o sistema formado pelas Equações 1 e 4 e relacione a solução com o que você observou no gráfico?

\end{enumerate}

\ifdefined\prof
\begin{solucao}

\begin{enumerate}
\item As retas são iguais.
\item São as mesmas retas. Isso significa que o conjunto solução das equações são iguais.
\item A equação $2$ é a equação $1$ multiplicada por $-3.$
\item São paralelas. Não, pois tal ponto estaria na interseção das retas.
\item As equações diferem apenas no termo independente. Basta alterar a constante isolada, isto é, $2x+3y=k, k\in \mathbb{R}$.
\item Concorrente
\item O sistema possui solução única, que significa que as retas são concorrentes em um ponto
\end{enumerate}

\end{solucao}
\fi

\end{document}