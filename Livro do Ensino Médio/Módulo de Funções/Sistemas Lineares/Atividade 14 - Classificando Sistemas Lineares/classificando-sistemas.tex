\documentclass[10 pt,usenames,dvipsnames, oneside]{article}
\usepackage{../../../modelo-ensino-medio}



\begin{document}

\begin{center}
  \begin{minipage}[l]{3cm}
\includegraphics[width=2cm]{logo}    
\end{minipage}\hfill
\begin{minipage}[r]{.8\textwidth}
 {\Large \scshape Atividade: Classificando Sistemas Lineares}  
\end{minipage}
\end{center}
\vspace{.2cm}

\ifdefined\prof
%Habilidades da BNCC
% \begin{objetivos}
% \item 
% \end{objetivos}

%Caixa do Para o Professor
\begin{goals}
%Objetivos específicos
\begin{enumerate}
\item Resolver um sistema linear por escalonamento.
\item Classificar um sistema linear em relação à cardinalidade de seu conjunto solução.
\end{enumerate}

\tcblower

%Orientações e sugestões
Exercício para resolução de sistemas já escalonados e sua classificação como sistema possível determinado (SPD), sistema possível indeterminado (SPI) ou sistema impossível (SI).
\end{goals}

\bigskip
\begin{center}
{\large \scshape Atividade}
\end{center}
\fi

Os sistemas lineares abaixo estão escalonados. Classifique-os em relação ao número de soluções (S.P.D., S.P.I. ou S.I.).

\begin{minipage}{\linewidth}
\begin{enumerate}[leftmargin=0pt]
\begin{multicols}{2}
\centering
\item

$
\left \{
\begin{aligned}
2x+3y-2z&=5\\
y-4z&=2\\
-3z&=9
\end{aligned}
\right.
$

\columnbreak

\item
$
\left \{
\begin{aligned}
-x-2y+z&=2\\
2y-z&=5
\end{aligned}
\right.
$

\end{multicols}
\begin{multicols}{2}
\centering

\item
$
\left \{
\begin{aligned}
x+z-z+w&=5\\
5y-3z+2w&=-8\\
-3z-w&=-3
\end{aligned}
\right.
$


\columnbreak

\item
$
\left \{
\begin{aligned}
x+y+3z-4w&=9\\
y-4z+w&=5\\
-z+4w&=2\\
2w&=-1
\end{aligned}
\right.
$
\end{multicols}
\end{enumerate}
\end{minipage}

\ifdefined\prof
\begin{solucao}

\begin{enumerate}
\item $(x,y,z)=(\dfrac{29}{2},-10,-3)$ , SPD.
\item $(x,y,z)=(-7;\dfrac{z+5}{2},z);z\in \mathbb{R}$, SPI.
\item $(x,y,z,w)=(\dfrac{97-11w}{15},\dfrac{-3w-9}{5}, \dfrac{-w-1}{3},w); w\in \mathbb{R}$, SPI.
\item $(x,y,z,w)=(\dfrac{75}{8},\dfrac{21}{8},\dfrac{-1}{2},\dfrac{3}{8})$, SPD.  
\end{enumerate}

\end{solucao}
\fi

\end{document}