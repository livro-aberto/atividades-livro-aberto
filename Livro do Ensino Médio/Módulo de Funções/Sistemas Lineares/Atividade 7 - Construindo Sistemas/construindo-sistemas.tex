\documentclass[10 pt,usenames,dvipsnames, oneside]{article}
\usepackage{../../../modelo-ensino-medio}



\begin{document}

\begin{center}
  \begin{minipage}[l]{3cm}
\includegraphics[width=2cm]{logo}    
\end{minipage}\hfill
\begin{minipage}[r]{.8\textwidth}
 {\Large \scshape Atividade: Construindo Sistemas}  
\end{minipage}
\end{center}
\vspace{.2cm}

\ifdefined\prof
%Habilidades da BNCC
% \begin{objetivos}
% \item 
% \end{objetivos}

%Caixa do Para o Professor
\begin{goals}
%Objetivos específicos
\begin{enumerate}
\item Fazer o aluno propor problemas.
\item Relacionar intercessões de curvas com soluções de sistemas. 
\end{enumerate}

\tcblower

%Orientações e sugestões
Nessa atividade, faça o aluno entender a relação entra as curvas e suas respectivas equações e consolidar a ideia que a intercessão entre as curvas (quando existente) correspondem às soluções dos sistemas constituídos por essas equações. 
\end{goals}

\bigskip
\begin{center}
{\large \scshape Atividade}
\end{center}
\fi

Utilizando apenas equações que aparecem na figura do último exemplo, construa sistemas de equações com as propriedades pedidas abaixo, indicando na figura quais os pontos que correspondem às suas respectivas soluções.

\begin{enumerate}
\item{}
Um sistema com três equações da figura que não possua solução;

\item{}
Dois sistemas que tenham exatamente duas soluções;


\end{enumerate}

\ifdefined\prof
\begin{solucao}

\begin{enumerate}
\item Basta tomar um conjunto de três equações que não tenham um ponto em comum a todas simultaneamente. Por exemplo, o sistema formado por $f, g$ e $h$.
\item O sistema composto pelas equações $f$ e $g$ e o sistema formado por $eq_1$ e $eq_2$.
\end{enumerate}

\end{solucao}
\fi

\end{document}