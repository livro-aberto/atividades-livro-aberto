\documentclass[10 pt,usenames,dvipsnames, oneside]{article}
\usepackage{../../../modelo-ensino-medio}



\begin{document}

\begin{center}
  \begin{minipage}[l]{3cm}
\includegraphics[width=2cm]{logo}    
\end{minipage}\hfill
\begin{minipage}[r]{.8\textwidth}
 {\Large \scshape Atividade: Construindo Retas no Plano}  
\end{minipage}
\end{center}
\vspace{.2cm}

\ifdefined\prof
%Habilidades da BNCC
% \begin{objetivos}
% \item 
% \end{objetivos}

%Caixa do Para o Professor
\begin{goals}
%Objetivos específicos
\begin{enumerate}
\item Estudar equações equivalentes.
\item Consolidar o significado geométrico de equações equivalente. 
\end{enumerate}

\tcblower

%Orientações e sugestões
\begin{itemize}
\item Nesse exercício, o objetivo é que o aluno perceba que é possível manipular a equação e escrever equações equivalentes a ela com os coeficientes que forem convenientes. Se possível, incentive seus alunos a plotar cada uma dessas equações com o GeoGebra, o que reforçará a ideia da equivalência entre elas, pois as suas representações ficarão sobrepostas quando construídas em um mesmo sistema de eixos
\end{itemize}
\end{goals}

\bigskip
\begin{center}
{\large \scshape Atividade}
\end{center}
\fi

Considere a reta $r$ cuja equação é $3x + 4y = 1$. Em cada item abaixo, encontre uma equação linear em $x$ e $y$ tal que ela:


\begin{enumerate}
\item{}
Seja equivalente à equação do enunciado, com coeficiente de $x$ igual a $-1$;

\item{}
Seja equivalente à equação do enunciado, com coeficiente de $y$ igual a $2$;

\item{}
Seja equivalente à equação do enunciado, com coeficiente de $x$ igual a $\frac{1}{3}$;

\item{}
Seja equivalente à equação do enunciado, com termo independente igual a $5$;

\item{}
Represente uma reta paralela a $r$ e que passa pelo ponto $(2,5)$;

\item{}
Tenha o coeficiente do $x$ igual a $6$, represente uma reta paralela a $r$ e passe pelo ponto $(3,-7)$;

\item{}
Tenha coeficiente do $y$ igual a $9$, represente uma reta concorrente a $r$ e passe pelo ponto $(8,1)$.

\item{}	
Você consegue encontrar mais de uma equação linear que atenda o que foi pedido no item \titem{g)}? O mesmo é possível para os itens \titem{e)} e \titem{f)}? Por quê?

\item{}
Utilize o GeoGebra para plotar as retas correspondentes às equações obtidas por você nos itens \titem{e)}, \titem{f)} e \titem{g)} e confirmar se elas cumprem as propriedades pedidas nesses itens.
\end{enumerate}

\ifdefined\prof
\clearpage
\begin{solucao}

\begin{enumerate}
\item Multiplicando toda a equação por $-\dfrac{1}{3}$, encontramos $x-\dfrac{4}{3}y = -\dfrac{1}{3}$.
\item Multiplicando toda a equação por $\dfrac{1}{2}$, encontramos $\dfrac{3}{2x} + 2y =\dfrac{1}{2}$.
\item Multiplicando toda a equação por $\dfrac{1}{9}$, encontramos $\dfrac{1}{2}x + \dfrac{4}{9}y = \dfrac{1}{9}$.
\item Multiplicando toda a equação por $-\frac{7}{4}$, encontramos $-\dfrac{21}{4}x - 7y = - \dfrac{7}{4}$.
\item Multiplicando toda a equação por $5$, obtemos $15x + 20y = 5$.
\item Multiplicando toda a equação por $-\dfrac{2}{5}$, obtemos $-\dfrac{6}{5}x - \dfrac{8}{5}y = -\dfrac{2}{5}$.
\item $-12x+9y=-87$.
\item Para atender o que foi pedido no \titem{g}), pegue qualquer reta concorrente a $r$ que passe por $(8,1)$, divida a equação pelo coeficiente do $y$ e multiplique por $9$. Para atender as condições do item \titem{e)}, multiplique a equação encontrada em \titem{e)} por qualquer numero real não nulo. Para o item \titem{f)} é impossível achar outra opção, pois só existe uma única reta que cumpra as condições exigidas, assim, todas as equações são múltiplas umas das outras, portanto, apenas uma tem o coeficiente do $x$ igual a $6$.
\end{enumerate}

\end{solucao}
\fi

\end{document}