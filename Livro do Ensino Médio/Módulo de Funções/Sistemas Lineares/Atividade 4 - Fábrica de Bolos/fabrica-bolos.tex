\documentclass[10 pt,usenames,dvipsnames, oneside]{article}
\usepackage{../../../modelo-ensino-medio}



\begin{document}

\begin{center}
  \begin{minipage}[l]{3cm}
\includegraphics[width=2cm]{logo}    
\end{minipage}\hfill
\begin{minipage}[r]{.8\textwidth}
 {\Large \scshape Atividade: Fábrica de Bolos}  
\end{minipage}
\end{center}
\vspace{.2cm}

\ifdefined\prof
%Habilidades da BNCC
% \begin{objetivos}
% \item 
% \end{objetivos}

%Caixa do Para o Professor
\begin{goals}
%Objetivos específicos
\begin{enumerate}
\item Levar o aluno a desenvolver a capacidade de modelar fenômenos reais por meio de equações lineares de duas variáveis.
\item Encontrar soluções condizentes com a realidade (o aluno deve ser capaz de saber que, nesse problema, só são admitidas soluções inteiras positivas
\end{enumerate}

\tcblower

%Orientações e sugestões
\begin{itemize}
\item A abordagem a essa questão pode começar com tentativa  e erro e depois, com a condução do profesor, o aluno pode chegar a equação $15f+18c=684$ que relaciona as quantidades de bolos vendidos, onde $c$ representa a quantidade de bolos de chocolate e $f$ os de fubá;.
\item Aproveite as diferentes respostas que os estudantes poderão encontrar. Todas elas estão corretas? Existiria mais de uma resposta correta? Note que esta última questão admite uma discussão numérica:
$f=\frac{684-18c}{15}$ é inteiro. O numerador é múltiplo de $3$ pois $684$ e $18$ o são, logo, temos que $684 -18c$ é múltiplo de 5. Segue que $684-18c$ deve terminar em $0$ ou em $5$, o que equivale a dizer que $18c$ deve terminar em $4$ ou em $9$. Como $18c$ é par,  isso equivale a dizer que $18c$ deve terminar em $4$, isto é, $c$ termina com $3$ ou $8$. Isso possibilita refinar e determinar rapidamente as possibilidades de soluções inteiras: $c = 3, 8, 13, 18, 23, 28, 33, 38.$
\end{itemize}
\end{goals}

\bigskip
\begin{center}
{\large \scshape Atividade}
\end{center}
\fi

A fábrica de bolos “Tia Tatá”{} produz e vende bolos de diversos tipos, mas os mais solicitados são o de fubá e o de chocolate, que custam $15$ reais e $18$ reais cada um, respectivamente. Em um determinado dia, a venda dos bolos de fubá e de chocolate gerou $684$ reais em caixa.

\begin{enumerate}

\item{}
Quantos bolos de cada tipo podem ter sido vendidos nesse dia?

\item{}
Use a construção no GeoGebra disponível em \url{https://www.geogebra.org/classic/juvucjuh}, para movimentar o controle deslizante $k$ e checar a sua resposta dada no item \titem{a)}. Diga como você pode usar essa ferramenta para encontrar outras respostas para o item \titem{a)}.

\item{}
Seria possível que a venda de bolos de chocolate nesse dia fosse de $50$ bolos? Justifique.
\end{enumerate}

\ifdefined\prof
\clearpage
\begin{solucao}

\begin{enumerate}
\item Denotando as possibilidades por pares do tipo $(c,f)$, onde $c$ é a quantidade de bolos de chocolate e $f$ a de fubá, temos as seguintes possibilidades $(3,42), (8,36), (13,30), \newline (18,24), (23,18), (28,12), \newline(33,6),(38,0)$;
\item Atividade exploratória.
\item Não, pois essa venda renderia R\$ $900{,}00$, o que excede os vencimentos do dia.
\end{enumerate}

\end{solucao}
\fi

\end{document}