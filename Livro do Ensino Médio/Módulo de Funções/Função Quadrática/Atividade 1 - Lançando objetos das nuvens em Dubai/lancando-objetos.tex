\documentclass[10 pt,usenames,dvipsnames, oneside]{article}
\usepackage{../../../modelo-ensino-medio}



\begin{document}

\begin{center}
  \begin{minipage}[l]{3cm}
\includegraphics[width=2cm]{logo}    
\end{minipage}\hfill
\begin{minipage}[r]{.8\textwidth}
 {\Large \scshape Atividade: Lançando objetos das nuvens em Dubai}  
\end{minipage}
\end{center}
\vspace{.2cm}

\ifdefined\prof
% Habilidades da BNCC
\begin{objetivos}
\item \textbf{EM12MT09} Reconhecer função quadrática e suas representações algébrica e gráfica, compreendendo o
modelo de variação determinando domínio, imagem, máximo e mínimo, e utilizar essas noções e
representações para resolver problemas como os de movimento uniformemente variado.
\end{objetivos}

%Caixa do Para o Professor
\begin{goals}
%Objetivos específicos
\begin{enumerate}
\item Reconhecer que a relação matemática entre a distância percorrida por um objeto em queda livre e o tempo de queda não pode ser modelada por uma função afim.
\item Relacionar o movimento de queda livre de um objeto a existência de uma aceleração na velocidade de queda.
\item Inferir que o tempo é uma grandeza contínua, mesmo sendo finito o número de dados coletados.
\item Reconhecer que o movimento pode ser descrito por uma curva e não por um conjunto de pontos desconectos.
\end{enumerate}

\tcblower

%Orientações e sugestões
\begin{itemize}
\item Sugerimos resolver a atividade anteriormente para definir o tempo necessário de sua aplicação.
\item Orientamos que seja feito um acompanhamento por parte do professor, durante a confecção da tabela apresentada no item a, com a finalidade de ter a certeza que os estudantes estejam compreendendo o significado dos valores gerados por ela.
\item Caso seja necessário, reforce as principais caracteríticas da função afim, como por exemplo: a sua taxa de variação constante.
\item No item \titem{d}, recomendamos que o professor chame a atenção dos estudantes para o fato de que, o gráfico seja apenas um conjunto de sete pontos, partindo da origem, e não uma curva contínua.
\item Para o item e, orientamos que o professor enfatize aos alunos que o registro fotográfico foi feito em intervalos de $1$ s, mas que o fenômeno continuou mesmo sem os registros.
\end{itemize}
\end{goals}

\bigskip
\begin{center}
{\large \scshape Atividade}
\end{center}
\fi

No topo do hotel Burj Al Arab, em Dubai, encontra-se a quadra de tênis mais alta do mundo, com aproximadamente \(200\) metros de altura. Em 2005, os campeões Roger Federer e Andre Agassi disputaram uma partida de exibição. Considere que por um descuido, uma das bolinhas usadas nesse jogo caiu \(200\)m, verticalmente e em queda livre. Vamos aproveitar essa situação para investigar a matemática por trás desse fenômeno físico. A imagem a seguir traduz a situação no início da queda da bola.

\begin{figure}[H]
\centering
% \capstart

\noindent\includegraphics[width=150bp]{{fig_1}.jpg}
\caption{Hotel e a bolinha de tênis (\textbf{credito da imagem aqui}).}\label{\detokenize{AF209-0:id125}}\end{figure}

Um observador registra com seu equipamento fotográfico a queda da bolinha, disparando fotos a cada intervalo de \(1\) segundo, até a mesma atingir o solo. Os registros fotográficos encontram-se agrupados e animados na simulação da queda, que pode ser visualizada no Geogebra: Bola de Tenis (\url{https://ggbm.at/hvnNHMY2})

A tabela a seguir descreve a altura da bolinha ao longo do tempo.

\begin{table}[H]
\centering
\begin{tabu} to \textwidth{|c|c|c|}
\hline
\thead
\(t\) & Tempo (s) & Altura (m) \\
\hline
\(t_0\) & \(0\) & \(200\) \\ 
\hline
\(t_1\) & \(1\) & \(195\) \\
\hline
\(t_2\) & \(2\) & \(180\) \\
\hline
\(t_3\) & \(3\) & \(155\) \\
\hline
\(t_4\) & \(4\) & \(120\) \\
\hline
\(t_5\) & \(5\) & \(75\) \\
\hline
\(t_6\) & \(6\) & \(20\) \\
\hline
\end{tabu}
\end{table}

\begin{enumerate}
\item {} 
Numa folha de papel ou similar, reproduza a tabela a seguir e preencha o que falta, informando a distância total percorrida pela bolinha na queda, a partir de \(t_0\).

\begin{table}[H]
\centering
\begin{tabu} to \textwidth{|c|l|}
\hline
\thead
Tempo de Queda & Distância percorrida pela bolinha \\
\hline
De \(t_0\) a \(t_0 = 0\)s & \(d_0 = 200 - 200 = 0\)m \\
\hline
De \(t_0\) a \(t_1  = 1\)s & \(d_1 = 200 - 195 = 5\)m \\
\hline
De \(t_0\) a \(t_2 = 2\)s & \(d_2 =\) \\
\hline
De \(t_0\) a \(t_3 = 3\)s & \(d_3 =\) \\
\hline
De \(t_0\) a \(t_4 = 4\)s & \(d_4 =\) \\
\hline
De \(t_0\) a \(t_5 = 5\)s & \(d_5 =\) \\
\hline
De \(t_0\) a \(t_6 = 6\)s & \(d_6 =\) \\
\hline
\end{tabu}
\end{table}

\needspace{5em}
\item {} 
As distâncias percorridas pela bolinha ao longo do tempo de queda aumentam com a mesma taxa de variação?

\item {} 
É possível obter uma função afim que relaciona a distância percorrida \(d_n\) (em metros) com o tempo de queda \(t\) (em segundos)? Justifique.

\item {} 
Em uma folha de papel ou similar, copie o plano cartesiano abaixo e, em seguida, represente os pares ordenados \((t;d_n)\) em que \(t\) representa o tempo de queda em segundos e \(d_n\) a distância, em metros, percorrida pela bolinha na queda:


\begin{figure}[H]
\centering

\begin{tikzpicture}[yscale=.8, xscale=1.2, scale=.75]
Gráfico
\tikzstyle{ponto}=[circle, minimum size=5pt, inner sep=0, draw=black, fill=black, shift only, label={}]
\draw [help lines, secundario!10, step=0.2] (0,0) grid (7,11);
\draw [help lines, secundario!40] (0,0) grid (7,11);
\draw [very thick, <->] (7.1,0) -- (0,0) -- (0, 11.1);
\node [below ] at (6.3,-0.5) {Tempo (s)};
\node [left] at (-0.5, 10.7) {Distância (m)};
\node [below] at (0,0) {0};
\node [below] at (1,0) {1};
\node [below] at (2,0) {2};
\node [below] at (3,0) {3};
\node [below] at (4,0) {4};
\node [below] at (5,0) {5};
\node [below] at (6,0) {6};
\node [left] at (0,1) {20};
\node [left] at (0,2) {40};
\node [left] at (0,3) {60};
\node [left] at (0,4) {80};
\node [left] at (0,5) {100};
\node [left] at (0,6) {120};
\node [left] at (0,7) {140};
\node [left] at (0,8) {160};
\node [left] at (0,9) {180};
\node [left] at (0,10) {200};
\end{tikzpicture}
\end{figure}

\item {} 
O domínio da função que descreve a queda da bolinha ao longo do tempo é \(D = \{0 ; 1 ; 2 ; 3 ; 4 ; 5 ; 6 \}\). A mesma situação poderia ser descrita por uma função de domínio contínuo?

\item {} 
Neste caso, ao ligarmos todos os pontos do gráfico do item \(d\) teríamos um segmento de reta ou uma curva?

\item {} 
Dentre as alternativas a seguir, qual relação atende aos valores descritos no gráfico sendo \(d(t)\) a distância percorrida pela bolinha na queda (em metros) com o tempo de queda \(t\) (em segundos).

\(\Box \; d(t)= -t^2\)

\(\Box \; d(t)= 10t+10\)

\(\Box \; d(t)= 20t\)

\(\Box \; d(t)= 5t^2\)

\(\Box \; d(t)= 10t^2\)

\end{enumerate}

\ifdefined\prof
\begin{solucao}

\begin{enumerate}
\item $d_0=0\text{ m}; d_1=5\text{ m}; d_2=20\text{ m}; d_3=45\text{ m}; d_4=80\text{ m}; d_5=125\text{ m}; d_6=180\text{ m}$.
\item Não. Para verificar, basta calcular a razão entre a variação das distâncias em dois intervalos distintos de um segundo, por exemplo: $\frac{5−0}{1−0}=5\neq\frac{20−5}{2−1}=15$, pois a função afim é caracterizada por uma variação constante.
\item Não, pois a taxa de variação não é constante.

\item \adjustbox{valign=t}
{
\begin{tikzpicture}[yscale=.8, xscale=1.2, scale=.75]
\tikzstyle{ponto}=[circle, minimum size=5pt, inner sep=0, draw=black, fill=black, shift only, label={}]
\draw [help lines, secundario!10, step=0.2] (0,0) grid (7,11);
\draw [help lines, secundario!40] (0,0) grid (7,11);
\draw [very thick, <->] (7.1,0) -- (0,0) -- (0, 11.1);
\node [below ] at (6.3,-0.5) {Tempo (s)};
\node [left] at (-0.5, 10.7) {Dist\^ancia (m)};
\node [below] at (0,0) {0};
\node [below] at (1,0) {1};
\node [below] at (2,0) {2};
\node [below] at (3,0) {3};
\node [below] at (4,0) {4};
\node [below] at (5,0) {5};
\node [below] at (6,0) {6};
\node [left] at (0,1) {20};
\node [left] at (0,2) {40};
\node [left] at (0,3) {60};
\node [left] at (0,4) {80};
\node [left] at (0,5) {100};
\node [left] at (0,6) {120};
\node [left] at (0,7) {140};
\node [left] at (0,8) {160};
\node [left] at (0,9) {180};
\node [left] at (0,10) {200};
\node [ponto, color=primario] at (1,0.25) {};
\node [ponto, color=primario] at (2,1) {};
\node [ponto, color=primario] at (3,2.25) {};
\node [ponto, color=primario] at (4,4) {};
\node [ponto, color=primario] at (5,6.25) {};
\node [ponto, color=primario] at (6,9) {};
\end{tikzpicture}
}
\item Sim, pois o tempo é contínuo.
\item Curva
\item $d(t)=5t^2$
\end{enumerate}

\end{solucao}
\fi

\end{document}