\documentclass[10 pt,usenames,dvipsnames, oneside]{article}
\usepackage{../../../modelo-ensino-medio}



\begin{document}

\begin{center}
  \begin{minipage}[l]{3cm}
\includegraphics[width=2cm]{logo}    
\end{minipage}\hfill
\begin{minipage}[r]{.8\textwidth}
 {\Large \scshape Atividade: Menino Gauss}  
\end{minipage}
\end{center}
\vspace{.2cm}

\ifdefined\prof
%Habilidades da BNCC
\begin{objetivos}
\item \textbf{EM12MT09} Reconhecer função quadrática e suas representações algébrica e gráfica, compreendendo o
modelo de variação determinando domínio, imagem, máximo e mínimo, e utilizar essas noções e
representações para resolver problemas como os de movimento uniformemente variado.
\end{objetivos}

%Caixa do Para o Professor
\begin{goals}
%Objetivos específicos
\begin{enumerate}
\item Reconhecer a função quadrática na expressão que dá a soma dos primeiros termos de uma progressão aritmética.
\item Resolver o problema de somar os primeiros termos de uma progressão aritmética com as ferramentas da função quadrática.
\end{enumerate}

% \tcblower

% %Orientações e sugestões
% \begin{itemize}
% \item 
% \end{itemize}
\end{goals}

\bigskip
\begin{center}
{\large \scshape Atividade}
\end{center}
\fi

No livro \textit{Antologia Matemática} de Malba Tahan, conta um episódio cuja personagem principal seria o “príncipe da matemática” Carl Frederick \textbf{Gauss} (\(\star 1777- \dagger 1855\)). Não se sabe se o episódio é real, mas conta-se que aos sete anos de idade, chegando para mais um dia de aula, \textit{Gauss} e seus colegas teriam encontrado o professor com pouca paciência. Assim, o professor, com o intuito de entreter seus alunos por longo tempo e não precisar dar-lhes qualquer atenção, pediu para que todos somassem os números naturais desde \(1\) até \(100\). Contudo, o jovem \textit{Gauss} em pouco tempo levou o resultado do exercício para o professor e este, incrédulo do feito, teria mandado \textit{Gauss} para a direção. Mais tarde, tudo se esclareceu e o professor reconheceu o acerto no método e no resultado dado pelo jovem e desculpou-se.

Como o jovem \textit{Gauss} teria obtido este resultado por um método aparentemente desconhido do enfurecido professor e com tanta rapidez?

Com a finalidade de responder a essa pergunta sugerimos uma atividade. Ela necessitará de uma fita métrica.

\begin{figure}[H]
\centering


\noindent\includegraphics[width=200bp]{{plastic-tape-measure}.jpg}
\caption{Imagem de \href{https://commons.wikimedia.org/wiki/File:Plastic\_tape\_measure.jpg}{Pastorius} CC-BY}\label{\detokenize{AF209-4:id1}}\end{figure}

Como as fitas métricas comercializadas tem um tamanho padrão, em nossa atividade vamos entender como o jovem \textit{Gauss} fez a soma começando por somar os números da fita métrica, ou seja, vamos começar resolvendo a expressão
\begin{equation*}
\begin{split}1+2+3+4+5+ \cdots +147+148+149+150\end{split}
\end{equation*}\begin{enumerate}
\item {} 
De posse da fita métrica, perceba que ela tem os dois lados numerados. Cada um desses lados tem todos os números que queremos somar?

\item {} 
Qual o número que corresponde ao verso (outro lado da fita) do número \(1\)? E quais são os números dos versos correspondentes de \(18\) e \(75\)?

\item {} 
Agora, vamos fazer algumas somas de um número com o seu correspondente no verso da fita. Faça:

\begin{enumerate}[label=\Roman*)]
\item \(1 + \,\;\) seu correspondente;

\item \(15 + \;\) seu correspondente;

\item \(31 + \;\) seu correspondente;

\item \(49 + \;\) seu correspondente;

\item \(75 + \;\) seu correspondente.
\end{enumerate}

\item {} 
Qual o resultado obtido sempre que se soma um número com o seu correspondente no verso desta fita?

\item {} 
Com base na resposta do item anterior, qual o resultado da soma de todos os números dos dois lados dessa fita?

\item {} 
A soma de todos os números em ambos os lados da fita é o resultado que queríamos obter?

\item {} 
Que operação devemos fazer com a soma de todos os números da fita para que ele seja o resultado da expressão
\begin{equation*}
\begin{split}1+2+3+4+5+ \cdots +147+148+149+150 \text{ ?}\end{split}
\end{equation*}
Qual é o valor dessa expressão?

\item {} 
Imagine agora uma outra fita que tenha em cada lado, todos os números de 1 até 100.

\begin{figure}[H]
\centering

\noindent\includegraphics[width=250bp]{{fita1}.jpg}
\end{figure}

Utilizando o mesmo raciocício, tente responder a mesma pergunta feita para a turma do jovem \textit{Gauss}, ou seja, quanto dá \(1+2+3+ \cdots +97+98+99+100\)?

\item {} 
E se a fita fosse até o número natural \(n\)?

\begin{figure}[H]
\centering

\noindent\includegraphics[width=250bp]{{fita2}.jpg}
\end{figure}

Com o que foi aprendido, obtenha uma expressão para o resultado da soma dos \(n\) primeiros números naturais. Ou seja, tente expressar em função de \(n\), o resultado de \(1+2+3+4+5+ \cdots +(n-3)+(n-2)+(n-1)+n\).

\end{enumerate}

\ifdefined\prof
\begin{solucao}

\begin{enumerate}
\item {} 
Sim.

\item {} 
\(150\); \(133\) e \(76\).

\item {} 
\begin{enumerate}[label=\Roman*)]
\item \(1+150=151\);

\item \(15+136=151\);

\item \(31+120=151\);

\item \(49+102=151\);

\item \(75+76=151\).
\end{enumerate}

\item {} 
\(151\).

\item {} 
\(150 \cdot 151=22650\).

\item {} 
Não.

\item {} 
Devemos dividir a soma obtida por \(2\); \(22650 \div 2=11325\).

\item {} 
As somas de cada número com seu correspondente no verso dá, agora, \(101\). Com isso, a soma de todos os números de ambos os lados dessa fita será \(100 \cdot 101\) e \((100 \cdot 101) \div 2 = 10100 \div 2 = 5050\).

\item {} 
As somas de cada número com seu correspondente no verso dá, agora, \(n+1\). Com isso, a soma de todos os números de ambos os lados dessa fita será \(n \cdot (n+1)\) e
\begin{equation*}
1+2+3+4+5+ \cdots +(n-3)+(n-2)+(n-1)+n=\frac{n \cdot (n+1)}{2} = \frac{n^2 + n}{2}=\frac{n^2}{2} + \frac{n}{2}
\end{equation*}

\end{enumerate}

\end{solucao}
\fi

\end{document}