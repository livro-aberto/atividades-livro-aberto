\documentclass[10 pt,usenames,dvipsnames, oneside]{article}
\usepackage{../../../modelo-ensino-medio}



\begin{document}

\begin{center}
  \begin{minipage}[l]{3cm}
\includegraphics[width=2cm]{logo}    
\end{minipage}\hfill
\begin{minipage}[r]{.8\textwidth}
 {\Large \scshape Atividade: Altura do arco da praça da Apoteose}  
\end{minipage}
\end{center}
\vspace{.2cm}

\ifdefined\prof
%Habilidades da BNCC
\begin{objetivos}
\item \textbf{EM12MT09} Reconhecer função quadrática e suas representações algébrica e gráfica, compreendendo o
modelo de variação determinando domínio, imagem, máximo e mínimo, e utilizar essas noções e
representações para resolver problemas como os de movimento uniformemente variado.
\end{objetivos}

%Caixa do Para o Professor
\begin{goals}
%Objetivos específicos
\begin{enumerate}
\item Relacionar, a partir de dados gráficos, qual a forma da função quadrática que melhor descreve a situação.

\item {} 
Associar situações concretas à forma da parábola e buscar soluções a partir da aplicação das ferramentas da função quadrática.

\item {} 
Inferir sobre a utilidade da função quadrática no cotidiano.

\item {} 
Distinguir em problemas concretos o papel de abscissa e ordenada para a representação gráfica da parábola.
\end{enumerate}

\tcblower

%Orientações e sugestões
Como um dos objetivos principais deste conjunto de atividades é trazer alguma sugestão de aplicação ou contextualização do tema, todos os enunciados apresentam um texto de motivação ou ambientação.

Além disso, as atividades tem um apelo mais visual, no sentido de conectar a função quadrática à sua representação gráfica, a parábola. Sendo assim, em todos os casos apresentados, o estudante deve fazer a escolha do sistema de coordenadas cartesiana q      ue melhor se adequa à situação apresentada e, a partir daí usar os conteúdos estudados para responder as questões sobre a situação apresentada.

Orientamos o professor a calcular o tempo das atividades levando em conta a possibilidade de um aluno não seguir o caminho que as respostas das atividades sugerem para que eles entendam as dificuldades que essas escolhas trazem. Isso será, certamente, tão importante quanto a resposta prevista em si.

Para cada sugestão gráfica apresentada para desenvolver o problema, deixe que os alunos argumente o porquê de suas escolhas e peça para aqueles que escolheram a forma que sugerimos nas respostas argumente sobre sua escolha.

Mesmo esse grupo de atividades sendo único, adicionamos à última atividade um texto sobre jogos eletrônicos e seus uso em sala de aula. Posto aqui, as orientações ficariam maiores do que o necessário e no tempo errado.
\end{goals}

\clearpage
\bigskip
\begin{center}
{\large \scshape Atividade}
\end{center}
\fi

A passarela Professor Darcy Ribeiro \((\star 1922, \dagger 1997)\), mais conhecida como Sambódromo, fica na cidade do Rio de Janeiro e foi construida em 1984. Com projeto arquitetônico de Oscar Niemeyer \((\star 1907, \dagger 2012)\), ela foi concebida para ser o local fixo de uma das maiores festas populares do Brasil, o Carnaval. Ao final da passarela, encontra-se a praça da apoteose, com o museu do samba e um enorme arco cujo formato lembra o de uma parábola.

\begin{figure}[H]
\centering


\noindent\includegraphics[width=275bp]{{Apoteose_do_Tiro_com_Arco}.jpg}
\caption{Foto de \href{https://commons.wikimedia.org/wiki/File:Apoteose\_do\_Tiro\_com\_Arco.jpg}{Jorge Mello} CC-BY-SA}\label{\detokenize{AF209-9:id3}}\end{figure}

Em \(2011\), pela primeira vez desde a construção, a prefeitura providenciou a limpeza do arco.

\begin{figure}[H]
\centering


\noindent\includegraphics[width=150bp]{{02_15_gvg_rio_lavagem10}.jpg}
\caption{\href{https://extra.globo.com/noticias/rio/banho-nos-arcos-do-sambodromo-1077277.html}{Banho nos arcos do Sambódromo}}\label{\detokenize{AF209-9:id4}}\end{figure}

A empresa que foi contratada para fazer essa limpeza, precisou ter uma estimativa da altura do arco, com a finalidade de saber se seu equipamento seria suficiente para a tarefa, já que a altura máxima que o equipamento suportaria, seria de \(40\) m de altura. Uma busca rápida na internet não forneceu o resultado esperado, apenas que o comprimento da base é de \(50\) m. Sendo assim, a estimativa teve que ser feita através de cálculos. Admitindo por aproximação que o arco seja parabólico, faça o que se pede:
\begin{enumerate}
\item {} 
Quantas informações concretas são fornecidas para esta parábola?

\item {} 
Caso você soubesse a função que descreve essa parábola, você seria capaz de  determinar a altura aproximada do arco?

\item {} 
Dentre as opções a seguir marque a que faz o rascunho do arco no plano cartesiano.


\begin{multicols}{3}
\begin{center}\begin{tikzpicture}[scale=.5]

  \draw [help lines, secundario!30, step=.5] (-2,-1.5) grid (6,5);
       \draw [, ->] (-2,0) -- (6,0) node [below left, scale=0.5] {$x$};
       \draw [, ->] (0,-1.5) -- (0,5) node [below left, scale=0.3] {$y$};
  \draw [color=\currentcolor!80,  , domain=-0.35:5.35] plot (\x,{-0.8*(\x)^2+4*(\x)});
  \node [above, align=center] at (2.5,5) {Figura 1};
\end{tikzpicture}\end{center}\begin{center}\begin{tikzpicture}
[scale=.5]
  \draw [help lines, secundario!30, step=.5] (-4,-1.5) grid (4,5);
  \draw [, ->] (-4,0) -- (4,0) node [below left, scale=0.5] {$x$};
       \draw [, ->] (0,-1.5) -- (0,5) node [below left, scale=0.3] {$y$};
             \draw [color=\currentcolor!80,, domain=-3.15:3.15] plot (\x,{-0.4*(\x)^2+4});
       \node [above, align=center] at (0,5) {Figura 2};
\end{tikzpicture}\end{center}\begin{center}\begin{tikzpicture}
[scale=.5]
       \draw [help lines, secundario!30, step=.5] (-4,-5) grid (4,1.5);
       \draw [, ->] (-4,0) -- (4,0) node [below left, scale=0.5] {$x$};
       \draw [, ->] (0,-5) -- (0,1.5) node [below left, scale=0.3] {$y$};
       \draw [color=\currentcolor!80, , domain=-3.5:3.5] plot (\x,{-0.4*(\x)^2});
  \node [above, align=center] at (0,1.5) {Figura 3};
\end{tikzpicture}\end{center}

\end{multicols}


\item {} 
Para essa escolha, qual o significado dos valores de \(x\) e de \(y\)?

\item {} 
Que pontos do plano cartesiano são conhecidos, se juntarmos a escolha gráfica com os dados fornecidos sobre o arco?

\item {} 
Com base em sua escolha do rascunho gráfico mais adequado e considerando os pontos conhecidos da parábola, qual forma da função quadrática resulta em maior quantidade de informações conhecidas?

\(\Box \; f(x)=ax^2+bx+c\)

\(\Box \; f(x)=a(x-p)^2+q\)

\(\Box \; f(x)=a(x-x_1)(x-x_2)\)

\item {} 
Quantos dados estão faltando para que seja conhecida a função que descreve esta parábola?

\item {} 
Com o auxílio da calculadora gráfica em: Estimando a parábola(\url{https://ggbm.at/VFR6nWHM}) obtenha a informação que falta para obter a função que descreve a parábola.

\item {} 
Qual a altura estimada para a altura do arco?

\item {} 
A empresa contratada para a limpeza do arco teve capacidade de concluir o serviço com o equipamento que possuia?

\end{enumerate}

\ifdefined\prof
\begin{solucao}

\begin{enumerate}
\item {} 
Apenas o comprimento da base, de \(50\) m.

\item {} 
Sim, seria a imagem do vértice.

\item {} 
Figura 2, pois a base do “arco” foi rascunhado sobre o eixo \(x\) e a altura procurada está sobre o eixo \(y\).

\item {} 
\(x\) pontos na base do arco e \(y\) medidas referentes às alturas de cada ponto da base do arco.

\item {} 
\((-25,0)\) e \((25,0)\).

\item {} 
\(f(x)=a(x-x_1)(x-x_2)\)

\item {} 
Apenas um, o \(a\).

\item {} 
\(a=-0\text{,}04\), portanto \(f(x)=-0\text{,}04(x+25)(x-25)\).

\item {} 
A altura aproximada do arco acontece para \(x=0\). Assim, \(f(0)=-0\text{,}04(0+25)(0-25)=-0\text{,}04 \dot (-625)=25\) m.

\item {} 
Sim.

\end{enumerate}

\end{solucao}
\fi

\end{document}