\documentclass[10 pt,usenames,dvipsnames, oneside]{article}
\usepackage{../../../modelo-ensino-medio}



\begin{document}

\begin{center}
  \begin{minipage}[l]{3cm}
\includegraphics[width=2cm]{logo}    
\end{minipage}\hfill
\begin{minipage}[r]{.8\textwidth}
 {\Large \scshape Atividade: A jogada vencedora}  
\end{minipage}
\end{center}
\vspace{.2cm}

\ifdefined\prof
%Habilidades da BNCC
% \begin{objetivos}
% \item 
% \end{objetivos}

%Caixa do Para o Professor
\begin{sugestions}
%Objetivos específicos

Os jogos eletrônicos constituem ótimos laboratórios de aprendizagem por simular situações que podem ir assumindo toda a complexidade da realidade aos poucos, uma “variável” por vez. De acordo com {[}WANG{]}, jogos de computador podem criar ambientes e mundos que de outra forma seriam inacessíveis aos estudantes.

Existe disponível na internet diversos projetos que envolvem o uso do jogo Angry Birds para o estudo das parábolas e lançamentos oblíquos. Por exemplo, \href{https://algebra2coach.com/transforming-parabolas-angry-birds-project/}{Transforming Parabolas \textendash{} The Angry Birds Project} e \href{https://www.tes.com/teaching-resource/angry-bird-parabolas-graphing-quadratic-equations-6165424}{Transforming Parabolas \textendash{} The Angry Birds Project}.
\end{sugestions}

\bigskip
\begin{center}
{\large \scshape Atividade}
\end{center}
\fi

Vamos trabalhar aqui com um famoso jogo que simula lançamento de objetos. No caso, são “pássaros” caricaturados em formato de personagens de cinema que tem que impedir o plano dos “porcos verdes” de roubarem seus ovos e trazer destruição ao universo. A “variável” resistência do ar, por exemplo, não está incluída em boa parte das fases deste jogo.

Digamos que o programador de uma das fases decida, dentre todos os possíveis lançamentos, um que forneça a maior quantidade de pontos possível para a fase. Entendendo a tela como um plano cartesiano, o programador deve escolher a parábola que representará a “Jogada Vencedora”. A figura a seguir ilustra a situação.

\begin{figure}[H]
\centering

\noindent\includegraphics[width=300bp]{{AB_Plano_Cartesiano}.png}
\caption{Imagem de divulgação.}\label{\detokenize{AF209-9:id6}}\end{figure}

Com a finalidade de inserir na programação a função que descreve a “Jogada Vencedora” o programador usou três coordenadas como referência: o pássaro e os dois “sóis”, cujas coordenadas estão destacadas a seguir.

\begin{figure}[H]
\centering

\noindent\includegraphics[width=300bp]{{AB_Coordenadas}.png}
\end{figure}
\begin{enumerate}
\item {} 
Quais são as coordenadas indicadas no gráfico pelo programador?

\item {} 
Quais os significados dos valores de \(x\) e de \(y\) neste contexto?

\item {} 
Das formas da função quadrática apresentadas a seguir, qual delas parece mais adequada diante das informações fornecidas?

\(\Box \; f(x)=ax^2+bx+c\)

\(\Box \; f(x)=a(x-p)^2+q\)

\(\Box \; f(x)=a(x-x_1)(x-x_2)\)

\item {} 
Substituido a origem na forma escolhida do item anterior, qual a conclusão?

\item {} 
Faça o mesmo para as outras duas coordenadas, mas considere também o que você concluiu no item anterior, e obtenha duas equações diferentes com variáveis \(a\) e \(b\).

\item {} 
Nas equações apresentadas no item anterior, uma tem o \(49\) e a outra tem o \(25\). Na que tem o \(49\), multiplique toda ela por \(25\) e, na outra, a que tem o \(25\), multiplique toda ela por \(49\). Feito isso, subtrai, membro a membro, as duas equações resultantes. Qual a conclusão?

\item {} 
Mais uma vez vamos pegar as equações do item ‘e’. Repare que uma tem um coeficiente \(7\) e a outra tem um coeficiente \(5\). Multiplique a que tem o \(7\) por \(5\) e a que tem o \(5\), por \(7\). Depois subtrai, membro a membro, as equações assim obtidas. Qual a conclusão?

\item {} 
Qual a função que o programador vai inserir como a “Jogada Vencedora”?

\end{enumerate}

\ifdefined\prof
\begin{solucao}

\begin{enumerate}
\item {} 
\((0,0)\), \((5,3)\) e \((7,1)\).

\item {} 
\(x\) será o deslocamento horizontal do pássaro após o lançamento e \(y\) será a altura do pássaro em relação ao eixo \(x\) durante o arremesso.

\item {} 
\(f(x)=ax^2+bx+c\).

\item {} 
\(f(0)=a \cdot 0^2+b \cdot 0+c=0 \Rightarrow c=0\).

\item {} \begin{align*}\!\begin{aligned}
f(5)& =a \cdot 5^2+b \cdot 5+0=3 \Rightarrow 25a+5b=3 \\\\
f(7)& =a \cdot 7^2+b \cdot 7+0=1 \Rightarrow 49a+7b=1 \\\\
\end{aligned}\end{align*}
\item {} \begin{align*}\!\begin{aligned}
49 \cdot 25a+ 49 \cdot 5b= 49 \cdot 3 & \Rightarrow 1225a+245b=147 \\\\
25 \cdot 49a+ 25 \cdot 7b = 25 \cdot 1 & \Rightarrow 1225a+175b=25 \\\\
(245-175) \cdot b = 147-25 & \Rightarrow b= \frac{122}{70} \Rightarrow b= \frac{61}{35} \\\\
\end{aligned}\end{align*}
\item {} \begin{align*}\!\begin{aligned}
7 \cdot 25a+ 7 \cdot 5b= 7 \cdot 3 & \Rightarrow 175a+35b=21 \\\\
5 \cdot 49a+ 5 \cdot 7b = 5 \cdot 1 & \Rightarrow 245a+35b=5 \\\\
(245-175) \cdot a = 5-21 & \Rightarrow a= - \frac{16}{70} \Rightarrow a=- \frac{8}{35} \\\\
\end{aligned}\end{align*}
\item {} 
\(f(x)= - \frac{8}{35}x^2+ \frac{61}{35}x\).

\end{enumerate}
\end{solucao}
\fi

\end{document}