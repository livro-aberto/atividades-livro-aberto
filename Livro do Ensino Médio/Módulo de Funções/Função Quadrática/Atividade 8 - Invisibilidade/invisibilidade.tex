\documentclass[10 pt,usenames,dvipsnames, oneside]{article}
\usepackage{../../../modelo-ensino-medio}



\begin{document}

\begin{center}
  \begin{minipage}[l]{3cm}
\includegraphics[width=2cm]{logo}    
\end{minipage}\hfill
\begin{minipage}[r]{.8\textwidth}
 {\Large \scshape Atividade: Invisibilidade}  
\end{minipage}
\end{center}
\vspace{.2cm}

\ifdefined\prof
%Habilidades da BNCC
\begin{objetivos}
\item \textbf{EM12MT09} Reconhecer função quadrática e suas representações algébrica e gráfica, compreendendo o
modelo de variação determinando domínio, imagem, máximo e mínimo, e utilizar essas noções e
representações para resolver problemas como os de movimento uniformemente variado.
\end{objetivos}

%Caixa do Para o Professor
\begin{goals}
%Objetivos específicos
\begin{enumerate}
\item {} 
Construir estratégia de resolução que dependa da identificação do que não é solução antes da conclusão.

\item {} 
Aplicar a definição geométrica de parábola numa situação prática;
\end{enumerate}

\tcblower

%Orientações e sugestões
Problemas geométricos que envolvem distâncias de um ponto a outro são, em geral, tratados com o recurso das circunferências e suas propriedades. Aqui desejamos ampliar as ferramentas de solução de problemas desse tipo, ajudando o estudante a incorporar como método de solução de problemas de distâncias, a parábola.
\begin{itemize}
\item {} 
Comece a resolver os pontos que parecem óbvios.

\item {} 
Quando o estudante tender a achar que todos são óbvios, sugira que ele utilize régua ou compasso para tentar comprovar a sua intuição.

\item {} 
Para motivar a solução dada pelos autores, sugira outros pontos, próximos da interseção da parábola com a linha contínua na parte superior da imagem.

\item {} 
Questione se existe uma região onde a resposta seria: “Tanto faz”! Estimule os estudantes a exibir ou descrever essa região. Se necessário, fornaça-lhe como opções as definições de circunferência, reta mediatriz e parábola.

\end{itemize}
\end{goals}

\bigskip
\begin{center}
{\large \scshape Atividade}
\end{center}
\fi

Num jogo eletrônico em que você controla um oficial militar infiltrado. Dentre as fases de treinamento tático há uma que exibe um salão vigiado por câmeras.

\begin{figure}[H]
\centering

\noindent\includegraphics[width=200bp]{{MGS_1998_PS_Espreita}.jpg}
\caption{Imagem de divulgação.}\label{\detokenize{AF209-6:id2}}\end{figure}

Como as câmeras fazem movimento de vai e vem, é possível atravessar o salão sem ser detectado, e esse é o objetivo desta fase. A imagem a seguir mostra a vista de cima desta fase.

\begin{figure}[H]
\centering

\begin{tikzpicture}[every node/.style={scale=2.5}, scale=.9]

\draw [color=black,fill=black, fill opacity=1]  (0,0) rectangle (10,6.02);
\draw[color=secundario, fill=secundario, fill opacity=1] (0,1) rectangle (10,2.4);
\draw[color=secundario, fill=secundario, fill opacity=1] (0,2) rectangle (9.7,3);
\draw[color=secundario, fill=secundario, fill opacity=1] (0,3) rectangle (10,3.8);
\draw[color=secundario, fill=secundario, fill opacity=1] (0.3,3.8) rectangle (10,4.3);
\draw[color=secundario, fill=secundario, fill opacity=1] (0,4.3) rectangle (0.5,4.8);
\draw[color=secundario, fill=secundario, fill opacity=1] (0.5,4.3) rectangle (10,4.8);
\path [fill=secundario, fill opacity=1]  (0,4.8) to  (4,6) -- (4.5,6)--(4.5,5.4)-- (4.8,5.4) ;
\path [fill=secundario, fill opacity=1]  (10,4.8)  to  (6,6) -- (5.5,6) -- (5.5,5.8) --  (5.5,5.4);
\path [fill=secundario, fill opacity=1]  (0,4.8) to  (10,4.8) -- (5.3,5.43) --  (4.8,5.43) ;
\draw[color=secundario, fill=secundario, fill opacity=1] (4.5,5.4) rectangle (5.5,5.9);
\draw [color=\currentcolor!80, dashed] (0,1) -- (10,1);
\draw [color=\currentcolor!80] (0,4.8) -- (4,6) -- (4.5,6) -- (4.5,5.8);
\draw [color=\currentcolor!80] (10,4.8) -- (6,6) -- (5.5,6) -- (5.5,5.8);
\draw [color=\currentcolor!80]  (5.5,5.6) -- (5.5,5.4) -- (5.3,5.4);
\draw [color=\currentcolor!80]  (4.5,5.6) -- (4.5,5.4) -- (4.8,5.4);
\draw [color=\currentcolor!80]  (0,4.3) -- (0.3,4.3) -- (0.3,4.1);
\draw [color=\currentcolor!80]  (0,3.8) -- (0.3,3.8) -- (0.3,4);
\draw [color=\currentcolor!80]  (10,3) -- (9.7,3) -- (9.7,2.8);
\draw [color=\currentcolor!80]  (10,2.4) -- (9.7,2.4) -- (9.7,2.6);
\draw [color=black, fill opacity=1, fill=black] (5,5.65) circle (0.15);
\node [ponto, color=destacado] at (8,4) {};
\node [ponto, color=destacado] at (7.5,2.5) {};
\node [ponto, color=destacado] at (5,2.8) {};
\node [ponto, color=destacado] at (3,4) {};
\node [ponto, color=destacado] at  (2.5,5)  {};
\node [ponto, color=destacado] at  (2.2,2)  {};
\node [below right, color=white,scale=.4] at (5,5.4) {Q};
\node [above right, color=white,scale=.4] at (0.5,4.3) {S};
\node [above left, color=white,scale=.4] at (10,3) {E};
\node [above left, color=white,scale=.4] at (8,4) {1};
\node [above left, color=white,scale=.4] at (7.5,2.5) {2};
\node [below left, color=white,scale=.4] at (5,2.8) {3};
\node [above left, color=white,scale=.4] at (3,4) {4};
\node [above right, color=white,scale=.4] at (2.5,5) {5};
\node [above left, color=white,scale=.4] at (2.2,2) {6};
\node [right, color=white,scale=.4] at (4,0.5) {Regi\~ao H};
\end{tikzpicture}
\end{figure}

A região em cinza é uma região que, em algum momento, pode ser enxergado por câmera durante o movimento de vai e vem. A linha verde contínua representa alguma barreira intransponível; já as linhas tracejadas podem ser ultrapassadas pelo personagem para se abrigar das câmeras e terminar a fase. Em ‘E’ o personagem entra no cenário essa passagem se fecha, em ‘S’ ele sai e vence a fase.

Os pontos em vermelho são posições possíveis para o personagem que, percebendo a proximidade do olhar de alguma das câmeras deve correr e se esconder numa região em preto. Sendo assim, para cada posição do personagem, diga para onde ele deve correr: Região horizontal ‘H’ ou Região quadrada ‘Q’.

\ifdefined\prof
\begin{solucao}

Traçando uma reta no limite da região ‘H’ e usando o ponto ‘Q’, pode-se traçar os pontos do salão que equidistam de ‘H’ ou ‘Q’, que é a parábola. Assim, as melhores chances de fuga se dão para:
\begin{figure}[H]
\centering

\begin{tikzpicture}[every node/.style={scale=2.5}, scale=.9]

\draw [color=black,fill=black, fill opacity=1]  (0,0) rectangle (10,6.02);
\draw[color=secundario, fill=secundario, fill opacity=1] (0,1) rectangle (10,2.4);
\draw[color=secundario, fill=secundario, fill opacity=1] (0,2) rectangle (9.7,3);
\draw[color=secundario, fill=secundario, fill opacity=1] (0,3) rectangle (10,3.8);
\draw[color=secundario, fill=secundario, fill opacity=1] (0.3,3.8) rectangle (10,4.3);
\draw[color=secundario, fill=secundario, fill opacity=1] (0,4.3) rectangle (0.5,4.8);
\draw[color=secundario, fill=secundario, fill opacity=1] (0.5,4.3) rectangle (10,4.8);
\path [fill=secundario, fill opacity=1]  (0,4.8) to  (4,6) -- (4.5,6)--(4.5,5.4)-- (4.8,5.4) ;
\path [fill=secundario, fill opacity=1]  (10,4.8)  to  (6,6) -- (5.5,6) -- (5.5,5.8) --  (5.5,5.4);
\path [fill=secundario, fill opacity=1]  (0,4.8) to  (10,4.8) -- (5.3,5.43) --  (4.8,5.43);
\draw[color=secundario, fill=secundario, fill opacity=1] (4.5,5.4) rectangle (5.5,5.9);
\draw [color=primario, dashed] (0,1) -- (10,1);
\draw [color=primario] (0,4.8) -- (4,6) -- (4.5,6) -- (4.5,5.8);
\draw [color=primario] (10,4.8) -- (6,6) -- (5.5,6) -- (5.5,5.8);
\draw [color=primario]  (5.5,5.6) -- (5.5,5.4) -- (5.3,5.4);
\draw [color=primario]  (4.5,5.6) -- (4.5,5.4) -- (4.8,5.4);
\draw [color=primario]  (0,4.3) -- (0.3,4.3) -- (0.3,4.1);
\draw [color=primario]  (0,3.8) -- (0.3,3.8) -- (0.3,4);
\draw [color=primario]  (10,3) -- (9.7,3) -- (9.7,2.8);
\draw [color=primario]  (10,2.4) -- (9.7,2.4) -- (9.7,2.6);
\draw [color=black, fill opacity=1, fill=black] (5,5.65) circle (0.15);
\node [ponto, color=destacado] at (8,4) {};
\node [ponto, color=destacado] at (7.5,2.5) {};
\node [ponto, color=destacado] at (5,2.8) {};
\node [ponto, color=destacado] at (3,4) {};
\node [ponto, color=destacado] at  (2.5,5)  {};
\node [ponto, color=destacado] at  (2.2,2)  {};
\node [below right, color=white,scale=.4] at (5,5.4) {Q};
\node [above right, color=white,scale=.4] at (0.5,4.3) {S};
\node [above left, color=white,scale=.4] at (10,3) {E};
\node [above left, color=white,scale=.4] at (8,4) {1};
\node [above left, color=white,scale=.4] at (7.5,2.5) {2};
\node [below left, color=white,scale=.4] at (5,2.8) {3};
\node [above left, color=white,scale=.4] at (3,4) {4};
\node [above right, color=white,scale=.4] at (2.5,5) {5};
\node [above left, color=white,scale=.4] at (2.2,2) {6};
\node [right, color=white,scale=.4] at (4,0.5) {Regi\~ao H};              \draw[color=atento, very thick] (0,6) .. controls  (2.2,2.6) and  (8,2.6) .. (10,6);
\draw[color=atento, very thick] (-1,1)--(11,1);
\end{tikzpicture}
\end{figure}


\(1)\) Região ‘H’.

\(2)\) Região ‘H’.

\(3)\) Região ‘H’.

\(4)\) Região ‘Q’

\(5)\) Região ‘Q’

\(6)\) Região ‘H’

\end{solucao}
\fi

\end{document}