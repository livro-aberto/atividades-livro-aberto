\documentclass[10 pt,usenames,dvipsnames, oneside]{article}
\usepackage{../../../modelo-ensino-medio}



\begin{document}

\begin{center}
  \begin{minipage}[l]{3cm}
\includegraphics[width=2cm]{logo}    
\end{minipage}\hfill
\begin{minipage}[r]{.8\textwidth}
 {\Large \scshape Atividade: Perímetro Fixo}  
\end{minipage}
\end{center}
\vspace{.2cm}

\ifdefined\prof
%Habilidades da BNCC
% \begin{objetivos}
% \item 
% \end{objetivos}

%Caixa do Para o Professor
\begin{goals}
%Objetivos específicos
Prezado colega esta atividade tem como objetivo aplicar o conceito de otimização em função quadrática num contexto geométrico, sem a utilização do gráfico da função nem muito menos da curva denominada parábola, para isso pretendemos:

\begin{enumerate}
\item explorar a situação através do uso, já corriqueiro, de preenchimento de um quadro.
\item modelar a situação utilizando álgebra de maneira simples e guiada.
\item apresentar e explorar a técnica de completar quadrados para passarmos a função quadrática encontrada da forma polinomial para a forma canônica, sem obrigatoriamente citar esses termos.
\item utilizar a apresentação da forma canônica para identificarmos os valores de área máxima e os valores que maximizam essa área, convidando seu aluno à fazer inferências apenas aritméticas na forma encontrada.
\end{enumerate}


\end{goals}

\bigskip
\begin{center}
{\large \scshape Atividade}
\end{center}
\fi

Imagine que você tenha um pedaço de barbante de \(12\) cm de comprimento e queira moldar um retângulo com ele e calcular sua área. A figura abaixo ajuda a ilustrar a situação.

\begin{figure}[H]
\centering

\noindent\includegraphics[width=150bp]{{maos}.jpg}
\end{figure}
\begin{enumerate}
\item {} 
A situação em questão envolve quatro grandezas, aponte quais são.

\item {} 
Quais grandezas descritas acima variam e quais não variam?

\item {} 
Numa folha de papel ou similar, copie a tabela a seguir e complete-a.

\begin{table}[H]
\centering
\begin{tabu} to \textwidth{|c|c|c|}
\hline
\thead
Base & Altura & Área \\
\hline
$0$ & & \\
\hline
$2$ & & \\
\hline
$4$ & & \\
\hline
$6$ & & \\
\hline
\end{tabu}
\end{table}


\item {} 
O que ocorreu com a área para os valores da base iguais a \(0\) e \(6\)?  Esses valores devem ser considerados em nossa análise da situação?

\item {} 
Quais as medidas da base do retângulo que apresentaram área máxima no quadro acima?

\item {} 
Assumindo a base do retângulo como \(x\), e sua altura como \(h(x)\), exiba uma expressão algébrica que representa a medida da altura desse retângulo em função de \(x\). A expressão \(h(x)\), encontrada pode ser considerada uma função afim? Com que domínios e imagens?

\item {} 
Assumindo a base do retângulo como \(x\), a altura \(h(x)\) encontrada no item anterior e sua área como \(A(x)\), exiba uma expressão que apresente a área deste retângulo em função de \(x\).

\item {} 
Verifique se a relação encontrada pode ser dada por \(A(x)=-(x^2-6x)\), caso contrário refaça os itens anteriores.

\item {} 
A expressão \(A(x)\), encontrada pode ser considerada uma função afim? Por quê?

\item {} 
Observe que a relação apresentada no item \titem{h)}, possui dentro do parênteses um binômio que pode ser parte de um trinômio quadrado perfeito, qual seria o terceiro termo que faria o binômio se transformar num trinômio quadrado perfeito?

\item {} 
Agora repita a relação: \(A(x)=-(x^2-6x+\Box -\Box)\) acrescentando e retirando o número encontrado no item anterior.

\item {} 
Ao fatorar a relação do item anterior podemos recair na forma: \(A(x)=a(x-p)^2+q\), quais os valores de a, p e q, que foram encontrados neste processo de fatoração?

\item {} 
Levando em consideração a forma apresentada no item anterior, e ao analisarmos apenas o termo \((x-p)^2\), Existe algum valor de \(x\) que torne a expressão negativa? e qual valor de \(x\) torna a expressão nula?

\item {} 
Ao analisarmos \(A(x)=-(x-3)^2+9\), existe algum valor de \(x\) que faça \(A(x)\) ser maior que \(9\)? Por quê?

\item {} 
Qual a área máxima do Retângulo?

\item {} 
Qual o valor de \(x\), que gera a área máxima?

\end{enumerate}

\ifdefined\prof
\clearpage
\begin{solucao}

\begin{enumerate}
\item {} 
No retângulo temos as medidas de: \textbf{perímetro}, \textbf{área}, \textbf{base} e \textbf{altura}.

\item {} 
O perímetro não varia, e a área, a base e a altura variam.

\item {} 
Segue o quadro preenchido:

\begin{table}[H]
\centering
\begin{tabular}{|c|c|c|}
\hline
\tcolor{Base} & \tcolor{Altura} & \tcolor{Área} \\
\hline
$0$ & $6$ & $0$ \\
\hline
$2$ & $4$ & $8$ \\
\hline
$4$ & $2$ & $8$ \\
\hline
$6$ & $0$ & $0$\\
\hline
\end{tabular}
\end{table}

\item {} 
A área foi nula. Eles não devem ser considerados, pois não existem retângulos cujas medidas dos lados sejam nulas.

\item {} 
Considerando a base como \(x\) temos \(x=2\) ou \(x=4\).

\item {} 
\(h(x)=6-x\) . Sim, com: \(h:]0,6[\to]0,\infty[\).

\item {} 
\(A(x)=x(6-x) \to A(x)=-(x^2-6x)\).

\item {} 
Verificação.

\item {} 
Não por vários motivos, seguem alguns:

\begin{enumerate}[label=\arabic*)]
\item a função afim é sempre monótona (sempre crescente ou sempre decrescente), os valores da última coluna do quadro nos mostram que ora \(A(x)\) é crescente ora é decrescente.

\item a função afim apresenta taxa de variação constante, já \(A(x)\) não apresenta, pois: \(\frac{5-0}{1-0}=5\) e \(\frac{8-5}{2-1}=3\).
\end{enumerate}

\item {} 
\(9\).

\item {} 
\(A(x)=-(x^2-6x+9-9)\).

\item {} 
\(A(x)=-(x^2-6x+9-9)=-(x^2-6x+9)+9=-(x-3)^2+9\) , com \(a=-1\) ; \(p=3\) e \(q=9\).

\item {} 
Não existe. \(x=p\).

\item {} 
Não. Pois para quaisquer valores de \(x\), \((x-3)^2\) sempre será positivo, e consequentemente \(-(x-3)^2\) será sempre negativo, e se esse valor negativo for somado com \(9\) o resultado obrigatoriamente será menor que \(9\).

\item {} 
\(9cm^2\).

\item {} 
\(3cm\).

\end{enumerate}
\end{solucao}
\fi

\end{document}