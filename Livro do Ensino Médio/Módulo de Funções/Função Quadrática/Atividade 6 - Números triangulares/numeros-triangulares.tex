\documentclass[10 pt,usenames,dvipsnames, oneside]{article}
\usepackage{../../../modelo-ensino-medio}



\begin{document}

\begin{center}
  \begin{minipage}[l]{3cm}
\includegraphics[width=2cm]{logo}    
\end{minipage}\hfill
\begin{minipage}[r]{.8\textwidth}
 {\Large \scshape Atividade: Números triangulares}  
\end{minipage}
\end{center}
\vspace{.2cm}

\ifdefined\prof
% %Habilidades da BNCC
\begin{objetivos}
\item \textbf{EM12MT09} Reconhecer função quadrática e suas representações algébrica e gráfica, compreendendo o
modelo de variação determinando domínio, imagem, máximo e mínimo, e utilizar essas noções e
representações para resolver problemas como os de movimento uniformemente variado.
\end{objetivos}

% %Caixa do Para o Professor
\begin{goals}
%Objetivos específicos
\begin{enumerate}
\item Reconhecer a função quadrática na expressão que dá a soma dos primeiros termos de uma progressão aritmética.
\item Resolver o problema de somar os primeiros termos de uma progressão aritmética com as ferramentas da função quadrática.
\end{enumerate}

% \tcblower

% %Orientações e sugestões
% \begin{itemize}
% \item 
% \end{itemize}
\end{goals}

\bigskip
\begin{center}
{\large \scshape Atividade}
\end{center}
\fi

No \hyperref[chap-funcoes]{capítulo de Introdução às Funções}, uma das \hyperref[numeros-triangulares-funcoes]{atividades} sugere que você determine a relação entre uma sequência de figuras e a quantidade de pontos usados para compor cada figura.

\begin{figure}[H]
\centering

\begin{tikzpicture}
\begin{scope}
\draw [fill=black] (0.,0.) circle (1.0pt);
\draw [fill=black] (0.5,0.) circle (1.0pt);
\draw [fill=black] (0.5,0.5) circle (1.0pt);
\draw [fill=black] (0.,0.5) circle (1.0pt);
\draw [fill=black] (1.5,0.) circle (1.0pt);
\draw [fill=black] (2.,0.) circle (1.0pt);
\draw [fill=black] (2.,0.5) circle (1.0pt);
\draw [fill=black] (1.5,0.5) circle (1.0pt);
\draw [fill=black] (2.5,0.) circle (1.0pt);
\draw [fill=black] (2.5,1.) circle (1.0pt);
\draw [fill=black] (1.5,1.) circle (1.0pt);
\draw [fill=black] (2.,1.) circle (1.0pt);
\draw [fill=black] (2.5,0.5) circle (1.0pt);
\draw [fill=black] (3.5,0.) circle (1.0pt);
\draw [fill=black] (4.,0.) circle (1.0pt);
\draw [fill=black] (4.,0.5) circle (1.0pt);
\draw [fill=black] (3.5,0.5) circle (1.0pt);
\draw [fill=black] (4.5,0.) circle (1.0pt);
\draw [fill=black] (4.5,1.) circle (1.0pt);
\draw [fill=black] (3.5,1.) circle (1.0pt);
\draw [fill=black] (5.,0.) circle (1.0pt);
\draw [fill=black] (5.,1.5) circle (1.0pt);
\draw [fill=black] (3.5,1.5) circle (1.0pt);
\draw [fill=black] (4.,1.) circle (1.0pt);
\draw [fill=black] (4.5,0.5) circle (1.0pt);
\draw [fill=black] (5.,0.5) circle (1.0pt);
\draw [fill=black] (5.,1.) circle (1.0pt);
\draw [fill=black] (4.5,1.5) circle (1.0pt);
\draw [fill=black] (4.,1.5) circle (1.0pt);
\draw [fill=black] (-1.,0.) circle (1.0pt);
\end{scope}
\end{tikzpicture}\hfill
\begin{tikzpicture}
\begin{scope}
\draw [fill=black] (-1.,0.) circle (1.0pt);
\draw [fill=black] (0.,0.) circle (1.0pt);
\draw [fill=black] (0.5,0.) circle (1.0pt);
\draw [fill=black] (0.6545084971874737,0.4755282581475766) circle (1.0pt);
\draw [fill=black] (0.25,0.7694208842938133) circle (1.0pt);
\draw [fill=black] (-0.15450849718747367,0.4755282581475768) circle (1.0pt);
\draw [fill=black] (2.,0.) circle (1.0pt);
\draw [fill=black] (2.5,0.) circle (1.0pt);
\draw [fill=black] (2.6545084971874737,0.4755282581475766) circle (1.0pt);
\draw [fill=black] (2.25,0.7694208842938133) circle (1.0pt);
\draw [fill=black] (1.8454915028125263,0.4755282581475768) circle (1.0pt);
\draw [fill=black] (3.,0.) circle (1.0pt);
\draw [fill=black] (3.3090169943749475,0.9510565162951532) circle (1.0pt);
\draw [fill=black] (2.5,1.5388417685876266) circle (1.0pt);
\draw [fill=black] (1.6909830056250525,0.9510565162951536) circle (1.0pt);
\draw [fill=black] (4.,0.) circle (1.0pt);
\draw [fill=black] (3.1545084971874737,0.4755282581475766) circle (1.0pt);
\draw [fill=black] (2.9045084971874737,1.2449491424413899) circle (1.0pt);
\draw [fill=black] (2.0954915028125263,1.24494914244139) circle (1.0pt);
\draw [fill=black] (4.5,0.) circle (1.0pt);
\draw [fill=black] (4.654508497187473,0.4755282581475766) circle (1.0pt);
\draw [fill=black] (4.25,0.7694208842938133) circle (1.0pt);
\draw [fill=black] (3.8454915028125263,0.4755282581475768) circle (1.0pt);
\draw [fill=black] (5.,0.) circle (1.0pt);
\draw [fill=black] (5.3090169943749475,0.9510565162951532) circle (1.0pt);
\draw [fill=black] (4.5,1.5388417685876266) circle (1.0pt);
\draw [fill=black] (3.6909830056250525,0.9510565162951536) circle (1.0pt);
\draw [fill=black] (5.154508497187473,0.4755282581475766) circle (1.0pt);
\draw [fill=black] (4.904508497187473,1.2449491424413899) circle (1.0pt);
\draw [fill=black] (4.095491502812527,1.24494914244139) circle (1.0pt);
\draw [fill=black] (5.5,0.) circle (1.0pt);
\draw [fill=black] (5.963525491562422,1.42658477444273) circle (1.0pt);
\draw [fill=black] (4.75,2.3082626528814396) circle (1.0pt);
\draw [fill=black] (3.5364745084375793,1.4265847744427305) circle (1.0pt);
\draw [fill=black] (5.654508497187474,0.4755282581475767) circle (1.0pt);
\draw [fill=black] (5.837430563354646,0.9408663263400823) circle (1.0pt);
\draw [fill=black] (5.557545365872711,1.7215466012812657) circle (1.0pt);
\draw [fill=black] (5.174750541804863,2.046031522986149) circle (1.0pt);
\draw [fill=black] (4.3461569097741055,2.014853473191221) circle (1.0pt);
\draw [fill=black] (3.942313819548215,1.7214442935010055) circle (1.0pt);
\end{scope}
\end{tikzpicture}
\end{figure}

\begin{figure}[H]
\centering

\begin{tikzpicture}
\begin{scope}
\draw [fill=black] (-1.,0.) circle (1.0pt);\draw [fill=black] (0.,0.) circle (1.0pt);\draw [fill=black] (-0.5,0.) circle (1.0pt);\draw [fill=black] (0.25,0.43301270189221935) circle (1.0pt);\draw [fill=black] (0.,0.8660254037844388) circle (1.0pt);\draw [fill=black] (-0.5,0.8660254037844389) circle (1.0pt);\draw [fill=black] (-0.75,0.43301270189221974) circle (1.0pt);\draw [fill=black] (1.,0.) circle (1.0pt);\draw [fill=black] (1.5,0.) circle (1.0pt);\draw [fill=black] (2.,0.) circle (1.0pt);\draw [fill=black] (2.5,0.8660254037844387) circle (1.0pt);\draw [fill=black] (2.,1.7320508075688776) circle (1.0pt);\draw [fill=black] (1.,1.7320508075688779) circle (1.0pt);\draw [fill=black] (0.5,0.8660254037844395) circle (1.0pt);\draw [fill=black] (1.75,0.43301270189221935) circle (1.0pt);\draw [fill=black] (1.5,0.8660254037844388) circle (1.0pt);\draw [fill=black] (1.,0.8660254037844389) circle (1.0pt);\draw [fill=black] (0.75,0.43301270189221974) circle (1.0pt);\draw [fill=black] (2.25,0.43301270189221935) circle (1.0pt);\draw [fill=black] (2.25,1.2990381056766582) circle(1.0pt);\draw [fill=black] (1.5,1.7320508075688776) circle (1.0pt);\draw [fill=black] (0.75,1.2990381056766587) circle (1.0pt);\draw [fill=black] (3.5,0.) circle (1.0pt);\draw[fill=black] (4.,0.) circle (1.0pt);\draw [fill=black] (4.5,0.) circle (1.0pt);\draw [fill=black] (5.,0.) circle (1.0pt);\draw [fill=black] (5.75,1.299038105676658) circle (1.0pt);\draw[fill=black] (5.,2.5980762113533165) circle (1.0pt);\draw [fill=black] (3.5,2.598076211353317) circle (1.0pt);\draw [fill=black] (2.75,1.2990381056766593) circle (1.0pt);\draw [fill=black] (5.,0.8660254037844387) circle (1.0pt);\draw [fill=black] (4.5,1.7320508075688776) circle (1.0pt);\draw [fill=black] (3.5,1.7320508075688779) circle (1.0pt);\draw [fill=black] (3.,0.8660254037844395) circle (1.0pt);\draw [fill=black] (4.25,0.43301270189221935) circle (1.0pt);\draw [fill=black] (4.,0.8660254037844388) circle (1.0pt);\draw [fill=black] (3.5,0.8660254037844389) circle (1.0pt);\draw [fill=black] (3.25,0.43301270189221974) circle (1.0pt);\draw [fill=black] (3.25,1.2990381056766587) circle (1.0pt);\draw [fill=black] (4.,0.8660254037844388) circle (1.0pt);\draw [fill=black] (4.,1.7320508075688776) circle (1.0pt);\draw [fill=black] (4.75,1.2990381056766582) circle (1.0pt);\draw [fill=black] (4.75,0.43301270189221935) circle (1.0pt);\draw [fill=black] (5.25,0.4330127018922193) circle (1.0pt);\draw [fill=black] (5.5,0.8660254037844386) circle (1.0pt);\draw [fill=black] (5.5,1.7320508075688752) circle (1.0pt);\draw [fill=black] (5.25,2.1650635094610884) circle (1.0pt);\draw [fill=black] (4.5,2.5980762113533156) circle (1.0pt);\draw [fill=black] (4.,2.5980762113533156) circle (1.0pt);\draw [fill=black] (3.25,2.1650635094611155) circle (1.0pt);\draw [fill=black] (3.,1.7320508075689163) circle (1.0pt);
\end{scope}
\end{tikzpicture}
\end{figure}

As quantidades de pontos em cada figuras são comumente chamado de números poligonais. Assim, \((1,4,9,16, \cdots)\) são números quadrados; \((1,5,12,22, \cdots)\) são números pentagonais; etc.

Nesta atividade, vamos pensar sobre os números triângulares. A imagem a seguir exibe os cinco primeiros:
\begin{figure}[H]
\centering

\begin{tikzpicture}[scale=.6, every node/.style={scale=.6}]
\tikzstyle{circ}=[circle,draw,minimum size=1cm, fill=\currentcolor!80];
\begin{scope}
\node (A) [circ] {};

\end{scope}

\begin{scope}[xshift=1.3cm,node distance=1cm]

\node (A) [circ] {};
\node (B) [circ, right of=A] {};
\node (C) at ($(A)!.5!(B)$) [circ, 
yshift=.86602cm] {};


\end{scope}

\begin{scope}[xshift=3.4cm,node distance=1cm]

\node (A) [circ] {};
\node (B) [circ, right of=A] {};
\node (C) at ($(A)!.5!(B)$) [circ, 
yshift=.86602cm] {};
\node (D) [circ, right of=B] {};
\node (E) [circ, right of=C] {};
\node (F) at ($(C)!.5!(E)$) [circ, yshift=.86602cm] {};


\end{scope}

\begin{scope}[xshift=6.6cm,node distance=1cm]

\node (A) [circ] {};
\node (B) [circ, right of=A] {};
\node (C) at ($(A)!.5!(B)$) [circ, 
yshift=.86602cm] {};
\node (D) [circ, right of=B] {};
\node (E) [circ, right of=C] {};
\node (F) at ($(C)!.5!(E)$) [circ, yshift=.86602cm] {};
\node (G) [circ, right of=D] {};
\node (H) [circ, right of=E] {};
\node (I) [circ, right of=F] {};
\node (J) at ($(F)!.5!(I)$) [circ, yshift=.86602cm] {};


\end{scope}

\begin{scope}[xshift=10.8cm,node distance=1cm]

\node (A) [circ] {};
\node (B) [circ, right of=A] {};
\node (C) at ($(A)!.5!(B)$) [circ, 
yshift=.86602cm] {};
\node (D) [circ, right of=B] {};
\node (E) [circ, right of=C] {};
\node (F) at ($(C)!.5!(E)$) [circ, yshift=.86602cm] {};
\node (G) [circ, right of=D] {};
\node (H) [circ, right of=E] {};
\node (I) [circ, right of=F] {};
\node (J) at ($(F)!.5!(I)$) [circ, yshift=.86602cm] {};
\node (K) [circ, right of=G] {};
\node (L) [circ, right of=H] {};
\node (M) [circ, right of=I] {};
\node (N) [circ, right of=J] {};
\node (O) at ($(N)!.5!(J)$) [circ, yshift=.86602cm] {};


\end{scope}

\end{tikzpicture}
\end{figure}
\begin{enumerate}
\item {} 
Escreva a sequência de números triângulares até o sexto termo.

\item {} 
Os números triangulares formam uma progressão aritmética?

\item {} 
A figura a seguir, destaca as linhas de cada triângulo, uma de cada cor. Escreva o total de bolinhas de \textbf{cada um desses triângulos} como soma das quantidades das suas linhas. Exemplo: \(T_4 = 1 + 2 + 3 + 4\)

\begin{figure}[H]
\centering

\begin{tikzpicture}[scale=.6, every node/.style={scale=.6}]
\tikzstyle{circ}=[circle,draw,minimum size=1cm, fill=\currentcolor!80];
\begin{scope}
\node (A) [circ] {};

\end{scope}

\begin{scope}[xshift=1.3cm,node distance=1cm]

\node (A) [circ] {};
\node (B) [circ, right of=A] {};
\node (C) at ($(A)!.5!(B)$) [circ, 
yshift=.86602cm, fill=session4!80] {};

\end{scope}

\begin{scope}[xshift=3.4cm,node distance=1cm]

\node (A) [circ] {};
\node (B) [circ, right of=A] {};
\node (C) at ($(A)!.5!(B)$) [circ, 
yshift=.86602cm, fill=session4!80] {};
\node (D) [circ, right of=B] {};
\node (E) [circ, right of=C, fill=session4!80] {};
\node (F) at ($(C)!.5!(E)$) [circ, yshift=.86602cm, fill=session2!80] {};

\end{scope}

\begin{scope}[xshift=6.6cm,node distance=1cm]

\node (A) [circ] {};
\node (B) [circ, right of=A] {};
\node (C) at ($(A)!.5!(B)$) [circ, 
yshift=.86602cm, fill=session4!80] {};
\node (D) [circ, right of=B] {};
\node (E) [circ, right of=C, fill=session4!80] {};
\node (F) at ($(C)!.5!(E)$) [circ, yshift=.86602cm, fill=session2!80] {};
\node (G) [circ, right of=D] {};
\node (H) [circ, right of=E, fill=session4!80] {};
\node (I) [circ, right of=F, fill=session2!80] {};
\node (J) at ($(F)!.5!(I)$) [circ, yshift=.86602cm, fill=session3!80] {};

\end{scope}

\begin{scope}[xshift=10.8cm,node distance=1cm]

\node (A) [circ] {};
\node (B) [circ, right of=A] {};
\node (C) at ($(A)!.5!(B)$) [circ, 
yshift=.86602cm, fill=session4!80] {};
\node (D) [circ, right of=B] {};
\node (E) [circ, right of=C, fill=session4!80] {};
\node (F) at ($(C)!.5!(E)$) [circ, yshift=.86602cm, fill=session2!80] {};
\node (G) [circ, right of=D] {};
\node (H) [circ, right of=E, fill=session4!80] {};
\node (I) [circ, right of=F, fill=session2!80] {};
\node (J) at ($(F)!.5!(I)$) [circ, yshift=.86602cm, fill=session3!80] {};
\node (K) [circ, right of=G] {};
\node (L) [circ, right of=H, fill=session4!80] {};
\node (M) [circ, right of=I, fill=session2!80] {};
\node (N) [circ, right of=J, fill=session3!80] {};
\node (O) at ($(N)!.5!(J)$) [circ, yshift=.86602cm,fill=session1!80] {};


\end{scope}

\end{tikzpicture}
\end{figure}
\item Após o item anterior, que relação você percebe entre os números triangulares e o episódio do menino \textit{Gauss}?

\item Com base nessa relação, você seria capaz de determinar o centésimo número triangular? Determine-o.

\item Chamando de \(T_{n}\) o número triangular da posição \(n\), escreva a relação entre \(n\) e \(T_{n}\).
\end{enumerate}

\ifdefined\prof
\begin{solucao}

\begin{enumerate}
\item {} 
\((1,3,6,10,15,21)\)

\item {} 
Não; \(3-1 \neq 6-3 \neq 10-6 \neq 15-10 \neq 21-15\).

\item {} 
\(1\)

\(1+2\)

\(1+2+3\)

\(1+2+3+4\)

\(1+2+3+4+5\)

\item {} 
Um número triangular é soma dos primeiros números naturais, tal como o episódio do menino \textit{Guass}.

\item {} 
Sim; \(T_{100}=1+2+3+ \cdots +98+99+100=5050\).

\item {} 
\(T_{n}= \frac{n \cdot (n+1)}{2}= \frac{n^2}{2} + \frac{n}{2}\).

\end{enumerate}

\end{solucao}
\fi

\end{document}